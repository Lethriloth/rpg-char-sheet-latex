\documentclass[oneside]{article}
\usepackage{array}
\usepackage{amsmath}
\usepackage{mathtools}
\begin{document}
\title{\texttt{rpgcharsheet} manual}
\author{Nathanael Farley}
\maketitle

\section{Introduction}
\label{sec:introduction}

This class is designed to create dynamic RPG character sheets. Currently, a Pathfinder template is enclosed. This template will be used in all examples, with extensions discussed at the end.


\section{Brief Overview}
\label{sec:how-latex-reads}

Almost all numbers in the sheet are held as counters which can be changed with the \verb=\Setcounter= command as per normal. Text based variables are held in commands, and can be changed with \verb=\renewcommand=. A simple \verb=player.tex= would be
\begin{verbatim}
TODO this player.tex
\end{verbatim}

For those who are interested, the sheet is generated in this way:
\begin{enumerate}
\item All values are initalised to 0 or default values from the class file
\item Any changes to these values are read into the template via \verb=\include{player.tex]=.
\item The sheet is rendered.
\end{enumerate}


\section{Using a template}
\label{sec:using-template}

All variables can be changed with \verb=\renewcommand= and \verb=\Setcounter=, however some commands are provided to make things simpler.

\subsection{Naming convention}
\label{sec:naming-convention}

In general, all character variables are prefaced with \verb=char= as in \verb=charstr=, \verb=chardex=. Abilities are abreviated as in table \ref{tab:abbrev}. All others are given full names, without spaces or punctuation (e.g.\ a variable about `Knowledge (Arcana)' would be \verb=charknowledgearcana=). All numbers have the word \verb=count= put after them, demonstrated later.

\begin{table}[h]
  \centering
  \begin{tabular}{c c }
    Name & Abbreviation\\\hline
    Strength & \verb=str=\\
    Dexterity & \verb=dex=\\
    Constitution & \verb=con=\\
    Intelligence & \verb=int=\\
    Wisdom & \verb=wis=\\
    Charisma & \verb=cha=\\\hline
  \end{tabular}
  \caption{Abbreviation for abilities}
  \label{tab:abbrev}
\end{table}

When a feature of an ability/skill is used, such as temprary modifiers, the user simply adds these onto the end of the ability/skill name (e.g.\ `Knowledge (Arcana) Miscellaneous Modifier' would become \verb=\Charknowledgearcanamiscmodcount=. ). These variables are also abbreviated (detailed in table \ref{tab:otherabbrev})

\begin{table}[h]
  \centering
  \begin{tabular}{c c }
    Feature & Abbreviation \\\hline
    Temporary/Temp & \verb=tmp=\\
    Modifier & \verb=mod=\\
    Adjustmuent & \verb=adj=\\\hline
  \end{tabular}
  \caption{Other abreviations}
  \label{tab:otherabbrev}
\end{table}

To change a number, one uses \verb=\Setcounter= in the usual way. A counter name is made of 4 parts:
\protect\parbox{\linewidth}{\texttt{Char}$\langle$\emph{attribute/skill name}$\rangle\langle$\emph{type of modifier}$\rangle$\texttt{count}}
where the attribute/skill name and type of modifier are as above. An example of this is:
\[
\mathtt{\backslash setcounter\{\underbrace{\mathtt{Char}}_{\mathclap{\mathrm{\emph{Char}acter}}}\overbrace{\mathtt{intimidate}}^{\mathclap{\mathrm{\emph{Intimidate}\ Skill}}}\underbrace{\mathtt{miscmod}}_{\mathclap{\mathrm{\emph{Misc}ellaneous\ \emph{Mod}ifier}}}\overbrace{\mathtt{count}}^{\mathclap{\mathrm{\emph{Count}er}}}\}\{5\}}
\]
which would be used to set the character's intimidate skill miscellaneous modifier to $+5$. Totals are automatically calculated wherever a an equals sign is seen.

\section{Conclusion}
\label{sec:conclusion}

TODO this conclusion!
\end{document}

%%% Local Variables: 
%%% mode: latex
%%% TeX-master: t
%%% End: 

    
\DeclareSpell{Apparent Treachery}{enchantment (compulsion) [mind-affecting]|V,  S|1 standard action|close (25 ft. + 5 ft./2 levels)|Targetsone creature/level, no two of which can be more than 30 ft. apart|1 round/level (D)|Will negates|yes}[]
    \DeclareSpellDescription{Apparent Treachery}{You shroud a number of creatures in an aura of suspicion and fill them with extreme paranoia regarding their allies. Affected targets believe their companions are behaving erratically, nervously, and seemingly with an eye towards betrayal.  Creatures under the effect of apparent treachery do not have allies and are not considered to be an ally to any other creature, including other creatures affected by this spell. They cannot move freely through their allies' spaces, flank creatures with them, cooperate with them using teamwork feats, or give or receive benefits from the aid another action or any spells or effect that affects only allies. If creatures affected by this spell are able to take attacks of opportunity, they always do so against provoking opponents, including those who were their allies before being affected by this spell.  A creature not under the effects of the spell who is trying to cast a spell against an affected target must succeed at an attack roll to touch the target, even if the spell is harmless, though the affected creature is not forced to attempt saving throws against harmless effects.}
        
\DeclareSpell{Film Of Filth}{transmutation () [poison]|V,  S|1 standard action|touch|Targetscreature touched|1 round/level (D)|Fortitude negates|yes}[]
    \DeclareSpellDescription{Film Of Filth}{You cause the target's flesh to exude a layer of putrescent slime so foul that the target is sickened (Fortitude negates) for the duration of the spell and for 1d4 rounds thereafter. All creatures within 20 feet also become sickened (Fortitude negates), and remain sickened for as long as they remain within 20 feet of the target and for 1d4 rounds thereafter. A creature that strikes the target with a bite attack must succeed at an additional save or become nauseated for 1d4 rounds. Creatures immune to poison are unaffected.}
        
\DeclareSpell{Lightning Lash}{evocation () [electricity]|V,  S|1 standard action|personal|Area20-foot-radius spread|1 round/level (D)|Fortitude negates (harmless)|yes}[]
    \DeclareSpellDescription{Lightning Lash}{You create a crackling lash of unholy lightning that flickers and flashes in your hand like a whip, shifting color in response to your mood and will. Once per round, you can make a melee touch attack with the lightning lash against a target within 15 feet. If the attack is successful, it deals 1d6 points of electricity damage and 1d6 points of damage from divine power (similar to flame strike), and allows you to attempt a trip combat maneuver check as a free action against your target (using your caster level as your CMB).}
        
\DeclareSpell{Maw Of Chaos}{conjuration (teleportation) [chaotic]|V,  S,  F/DF (a gold-plated,  cold iron ring that was forged in the Abyss)|1 standard action|close (25 ft. + 5 ft./2 levels)|Area5-foot-radius spread|concentration (maximum 1 round/level)|see text|yes; see text}[]
    \DeclareSpellDescription{Maw Of Chaos}{This spell creates a rip in reality that plunges into the interspatial vortices that constantly churn with the raw destructive chaos of the Abyss. Each round at the beginning of your turn, the maw of chaos attempts a drag combat maneuver check against every creature within 40 feet, using your caster level plus your primary spellcasting ability modifier in place of a CMB. If a creature is dragged into a maw of chaos, the area erupts in a surge of chaotic energy and the creature takes 1d6 points of damage per caster level. Only one such eruption can occur per round.  Creatures dragged adjacent to the maw of chaos become entangled by the frayed strands of reality being torn apart at the rim of the maw of chaos. Escape requires a successful Escape Artist check or grapple check against a DC equal to 10 plus the spell's save DC. Every creature without the chaotic subtype that ends its turn adjacent to a maw of chaos takes 2 points of damage to each ability score. Creatures with the lawful subtype take double this amount of damage; creatures with the chaotic subtype take no damage.  Calling, summoning, and teleportation effects used within 30 feet of the maw of chaos or that cause a creature to appear within 30 feet of a maw of chaos are redirected, causing the creature to arrive adjacent to the maw of chaos rather than at its intended destination. Unattended objects (including dead bodies) adjacent to the maw of chaos are drawn into it and affected as by disintegrate at the beginning of the caster's next turn.}
        
\DeclareSpell{Summon Greater Demon}{conjuration (summoning) [chaotic,  evil]|V,  S,  F/DF (a tiny bag and a small candle)|1 round|close (25 ft. + 5 ft./2 levels)|Effectone summoned creature|1 round/level (D)|none|no}[]
    \DeclareSpellDescription{Summon Greater Demon}{This spell functions like summon monster, except it allows you to summon a single coloxus (Pathfinder RPG Bestiary 3 72), an omox demon (Pathfinder RPG Bestiary 2 79), or 1d3 kalavakus demons (Bestiary 2 78).}
        
\DeclareSpell{Summon Lesser Demon}{conjuration (summoning) [chaotic,  evil]|V,  S,  F/DF (a tiny bag and a small candle)|1 round|close (25 ft. + 5 ft./2 levels)|Effectone summoned creature|1 round/level (D)|none|no}[]
    \DeclareSpellDescription{Summon Lesser Demon}{This spell functions like summon monster, except it allows you to summon a single brimorak (Lords of Chaos 56), one incubus (Bestiary 3 73), one thoxel demon (see page 86), 1d3 schir demons (Bestiary 3 74), or 1d4+1 vermlek demons (Lords of Chaos 54).}
        
\DeclareSpell{Unleash Pandemonium}{conjuration () [chaotic]|V,  S|1 standard action|close (25 ft. + 5 ft./2 levels)|Area30-foot-radius spread|concentration (maximum 1 round/level) +1 round (D)|Will partial; see text|no}[]
    \DeclareSpellDescription{Unleash Pandemonium}{You call upon the wild winds of the Abyssal atmosphere, howling with the screams of damned and demented souls in torment. The area is filled with winds of windstorm strength (Core Rulebook 439), blowing in a random direction each round. Creatures within the area of effect are deafened as long as they remain within the area and for 1d4 rounds thereafter; however, they continue to hear the sounds of screams in their minds with painful intensity, causing them to become shaken for as long as they remain deafened. A successful Will save negates the shaken condition but not the deafness.}
        
\DeclareSpell{Vermicious Assumption}{conjuration (calling) [chaotic,  evil]|V,  S,  M (a handful of worms)|10 minutes|touch|Targetsone Medium humanoid corpse|instantaneous|none|no}[]
    \DeclareSpellDescription{Vermicious Assumption}{You call a single vermlek demon (Lords of Chaos 54) to invade and inhabit the body of the target corpse, taking on its likeness. The vermlek can remain on the Material Plane indefinitely as long as it has a body to inhabit; however, if it remains outside of a host for more than 1 minute, it's banished back to the Abyss. The vermlek's initial attitude towards you is friendly, but you must succeed at an opposed Charisma check to convince it to obey your commands, similar to a charmed creature. You gain a +2 circumstance bonus on this Charisma check if you offer it a fresh humanoid corpse to inhabit.}
    
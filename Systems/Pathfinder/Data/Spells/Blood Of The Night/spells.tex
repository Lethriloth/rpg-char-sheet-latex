    
\DeclareSpell{Display Aversion}{illusion (shadow) []|V,  S,  M (a drop of holy water)|1 standard action|personal|Targetsyou|concentration + 1d4 rounds||}[]
    \DeclareSpellDescription{Display Aversion}{This spell functions like minor image, except it always creates an animated illusion of you presenting to a vampire a material, object, or sound that it is averse to, such as garlic, a holy symbol, or bells ringing. You specify what aversion the illusion depicts when you cast the spell. The vampire reacts to the illusion as if it were real; it can overcome the effect by succeeding at a disbelief save or a normal Will save against the illusion's DC (instead of the normal DC 25 to overcome its revulsion). The illusion is only quasi-real and cannot otherwise affect creatures.}
        
\DeclareSpell{Domination Link}{divination () [mind-affecting]|V,  S,  F/DF (a copper piece)|1 standard action|60 ft.|Areacone-shaped emanation|concentration, up to 1 min./level (D)|Will negates; see text|no}[]
    \DeclareSpellDescription{Domination Link}{This spell functions like detect thoughts, with the additional ability to find echoes of the thoughts of a creature mentally controlling the target. For example, if the target has been dominated by a vampire, you can use evidence left in the target's mind to learn about that vampire. Each minute you concentrate on the spell, you can learn your choice of one of the following pieces of information.  Direction: The controller's general direction and distance.  Emotion: The controller's emotional state (gloating, sated, frightened, angry, and so on).  Image: A powerful iconic image relevant to the controller or its connection to the target, such as a symbol on a door or a name on a gravestone.  Location: The controller's general location, such as "in a large city" or "on a ship."  Name: The name by which the target knows its controller (if any).  All of this information is based on the last time the influencing creature linked itself to the target, either to issue a command or to receive sensory input from the target. For example, if at nightfall a vampire commanded a dominated victim to walk to a cemetery, this spell can reveal the vampire's general location at that time, though it may have moved since then.}
        
\DeclareSpell{Project Weakness}{necromancy () [curse,  evil]|V,  S|1 standard action|touch|Targetsliving creature touched|permanent|Will negates|yes}[]
    \DeclareSpellDescription{Project Weakness}{You curse the target with the weaknesses of your kind of vampirism. The creature reacts to garlic, mirrors, ringing bells, sunlight, and so on as if it were a vampire of the same type as you. This cannot kill the target; anything that would kill it (such as a lengthy exposure to sunlight if you are a moroi vampire) renders it helpless until the curse or the harmful effect is removed. The target gains none of the benefits of being a vampire (such as fast healing or requiring special ways to be permanently killed), only the penalties.}
        
\DeclareSpell{Steal Years}{transmutation () []|V,  S,  M (a handful of ash)|1 standard action|touch|Targetscreature touched|24 hours|Fortitude negates|yes}[]
    \DeclareSpellDescription{Steal Years}{You temporarily drain youth and vitality from the target and channel it into yourself. If the target fails its Fortitude save, it physically ages 1d4 years per two caster levels (maximum 5d4), and you decrease your age by the same number of years. If this changes the age category of you or the target, only adjust physical ability scores. This effect cannot bring your age to lower than the minimum age of adulthood for your race (see page 169 of the Core Rulebook). This stolen youth does not actually change your age or prolong your life; you will still die at your allotted time, no matter how youthful you appear. Likewise, the spell does not add to the target's true age, and cannot make the target die of old age.  When the spell ends, the sudden weight of aging makes you fatigued for 1d4 hours.}
        
\DeclareSpell{Steal Years, Greater}{transmutation () []|V,  S,  M (a handful of ash)|1 standard action|touch|Targetscreature touched|1 day/level|Fortitude negates|yes}[]
    \DeclareSpellDescription{Steal Years, Greater}{This spell functions like steal years, except you drain 1d6 years per two caster levels (maximum 10d6).}
        
\DeclareSpell{Transmute Wine To Blood}{transmutation () []|V,  S,  M (drop of animal blood)|1 standard action|touch|Targetsbottle of wine worth at least 10 gp|instantaneous|Fortitude negates (object)|yes (object)}[]
    \DeclareSpellDescription{Transmute Wine To Blood}{You transform one bottle of fine wine into 1 pint of animal blood, sufficient for a creature with the blood drain ability to feed upon as if it came from a Medium animal with 1 Hit Die. If you are using the optional hunger rules (see page 22), this blood satiates an undead creature's hunger, negating any withdrawal effects, but does not grant the creature a feeding bonus. The blood coagulates and spoils at the normal rate.}
    
    
\DeclareSpell{Bone Fists}{necromancy|V,  S,  M (the knucklebone of a dire animal)|1 standard action|close (25 ft. + 5 ft./2 levels)|Targets: 1 creature/level, no two of which can be more than 30 feet apart|1 minute/level|none (harmless)|no}[]
    \DeclareSpellDescription{Bone Fists}{The bones of your targets' joints grow thick and sharp, protruding painfully through the skin at the knuckles, elbows, shoulders, spine, and knees. The targets each gain a +1 bonus to natural armor and a +2 bonus on damage rolls with natural weapons, and they are treated as having armor spikes, with which they are proficient.}
        
\DeclareSpell{Flash Forward}{conjuration (teleportation)|V,  S,  F (a single gear or bit of clockwork)|1 standard action|personal|Targets: you|instantaneous|none (harmless)|no}[]
    \DeclareSpellDescription{Flash Forward}{You cheat the laws of time and enter into combat before reverting back to your original position. As part of the action to cast the spell, you make a charge attack against an enemy. You make this charge attack normally, accounting for terrain,  obstacles, attacks of opportunity, attack rolls, and damage rolls. At the end of your charge action, you instantly teleport back to your original location as a free action. Any damage or conditions dealt by you or to you during this action are real and remain when you return to your original location.}
        
\DeclareSpell{Particulate Form}{transmutation|V,  S,  M (a pinch of fine sand)|1 standard action|close (25 ft. + 5 ft./2 levels)|Targets: 1 creature/level, no two of which can be more than 30 feet apart|1 round/level (D)|none (harmless)|no}[]
    \DeclareSpellDescription{Particulate Form}{The targets' physical forms undergo a bizarre transformation. They look and function normally, but are composed of countless particles that separate and reconnect to remain whole. Each target gains fast healing 1 and is immune to bleed damage, critical hits, sneak attacks, and other forms of precision damage. The value of this fast healing increases by 1 at caster levels 10th, 15th, and 20th. Any target can end the spell effect on itself as a swift action; the target then regains 5d6 hit points and can attempt an additional saving throw against any one disease or poison affecting it (at the original save DC), ending that disease or poison with a successful saving throw.}
        
\DeclareSpell{Phasic Challenge}{transmutation|V,  S,  M (a scrap of a knight's banner)|1 standard action|short (25 ft. + 5 ft./2 levels)|Targets: two creatures within 60 feet of one another; see text|1 round/level (D)|Will negates (see below)|yes}[]
    \DeclareSpellDescription{Phasic Challenge}{You select one enemy and one willing ally as targets. The enemy gains a new Will saving throw at the beginning of each turn, and on a successful saving throw the spell ends. Both targets remain visible and audible, and can see and hear other creatures, but cannot physically interact with any creature save one another. Spells or weapon attacks from the affected creatures impact only each other, though spells might affect terrain or other factors not related to other creatures. If one of the creatures becomes unconscious or dies, or if the effect is dispelled, the effect ends for both of them.}
        
\DeclareSpell{Spellcurse}{necromancy [curseUM]|V,  S,  M (a fragment of a destroyed magical item)|1 standard action|medium (100 ft. + 10 ft./level)|Targets: one creature|instantaneous|Will half|yes}[]
    \DeclareSpellDescription{Spellcurse}{You disrupt any spell energy affecting your target, causing that energy to crackle with power and harm the target. The target takes 1d6 points of damage for each spell with a duration of 1 round or greater currently affecting it. The spells themselves are not dispelled or modified.}
        
\DeclareSpell{Warp Metal}{transmutation [earth]|V,  S,  M (a twisted wire)|1 standard action|close (25 ft. plus 5 ft./2 levels)|Targets: 1 Small wooden object/level, all within a 20-ft. radius; see text|instantaneous|Will negates (object)|yes (object)}[]
    \DeclareSpellDescription{Warp Metal}{You cause metal to bend and warp, permanently destroying its straightness, form, and strength. A warped door springs open (or becomes stuck, requiring a successful Strength check to open, at your option). A vehicle grinds to a halt and ceases to operate. Warped ranged weapons are useless. A warped melee weapon imposes a -4 penalty on attack rolls.  You can warp one Small or smaller object or its equivalent per caster level. A Medium object counts as two Small objects, a Large object as four, a Huge object as eight, a Gargantuan object as 16, and a Colossal object as 32.  Alternatively, you can unwarp metal (effectively warping it back to normal) with this spell. Make whole, on the other hand, does nothing to repair a warped item.  You can combine multiple consecutive castings of warp metal to warp (or unwarp) an object that is too large for you to warp with a single casting. Until the object is completely warped, it suffers no ill effects.}
        
\DeclareSpell{Biting Words}{evocation [language-dependent,  sonic]|V,  S|1 standard action|personal|Targets: you|1 minute/level|Will negates (harmless)|no}[]
    \DeclareSpellDescription{Biting Words}{Your voice becomes suffused with magic so that you can harm your opponents with but a word. As a standard action, you can target one  opponent within 30 feet with a ranged touch attack by speaking to it, dealing an amount of damage equal to 1d6 + your Strength or Charisma modifier, whichever is higher. The damage dealt is bludgeoning, piercing, and slashing damage and can be reduced by damage reduction. Each attack you make reduces the spell's remaining duration by 1 minute. If an attack reduces the remaining duration to 0 minutes or less, the spell ends after the attack resolves.  You can attack with biting words by shouting instead of speaking intelligibly. When doing so, the amount of damage dealt by the attack is reduced to 1d4 + 1/2 your Strength or Charisma modifier (whichever is higher), but the spell doesn't count as having the language-dependent descriptor for that attack.}
        
\DeclareSpell{Bouncing Bomb Admixture}{transmutation|V,  S|1 swift action|personal|Targets: you|1 round|Will negates (harmless)|no}[]
    \DeclareSpellDescription{Bouncing Bomb Admixture}{Upon drinking an extract created with this formula, you make a significant change to your magical reserve that modifies the nature of the next bomb you create and throw during this extract's duration. This effect on your magical reserve has no effect on any discoveries that you use to modify your bombs, but you can only have one admixture effect (a formula with "bomb admixture" in its title) active at a time. If you drink another bomb admixture, the effects of the former bomb admixture end and the new one becomes active.  When you throw your next bomb, choose one target that would normally be hit by the bomb's splash damage. The target is affected as if it suffered a direct hit from the bomb instead.}
        
\DeclareSpell{Release The Hounds}{conjuration (summoning)|V,  S,  M/DF (shards of a canine's fang)|1 standard action|close (25 ft. + 5 ft./2 levels)|Effect: one pack of canines|1 round/level (D)|none|no}[]
    \DeclareSpellDescription{Release The Hounds}{This spell summons a pack of canines that respond to the spellcaster's commands and act in perfect unison, causing them to function like a swarm. The pack uses the statistics for a winter wolf (Pathfinder RPG Bestiary 280), except it loses its breath weapon, the cold subtype, and its cold special attack and gains the swarm subtype, a swarm attack that deals 3d6 points of damage, and the distraction special attack (DC 17). Whenever the pack damages an opponent with its swarm attack, it can immediately attempt a trip combat maneuver check against that creature with its trip special attack. The pack does not gain any damage reduction or immunity to damage and can be attacked by effects that target a specific number of creatures, though such attacks deal 1/4 the normal amount of damage (25\%) and effects that don't deal hit point damage are only 25\% likely to work.}
        
\DeclareSpell{Roaming Pit}{conjuration (creation)|V,  S,  M (powered diamond dust worth 10 gp)|1 standard action|medium (100 ft. + 10 ft./level)|Effect: mobile 10-ft.-by-10-ft. hole, 10 ft. deep/2 levels|1 round/level|Reflex negates|no}[]
    \DeclareSpellDescription{Roaming Pit}{This spell functions as create pitAPG, except the pit is capable of movement. As a move action, you can direct the pit to move up to 20 feet, though it must always remain on a horizontal surface large enough to accommodate its area. If the pit's movement causes it to share a space with a creature on the same horizontal surface, that creature must succeed at a Reflex saving throw or fall into the pit. Any creature that avoids falling into the pit when it reaches its new destination moves to the nearest safe space. Creatures that fall into the pit move with it if it is relocated.}
        
\DeclareSpell{Wall Of Bone}{necromancy|V,  S,  M (a polished humanoid femur)|1 standard action|close (25 ft. + 5 ft./2 level)|Effect: solid wall of humanoid bones with an area of up to one 5-ft. square/level|1 minute/level|none|yes; see text}[]
    \DeclareSpellDescription{Wall Of Bone}{This spell creates a vertical wall of skeletal arms that attaches itself to any solid surface. The wall of bone works identically to wall of stone e xcept a s n oted a bove a nd i n t his s pell description. The wall of bone is 1 inch thick per 4 caster levels and composed of up to one 5-foot square per level. The wall created must be vertical, and must rest upon a firm foundation. It cannot be used to bridge a chasm, for instance, or to act as a ramp. Each 5-foot square of the wall has hardness 4 and 7 hit points per inch of thickness. A section of wall whose hit points drops to 0 is breached. If a creature tries to break through the wall with a single attack, the DC of the Strength check is equal to 15 + 2 per inch of thickness.  For each creature adjacent to the wall, the skeletal hands attempt a combat maneuver check to grapple it. The skeletal hands do not provoke attacks of opportunity. They make their attacks at the start of your turn, when the wall is summoned, or when an enemy first moves adjacent to the wall. The skeletal hands' CMB is equal to your caster level, and they can grapple a Huge or smaller creature with no penalty.  If the wall successfully grapples a foe, that foe takes 1d6 points of damage and gains the grappled condition. Grappled opponents cannot move without first breaking the grapple. All other movement is prohibited unless the creature breaks the grapple first. The wall cannot move or pin foes. Each round the wall succeeds at a grapple combat maneuver check, it deals 1d6 additional points of damage. The CMD of the wall, for the purposes of escaping the grapple, is equal to 10 + its CMB.}
        
\DeclareSpell{Bind Sage}{conjuration (calling)|V,  S|10 minutes|close (25 ft. + 5 ft./2 levels); see text|Targets: one caulborn (Pathfinder RPG Bestiary 3 48)|instantaneous|Will negates|no and yes; see text}[]
    \DeclareSpellDescription{Bind Sage}{This variant of planar binding is specifically used to call one of the most knowledgeable types of outsiders: the immortal and prophetic caulborn (Pathfinder RPG Bestiary 3 48). This spell calls a single caulborn into a specially prepared trap. The caster of this spell can compel a bound caulborn only to provide information (using its Knowledge skills or detect thoughts); attempts to compel the caulborn to fight, guard a location, or perform some other task automatically fail.  This spell otherwise functions as planar binding. The most effective gifts for a caulborn are unique books or intelligent creatures upon whose thoughts the caulborn can feed. The true names of caulborn are nearly impossible to discover, as they are not inherently individualistic, but certain reclusive scholar communities in Kaer Maga may know titles by which specific caulborn can be called.}
        
\DeclareSpell{Secluded Grimoire}{conjuration (summoning)|V,  S|1 round|touch|Targets: spellbook touched|instantaneous|none|no}[]
    \DeclareSpellDescription{Secluded Grimoire}{This spell sends a spellbook into a random but safe location on the Ethereal Plane, where it remains indefinitely. When you cast this spell, the target spellbook dissolves into quickly fading lines of energy and runes that reflect all of the spells stored within. Thereafter, you can retrieve the spellbook by concentrating as a standard action, causing it to reappear in your hands. You cannot cast this spell on another spellbook if you currently have a spellbook in the Ethereal Plane. No other creatures or objects accompany the spellbook you send away, and the duration of any spells currently affecting the spellbook continue normally.}
        
\DeclareSpell{Alaznist's Jinx}{evocation [curseUM]|V,  S|1 standard action|touch|Targets: creature touched|permanent|Will negates|yes}[]
    \DeclareSpellDescription{Alaznist's Jinx}{You inflict a curse similar to the spell burn spellblight (Pathfinder RPG Ultimate Magic 97) on a creature. Each time a spellcaster who is afflicted with this curse casts a spell or uses a spell-like ability, her skin seems to burn as though she were on fire. With a successful concentration check (DC = 15 + double the spell level of the spell cast or spell-like ability used), the spellcaster can ignore the pain of the effect, but if she fails, she loses the spell or spell slot and is staggered for a round.  Unlike with the spell burn spellblight, the burning sensation is a tangible effect, visible during the act of spellcasting as an incorporeal, violet flame surrounding the caster.}
        
\DeclareSpell{Flexile Curse}{transmutation [curseUM]|V,  S|1 standard action|touch|Targets: creature touched|permanent|Will negates|yes}[]
    \DeclareSpellDescription{Flexile Curse}{You curse your target with a withering aura that degrades its armor and shield (if any). This reduces the hardness, armor bonus, and enhancement bonus of any armor or shield worn by the target by 1. For every hour the target wears a suit of armor or a shield, the hardness, armor bonus, and enhancement bonus are reduced by an additional 1. If the armor or shield's hardness is reduced to 0, anytime the target of the curse is struck while wearing that item, there is a 20\% chance that it gains the broken condition. If the enhancement bonus of magic suit of armor or shield is reduced to 0, the armor or shield loses any other special abilities it had.  If the target removes the suit of armor or shield (even if it's broken or has been drained of all enhancement bonuses), the armor or shield regains its hardness, armor bonus, and enhancement bonus at the rate of 1 per 2 hours. A suit of armor or shield broken by this curse that regains its full hardness ceases to be broken. A magic suit of armor or shield that lost its special abilities regains them when its enhancement bonus is fully restored.}
        
\DeclareSpell{Irregular Size}{transmutation [curseUM]|V,  S|1 standard action|touch|Targets: creature touched|permanent|Fortitude negates|yes}[]
    \DeclareSpellDescription{Irregular Size}{You curse a creature so one set of its limbs (typically its arms, legs, or wings) shrivels in size.  Arms: The creature counts as one size category smaller for the purpose of determining the size of weapon it can wield. If the creature is capable of making natural attacks with its arms, the damage dealt by those attacks decreases as though the target were one size category smaller than its actual size.}
        
\DeclareSpell{Itching Curse}{necromancy [curseUM]|V,  S|1 standard action|close (25 ft. + 5 ft./2 levels)|Targets: one living creature with 5 HD or fewer|1 hour/level (D)|Will negates|yes}[]
    \DeclareSpellDescription{Itching Curse}{You curse the target with a distracting, unbearable itch. Unless the target scratches as a move action, it takes a -1 penalty on attack rolls, saving throws, skill checks, and ability checks.}
        
\DeclareSpell{Kalistocrat's Nightmare}{transmutation [curseUM]|V,  S,  M (a copper piece)|1 standard action|touch|Targets: creature touched|1 hour/level (see text)|Will negates|yes}[]
    \DeclareSpellDescription{Kalistocrat's Nightmare}{You temporarily curse a creature so its touch lessens the value of coins it touches. While under the effects of this curse, whenever the target touches a coin of higher value than copper piece, that coin changes into a copper piece. The change takes place over the course of the following minute, allowing the target to interact with multiple coins before the effect of the curse becomes apparent. The affected coins are permanently transmuted from their previous material (typically gold or silver) into copper coins, though remove curse (which can affect up to 50 coins with a single casting) or a similar spell can restore them to their previous material.}
        
\DeclareSpell{Lost Legacy}{enchantment [curseUM,  mind-affecting]|V,  S,  DF/F (a holy symbol of a dead or forgotten deity,  or flag of a fallen or forgotten nation)|1 standard action|touch|Targets: creature touched|permanent|Will negates|yes}[]
    \DeclareSpellDescription{Lost Legacy}{You inflict a powerful curse on a touched creature that causes others to quickly forget positive aspects of their interactions with the target. The target cannot attempt a Diplomacy (or wild empathy or similar ability) check to improve the attitude of other creatures. Any creature that normally has an attitude of friendly or helpful toward the target must succeed at a Will save at the spell's normal saving throw DC each time it interacts with the target, or its attitude becomes indifferent. Once this curse is removed, creatures coming back into contact with the target regain their pleasant memories of it. If their attitudes toward the target have not been reduced from indifferent, their original attitudes are restored.}
        
\DeclareSpell{Earsend}{necromancy|V,  S,  M (butterfly wing)|1 standard action|close (25 ft. + 5 ft./2 levels)|Targets: creature touched|10 minutes/level|Will negates (harmless)|no}[]
    \DeclareSpellDescription{Earsend}{You cause one of your ears to tear itself free of your body and transform into a fly-like magical creature you control. This functions like skinsendUM, except your ear is a Fine construct with a fly speed equal to your base speed and a bonus on Fly checks equal to half your caster level. Your sense of hearing functions from your animated ear as if it were connected to your head, allowing you to hear as well as you normally could from your animated ear's vantage point.}
        
\DeclareSpell{Hidden Blades}{illusion (glamer)|V,  S,  M (a shard of glass)|1 standard action|touch|Targets: weapon or ammunition touched; see text|10 minutes/level|Will negates (harmless, object)|yes (harmless, object)}[]
    \DeclareSpellDescription{Hidden Blades}{You render a target weapon or up to 50 pieces of ammunition invisible, granting the wielder a +20 bonus on Sleight of Hand checks made to conceal the weapon or ammunition and a +5 circumstance bonus on Bluff checks to feint with the weapon or ammunition.}
        
\DeclareSpell{Impenetrable Veil}{abjuration|V,  S,  M (dust from the Dimension of Dreams worth 1, 250 gp)|1 standard action|touch|Targets: creature touched|10 minutes/level|Will negates (harmless)|yes (harmless)}[]
    \DeclareSpellDescription{Impenetrable Veil}{You enchant the target so it is nearly impossible to detect by both magical and mundane means, granting it the following benefits. The target gains a bonus on Stealth checks equal to half  your caster level, and can use Stealth to hide from all creatures attempting to perceive it, even when it lacks concealment or cover. It can attempt a Stealth check to avoid detection from creatures using blindsight, blindsense, or any ability that functions as either (such as lifesense or tremorsense). Furthermore, the target leaves no trail and cannot be tracked unless it chooses to leave a trail. If a creature attempts to discern the target's presence or location using divination magic (including magic items with a divination aura such as a crystal ball), that creature must succeed at a caster level check with a DC equal to the target's Stealth bonus to discern any information about the target, and on a failed check cannot do so for the spell's duration.}
        
\DeclareSpell{Innocuous Shape}{transmutation (polymorph)|V,  S,  M (a handful of dandelion seeds)|1 standard action|touch|Targets: living creature touched|1 minute/level (D)|Will negates (harmless)|yes (harmless)}[]
    \DeclareSpellDescription{Innocuous Shape}{This spell transforms a creature into a Medium or smaller animal or humanoid of no more than 1 Hit Die. If you use this spell to cause the target to take on the form of an animal, the spell functions as beast shape II. If the form is that of a humanoid, the spell can function as alter self, youthful appearanceUM, or both, such that you can transform a creature into a generic, youthful humanoid of any type.  Any creature that interacts with the target of this spell must succeed at a Will saving throw (using the DC of the spell) or view the target in the most innocuous possible light. On a failed saving throw, the creature views all of the target's actions as inoffensive and no cause for concern unless the target becomes an obvious threat. The creature can otherwise act normally, and feels no compunction to obey or ignore the target; it simply assumes none of the target's actions are dangerous or malicious unless shown evidence that proves otherwise.}
        
\DeclareSpell{Lesser Nondetection}{abjuration|V,  S,  M (a drop of mercury)|1 standard action|close (25 ft. + 5 ft./2 levels)|Targets: creature touched|1 minute/level|Will negates (harmless)|no}[]
    \DeclareSpellDescription{Lesser Nondetection}{This spell functions like nondetection, except it blocks the effects of only divination spells and effects that target an area, rather than you or an object in your possession specifically. For example, lesser nondetection doesn't ward you against a spellcaster who is attempting to scry on you with the scrying spell or find an object in your possession with locate object, but it does protect  you against effects that target an area you happen to be in, such as detect spells or clairaudience/clairvoyance.\\\\

{\centering\bf Nondetection\hrule}

The warded creature or object becomes difficult to detect by divination spells such as clairaudience/clairvoyance, locate object, and detect spells. Nondetection also prevents location by such magic items as crystal balls. If a divination is attempted against the warded creature or item, the caster of the divination must succeed on a caster level check (1d20 + caster level) against a DC of 11 + the caster level of the spellcaster who cast nondetection. If you cast nondetection on yourself or on an item currently in your possession, the DC is 15 + your caster level.

If cast on a creature, nondetection wards the creature's gear as well as the creature itself.}
        
\DeclareSpell{Phantasmal Reminder}{illusion (phantasm) [mind-affecting]|V,  S|1 standard action|medium (100 ft. + 10 ft./level)|Targets: one living creature|1 round (see text)|Will disbelief, then Fortitude partial; see text|yes}[]
    \DeclareSpellDescription{Phantasmal Reminder}{You create a memory loop of a successful attack made against the target, forcing its conscious mind to recall the details of the attack in such excruciating detail that its physical body is racked by the recollection. This spell can affect only a creature that has taken damage since the end of your last turn. The target first can attempt a Will save to recognize the attack as unreal. If it fails that saving throw, the target must succeed at a Fortitude save or take an amount of damage equal to 1d6 x your caster level (maximum 10d6). Because the damage is a quasi-real memory of existing wounds, this damage can't be reduced or prevented (such as by the shield other spell).  If the target of a phantasmal reminder attack succeeds at the Will save to disbelieve the memory loop and either has natural telepathy or is wearing a helm of telepathy, the memory of damage automatically rebounds to affect you. You must immediately attempt a Will save to disbelieve; if you fail, you take half the spell's damage yourself.}
        
\DeclareSpell{Symbol Of Distraction}{enchantment (compulsion) [mind-affecting]|V,  S,  M (mercury and phosphorus,  plus powdered diamond and opal worth a total of 5, 000 gp)|10 minutes|0 feet; see text|Effect: one symbol|see text|Will negates|yes}[]
    \DeclareSpellDescription{Symbol Of Distraction}{This spell functions like symbol of death, except that all creatures within the radius of a symbol of distraction instead become fascinated by the symbol for 10 minutes per caster level. Unlike symbol of death, symbol of distraction has no hit point limit; once triggered, a symbol of distraction simply remains active for a duration of 10 minutes x your caster level. All fascinated creatures move toward the symbol of distraction, trying to remain within the symbol's area of effect. If the symbol leads affected creatures into a dangerous area, each fascinated creature can attempt an additional Will saving throw, with success indicating it is no longer fascinated. If a creature's view of the symbol is completely blocked, it is immediately freed of the symbol's effect.\\\\

{\centering\bf Symbol Of Death\hrule}

This spell allows you to scribe a potent rune of power upon a surface.

When triggered, a symbol of death kills one or more creatures within 60 feet of the symbol (treat as a burst) whose combined total current hit points do not exceed 150. The symbol of death affects the closest creatures first, skipping creatures with too many hit points to affect.

Once triggered, the symbol becomes active and glows, lasting for 10 minutes per caster level or until it has affected 150 hit points' worth of creatures, whichever comes first. A creature that enters the area while the symbol of death is active is subject to its effect, whether or not that creature was in the area when it was triggered. A creature need save against the symbol only once as long as it remains within the area, though if it leaves the area and returns while the symbol is still active, it must save again.

Until it is triggered, the symbol of death is inactive (though visible and legible at a distance of 60 feet). To be effective, a symbol of death must always be placed in plain sight and in a prominent location. Covering or hiding the rune renders the symbol of death ineffective, unless a creature removes the covering, in which case the symbol of death works normally.

As a default, a symbol of death is triggered whenever a creature does one or more of the following, as you select: looks at the rune; reads the rune; touches the rune; passes over the rune; or passes through a portal bearing the rune. Regardless of the trigger method or methods chosen, a creature more than 60 feet from a symbol of death can't trigger it (even if it meets one or more of the triggering conditions, such as reading the rune). Once the spell is cast, a symbol of death's triggering conditions cannot be changed.

In this case, "reading" the rune means any attempt to study it, identify it, or fathom its meaning. Throwing a cover over a symbol of death to render it inoperative triggers it if the symbol reacts to touch. You can't use a symbol of death offensively; for instance, a touch-triggered symbol of death remains untriggered if an item bearing the symbol of death is used to touch a creature. Likewise, a symbol of death cannot be placed on a weapon and set to activate when the weapon strikes a foe.

You can also set special triggering limitations of your own. These can be as simple or elaborate as you desire. Special conditions for triggering a symbol of death can be based on a creature's name, identity, or alignment, but otherwise must be based on observable actions or qualities. Intangibles such as level, class, HD, and hit points don't qualify.

When scribing a symbol of death, you can specify a password or phrase that prevents a creature using it from triggering the symbol's effect. Anyone using the password remains immune to that particular rune's effects so long as the creature remains within 60 feet of the rune. If the creature leaves the radius and returns later, it must use the password again.

You also can attune any number of creatures to the symbol of death, but doing this can extend the casting time. Attuning one or two creatures takes negligible time, and attuning a small group (as many as 10 creatures) extends the casting time to 1 hour. Attuning a large group (as many as 25 creatures) takes 24 hours. Attuning larger groups takes an additional 24 hours per 25 creatures. Any creature attuned to a symbol of death cannot trigger it and is immune to its effects, even if within its radius when it is triggered. You are automatically considered attuned to your own symbols of death, and thus always ignore the effects and cannot inadvertently trigger them.

Read magic allows you to identify a symbol with a Spellcraft check (DC 10 + the symbol's spell level). Of course, if the symbol is set to be triggered by reading it, this will trigger the symbol.

A symbol of death can be removed by a successful dispel magic targeted solely on the rune. An erase spell has no effect on a symbol of death. Destruction of the surface where a symbol of death is inscribed destroys the symbol but also triggers it.

Symbol of death can be made permanent with a permanency spell.

A permanent symbol of death that is disabled or has affected its maximum number of hit points becomes inactive for 10 minutes, but then can be triggered again as normal.

Note: Magic traps such as symbol of death are hard to detect and disable. A rogue (only) can use the Perception skill to find a symbol of death and Disable Device to thwart it. The DC in each case is 25 + spell level, or 33 for symbol of death.}
        
\DeclareSpell{Touch Of Slumber}{enchantment (compulsion) [mind-affecting]|V,  S,  M (a bit of seaweed)|1 standard action|touch|Targets: nonhostile creature touched|instantaneous|Will negates|yes}[]
    \DeclareSpellDescription{Touch Of Slumber}{This spell functions only against a creature with an attitude toward you of indifferent or better, and only if the target is not hostile toward you or your visible allies. A target touched by you (this touch usually requires a successful melee touch attack) must succeed at a Will saving throw or fall asleep for 2d6 hours. While asleep, the creature is helpless. For 1 minute after the creature is affected, any loud noise or rough contact causes it to awaken immediately. Thereafter, slapping or wounding an affected creature awakens it, but normal noise does not. Awakening a creature is a standard action (this is an application of the aid another action).}
    
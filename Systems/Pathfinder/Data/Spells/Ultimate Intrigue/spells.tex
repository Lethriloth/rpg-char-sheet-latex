    
\DeclareSpell{Absolution}{abjuration () []|V,  S,  M (a vial of holy water),  DF|1 round|touch|Targetsliving creature touched|instantaneous|none|no}[Removes enchantments and forgives actions taken under their effects.]
    \DeclareSpellDescription{Absolution}{You purge impure thoughts from the target's mind and fill him with exultant relief at the forgiveness of his sins. Absolution ends all charm or compulsion effects affecting the target (including harmless compulsions, such as heroism) as per break enchantment. If the target was forced to perform any actions contrary to his alignment, monk vows, paladin oath, or similar code of conduct by that charm or compulsion effect, that action doesn't cause him to lose access to class abilities, including divine spellcasting.  Unlike an atonement spell, absolution can't reverse alignment change or the effects of willing transgressions, induce a creature to change its alignment, or restore class abilities lost because of misdeeds performed in the past. Absolution automatically works if the caster and the target share the same alignment or the same patron deity. If they don't, but their alignments are within one step of each other, absolution has a 5\% chance of success per caster level. If neither of these is true, the spell automatically fails.  If using the honor subsystem (Pathfinder RPG Ultimate Campaign 160), casting absolution also eliminates the honor loss for events and actions committed by the target while he was affected by a charm or compulsion effect that the spell ended.}
        
\DeclareSpell{Aerial Tracks}{divination () [air]|V,  S|1 standard action|medium (100 ft. + 10 ft./level)|Areacircle centered on you, with a radius of 100 feet + 10 feet per level|1 hour/level|none|no}[Track f lying creatures through the air.]
    \DeclareSpellDescription{Aerial Tracks}{You cause the air in the area to ripple where creatures have flown through it up to 1 day ago per caster level. These aerial ripples are tinged by glowing wisps, providing enough illumination to follow the tracks without penalties due to poor lighting. The area moves with you, allowing you to follow the tracks through the air over long distances provided you can fly or follow the route along the ground within range to read the aerial tracks. Other creatures can also follow the trail as long as they move with you. The base DC of Survival checks to track creatures through the air with this spell is the same as tracking creatures across soft ground.}
        
\DeclareSpell{Animal Ambassador}{enchantment (compulsion) [mind-affecting]|V,  S,  M (a morsel of food the animal likes)|10 minutes|close (25 ft. + 5 ft./2 levels)|Targetsone Medium or smaller animal|1 day/level or until message is delivered|none (see text)|no}[Grant an animal messenger sentience to deliver your message.]
    \DeclareSpellDescription{Animal Ambassador}{You compel a single animal to travel to a spot you designate and deliver a message to a creature you identify. This spell is similar to animal messenger, but can affect larger animals. In addition, the target animal is temporarily awakened to sentience (as the awaken spell) for the duration of this spell, and it can use its increased mental acuity to come up with creative solutions to overcome obstacles to delivering its message. The awakened target animal speaks any one language you know. In addition, you can imbue the animal with up to 5 ranks in any of the following skills: Bluff, Diplomacy, Knowledge (local), Knowledge (nobility), Linguistics, or Sense Motive. It treats any of these skills that are class skills for you as class skills. The number of ranks you imbue in any of these skills can't exceed the target animal's Hit Dice nor the number of ranks you possess in that skill.  The message you send with your animal ambassador can be a verbal message, which the awakened target animal can speak using its own voice or deliver using your voice. It can engage in conversation using its own intellect, knowledge, and linguistic abilities. The animal ambassador is loyal to you, but it is otherwise susceptible to mind-affecting effects like any other creature. You can also send your animal ambassador with an object or container that is within its physical ability to carry, making it deliver the contents to the target of your intended message. If the object it is given to carry is poisonous, trapped, or otherwise inherently dangerous (even if it would normally be dangerous only to the creature receiving it rather than the animal), or if you or your allies attack the target animal, the animal ambassador spell fails and the animal becomes hostile toward you. The animal ambassador intelligently but single-mindedly attempts to deliver the message to its intended target, and you can't task it with other tasks like fighting, scouting, searching for traps, and so on. It leaves on its delivery once it receives its message.  Once the animal ambassador has located its target and delivered its message (and object, if desired), its enhanced mental abilities fade within 2d6 minutes. The spell then ends, even if its duration hasn't elapsed.}
        
\DeclareSpell{Aphasia}{enchantment (compulsion) [mind-affecting]|V,  S|1 standard action|close (25 ft.+ 5 ft./2 levels)|Targetsone creature|1 round/level|Will negates (see text)|yes}[Prevent a target from understanding language.]
    \DeclareSpellDescription{Aphasia}{You render the target unable to understand any language, including spoken language, written language, sign language, gestures attempting to mimic a crude language, or even truespeech and telepathy. The affected creature is unable to communicate, use command words, cast spells with verbal components, or use any other abilities that requires language.  At the end of each of its turns, the subject can attempt a new saving throw to end the effect.  Tongues counters and dispels aphasia, and a creature with the tongues spell active is immune to aphasia. Oracles with the tongues curse ignore aphasia in combat.}
        
\DeclareSpell{Auditory Hallucination}{illusion (phantasm) [mind-affecting]|S|1 standard action|long (400 ft. + 40 ft./level)|Targetsone creature/level, no two of which can be more than 30 ft. apart|concentration|Will disbelief|yes}[Create a phantasm with auditory effects.]
    \DeclareSpellDescription{Auditory Hallucination}{You cause the targets to believe they hear any sound you imagine. The sound can include intelligible speech. Instead of precisely imagining a sound, you can identify a sound the subjects know and they imagine it doing what you describe as you cast the spell. For example, you could cast this spell on orc warriors and have them imagine the sound of their chieftain calling for help, even if you've never heard their chieftain and even if the chieftain speaks in a language you don't understand. All targets hear the same hallucination. You can change the sound as part of concentrating on the spell.}
        
\DeclareSpell{Audiovisual Hallucination}{illusion (phantasm) [mind-affecting]|S|1 standard action|long (400 ft. + 40 ft./level)|Targetsone creature/level, no two of which can be more than 30 ft. apart|concentration + 3 rounds (D)|Will disbelief|yes}[Create a phantasm with auditory and visual effects.]
    \DeclareSpellDescription{Audiovisual Hallucination}{This spell functions as auditory hallucination, except that you can include the image of any object, creature, or force you imagine or identify for the targets to imagine. You can move the image while you concentrate. After you cease concentration, you can define simple movements or changes for the phantasm to perform that can be explained in 25 words or fewer.  The image disappears when struck by an opponent unless you cause the illusion to react appropriately or instruct it to do so. Its AC is equal to 10 + the level of this spell.}
        
\DeclareSpell{Aura Of The Unremarkable}{enchantment (compulsion) [mind-affecting]|V,  S,  M (a white feather)|1 standard action|30 ft.|Targetsnon-allied creatures within a 30-ft. emanation|1 minute/level (D) and instantaneous|Will negates|yes}[Make actions seem mundane to nearby creatures.]
    \DeclareSpellDescription{Aura Of The Unremarkable}{An invisible sphere of magic surrounds you, clouding the minds of creatures in the area so they regard even the strangest actions as innocuous. For example, if you and your allies are beating a member of the city guard for information, creatures within the area don't think this is unusual or cause for alarm; if your ally is aiming a crossbow at the queen from a balcony, the affected creatures accept this as normal and unworthy of concern. Any hostile actions by you or your allies against a creature or its allies break the effect of the spell for that creature. When the spell ends (or when the affected creatures move outside of the range of the emanation), observers see things normally but altered perceptions from the earlier events remain. Each mention of the events as noteworthy (such as being questioned about them by an authority figure) allows the target another Will save to break the effect and remember things normally.}
        
\DeclareSpell{Bountiful Banquet}{conjuration (creation) []|V,  S,  M (a turkey bone)|10 minutes|close (25 ft. + 5 ft./2 levels)|Effectfeast for two creatures/level|1 hour; see text|none|no}[Create a luxurious feast for two creatures/level.]
    \DeclareSpellDescription{Bountiful Banquet}{You conjure a beautiful and delicious feast with hors d'oeuvres, four courses worth of food, and plentiful drink. The food appears on ornate serving trays or in exquisite covered tureens, as appropriate to each type of dish. Place settings and serving utensils also appear-enough for each creature that will participate in the feast-along with elegant tablecloths and table linens. The spell doesn't create furniture, but the feast does adapt to appear on top of existing tables (or similar objects in the environment).  Though the feast and all the finery last only 1 hour, creatures that partake remain nourished and sated for 24 hours. Though you have little control over the fine details of the feast, you can specify what type of dish you want for each course and what sorts of beverages are provided. The feast automatically adjusts depending on the type of spellcaster you are. For instance, a druid casting this spell typically creates a spread of natural berries, whole roasted animals, and sweet (and possibly fermented) nectar for beverages, all on rough-hewn wooden plates with chopsticks instead of silverware and sizable leaves replacing napkins.}
        
\DeclareSpell{Break, Greater}{transmutation () []|V,  S,  M (an unbroken platinum tuning fork worth 100 gp)|1 round|30 ft.|Targetsall Medium or smaller objects in a 30-ft. burst centered on you|instantaneous|none|yes (object)}[Break all nearby objects.]
    \DeclareSpellDescription{Break, Greater}{You release a burst of destructive energy. Each Medium or smaller object in the area gains the broken condition unless it succeeds at a Fortitude saving throw. If a broken object fails this save, it is instead destroyed. Magic items can be broken by this spell, but not destroyed. Objects in your possession are not immune.}
        
\DeclareSpell{Build Trust}{divination () []|V,  S,  M (a gold piece)|1 standard action|close (25 ft. + 5 ft./2 levels)|Targetsone creature|1 day/level; see text|Will negates|yes}[Gain various bonuses when interacting with the target.]
    \DeclareSpellDescription{Build Trust}{You get a sense of the best way to interact with the target in order to encourage positive regard and fellowship toward you. You gain a +2 circumstance bonus on all Charisma checks and Charisma-based skill checks you attempt when interacting with the target. In addition, whenever you fail a Charisma  check or Charisma-based skill check when interacting with the target, you can reroll the check as an immediate action. Attempting this reroll grants the target a new saving throw to end the spell. The target doesn't become hostile to you when the spell ends, but it does become disillusioned of its new trust in you. Attacking the target or taking an obvious hostile action against it automatically ends the spell.  If you are using the contacts rules (Ultimate Campaign 148), your trust score with the target increases by 1 for the duration of the spell. If you're using the individual influence system (see pages 102-109), if the target fails its initial saving throw you learn one of its influence skills, strengths, or weaknesses as though you had succeeded at a discovery check.}
        
\DeclareSpell{Charm Person, Mass}{enchantment (charm) [mind-affecting]|V,  S|1 standard action|close (25 ft. + 5 ft./2 levels)|Targetsone or more humanoid creatures, no two of which can be more than 30 ft. apart|1 hour/level|Will negates|yes}[As charm person, but affects multiple creatures within 30 ft.]
    \DeclareSpellDescription{Charm Person, Mass}{This spell functions like charm person, except that mass charm person affects a number of humanoid creatures whose combined Hit Dice don't exceed twice your level. If there are more potential targets than you can affect, you choose them one at a time until you reach the limit of HD you can affect. If you cast mass charm person on only one creature, you ignore the spell's HD limit.}
        
\DeclareSpell{Codespeak}{transmutation () []|V,  S,  M (a complex rune inscribed on a slip of paper that is then placed under your tongue)|1 standard action|close (25 ft. + 5 ft./2 levels)|Targetsyou plus one willing creature per 2 levels, no two of which can be more than 30 ft. apart|10 min./level (D)|none|no}[Speak, read, and write a new code language.]
    \DeclareSpellDescription{Codespeak}{Upon casting this spell, all recipients gain the ability to speak a new language. This language sounds like random, babbling syllables to anyone not under the influence of the spell, but the targets understand each other perfectly.  Anyone using codespeak can read and write in this new language as well. Once the spell expires, however, any coded writing suddenly appears as gibberish. If the exact same group of individuals become the targets of a codespeak spell again, cast by the same caster, they can once again read any coded writings. A dedicated codebreaker can crack such writing's code, deciphering it one page at a time with a series of DC 30 Linguistics checks.  Comprehend languages doesn't enable a caster to understand the language of another's codespeak spell, but it does reveal that the targets are speaking a magical language. Tongues translates codespeak normally.}
        
\DeclareSpell{Complex Hallucination}{illusion (phantasm) [mind-affecting]|S|1 standard action|long (400 ft. + 40 ft./level)|Targetsone creature/level, no two of which can be more than 30 ft. apart|concentration + 3 rounds|Will disbelief|yes}[Create a phantasm with effects for all senses.]
    \DeclareSpellDescription{Complex Hallucination}{This spell functions as audiovisual hallucination (see page 204), except that the phantasm you create can also include olfactory, tactile, and thermal effects.}
        
\DeclareSpell{Compulsive Liar}{enchantment (compulsion) [mind-affecting]|V,  S|1 standard action|touch|Targetscreature touched|1 hour/level (D)|Will negates|yes}[Prevent target from speaking the truth.]
    \DeclareSpellDescription{Compulsive Liar}{The target becomes unable to speak the truth. Lies the target speaks don't need to be convincing, nor do they even need to be consistent, but they can't be true as far as the target is aware. This extends to non-verbal communication, such as hand signs or written notes. The spell allows talking in metaphors and talking about fictional figures.  The spell doesn't affect the target's ability to say things that are neither true nor false, such as questions, commands, or verbal spell components. If the target of this spell is simultaneously compelled to tell the truth (for instance, by being within a zone of truth), the target is only able to say things that are neither true nor false.}
        
\DeclareSpell{Conditional Curse}{necromancy () [curseUM]|V,  S|1 standard action|close (25 ft. + 5 ft./2 levels)|Targetsone creature|permanent (see text)|Will negates|yes}[Bestow a curse that is difficult to remove without fulfilling a condition.]
    \DeclareSpellDescription{Conditional Curse}{This spell functions as bestow curse, except that you must state a condition under which the curse is broken, ending its effect. An intelligent target, even one of animal intelligence, innately understands this condition even if it doesn't understand your language. The condition must be possible for the target to bring about within a year and a day without ensuring its own death and stated in 25 or fewer words. The curse is more difficult to remove via magic. The DC to remove conditional curse with break enchantment or remove curse increases by 5.}
        
\DeclareSpell{Conditional Favor}{abjuration () []|V|1 swift action||Targets1 creature|1 day/level (D)|none (see below)|yes}[Provide another spell whose effects reverse if the target breaks a restriction.]
    \DeclareSpellDescription{Conditional Favor}{You must cast this spell immediately before casting another spell on the same creature, eliciting a promise or warning against a behavior and binding the target to the paired spell. If you don't cast a paired spell, conditional favor has no effect. The paired spell must be from the abjuration, conjuration (healing), enchantment, or transmutation school or subschool, and must be cast on a willing creature. If the spell's recipient violates the oath or prohibition while conditional favor remains in effect, the paired spell is undone as if never cast. If the spell was a healing spell, the hit point damage or condition you removed returns immediately, even if the subject has enjoyed subsequent rest or healing. Poisons, diseases, curses, restored ability damage, and negative levels removed by the paired spell return as well.  Conditional favor recognizes the spirit of your condition and doesn't trigger a violation due to unintended consequences or circumstances that the subject could not predict with her current knowledge of the situation. For instance, if the prohibition prevented the subject from laying a finger on royalty, touching a disguised prince would not count as a violation if the subject did not recognize the prince, nor would touching a member of royalty while dominated. The subject of the spell intuitively knows beforehand whether an action will cause it to lose the paired spell's benefit.}
        
\DeclareSpell{Conjuration Foil}{abjuration () []|S|1 immediate action|medium (100 ft. + 10 ft./level)|Area20-foot radius spread|1 round|Will partial (see text)|yes (object)}[Interfere with nearby teleportation effects.]
    \DeclareSpellDescription{Conjuration Foil}{All creatures in the area gain a +4 bonus on saving throws against teleportation effects. If any creature would enter or depart the area via a summoning or teleportation effect, that creature takes 1d6 points of damage per spell level of the triggering effect (or half the HD of the originating creature if the effect has no spell level) and arrives in a random similar location within the triggering effect's range, rather than the intended destination. A successful Will save halves the damage and negates the altered destination.}
        
\DeclareSpell{Conjure Carriage}{conjuration (creation) []|V,  S,  M (a gourd)|1 round|close (25 ft. + 5 ft./2 levels)|Effectone quasi-real carriage, horses, and driver|1 hour/level (D)|none|no}[Create a fine carriage.]
    \DeclareSpellDescription{Conjure Carriage}{You create a fine wooden carriage with whatever cosmetic embellishments you desire. It is well constructed, although not exceptionally ostentatious. The carriage can carry up to six Medium or Small passengers. When conjured, the carriage comes with a team of two quasi-real light horses, which are already harnessed to the carriage. At your command, an invisible coachman similar to an unseen servant can assume the role of driver and direct the carriage, although it can't perform any complex or dangerous driving, and fails any checks made to drive the carriage in such conditions. At the end of the spell's duration, the carriage, horses, and coachman disappear into nothingness, depositing everything on or in it on the ground in its space.}
        
\DeclareSpell{Contingent Venom}{necromancy () [poisonUM]|V,  S,  M (herbs used in antitoxins worth 100 gp)|1 standard action|touch|Targetsone dose of poison or one venomous creature|permanent until discharged (D)|Fortitude negates|yes}[As languid venom, but with a triggering condition.]
    \DeclareSpellDescription{Contingent Venom}{This spell functions as languid venom (see page 218), but you can stipulate a specific condition or circumstance that will end the poison's onset time and cause it to take effect. The conditions for triggering the poison can be as general or as detailed as desired, but the triggers must be visual or audible (as per magic mouth) or else based on physical contact with or consumption of a specific object, substance, or creature. This triggering condition can either result in the immediate onset of the poison, or cause the poison to take effect a number of rounds after being triggered no greater than 1 round per caster level. You must make all decisions involving triggering when you cast contingent venom, and you can't change those decisions later.}
        
\DeclareSpell{Controlled Fireball}{evocation () [fire,  ruse]|V,  S,  M (a ball of bat guano and sulfur)|1 standard action|long (400 ft. + 40 ft./level)|Area20-foot-radius spread|instantaneous|Reflex half|yes}[As fireball, but secretly deals less damage to your allies.]
    \DeclareSpellDescription{Controlled Fireball}{This spell functions as fireball except you can cause the bead of fire to originate from anywhere you can see within range. You can choose a number of squares within the area up to your Intelligence bonus (for magi, occultists, or wizards) or Charisma bonus (for bloodragers or sorcerers) to be struck by weaker flames; the controlled fireball deals minimum damage in those squares.  Attempts to identify controlled fireball with a skill check incorrectly identify it as fireball (see the ruse descriptor o n page 192).}
        
\DeclareSpell{Crime Of Opportunity}{enchantment (compulsion) [mind-affecting]|V,  S,  M (a slit purse)|1 standard action|close (25 ft. + 5 ft./2 levels)|Targetsone creature|1 round|Will negates|yes}[Compel a target to take a criminal action.]
    \DeclareSpellDescription{Crime Of Opportunity}{You awaken a sudden criminal impulse in the target, compelling it to commit a criminal act as if affected by the crime wave spell.}
        
\DeclareSpell{Crime Wave}{enchantment (compulsion) [mind-affecting]|V,  S,  M (a tarnished coin)|1 standard action|medium (100 ft. + 10 ft./level)|Targetsone creature/level, no two of which can be more than 30 ft. apart|1 round/level (D)|Will negates|yes}[Compel targets to commit criminal actions.]
    \DeclareSpellDescription{Crime Wave}{You instill overwhelming avarice in the targets and impel them toward a wild spree of larceny. Creatures affected by a crime wave must roll percentile dice each round to determine what action they take.  d\% Behavior  1-25 Act normally, but with suspicion toward others. The target doesn't benefit from or provide benefits with teamwork feats and the aid another action. If a creature attempts to use a harmless spell or effect on the target, there is a 50\% chance the target tries to avoid that effect as best as possible (taking an attack of opportunity against a spellcaster, requiring a successful attack roll on a touch spell, or attempting a saving throw).  26-50 Attempt a stealAPG combat maneuver or Sleight of Hand check to steal a random valuable object from the nearest creature (or a nearby unattended object, if obviously of great value), moving adjacent to that creature or object as needed. Once an affected creature has stolen an item, further results of 26-50 cause the affected creature to flee, focusing all of its efforts on escaping with its loot and fighting to prevent the stolen object from being taken.  51-75 Attempt to break, destroy, or deface the nearest unattended manufactured object or structure. If the object or structure is too difficult to damage, the affected creature instead vandalizes or otherwise defaces its appearance.  76-100 Attack the nearest creature (for this purpose, a familiar counts as part of the affected creature's self).  A character affected by a crime wave who is unable to carry out the indicated action moves toward the nearest source of cover or concealment and attempts a Stealth check to hide. Affected creatures with the ability to turn invisible (including through the use of magic items or spells) do so instead of attempting a Stealth check.}
        
\DeclareSpell{Cultural Adaptation}{divination () []|V,  S,  M/DF (a document written in the language of the culture to be emulated)|1 standard action|personal|Targetsyou|10 minutes/level||}[Adapt to fit the local culture.]
    \DeclareSpellDescription{Cultural Adaptation}{When casting this spell, you must concentrate on a culture or subculture to which you wish to adapt. If you speak the native language of the culture in question, then for the duration of this spell, you speak the language with a native accent. The  spell doesn't teach you the language in question, but can be combined with tongues or a similar spell. Your body language and gestures mark you as a native of the culture, and you unconsciously make small decisions that help you blend in. Combined, these grant you a +2 circumstance bonus on Diplomacy checks to influence members of the culture to which you have adapted, which doesn't stack with other circumstance bonuses you might possess by virtue of being a member of the chosen culture. You also gain a +2 circumstance bonus on Disguise checks to pass yourself off as if you were a member of the culture, if you are not.  This doesn't provide benefits when disguising yourself as a specific member of the culture, though it negates any circumstance penalties you might otherwise have taken due to not acting appropriately for that person's culture. Finally, the DCs of enchantment (charm) spells you cast against natives of the culture to which you are attuned increase by 1.}
        
\DeclareSpell{Curse Of The Outcast}{enchantment (compulsion) [curseUM,  emotionUM,  mind-affecting]|V,  S,  M (a handful of earthworms)|1 standard action|close (25 ft. + 5 ft./2 levels)|Targetsone creature|permanent|Will negates|yes}[Curse someone to rub people the wrong way.]
    \DeclareSpellDescription{Curse Of The Outcast}{Everything about the target seems off-putting and grating, and everyone he meets is compelled to see the worst in him.  Whenever the target attempts a Bluff, Diplomacy, Intimidate, or Perform check, he must roll twice and take the lower result. Additionally, each creature he encounters has its initial attitude toward him reduced by one step (helpful becomes friendly, friendly becomes indifferent, and so on).}
        
\DeclareSpell{Dark Whispers}{illusion (shadow) [language-dependent,  shadowUM]|V,  S,  F/DF (a scrap of black cloth)|1 standard action|long (400 ft. plus 40 ft./level)|Targetsone creature/level|10 minutes/level (D)|none|yes}[Whisper through the shadows.]
    \DeclareSpellDescription{Dark Whispers}{You communicate through the shadows of one or more targets within range. The shadows have no physical presence and don't move or animate. Instead, your words emerge from the shadow as a clear whisper, absent any accent or other identifying features. The targets can make conversation with the shadow, but must speak aloud to do so. The targets' voices emerge from your own shadow only when they intend to speak to the shadow, but you hear no other sounds from the target's immediate area. Their responses also emerge as clear whispers, absent identifying features, but you can instinctively identify which target is speaking to you through the shadow. Once the spell has been cast, you don't need to have line of effect to the targets or their shadows to communicate back and forth.  The shadow communication is audible, so it can be intercepted by adversaries who succeed at a DC 25 Perception check. The spell can be silenced. You can't cast spells on subjects or otherwise establish line of effect through the shadows, but spells that allow you to speak or understand languages work normally across dark whispers.}
        
\DeclareSpell{Deadman's Contingency}{evocation () []|V,  S,  M (a scorpion's tail),  F (ivory statuette of you worth 1, 500 gp)|10 minutes or more; see text|personal|Targetsyou|up to 1 hour/level plus 1d6 rounds (D); see text||}[Set one of a list of contingencies for your demise.]
    \DeclareSpellDescription{Deadman's Contingency}{This spell functions as contingency, except as noted above. This spell also only comes into effect after your death and works only with certain spells. The companion spell triggers 1d6 rounds after your death. All decisions made involving the companion spell must be made when deadman's contingency is cast (including messages and recipients for spells like magic mouth or sending). If the spell targets an object or appears in a certain location, it must target or be centered on your corpse.  The following spells can be companion spells for deadman's contingency: animate dead (animating your corpse as an uncontrolled skeleton or zombie), disintegrate, fireball, gentle repose, magic mouth, major image (with a duration of 3 rounds), permanent image, sending, stinking cloud, and teleport object.}
        
\DeclareSpell{Deceitful Veneer}{illusion (glamer) []|V,  S|1 standard action|close (25 ft. + 5 ft./2 levels)|Targetsone creature|10 minutes/level (D)|Will negates|yes}[Make someone seem like an obvious liar.]
    \DeclareSpellDescription{Deceitful Veneer}{You subtly alter both the target's aura and subtle cues in its body language, tone of voice, and word choice, which makes everything that the target says seem to be a lie. Every statement that the target makes appears to be a lie under both magical scrutiny (such as discern lies) and mundane scrutiny (such as using the Sense Motive skill). Someone who closely scrutinizes the target can determine when it is actually telling the truth with a successful Sense Motive check (DC = 15 + your caster level).  As long as you are within close range of the target, as a standard action you can suppress or resume the effects of this spell, allowing you to let the target seem to be telling the truth at some times and still seem to be lying at others.}
        
\DeclareSpell{Deflect Blame}{enchantment (compulsion) [mind-affecting]|V,  S|1 immediate action|close (25 ft. + 5 ft./2 levels)|Targetsone creature|instantaneous|Will negates|yes}[Blame someone else for your action.]
    \DeclareSpellDescription{Deflect Blame}{You can cast this spell immediately after attacking a creature, causing that creature to believe that a different creature that threatens it was responsible for the attack rather than you. You can instead cast this spell immediately after a failed Bluff, Diplomacy, or Intimidate check, causing the target of that check to believe that a different creature you designate within spell range was responsible for the content of that failed check. Using the spell in these ways doesn't compel the target to undertake a specific action in response to its belief of where the blame lies.}
        
\DeclareSpell{Demanding Message}{enchantment (compulsion) [language-dependent,  mind-affecting]|V,  S,  F (a piece of copper wire)|1 standard action|medium (100 ft. + 10 ft./level)|Targetsone creature/level|10 minutes/level, then 1 hour/level or until completed (D); see text|Will negates (see text)|yes; see text}[Send messages as per message with a suggestion for one creature.]
    \DeclareSpellDescription{Demanding Message}{This spell initially functions as message (allowing no save or spell resistance). Once during the message effect, you can concentrate as a standard action to issue a suggestion to one target as part of delivering a message. Spell resistance and a Will save apply to the suggestion, and it lasts for 1 hour per level or until completed.}
        
\DeclareSpell{Demanding Message, Mass}{enchantment (compulsion) [language-dependent,  mind-affecting]|V,  S,  F (a piece of copper wire)|1 standard action|medium (100 ft. + 10 ft./level)|Targetsone creature/level|10 minutes/level, then 1 hour/level or until completed (D); see text|Will negates (see text)|yes; see text}[Send messages as per message with one suggestion for each creature.]
    \DeclareSpellDescription{Demanding Message, Mass}{This spell functions as demanding message, except that you can issue one suggestion to each of the spell's targets instead of just one. Each time you do so, it takes a standard action. You can issue a different suggestion to each target.}
        
\DeclareSpell{Desperate Weapon}{conjuration (creation) []|V|1 swift action|personal|Effectone-handed improvised weapon|1 minute/level|none|no}[Create an improvised weapon.]
    \DeclareSpellDescription{Desperate Weapon}{You create a one-handed object that you might expect to see in your current surroundings, which you can then use as an improvised weapon. The spell conjures such an object near your hand such that you can retrieve it as you complete the spell.  No matter what sort of object you picked, it functions as a one-handed improvised weapon appropriate for your size and that deals 1d6 points of damage for a Medium creature (1d4 for Small creatures). The item deals the type of damage you choose (bludgeoning, piercing, or slashing) when casting the spell, though the object you request must conform to the damage type.  The spell ends prematurely if the improvised weapon leaves your grasp. The object has no value and can't be used for other functions other than as an improvised weapon (for instance, this spell doesn't allow you to conjure an expensive spyglass and sell it or use its other abilities, but you could still use it to beat someone over the head). The conjured object can't already be a manufactured weapon, even in a location where you might expect to see manufactured weapons. It can be an object that would normally make for an unusual improvised weapon, like a herring at a fish market, and it still deals its full damage.}
        
\DeclareSpell{Detect Anxieties}{divination () [mind-affecting]|V,  S,  F/DF (a medallion)|1 standard action|60 ft.|Areacone-shaped emanation|concentration, up to 1 minute/level (D)|Will negates (see text)|no}[Learn what makes creatures anxious.]
    \DeclareSpellDescription{Detect Anxieties}{This spell functions as detect thoughts except that you sense significant anxieties of creatures with an Intelligence score of 1 or higher, regardless of whether they are conscious or not.  Instead of Intelligence, the second round of concentration reveals each mind's Wisdom score and current degree of fear (shaken, frightened, panicked, cowering, or paralyzed with fear). If the highest Wisdom score is 26 or higher (and at least 10 points higher than your own Wisdom score), you are stunned for 1 round and the spell ends.  Instead of surface thoughts, the third round of concentration reveals the most pressing current anxiety of any mind in the area (Will negates).  Presenting a creature with the threat of its anxiety grants you a +2 bonus (or higher, at the GM's discretion) on checks to Intimidate that creature.}
        
\DeclareSpell{Detect Desires}{divination () [mind-affecting]|V,  S,  F/DF (a medallion)|1 standard action|60 ft.|Areacone-shaped emanation|concentration, up to 1 minute/level (D)|Will negates (see text)|no}[Learn what creatures desire.]
    \DeclareSpellDescription{Detect Desires}{This spell functions as per detect thoughts, except you sense significant desires of creatures with an Intelligence score of 1 or higher, regardless of whether they are conscious or not.  Instead of Intelligence, the second round of concentration reveals each mind's Charisma score. If the highest Charisma score is 26 or higher (and at least 10 points higher than your own Charisma score), you are stunned for 1 round and the spell ends.  Instead of surface thoughts, the third round of concentration reveals the most pressing current desire of any mind in the area (Will negates).  Presenting a creature with an opportunity to fulfill a significant desire grants you a +2 circumstance bonus (or higher, at the GM's discretion) on Diplomacy checks to influence it.}
        
\DeclareSpell{Detect Magic, Greater}{divination () []|V,  S|1 standard action|60 ft.|Areacone-shaped emanation|concentration, up to 1 minute/level (D)|none|no}[As detect magic, but learn more information.]
    \DeclareSpellDescription{Detect Magic, Greater}{This spell functions as detect magic, except that you can glean much more information from the magical auras that you find, and those auras can be found after a much greater length of time. You can detect a lingering aura for up to 1 day per caster level you have, regardless of the aura's original strength.  Additionally, when you use a standard action to concentrate on this spell, you can also study a creature within the spell's area and attempt a Spellcraft check in order to determine the last spell that the creature cast by identifying lingering traces that the spell left in the caster's aura. The DC to identify the spell is equal to 20 + the creature's caster level.  Finally, you are able to locate and analyze the signature flourishes in a magical aura that allow you to match a spell to the person who cast it. In order to find these identifiers in a spell's aura, you must spend 1 round focusing on that spell in particular, and succeed at an opposed Knowledge (arcana) check against the caster (or a Knowledge [arcana] check with a DC equal to 15 + the spell level if the caster wants her work to be identified and emphasizes these unique elements rather than obscuring them). Once you learn a caster's set of identifiers, you can remember them as easily as a face or a voice. You can recognize this signature if you succeed at a Spellcraft check when later identifying a spell to determine whether or not that spell was cast by the same individual. The spell greater magic aura (see page 219) can obfuscate this information, making it seem that someone else cast the spell. Greater detect magic grants a saving throw against magic aura (but not greater magic aura).}
        
\DeclareSpell{Detect The Faithful}{divination () []|V,  S,  DF|1 standard action|60 ft.|Areacone-shaped emanation|concentration, up to 1 minute/level (D)|none|no}[Find others of the same faith.]
    \DeclareSpellDescription{Detect The Faithful}{You can detect other worshipers of your deity (mortal worshipers, outsider servants, and so on). The amount of information revealed depends on how long you focus on a particular area or subject.  1st Round: Presence or absence of the faithful.  2nd Round: Number of individual faithful in the area.  3rd Round: The exact location of each worshiper. If a fellow worshiper is outside your line of sight, then you discern his direction but not his exact location.  Each round, you can rotate to detect worshipers in a new area. The spell can penetrate barriers, but a sheet of lead, 1 foot of stone, 1 inch of common metal, or 3 feet of wood or dirt blocks it. A creature's personal interpretation of its beliefs determines whether or not it is of the same faith as you-hence heretics and splinter cultists of your deity still count as worshipers of that deity. Furthermore, since the spell picks up a creature's current beliefs and feelings, a creature actively pretending to be a member of the same faith also appears to the spell to be a member. Thus, the spell is still useful in locating potential hidden members of the same faith among the general populace, but on its own, it doesn't weed out spies.}
        
\DeclareSpell{Disrupt Silence}{abjuration () []|S,  M (tiny silver bell,  chime,  or gong)|1 standard action|touch|Area10-ft.-radius emanation centered on a creature, object, or point in space|1 round/level (D)|Will negates|yes}[Disrupt all silence effects in an area.]
    \DeclareSpellDescription{Disrupt Silence}{You suppress magical sound-dampening effects within the area. Disrupt silence temporarily negates magical silence within its area, so that normal sounds can be heard within the overlapping areas of effect. Additionally, disrupt silence can automatically counter or dispel any magical silence effect of equal or lower level cast upon the same target, such as silence. If you cast disrupt silence on the target of a higher-level silence effect, it functions as dispel magic instead of its normal function.}
        
\DeclareSpell{Dress Corpse}{necromancy () []|V,  S,  M (a pickled herring)|1 standard action|touch|Targetscorpse touched|instantaneous|none|no}[Doctor the evidence on a corpse.]
    \DeclareSpellDescription{Dress Corpse}{You cause the flesh and bones of a corpse to shift themselves to suit a narrative of your choosing. This spell can hide or create telltale wounds, bruising, and other subtle clues as to the nature of the target's death, and the final hours leading up to it, allowing you to make the corpse appear to have died in just about any way. You could, for example, make stab wounds close up as though they were never there, rearrange bruises on the neck, evaporate traces of poison within the body into nothingness, make burn marks grow to cover the corpse's skin, or shrivel the target's body  as though the creature had starved. This spell can't hide extreme alterations to the body (such as the loss of a limb), nor can it restore flesh to a skeletal corpse or strip a corpse down to skeletal form. It is also unable to change the apparent identity of the corpse.  Anyone who closely examines the corpse can attempt a Perception check (DC = 10 + your caster level) to notice that the corpse's wounds (or lack thereof) don't look natural, but this doesn't allow the observer to determine what the corpse looked like before this spell was cast. Closely examining the corpse with a successful Heal check (DC = 15 + your caster level) not only reveals that the target's apparent wounds are false, but also what the originally obscured wounds were.}
        
\DeclareSpell{Entice Fey}{conjuration (calling) []|V,  S,  M (offerings worth 1, 250 gp plus payment),  DF|10 minutes|close (25 ft. + 5 ft./2 levels)|Effectone or two called fey, totaling no more than 12 Hit Dice, which can't appear more than 30 ft. apart|instantaneous|none|no}[Entice service from a fey of 12 Hit Dice or fewer.]
    \DeclareSpellDescription{Entice Fey}{This spell functions as lesser entice fey, except that the spell's whimsical calling can produce a single fey of 12 Hit Dice or less, or two fey of the same kind whose Hit Dice total no more than 12.}
        
\DeclareSpell{Entice Fey, Greater}{conjuration (calling) []|V,  S,  M (offerings worth 2, 500 gp plus payment),  DF|10 minutes|close (25 ft. + 5 ft./2 levels)|Effectup to three called fey, totaling no more than 18 Hit Dice, no two of which can appear more than 30 ft. apart|instantaneous|none|no}[Entice service from a fey of 18 Hit Dice or fewer.]
    \DeclareSpellDescription{Entice Fey, Greater}{This spell functions as lesser entice fey, except the spell's whimsical calling can produce a single fey of 18 Hit Dice or less,  or up to three fey of the same kind whose Hit Dice total no more than 18.}
        
\DeclareSpell{Entice Fey, Lesser}{conjuration (calling) []|V,  S,  M (offerings worth 500 gp plus payment,  see text),  DF|10 minutes|close (25 ft. + 5 ft./2 levels)|Effectone called fey with 6 Hit Dice or fewer|instantaneous|none|no}[Entice service from a fey of 6 Hit Dice or fewer.]
    \DeclareSpellDescription{Entice Fey, Lesser}{This spell functions as lesser planar ally, except that you entice a fey of 6 HD or fewer to lend you its aid with an offering of music or something else it finds appealing. Like lesser planar ally, this spell is unpredictable, and the fey who answers the calling is up to the whims of nature and the fey, not your own choice. You must succeed at a Knowledge (nature) check or Perform check (DC = 20 + target's HD) in addition to the spell's material component to entice the fey into appearing, after which you can negotiate for the service and your payment. The maximum HD of fey that you can call with that casting is equal to the result of your check - 20. For example, if your check result is a 24, the maximum HD for the called fey is 4. A high result doesn't allow you to break the HD maximum for the spell, and a result of 20 or less means you can't call a fey at all. If the fey doesn't like the sound of your offer, it can simply choose to refuse, in which case you don't expend any of the material components for the spell, either the offerings or the payment.}
        
\DeclareSpell{Fabricate Disguise}{transmutation () []|S|1 standard action|personal|Targetsyou|instantaneous||}[Create a disguise in an instant.]
    \DeclareSpellDescription{Fabricate Disguise}{You change outfits or create a disguise out of materials you are wearing or carrying (potentially including a disguise kit). The spell can't alter your body or change the structure of objects, but can style wigs, apply makeup or piercings, and otherwise make use of tools to make superficial changes. In an instant, you have a nonmagical disguise or clothing change. Attempt a Disguise check to determine the effectiveness of the disguise.}
        
\DeclareSpell{False Belief}{enchantment (compulsion) [mind-affecting]|V,  S,  M (lemon juice and a scrap of parchment)|10 minutes, plus length of memory to be altered|touch|Targetswilling creature touched|1 hour/level|none|yes}[Temporarily plant a false memory.]
    \DeclareSpellDescription{False Belief}{You temporarily alter the target's memory (similar to modify memory) to eliminate, change, or implant a memory of up to 1 hour in length. When the duration of this expires, the target's real memory returns, and the false memory fades to little more than a vague outline, like a dream. This false memory seems true to the target, so effects that detect lies or force the target to speak the truth (as the subject understands it) don't detect the falsehood.}
        
\DeclareSpell{False Future}{illusion (glamer) []|V,  S,  M (crushed jade worth 100 gp)|1 standard action|touch|Targetscreature or object touched|1 hour/level (D)|Will negates or Will disbelief (see text)|yes}[Cause divinations of the future to reveal the result you choose.]
    \DeclareSpellDescription{False Future}{You interfere with attempts to predict the target's future by preventing divinations from revealing what the target will do and what will befall the target while under the spell's effects. Instead of the target's true actions or experiences, divinations resolve as if the target will experience some different future you describe as you cast false future. The target creature can attempt a Will save to avoid the initial effect, and creatures using divinations get a Will save to disbelieve the illusion. This spell doesn't prevent divinations cast after the duration's end from determining what the subject actually did during the time you obscured using false future.  False future can't be detected by detect magic or identify, but greater detect magic (see page 212) can detect it.}
        
\DeclareSpell{False Resurrection}{conjuration (calling) [chaotic,  evil,  ruse]|V,  S,  M (diamond worth 10, 000 gp),  DF|1 minute|touch|Targetsdead creature touched|1 day/level|none|yes (harmless)}[Appear to resurrect someone but instead allow a shadow demon to possess the corpse.]
    \DeclareSpellDescription{False Resurrection}{This spell functions as resurrection except that instead of recalling the dead soul to life, the spell calls a shadow demon with the advanced simple template (Pathfinder RPG Bestiary 67, 294) to possess the body. If you fail to overcome the subject's spell resistance, the subject's soul can negate the calling and gain the option to return to life (as resurrection) in the restored body. The possession otherwise functions (as possessionOA) except that the subject's soul is not present to resist. When false resurrection ends or the demon is removed from the subject, the demon returns to the Abyss, leaving the body alive but soulless (like that created by clone). If the soul hasn't been returned to life already, it has the option to return in the now-empty body if it still lives (as resurrection). If the body was killed, the demon is sent back to the Abyss but the subject remains dead.  Attempts to identify false resurrection with a skill check incorrectly identify it as resurrection (see the ruse descriptor on page 192). A fooled viewer mistakes false resurrection's aura as the lingering aura of an instantaneous conjuration effect.}
        
\DeclareSpell{False Resurrection, Greater}{conjuration (calling) [evil,  lawful,  ruse]|V,  S,  M (diamond worth 25, 000 gp),  DF|1 minute|touch|Targetsdead creature touched|permanent|none|yes (harmless)}[Appear to use true resurrection on someone but instead allow a belier devil to possess the corpse.]
    \DeclareSpellDescription{False Resurrection, Greater}{This spell functions as false resurrection except that i t calls a belier devil with the young simple template (Pathfinder RPG Bestiary 2 85, 292) to possess the body and all parts of false resurrection that function like resurrection instead function like true resurrection.  Attempts to identify greater false resurrection with a skill check incorrectly identify it as true resurrection (see the ruse descriptor on page 192). A fooled viewer mistakes greater false resurrection's aura as the lingering aura of an instantaneous conjuration effect.}
        
\DeclareSpell{False Vision, Greater}{illusion (glamer) []|V,  S,  M (a jade sphere worth 500 gp)|1 standard action|see text|Targetsone creature|1 hour/level (D)|Will negates|no}[As false vision, but moves with the target.]
    \DeclareSpellDescription{False Vision, Greater}{This functions similarly to false vision, but instead of placing the illusion on a nearby area, you can tie it to a specific individual, and can do so from great distances. The spell can be cast at any distance. The difficulty of the save depends on your knowledge of the subject and what sort of physical connection (if any) you have to that creature. The target gains the same bonuses and penalties on its Will save to resist this spell as the bonuses and penalties that apply to the scrying spell (including a +5 bonus if the target is on another plane).  The illusion created by the spell moves with the target, and is not stationary. The image can affect the way the target is perceived, the way the target's surroundings are perceived, and the way that specific creatures or objects around the target are perceived. For the target, and any other specific creatures or objects you specify, you can cause them to appear as other creatures or objects that you designate, not to appear at all, or to have their appearance unaltered. For the target's surroundings, you can choose to make the target appear to be somewhere else, either a specific location with which you are familiar, or a generic location conjured from your imagination. You can change the way that the spell affects the appearance of any of these things by concentrating on the spell. You can also cause creatures or objects to appear in the illusion that are not really there, or to make a creature or object seem to act in a way other than it is actually acting. In this case, you must concentrate on the spell, or these aspects of the illusion simply remain static. The spell can provide visual, auditory, olfactory, tactile, and thermal sensations as needed.  The illusion applies to only those who observe the target via a divination (scrying) spell, and has no effect on viewers who are there in person.}
        
\DeclareSpell{Ghost Brand}{transmutation () [shadowUM]|V,  S,  M (a branding iron and a strip of white silk worth 10 gp)|10 minutes|touch|Targetsone willing creature and one object touched|1 day/level (D)|none|no}[Allow an item to transform into a brand and back.]
    \DeclareSpellDescription{Ghost Brand}{You alter the fundamental substance of a single object up to 2 cubic feet per level in size and at least one size category smaller than the target creature, causing it to become shadowy and intangible, as though made out of quasi-real shadowstuff. You link the intangible item to the target's flesh by making a brand shaped like the item on the target's skin. The intangible item merges with the target's flesh and is contained within the target's body, moving with the target wherever it goes. The creature can retrieve the item or reabsorb it as a full-round action, and can do so as many times as it wants. When worn or wielded by the target, the item regains its solidity and functions normally, though the target can't drop or remove the item (other than by using the full-round action), nor can it be disarmed or stolen. If the item is destroyed, the spell ends. If the ghost brand spell is dispelled while the object is inside the target's body,  the object bursts out of the target's flesh, dealing 3d6 points of piercing and slashing damage to the target that bypasses DR, as well as 1d6 points of bleed damage.}
        
\DeclareSpell{Glimpse Of Truth}{divination () []|V,  S,  M (a tourmaline costing 50 gp)|1 standard action|personal|Targetsyou|1 round|Will negates (harmless)|yes (harmless)}[Gain true seeing for 1 round.]
    \DeclareSpellDescription{Glimpse Of Truth}{This spell functions like true seeing, except as noted above.}
        
\DeclareSpell{Handy Grapnel}{transmutation () []|V,  S,  M (an arrowhead)|1 standard action|touch|Targetsone ropelike object, length up to 50 ft. + 5 ft./level|1 minute/level (D)|Fortitude negates (object)|no}[Transform a ropelike object into a retracting grapple.]
    \DeclareSpellDescription{Handy Grapnel}{You cause the target rope to shrink and reshape itself into an arrow, bolt, or similar piece of ammunition, which you can shoot at any Medium or larger object. With a successful attack roll against an AC equal to 5 plus the hardness of the target object, the handy grapnel strikes and embeds itself in that object with the strength of an iron grappling hook.  As a move action, you can command the rope to extend from the arrow. If you are within a distance equal to the length of the rope and have a hand free, the end of the rope swings directly into your hand. With another move action, you can command the handy grapnel to retract itself up to the grappling hook, pulling up any creature or object supported by the rope at a speed of 50 feet per round.  Once embedded, the rope can be used for climbing or any other purpose a rope could serve, and it can bear up to 200 pounds per level of the caster at a time (maximum 1,000 pounds). If this weight limit is exceeded, the handy grapnel comes loose and any creature or object supported by the rope falls.}
        
\DeclareSpell{Hidden Presence}{enchantment (compulsion) [mind-affecting]|V,  S,  M (eye drops made with extract of poppy)|1 standard action|close (25 ft. + 5 ft./2 levels)|Targetsup to one creature per 3 caster levels|1 minute/level (D)|Will negates|yes}[Prevent creatures from noticing your presence.]
    \DeclareSpellDescription{Hidden Presence}{You prevent the targets from having conscious awareness of your presence. You make yourself completely undetectable to the subjects by erasing all awareness of your presence from their minds. The targets can't see, hear, smell, feel, or taste you, including with extraordinary or supernatural senses such as blindsense, blindsight, scent, or tremorsense. They can't pinpoint your location by any means, including detect spells.  The targets remain unaware of your actions, provided you don't make any attacks or cause any obvious or directly threatening changes in the targets' environment. If you attack any of the target creatures, the effect ends. If you take an action that creates a sustained and obvious change in the target's environment-for example, attacking a creature other than a target or moving a sizeable or attended object the target can see-the target immediately receives a new saving throw.}
        
\DeclareSpell{Hollow Heroism}{enchantment (compulsion) [mind-affecting,  ruse]|V,  S|1 standard action|touch|Targetscreature touched|10 minutes/level (D)|Will negates (harmless)|yes (harmless)}[Provide a heroism effect that you can reverse at any time.]
    \DeclareSpellDescription{Hollow Heroism}{This spell functions as heroism, except that you can reverse the spell by issuing a special command as a standard action if you are within medium range (100 feet + 10 feet per caster level) of the target. The target loses the bonuses and instead takes a -2 penalty on attack rolls, saving throws, and skill checks until the spell's duration ends (no save).  Attempts to identify hollow heroism with a skill check incorrectly identify it as heroism (see the ruse descriptor on page 192).}
        
\DeclareSpell{Hollow Heroism, Greater}{enchantment (compulsion) [mind-affecting,  ruse]|V,  S|1 standard action|touch|Targetscreature touched|1 minute/level (D)|Will negates (harmless)|yes (harmless)}[Provide a greater heroism effect that you can reverse at any time.]
    \DeclareSpellDescription{Hollow Heroism, Greater}{This spell functions as greater heroism, except that you can reverse the spell by issuing a special command as a standard action if you are within medium range (100 feet + 10 feet per caster level) of the target. The target loses the bonuses and instead takes a -4 penalty on attack rolls, saving throws, and skill checks until the spell's duration ends (no save), increasing to a -8 penalty against fear effects. Additionally, she takes damage equal to your caster level when you reverse the spell.  Attempts to identify greater hollow heroism with a skill check incorrectly identify it as greater heroism (see the ruse descriptor on page 192).}
        
\DeclareSpell{Illusion Of Treachery}{illusion (figment) []|S,  F (a tiny marionette)|1 standard action|close (25 ft. + 5 ft./2 levels)|Targetsone creature|1 round/level (D)|Will disbelief|yes}[Make it seem like another is also responsible for your attacks.]
    \DeclareSpellDescription{Illusion Of Treachery}{You create an illusion that takes the same space as a foe and mimics its movements perfectly. Whenever you cast a spell or throw a weapon, the illusion ceases mimicking the target's actions just long enough to make it look like the target cast the spell or threw the weapon simultaneously with you, such that witnesses who can see both you and the target can't tell with certainty who truly cast the spell or threw the weapon. Witnesses who can see only the target see it as the only apparent source. The subject of this spell doesn't provoke attacks of opportunity from these illusory actions. Each target of the attack or spell counts as interacting with the illusion and thus receives a save to disbelieve.}
        
\DeclareSpell{Illusion Of Treachery, Greater}{illusion (figment) []|S,  F (a tiny marionette)|1 standard action|close (25 ft. + 5 ft./2 levels)|Targetsyou and one creature|1 round/level (D)|Will disbelief|yes}[Make it seem like another is responsible for your attacks while concealing your own actions.]
    \DeclareSpellDescription{Illusion Of Treachery, Greater}{This spell functions as illusion of treachery except that it also conceals your own actions (as illusion of calmUC) and renders invisible all spell effects and ranged attacks originating from you until they reach the attack's target (this doesn't cause the attack's target to be unable to avoid the attack, as they still see the attack originating from the spell's target).}
        
\DeclareSpell{Insect Spies}{divination () []|V,  S,  M (a drop of honey)|1 round|close (25 ft. + 5 ft./2 levels)|Effectup to one insect spy/4 levels|10 minutes/level (D)|none|no}[Use magic beetles as spies.]
    \DeclareSpellDescription{Insect Spies}{You summon one or more glossy black beetles, which have a measure of intelligence and make for excellent spies. When they are in your presence, the insects obey your mental commands, and you can issue orders to any number of them as a single standard action, provided that you issue the same orders to each one. In order to issue different orders to different insects, you must spend a separate standard action for each set of orders. An insect in physical contact with you can answer simple questions about what it has observed, at a rate of one question per round. It can relate only what it perceived with its senses, and can't repeat speech. It has difficulty making subjective judgments, and questions that demand such reasoning are unlikely to yield a clear answer. For example, an insect is unable to relay someone's emotional state or determine who among several people it saw might be in charge.  Each insects' size is Fine. Each insect has 1 hit point, AC 20 (+2 Dexterity, +8 size), a movement speed of 5 feet, a climb speed of 5 feet, and a fly speed of 20 feet (perfect maneuverability). The insects use your saving throw bonuses, have a total Perception skill bonus equal to 5 + 1/2 your caster level, and can't make attacks. Due to their incredibly small size and magical nature, they can make Stealth checks to avoid being noticed even if they lack a source of cover or concealment, and they have a total Stealth skill bonus equal to 18 + 1/2 your caster level. The insects can even climb onto creatures of Tiny or larger size while using Stealth, possibly riding on those creatures unnoticed. A Tiny creature gains a +16 bonus on Perception checks made to notice one of these insects currently climbing on it. For each size category larger than Tiny the creature being climbed is, this bonus is reduced by 4 (to a minimum of +0 for Huge or larger creatures).  You also maintain a faint mystical connection with these insects, which allows you to sense where they are. As a full-round action, you can concentrate on the spell in order to learn the direction and relative distance of each of the insects.}
        
\DeclareSpell{Insect Spies, Greater}{divination () []|V,  S,  M (a drop of honey)|1 round|close (25 ft. + 5 ft./2 levels)|Effectup to one insect spy/4 levels|10 minutes/level (D)|none|no}[Use magic beetles as spies and also share their senses.]
    \DeclareSpellDescription{Insect Spies, Greater}{As insect spies, but you can also borrow the senses of the summoned insects. As a move action, you can choose to receive sensory input from one of the insects, seeing what it sees and hearing what it hears. While doing so, you are treated as being blind and deaf. You can change to another insect, or return to your own senses, with another move action.}
        
\DeclareSpell{Instant Fake}{illusion (figment) []|V,  S,  M (a piece of costume jewelry)|1 standard action|1 object touched|Targetsone object weighing no more than 1 lb./level|1 minute/level|Will disbelief (if interacted with)|no}[Provide a short-term replica of an object.]
    \DeclareSpellDescription{Instant Fake}{You create an illusory duplicate of the target item. If you hold the charge on this spell, you can deliver it while touching an object you steal with Sleight of Hand or a stealAPG combat maneuver; in this case, the illusion phases into existence exactly as you remove the genuine article, allowing you to instantaneously replace a protected or guarded item with no change in appearance, weight, or other factors.  The illusion appears to be a perfect replica. Actively examining the fake with an Appraise or Perception check grants a creature a Will save, but on a failed saving throw, it concludes that the fake is the genuine article. The illusion isn't a functional item, nor does it have any magical properties of the original. For example, an instant fake of a set of thieves' tools can't be used to pick a lock, a false warhammer can't harm a person or break an object, a suit of unreal chainmail offers no actual protection, and an illusory potion of cure light wounds doesn't heal any hit points when imbibed.}
        
\DeclareSpell{Instant Summons, Greater}{conjuration (summoning) []|V,  S,  M (sapphires worth 1, 000 gp each)|1 standard action|see text|Targetsup to one object per 3 caster levels, each weighing 10 lbs. or less whose longest dimension is 6 ft. or less|permanent or until discharged|none|no}[As instant summons, but for multiple objects and creatures.]
    \DeclareSpellDescription{Instant Summons, Greater}{This spell functions as instant summons, except that you can target multiple objects. You must use a separate sapphire worth 1,000 gp for each one. For each item you target, you can touch a creature, granting that creature the ability to speak the special word for that item (each item has its own special word) while crushing the matched gem to call the item to hand. Only you or the touched creature can activate the gem or see the arcane mark upon it.}
        
\DeclareSpell{Know Peerage}{divination () [mind-affecting]|V,  S,  M (thread from a tabard or livery)|1 standard action|touch|Targetscreature touched|10 minutes/level|Will negates (harmless)|yes (harmless)}[Target uses your Knowledge (nobility) ranks.]
    \DeclareSpellDescription{Know Peerage}{You impart your knowledge of nobility and peerage to the target, allowing her to recognize members of noble households, differentiate one set of heraldry from another, and otherwise identify who's who at a royal gala or other noteworthy social event. The target is able to identify noble individuals, noble family names, and noble crests, signets, heraldry, and other symbols. The target treats her number of ranks in Knowledge (nobility) as though it were equal to your number of ranks in Knowledge (nobility), to a maximum of 5 ranks and a minimum of 0. If the target's number of ranks is greater than yours, she uses her own number of ranks instead. In addition, if the target's new total skill bonus on Knowledge (nobility) checks is at least +0, she automatically succeeds on all Knowledge (nobility) checks with a DC of 10 or lower.}
        
\DeclareSpell{Languid Venom}{necromancy () [poison]|V,  S,  M (herbs used in antitoxins worth 25 gp)|1 standard action|touch|Targetsone dose of poison or one venomous creature|permanent until discharged (D)|Fortitude negates|yes}[Delay a poison’s onset and hide its presence.]
    \DeclareSpellDescription{Languid Venom}{You greatly extend the time it takes for the poison you touch to take effect, giving that poison an onset time up to 1 hour per caster level. (You touch the poison's container, so you don't risk exposing yourself to a contact poison.) The target doesn't attempt a saving throw when initially exposed to the languid venom, but instead saves at the end of the poison's onset time. If the poison is neutralized or otherwise cured prior to the end of its onset time, it is rendered harmless. Failing saves against multiple doses of languid venom have the normal cumulative effect for poisons (Pathfinder RPG Core Rulebook 558).  Languid venom is difficult to detect or identify. Detect poison and similar effects detect languid venom only with a successful caster level check against a DC equal to 11 + your caster level (rolled secretly by the GM). Even if the poison is detected, the DC of Craft (alchemy) or Wisdom checks to identify the poison is increased by 10. If a poison is affected by an additional effect that requires a caster level check to detect the poison or increases the DC to identify it-such as obscure poison (see page 220)-those effects don't stack. Use only the caster level check with the higher DC and increase the DC of the check to identify the poison by the higher of the two.  If cast upon a venomous creature, languid venom delays the onset of that creature's poison when the creature next delivers its natural poison.}
        
\DeclareSpell{Life Of Crime}{enchantment (compulsion) [curseUM,  mind-affecting]|V,  S,  M (a black mask)|1 standard action|close (25 ft. + 5 ft./2 levels)|Targetsone living creature|permanent|Will negates|yes}[Permanently turn someone into a crazed criminal.]
    \DeclareSpellDescription{Life Of Crime}{You unleash the basest instincts of iniquity in the target and cause them to become his overriding reason for being. The target neither gains nor provides benefit from teamwork feats or the aid another action and can't willingly accept harmless magical effects from others.  The target moves by Stealth whenever possible, and lies and deceives others instinctively to further its personal agenda. In addition, when the target is conscious, it must succeed at a Will save against the spell's save DC each hour (or each round during combat or a similarly stressful situation) or behave as if affected by a crime wave spell for 1 round.  Life of crime is particularly difficult to remove. Only a remove curse with a higher caster level than life of crime's caster level, or a limited wish, wish, or miracle can remove its effects.}
        
\DeclareSpell{Mage's Decree}{evocation () []|V,  S,  F (a brass cone or trumpet)|1 standard action|up to 1 mile/level; see text|Targetssee text|instantaneous|none|no}[Send a message to creatures within miles.]
    \DeclareSpellDescription{Mage's Decree}{You speak a short message (up to 25 words), and it is immediately transmitted to each target, who hear it as clearly as if you were standing next to them. By default, the spell targets every creature with an Intelligence score of 3 or greater that is within the spell's range, but at your discretion, you can choose to restrict the spell to certain creatures, causing it to either only deliver its message to creatures meeting a certain criteria, or to deliver it to all creatures except those meeting that criteria. The criteria must be something objective and observable. For example, you could cause the mage's decree to reach only creatures of a certain race. You can't choose recipients that rely on unobservable information, such as creatures of a certain alignment or of a particular class.  You can't pick and choose individual creatures to target or exclude. While the spell's range defaults to 1 mile per caster level, you can choose to reduce it to a smaller radius, although the spell's area can't be shaped.  The nature of the spell prevents the message it carries from having any magical power; the message can't be used to transmit spells or abilities that are conveyed via speech. The message is transmitted in your voice in whatever language you use to speak it, and is not automatically translated. Any steps you take to disguise your voice are just as effective for messages delivered via this spell as they are for your normal speech. Mage's decree isn't a language-dependent spell; all targeted creatures receive the message, but might not understand it if they don't understand the language in which you spoke the message.}
        
\DeclareSpell{Magic Aura, Greater}{illusion (glamer) []|V,  S,  F (a woolen handkerchief)|1 standard action|close (25 ft. + 5 ft./2 levels)|Targetsone creature, or one object weighing up to 20 lbs./level|1 day/level (D)|none (see text)|no}[As magic aura, but also affects creatures and allows more options.]
    \DeclareSpellDescription{Magic Aura, Greater}{If cast on an object, this spell functions as magic aura, except that if you have identified the unique spellcasting signatures of a specific individual with greater detect magic (see page 212) or a similar spell, you can make the magic aura appear to have been created by that individual. Alternatively, you can simply obscure all identifiers, making it more difficult to determine who cast the spell. In either case, if the object is the subject of a greater detect magic spell, any Spellcraft check made to identify the unique spellcasting identifiers of the aura automatically produce the  result you chose unless the observer disbelieves the spell with a successful Will save (as with magic aura, however, detect spells don't grant a save to disbelieve).  If cast on a creature, you can make that creature register to detect spells (and spells with similar capabilities) as though it were the subject of any number of spells that you specify, when the spell is cast. Alternatively, you can make the creature register as nonmagical, hiding all spell effects that he is currently affected by from such scrutiny. If you choose to make the creature register as being the subject of one or more spells, you can also alter the unique spellcasting identifiers of those spell auras, in the same fashion as described for objects.  If the target is a creature, you can also alter how the creature registers to arcane sight, making the creature appear to have or not have spellcasting or spell-like abilities, whether those abilities are arcane, divine, or psychic in nature, and the strength of the most powerful spell or spell-like ability they currently have available for use. Similarly, you can alter the way the target appears when viewed with greater detect magic, causing the last spell that he cast to seem to be any spell of your choice.}
        
\DeclareSpell{Majestic Image}{transmutation () []|V,  S,  M (a drop of paint and a ball of clay)|1 standard action|200 ft./level|Effecttransfer consciousness to an object bearing your likeness|concentration|Will negates (see text)|yes}[As enter image, but also gain bonuses on social skills while in the image.]
    \DeclareSpellDescription{Majestic Image}{You cast your consciousness into a single object within range that bears your likeness, as if choosing a specific image with the spell enter imageAPG. In addition to observing your surroundings, speaking, and manipulating the image you inhabit, however, you can converse with nearby creatures and use your normal social skills.  You gain a +5 bonus on Bluff checks to tell lies and Diplomacy checks to make a request. You gain a +2 bonus on Intimidate checks and use the object's size to determine whether you gain a bonus or penalty on Intimidate checks for size.}
        
\DeclareSpell{Matchmaker}{enchantment (charm) [mind-affecting]|S,  M (a rose petal)|1 standard action|medium (100 ft. + 10 ft./level)|Targetstwo living creatures|1 hour/level|Will partial, see text|yes}[Cause two creatures to fall in love.]
    \DeclareSpellDescription{Matchmaker}{You entice the target creatures to become romantically interested in one another. Each creature saves and applies spell resistance separately. Both must be affected for the spell to have an effect. If either creatures has a prior unfriendly or hostile attitude toward the other, it receives a +4 bonus on its saving throw.  This spell doesn't override the targets' normal sexual preferences or other limitations. If romantic feelings are incompatible for this reason, the creature instead feels an intimate platonic bond with the other.}
        
\DeclareSpell{Meticulous Match}{divination () []|V,  S|10 minutes|touch|Targetstwo objects touched|?|Fort negates (object)|yes (object)}[Determine if two things are identical.]
    \DeclareSpellDescription{Meticulous Match}{You compare two similar items and know if they are identical to one another or not. The spell can indicate an identical match, a categorical match, or no match. For instance, blood samples are identical if they are from the same creature. They are categorical if they are from the same species. There is no match if they are from different species, or if one sample is merely stage blood.  Alternatively, you can compare dissimilar items and know if they have a potential relationship. For instance, you can compare a creature's tooth against a bite mark and know if that creature could have caused the bite mark.  This spell is not infallible-an identical match can result from comparing items or creatures that are duplicates of one another. For instance, a knife might have an identical match with a stab wound if an identical knife was used to inflict the wound, and twins might have identical blood or tissues.}
        
\DeclareSpell{Obscure Poison}{abjuration () []|S,  M (herbs used in antitoxins worth 10 gp)|1 standard action|touch|Targetsone dose of poison or one venomous creature touched|1 hour/level|none|no}[Make it harder to detect a poison or a venomous creature.]
    \DeclareSpellDescription{Obscure Poison}{You make the touched poison difficult to detect or identify. Detect poison and similar effects detect an obscured poison only with a successful caster level check against a DC equal to 15 + your caster level (rolled secretly by the GM). Even if the poison is detected, the DC of Craft (alchemy) or Wisdom checks to identify the poison is increased by 10.  If cast upon a venomous creature, obscure poison disguises all of the creature's natural poisons in the same way.}
        
\DeclareSpell{Open And Shut}{illusion (glamer) []|V,  S,  F (a doornail,  doorknob,  or hinge)|1 swift action|touch|Targetsone door, window, or similar portal no more than 10 feet by 10 feet in area|1 round/level (D)|Will disbelief|no}[Obfuscate whether a door is open or closed.]
    \DeclareSpellDescription{Open And Shut}{You alter the appearance of a door and disguise whether it is open or closed. You can cause the touched door to appear closed regardless of whether it is open or closed, to appear open regardless of whether it is open or closed, or to appear to open or close. After you cast the spell, you can change between these options as a move action. Creatures using a move action to open or shut the door can attempt a Will save to disbelieve the illusion.  Regardless of how you alter the appearance of the door, creatures that believe the illusion take a -5 penalty on Perception checks regarding the door itself or creatures on the other side of the door.  This spell affects windows, gates, and similar openings in the same way it affects doors.}
        
\DeclareSpell{Open Book}{divination () [curseUM]|V,  S,  M (a page torn from a book)|1 standard action|close (25 ft. + 5 ft./2 levels)|Targetsone creature|permanent|Will negates|yes}[Make it permanently easier to learn more about a target.]
    \DeclareSpellDescription{Open Book}{You lay bare not only the mind of a target, but the target's history as well. The target takes a -2 penalty on saving throws against  divinations, and Diplomacy checks to gather information about the target gain a bonus equal to half your caster level (maximum +10).}
        
\DeclareSpell{Overwhelming Poison}{necromancy () [poison]|V,  S,  M (an adder's fang)|1 standard action|close (25 ft. + 5 ft./2 levels)|Targetsone creature or one dose of poison; see text|10 minutes/level|none|no}[Make a poison more difficult to resist.]
    \DeclareSpellDescription{Overwhelming Poison}{This spell increases the virulence of the targeted dose of poison, making the poison more difficult to resist. The poison is unaffected by delay poison, and the DC to remove it with neutralize poison is increased by 5. Additionally, the poison uses its own saving throw DC or overwhelming poison's DC, whichever is higher.  If cast on a creature that is currently suffering from exposure to one or more doses of poison, the spell applies to one of the doses of your choice, or a random dose of poison affecting the target if you don't know what poisons are afflicting the target. If cast on a creature that is venomous, this spell affects the first dose of poison that creature delivers before the end of the spell's duration.}
        
\DeclareSpell{Pack Empathy}{divination () []|V,  S|1 standard action|close (25 ft. + 5 ft./2 levels)|Targetsyou plus one willing living creature per 3 levels, no two of which can be more than 30 ft. apart|1 hour/level (D)|none|no}[Create an empathic bond with allies.]
    \DeclareSpellDescription{Pack Empathy}{You create an instinctual connection between the targets. Each can sense the others' overall emotional states, which allows them to communicate basic emotional concepts (such as alerting each other of danger due to increased stress). Once the spell has been cast on the subjects, the distance between them and the caster doesn't affect the spell as long as they are on the same plane of existence. If a subject leaves the plane, or if it dies, the spell ceases to function for it.}
        
\DeclareSpell{Peacebond, Greater}{abjuration () []|S|1 standard action|medium (100 ft. + 10 ft./level)|Targetsup to one weapon/level, no two of which can be more than 30 ft. apart|1 minute/level|Will negates (object)|yes (object)}[As peacebond, but on multiple weapons, even if they aren’t sheathed.]
    \DeclareSpellDescription{Peacebond, Greater}{If a target weapon is sheathed or slung as the spell is cast, this functions as peacebondUC, locking the target's weapon in place on its owner's body or within the weapon's sheath or holster. Anyone who then tries to draw the weapon must spend a standard action and succeed at a Strength check to do so, provoking attacks of opportunity whether the attempt succeeds or fails. The DC for Strength checks required by this spell is equal to the spell's save DC.  If a target weapon is not currently sheathed or slung as the spell is cast, the weapon immediately attempts to sheathe itself, and its wielder must succeed at a Strength check to prevent it from doing so. Once sheathed or slung in this way, the weapon is difficult to draw, as previously noted above. Unattended weapons that are not currently sheathed or slung are anchored in place by the spell, and require a successful Strength check to pick up; each attempt requires a standard action. If the wielder doesn't have a sheath or sling available for the weapon, failure on the Strength check causes the weapon to fall to the ground, at which point it requires a Strength check to pick up, as with unattended weapons.}
        
\DeclareSpell{Permanent Hallucination}{illusion (phantasm) [mind-affecting]|S|1 standard action|long (400 ft. + 40 ft./level)|Targetsone creature|permanent (D)|Will disbelief|yes}[As scripted hallucination, but permanent.]
    \DeclareSpellDescription{Permanent Hallucination}{This spell functions as audiovisual hallucination (see page 204), except that the phantasm you create includes visual, auditory, olfactory, tactile, and thermal components, and the phantasm follows a complex script. The phantasm follows that script without your having to concentrate on it and can react to stimuli the target perceives, as appropriate for the script. Unlike most illusions with a save to disbelieve, if the target disbelieves a permanent hallucination, she can choose to end the effect entirely at any time.}
        
\DeclareSpell{Phantasmal Affliction}{illusion (phantasm) [mind-affecting]|V,  S,  M (a drop of cod liver oil)|1 standard action|close (25 ft. + 5 ft./2 levels)|Targetsliving creature|see text|Will disbelief, then Fortitude or Will negates (see text)|yes}[Convince a target that it contracted an aff liction.]
    \DeclareSpellDescription{Phantasmal Affliction}{You cause the creature to believe she has a debilitating affliction. The target can attempt a Will save to recognize the affliction as unreal. If that save fails, the creature suffers an imaginary affliction of your choice.  Curse: The target believes she has been cursed. She takes a permanent -4 penalty on attack rolls, saves, ability checks, and skill checks for 1 hour per caster level. After her save to disbelieve, the target attempts a second Will save to negate this effect.  Poison: The target believes she has been poisoned. Choose a physical ability score. Each round for 6 rounds, plus 1 round per 5  caster levels, the target takes 1d3 points of damage to the chosen ability score. Each turn, the creature can attempt a Fortitude save to negate the damage and end the ongoing damage.  Wasting: The target believes she has contracted a wasting disease. Each day, the creature takes 1d4 points of Constitution damage and becomes fatigued. A successful Fortitude save prevents this damage. Two consecutive successful saves end the effect.  Since the affliction exists entirely in the creature's mind, phantasmal affliction is not affected by normal cures like neutralize poison or remove disease, or other effects like delay poison or the Heal skill. Ordinary immunities do not apply in this case (though a creature immune to the affliction receives a +4 bonus on the Will save to disbelieve the illusion). Constitution damage from the affliction can't kill the target. Instead, it causes the target to fall unconscious like other forms of ability damage. Phantasmal affliction is a spell effect and can be dispelled normally.  Placebo effectOA counters and dispels phantasmal affliction.}
        
\DeclareSpell{Pocketful Of Vipers}{conjuration (summoning) []|V,  S,  M (a snake scale and fang)|1 round|touch|Targetsobject touched|1 hour/level or until discharged (D)|Fortitude negates (object)|no}[Ward a container with summoned vipers.]
    \DeclareSpellDescription{Pocketful Of Vipers}{You set a magical ward upon the object touched, which must be a container such as a pouch, bag, backpack, or pocket. If any creature opens the container without first speaking a command word, 1d3 summoned venomous snakes (Pathfinder RPG Bestiary 255) appear, slithering out of the container and attacking that creature for 1 round/level before disappearing. They attack other creatures only if they themselves are attacked.}
        
\DeclareSpell{Poisonous Balm}{conjuration (healing) [poisonUM,  ruse]|V,  S|1 standard action|close (25 ft. + 5 ft./2 levels)|Targetsone creature|instantaneous, then 1 hour or until triggered plus 6 rounds; see text|Will partial (harmless), then Fortitude negates (see text)|yes}[As cure serious wounds, but leave behind a latent venom.]
    \DeclareSpellDescription{Poisonous Balm}{You mend the target's injuries, curing 3d8 hit points + 1 point per caster level (maximum +15) as cure serious wounds, but leaving its body laced with a subtle toxin that remains inert until you activate it. A target that decides to attempt the Will save and succeeds is healed for half (as cure serious wounds) and negates the toxin. Otherwise, you can activate the toxin by concentrating on the spell as a standard action, at which point the victim takes 1d3 points of Strength damage per round for 6 rounds. Once the poison is active, the target can attempt a Fortitude save each round to negate that round's damage and end the affliction. If you don't trigger the poison for 1 hour, the spell ends and leaves the target unharmed.  Detect poison reveals an inert poisonous balm only if the caster succeeds at a caster level check against a DC equal to 15 + your caster level. Attempts to identify poisonous balm with a skill check incorrectly identify it as cure serious wounds (see the ruse descriptor on page 192). A fooled viewer mistakes poisonous balm's aura as the lingering aura of an instantaneous effect.}
        
\DeclareSpell{Pox Of Rumors}{enchantment (compulsion) [curseUM,  mind-affecting]|V,  S,  M (a physical connection to the target; see text)|8 hours|see text|Targetsone creature|1 day/level|Will negates, then Will partial (see text)|yes}[Curse a creature to attract nasty rumors.]
    \DeclareSpellDescription{Pox Of Rumors}{You curse the target to attract negative assumptions and rumors of a sort you specify when you cast the spell. If the target fails the initial save to negate the curse, every day that it spends in a settlement, it must attempt a Will save. If it fails, it accidentally says or does something that makes others assume the rumor you specified is true in some way that is unflattering or incriminating. If the creature is not aware of the nature of the rumors, it takes a -4 penalty on these secondary saves. After the first failed save, the attitude each resident in the settlement has regarding the target is worsened by one step. For each additional failure, the target becomes the victim of focused harassment. A group of residents taunts or attacks the creature, potentially sending the authorities to investigate if the rumor indicates criminal guilt.  The spell can be cast at any distance. The difficulty of the save depends on your knowledge of the subject and what sort of physical connection you have to that creature. The target gains the same bonuses and penalties on its Will save to resist this spell as the bonuses and penalties that apply to the scrying spell (including a +5 bonus if the target is on another plane), except that you can't cast pox of rumors without at least a possession or garment to use as the material component, and the target takes no penalty when you use a possession or garment and only a -5 penalty when you use a piece of the target's body. Pox of rumors is a curse, and until its duration expires, it can be removed only by remove curse or similar magic.}
        
\DeclareSpell{Prognostication}{divination () []|V,  S,  M (rare incense and tonics worth 250 gp)|8 hours|personal|Targetsyou|instantaneous||}[Gain cryptic information from further in the future than divination.]
    \DeclareSpellDescription{Prognostication}{You glimpse the future. Prognostication functions as divination except that the spell can see up to a year and a day into the future. Because of the increased unpredictability of the distant future, prognostication is significantly more cryptic than the already-cryptic divination spell.}
        
\DeclareSpell{Quieting Weapons}{illusion (glamer) []|S|1 standard action|close (25 ft. + 5 ft./2 levels)|Targetsup to one natural or manufactured weapon per 3 caster levels|10 minutes/level|Will negates (object)|yes (object)}[Weapons make no sound and quiet their victims.]
    \DeclareSpellDescription{Quieting Weapons}{The target weapons and any ammunition they fire make no sound as part of their normal functions as a weapon. For instance, a firearm's firing would not make an explosive sound, but if you  cast this spell on a creature's bite attack, it would not prevent it from vocalizing from its mouth. The first time a creature is struck by a weapon affected by this spell, it must succeed at a Will save (SR applies to this effect) or it becomes unable to make noise louder than a whisper (Perception DC 10 to hear) whether vocally or by other means for the duration of the effect. Because the creature can still whisper, this doesn't interfere with verbal spell components. Whether it succeeds or fails its saving throw, the creature is immune to further effects from this casting of quieting weapons.}
        
\DeclareSpell{Red Hand Of The Killer}{necromancy () []|V,  S,  M (a black candle),  F (a corpse slain no more than 1 day ago per caster level)|1 standard action|see text|Targetsone creature|1 day/level (D)|Will negates|yes}[Stain the hand of a creature’s killer red.]
    \DeclareSpellDescription{Red Hand Of The Killer}{Drawing upon the spiritual link between a corpse and its killer, you reach out across space to brand the killer of the corpse you used as a focus for this spell, creating a physical manifestation of the killer's guilt.  The killer's right hand becomes stained indelibly red, and this stain can't be removed (although it can be hidden by magical or mundane means, such as disguise self or by wearing gloves). If the killer is not humanoid, or doesn't have a right hand for some other reason, the spell instead causes a red stain in the shape of a hand to appear elsewhere on the creature's body (typically on the chest).  This spell affects only the creature that directly killed the targeted corpse. Other individuals that contributed to the target's death are unaffected, and if the victim did not die from violence or died indirectly (for instance, if the creature died from suffocating after someone trapped it in a room filling with water), then the spell has no effect. The killer can attempt a Will save to resist the spell's effects. Distance is not a factor, but the killer must be on the same plane as you at the time you cast the spell, or the spell fails. Once a corpse has acted as the focus for red hand of the killer, it can never act as the focus for another casting of red hand of the killer.}
        
\DeclareSpell{Reincarnate Spy}{conjuration (healing) []|V,  S,  DF,  M (oils worth 2, 500 gp and a possession or piece of the body of the creature to resemble)|10 minutes|touch|Targetsdead creature touched|instantaneous|none (see text)|yes}[As reincarnate, but creating a body similar to that of a chosen creature, and you secretly keep part of the body.]
    \DeclareSpellDescription{Reincarnate Spy}{This spell functions as reincarnate except that you can cause the new body to resemble a particular creature, matching its age  category and sex and rerolling any race result that would be the wrong size category. The subject further gains a +5 bonus on Disguise checks to impersonate the chosen creature due to similar features, although it might take a penalty for being the wrong race.  The spell automatically leaves you with a small piece of the creature's new body, typically a lock of hair (useful for scrying and other such spells).}
        
\DeclareSpell{Resplendent Mansion}{conjuration (creation) []|V,  S,  F (a miniature cornerstone carved from precious gemstones worth 500 gp)|1 minute|long (400 ft. + 40 ft./level)|Effectopulent mansion, up to 300 feet on a side and one story tall/4 levels|1 day/level (D)|none|no}[Conjure an opulent mansion several stories tall.]
    \DeclareSpellDescription{Resplendent Mansion}{This spell creates a towering mansion. While casting the spell, you hold an image of the mansion and its desired appearance in your mind. The mansion can contain as many or as few rooms as you desire, and is decorated to match your image. You can imagine a purpose for each room of the mansion, and the proper accouterments appear within. Any furniture or other mundane fixtures function normally for anyone inside the mansion, but cease to exist if taken beyond its walls. No fixture created with this spell can create magical effects, but magical devices brought into the mansion function normally.  A resplendent mansion contains the same types of foodstuffs and servants as a mage's magnificent mansion.  Each of the mansion's exterior doorways and windows are protected by alarm spells. You choose whether each alarm is audible or mental as you cast the spell, and each alarm has a different sound (for an audible alarm) or sensation (for a mental one), allowing you to instantly determine which portal has been used.  The mansion must be created on a plot of land free of other structures. It adapts to the natural terrain, adopting all structural requirements for being built on, for example, a mountainside. The mansion adjusts around small features such as ponds or spires of rock, but can't be created on water or other nonsolid surfaces. If created on snow, sand dunes, or other soft surfaces with a solid surface underneath, the foundation reaches the solid ground. If created on a solid but unstable surface, such as a swamp or an area plagued by tremors, there's a 10\% chance each day that the mansion begins to sink or collapse.  The mansion doesn't harm creatures within the area when it appears, and can't be created among a crowd or in a densely populated area. Any creature inadvertently caught inside the mansion when the spell is cast ends up unharmed inside the complete mansion.}
        
\DeclareSpell{Rumormonger}{divination () []|V|1 standard action|long (400 ft. + 40 ft./level)|Effect1 rumor|1 day/level (D)|Will negates (see text)|yes}[Follow a rumor to see where it spreads.]
    \DeclareSpellDescription{Rumormonger}{You utter a brief anecdote or bit of news as the verbal component of this spell and track its progress through a social gathering or other crowd. When someone who heard the rumor directly from you and repeated the rumor is within your range, they glow silver to your sight, though this glow doesn't occur if the creature is in disguise (unless it was in the same disguise at the time of casting). You can choose to follow the rumor by selecting any such creature in range, at which point the creatures who heard the rumor from you no longer glow silver, and now the creatures who heard the rumor from your chosen creature glow silver instead. You can follow the rumor's path until you reach a creature that heard the rumor but didn't repeat it (or repeated it incorrectly). The glow identifies only creatures who heard the same information you conveyed. Minor cosmetic changes in the rumor don't interrupt the chain, but when the rumor no longer resembles the information you imparted, the trail stops and the spell ends.  You can instead use this spell to trace a rumor back to its source once you hear it. In this case, you repeat the rumor as you heard it for the spell's verbal component. The person who told you the rumor can attempt a Will save to end the effect. Failure indicates the silver glow leads to the individual that told her. This process continues each time you locate the next individual spreading the same information. Each individual attempts the save until one of them succeeds (in which case the spell ends and you can't attempt to trace this particular rumor again) or you trace the rumor to its original source. As with the other application of the spell, tracing a rumor back fails to cause a creature to glow if it is in disguise, unless it was in the same disguise at the time of speaking the rumor.}
        
\DeclareSpell{Scripted Hallucination}{illusion (phantasm) [mind-affecting]|S|1 standard action|long (400 ft. + 40 ft./level)|Targetsone creature/level, no two of which can be more than 30 ft. apart|1 minute/level (D)|Will disbelief|yes}[As complex hallucination, but without concentration.]
    \DeclareSpellDescription{Scripted Hallucination}{This spell functions as audiovisual hallucination (see page 204), except that the phantasm includes visual, auditory, olfactory, tactile, and thermal components, and the phantasm follows a complex script. The phantasm follows that script without y}
        
\DeclareSpell{Selective Alarm}{abjuration () []|V,  S,  F/DF (a slender iron rod 1 foot in length)|1 standard action|close (25 ft. + 5 ft./2 levels)|Areaup to 20-ft.-radius emanation centered on a point in space|2 hours/level (D)|none|no}[As alarm, but only against selected creatures.]
    \DeclareSpellDescription{Selective Alarm}{This spell functions as alarm, except that you can tweak the spell to be more discerning in what types of creatures or objects trigger the alarm. Instead of being triggered whenever a creature of Tiny size or larger enters the warded area, you can set whatever triggering conditions you wish, as long as they are based on observable phenomenon. For example, you could cause the selective alarm to trigger when creatures of a certain race (such as orcs, bugbears, or kobolds) enter the area, or whenever a group of four or more creatures enters the area together, or when a metal object is brought into the area. You can't choose triggering conditions that rely on unobservable information, such as having it triggered when creatures of a certain alignment enter the area, nor could you have the alarm be triggered by "something worth more than 5,000 gp" entering the area, or even "a weapon" entering the area, because an item's value and classification as a weapon is subjective, and might vary from one person to the next.}
        
\DeclareSpell{Shamefully Overdressed}{enchantment (compulsion) [mind-affecting]|V,  S|1 standard action|close (25 ft. + 5 ft./2 levels)|Targetscreature touched|1 round/level (D)|Will negates|yes}[Force target to remove equipment.]
    \DeclareSpellDescription{Shamefully Overdressed}{The target sees its own attire as hopelessly out of fashion, ostentatious, and embarrassing, and is filled with a compulsion  to strip off all clothing. Each round, the target must spend a move action to remove a worn item that can be removed with a move action, dropping the item once it is removed. The target doesn't distinguish between magical and nonmagical items when removing them. Each round it removes an item as determined randomly from the following slots: belt, body, chest, eyes, feet, hands, head, headband, shoulders, or wrists. When determining randomly, don't include any slots if the character has no item of that sort, and don't include items that take more than one move action to remove. Though creatures can't have more than one magic item in any of those slots, they can have multiple mundane items that fit each slot, in which case randomly decide which one they remove. If a target is wearing clothing that doesn't fit in any of those slots, such as breeches or a quiver, add it to the list of possibilities at the GM's discretion.  The target regards the discarded items with revulsion, and if forced to touch such an item (such as with a melee or ranged touch attack using the item as an improvised weapon), the target becomes sickened for 1d3 rounds. Other than the move action to remove items, the character can take whatever actions it chooses.}
        
\DeclareSpell{Shifted Steps}{illusion (glamer) [sonic]|V,  S|1 standard action|close (25 ft. + 5 ft./2 levels)|Targetsone creature or object up to 10 feet across|concentration + 1 round/level (D)|Will negates (harmless) and Will disbelief (if interacted with); see text|no}[Make a target sound as if elsewhere.]
    \DeclareSpellDescription{Shifted Steps}{You cause the target to sound as if it is elsewhere within range, including its movements, speech, and all other sounds. As long as you concentrate, you can cause the sound's apparent location to change as you see fit within range from your current location. Once you cease concentrating, the sound moves so that it remains the same relative distance and direction from the target. This spell can fool any sound-based blindsense or blindsight (including echolocation), but it can't fool other forms of detection such as other forms of blindsense or blindsight, lifesense, normal vision, and tremorsense. The target receives a saving throw against the effect if it doesn't wish for you to shift its sound, and any creature that interacts with the illusion receives a Will save to disbelieve the glamer.}
        
\DeclareSpell{Swallow Poison}{transmutation () []|V,  S,  DF|1 standard action|personal|Targetsyou|1 hour/level or until discharged||}[Protect yourself from ingested poison, then spit it out in a cone.]
    \DeclareSpellDescription{Swallow Poison}{A special gland grows on the inside of your throat, which absorbs any poisons that you ingest, and can be used to expel them in a toxic spray. Any time you consume ingested poison during the spell's duration, you can roll a caster level check against the poison's save DC in order to harmlessly absorb the poison, ignoring its effects. The spell can absorb up to 1 dose of poison per 3 caster levels, after which the gland is unable to process any more poison, and any further doses of ingested poison affect you normally.  You store each dose of absorbed poison within the gland, and you can end the spell to spray one dose of absorbed poison out of your mouth as a standard action. This sprays the poison in a 15-foot cone. Each creature in the area must succeed at a Reflex save (at swallow poison's DC) or be exposed to the sprayed poison, which is treated as though it were a contact poison for this purpose. Everything about the poison other than its type, including its frequency, effect, and saving throw DC, are unaffected by this spell. Any other doses of poison you had absorbed instantly become inert when you end the spell. If the spell's duration ends without you spraying a poison, all poisons you had absorbed become inert.}
        
\DeclareSpell{They Know}{enchantment (compulsion) [emotionUM,  fear]|V,  S,  F/DF,  M (a drop of black ink)|1 standard action|medium (100 ft. plus 10 ft./level)|Targetsone intelligent creature|1 minute/level (D)|Will negates|yes}[Convince target that a nearby creature knows her greatest secret.]
    \DeclareSpellDescription{They Know}{You instill a target with the fear that the nearest other intelligent creature knows her darkest secret. If she is in the midst of another situation as vital as protecting her darkest secret, such as combat, this doesn't cause her to cease participating, but otherwise, she becomes compelled to use her abilities and skills to determine exactly how much the other individual knows. Even if she isn't around other creatures or otherwise avoids confronting her paranoia for a time, the nagging fear causes the target to become shaken for the duration of the spell (though this doesn't stack with other fear effects to make the target frightened or panicked).  The subject's paranoia increases over time, forcing her to take additional steps to protect herself from the other creature. Depending on her personality (or subject to the GM's discretion), she might confess, publicly demand to know what the other creature knows, attack the other creature to silence it, or offer the other creature a bribe.  The spell fails if the target truly feels she has nothing to hide.}
        
\DeclareSpell{Trace Teleport}{divination () []|V,  S,  F (a magnifying lens)|1 standard action|40 feet|Area40-ft.-radius emanation centered on you|1 minute/level|none|no}[Determine where and when teleportation occurred, and glimpse the origin or destination.]
    \DeclareSpellDescription{Trace Teleport}{You immediately become aware of any teleportation effects that begin or end within the spell's area. The spell's area radiates from you and moves as you move. You know the exact origin point of any teleportation effect that originates within the spell's area, and the exact end point of any teleportation effect that terminates within the spell's area. Further, you can detect the lingering traces of any teleportation effect that occurred up to 1 hour previously, in the same fashion. You intuitively know, to the nearest minute, when the teleportation effect occurred.  Whenever you detect the origin point or termination point of any teleportation effect with this spell, you can study that origin point or termination point for 1 round. If you do, you can attempt a caster level check (DC = 11 + the teleportation effect's caster level), taking a -5 penalty if the effect occurred more than 1 minute ago. If you succeed, you gain a glimpse of the teleportation effect's termination point (if you detected an origin point) or origin point (if you detected a termination point). This glimpse lasts long enough for you to get a brief look at the area, but not long enough to scrutinize it in detail. It doesn't come with any geographic knowledge of the location, so it is not sufficient for teleport or similar magic. You can't retry the caster level check, even if you cast trace teleport again.}
        
\DeclareSpell{Trade Items}{conjuration (teleportation) []|V,  S,  F (any held object weighing no more than 5 lbs./level)|1 standard action|close (25 ft. + 5 ft./2 levels)|Targetsone object weighing no more than 5 lbs./level|instantaneous|Will negates (object)|yes (object)}[Swap a focus object with a target object.]
    \DeclareSpellDescription{Trade Items}{You cause the target object and the object used as a focus for the spell to immediately swap places. The target object appears in your hand (or falls to the ground in your square if you are unable to hold it). The stronger the connection between the two objects, the more difficult the spell is to resist, as indicated on the table below. The modifiers are cumulative.  Similarity DC  The two objects are the same type of object (such as +1 "keys," "swords," "scrolls")  The two objects are made of the same material +1  The two objects have the same weight +1 (accurate to the nearest ounce)  The two objects are part of the same set or made +2 in the exact same mold  The target object has a higher gp value than the focus item -2  The target object is tied its holder, such as a bonded item -2}
        
\DeclareSpell{Treacherous Teleport}{conjuration (teleportation) [ruse]|V|1 standard action|personal and touch|Targetsyou and touched objects or other touched willing creatures|instantaneous|none and Will negates (object)|no and yes (object)}[As teleport, except you choose some creatures to suffer a mishap or go elsewhere.]
    \DeclareSpellDescription{Treacherous Teleport}{This spell functions as teleport except that you can opt to intentionally cause any number of the creatures traveling with you to suffer a mishap or arrive in a specific different location you visualize simultaneously with the original destination, or both.  Attempts to identify treacherous teleport with a skill check incorrectly identify it as teleport ( see t he r use d escriptor o n page 192).}
        
\DeclareSpell{Triggered Hallucination}{illusion (phantasm) [mind-affecting]|V,  S,  M (jade dust worth 25 gp)|1 standard action|long (400 ft. + 40 ft./level)|Targetsone creature|permanent until triggered, then 1 minute/level|Will disbelief|yes}[As scripted hallucination, but it only appears when triggered.]
    \DeclareSpellDescription{Triggered Hallucination}{This spell functions as audiovisual hallucination (see page 204), except that this spell's phantasm has no apparent effect until a specific condition occurs. You must overcome the target's spell resistance to plant the triggered hallucination, but the target doesn't attempt a Will save to disbelieve the illusion until the condition occurs (at which point it receives an automatic Will save, as with audiovisual hallucination). The phantasm can include auditory, olfactory, visual, tactile, and thermal elements, including intelligible speech.  You set the triggering condition (such as hearing a certain word or seeing a type of creature) when casting the spell. The event that triggers the illusion can be as general or detailed as desired but must be based on an audible, tactile, olfactory, or visual trigger. The trigger can't be based on any quality not  normally obvious to the senses, such as alignment. Triggered hallucination uses the target's senses to notice triggers.}
        
\DeclareSpell{True Prognostication}{divination () []|V,  S,  M (rare incense and tonics worth 1, 000 gp)|1 week|personal|Targetsyou|instantaneous||}[Gain incredibly cryptic information from the distant future.]
    \DeclareSpellDescription{True Prognostication}{True prognostication functions as divination except that the spell can see up to 100 years into the future. Because of the extreme unpredictability of the far-distant future, true prognostication is incredibly cryptic when used to learn about events on such large a scale.}
        
\DeclareSpell{Underbrush Decoy}{transmutation () []|S|1 swift action|close (25 ft. + 5 ft./2 levels)|Targetsone non-creature plant of size Tiny, Small, or Medium|1 round|Will negates (object)|yes (object)}[Create a rustling distraction to hide.]
    \DeclareSpellDescription{Underbrush Decoy}{You cause a plant to rustle noisily, distracting nearby creatures. You can attempt a Bluff check to create a distraction to hide, using your caster level + your Wisdom modifier in place of your total Bluff skill bonus and applying the result to all creatures within 30 feet. You count the target as distracted, as do any creatures that knew about your distraction in advance. Creatures might not be distracted if they detect you casting the spell or otherwise anticipate your subterfuge.}
        
\DeclareSpell{Undetectable Trap}{illusion (glamer) []|V,  S,  M (a square of black silk worth 50 gp)|10 minutes (see text)|touch|Targetstrap touched|1 day/level (D)|none|no}[Make a trap extremely difficult to find.]
    \DeclareSpellDescription{Undetectable Trap}{You shroud a single trap with a powerful illusion to make it more difficult to locate. Spells like detect magic can't locate any magic aura from either the target trap or from undetectable trap. Furthermore, a character under the effect of the find traps spell doesn't receive an automatic chance to locate the target trap when she comes within 10 feet of it, and the bonus on Perception checks from find traps doesn't apply to attempts to notice the target trap. Add 1/2 your caster level to the DC for any creature without the trapfinding class ability to notice the target trap with Perception checks.  A ranger with the ranger traps class feature (Pathfinder RPG Ultimate Magic 64) can cast this spell on a ranger trap he creates as part of the same action he uses to prepare a ranger trap. This doesn't reduce the spell's casting time if such a ranger casts this spell on an ordinary trap, even one that he created himself using Craft (traps).}
        
\DeclareSpell{Unerring Tracker}{divination () []|V,  S,  DF|10 minutes|personal|Targetsyou|10 minutes/level|none|no}[Follow an entire trail unerringly.]
    \DeclareSpellDescription{Unerring Tracker}{During this spell's duration, you can touch the sign of a creature's passage that you have identified using the Survival skill to make the other steps in the creature's path perfectly clear to you, no matter how minute. This trail can't be more than 24 hours old. You can follow the trail at any speed, provided you have line of sight to the trail. You can distinguish the trail of the particular creature followed even if it joins and splits with other trails. Once you have selected a trail to follow using this spell, it can't be changed.  The spell is unable to follow teleportation or interplanar travel for any distance. The trail appears to end where the creature teleported (though a successful Spellcraft check allows you to determine the method of teleportation, if a spell was used). Unerring tracker can't track creatures under the effect of a pass without trace spell, as those creatures leave no trail at all, but can track creatures using nondetection (though not mind blank).  This spell can be used to track flying creatures, but the trail must at least begin on a solid surface.  This spell doesn't reveal the creature's current position or any shorter path than the one it followed (for instance, it will not reveal that the creature doubled back upon the trail until you reach the point where the creature turned around.) It doesn't reveal traps or other hazards along the trail.}
        
\DeclareSpell{Urban Step}{conjuration (teleportation) []|V,  S,  M (scrap of cobweb)|1 standard action|medium (100 ft. + 10 ft./level)|Targetstwo doors or other portals in range|1 round||}[Step into one doorway and out another.]
    \DeclareSpellDescription{Urban Step}{You set up a magical connection between two doors (or other physical portals, such as windows) that both must be within range, line of sight, and line of effect. During the spell's duration, you can move through one of the two portals. When you do, you teleport to the other portal, emerging in either direction. Both portals must be open and unobstructed when you cast the spell and enter the first portal, and they both must be large enough for you to fit through, otherwise the spell ends and the teleportation fails. You can bring along objects as long as their combined weight doesn't exceed your maximum load. You can't bring other creatures with you, and other creatures that go through the portals don't teleport. Once you step through, the spell ends and you can't take any other actions until your next turn.}
        
\DeclareSpell{Vicarious View}{divination (scrying) []|V,  S|1 standard action|touch|Effectmagical sensor|1 minute/level (D)|Will negates (object)|yes (object)}[Plant a scrying sensor that you can use to spy on a creature, object, or location.]
    \DeclareSpellDescription{Vicarious View}{You plant a scrying sensor on a touched creature, object, or point in space, allowing you to see and hear the creature, object, or point and its surroundings (approximately 10 feet in all directions). If the creature or object on moves, the sensor moves with it. Unlike other scrying spells, vicarious view doesn't allow magically or supernaturally enhanced senses to work through it.}
        
\DeclareSpell{Voluminous Vocabulary}{divination () []|V,  S,  M (a quill)|1 standard action|touch|Targetscreature touched|8 hours (D)|Will negates (harmless)|yes (harmless)}[Grant ability to speak, read, and write one or more languages for 8 hours.]
    \DeclareSpellDescription{Voluminous Vocabulary}{You choose any language (except for secret languages, such as Druidic). The target gains the ability to speak, understand, read, and write that language. When you cast this spell, you can attempt a DC 15 Linguistics check. If you succeed, choose an additional language, plus one more language for every 10 by which your check result exceeded the DC.  Written material can be read at the rate of one page (250 words) per minute. As with comprehend languages, this spell doesn't impart insight into material the target read, just the literal meaning, and it doesn't allow the target to read magical writing or decipher codes.  If the target lacks the mental capacity to grasp a language, it still gains enough knowledge to respond to and carry out even extremely complex commands or suggestions coached in the language (whether written or spoken). However, since this spell endows the target merely with a temporarily enhanced vocabulary, the person offering instructions to nonsentient creatures must take care to avoid metaphors or any other ambiguity.}
        
\DeclareSpell{Wizened Appearance}{transmutation (polymorph) []|V,  S|1 standard action|touch|Targetscreature touched|1 hour/level|Fortitude negates|yes}[Make a target appear as an older version of itself.]
    \DeclareSpellDescription{Wizened Appearance}{You polymorph your target to look like an older version of itself. You select how much older (for example, "10 years older" or "as an adult"). You can't otherwise change the target's appearance other than those details directly associated with aging (for example, a target's hair might turn gray or the target might develop liver spots). This spell allows children of creatures that are Medium or smaller when fully grown to grow one size category to the normal, adult size of that type of creature, but otherwise the selected age increase doesn't alter the creature's size. A change in size doesn't alter the target's ability scores. This spell doesn't affect or cause any age-based modifications to ability scores or other age-related effects like dragon age categories or natural metamorphoses.  Wizened appearance and youthful appearanceUM counter and dispel each other.}
    
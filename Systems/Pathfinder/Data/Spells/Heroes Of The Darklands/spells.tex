    
\DeclareSpell{Mage's Crawl Space}{transmutation|V,  S,  M (a piece of clay)|1 standard action|personal|Effect: 5-foot sphere centered on yourself|1 hour/level (D)|none|no}[]
    \DeclareSpellDescription{Mage's Crawl Space}{When you cast this spell, you merge with an adjacent natural stone surface, forming a 5-foot pocket within. If the stone surface does not have enough volume to safely contain a 5-foot sphere of empty space, the spell fails. When created, there is sufficient air within the pocket for up to one Medium creature to survive for the spell's duration, including one additional Tiny or smaller creature, such as a familiar. At the end of the spell's duration, you are immediately expelled into the nearest available open space adjacent to the surface you merged with.}
        
\DeclareSpell{Rock Whip}{conjuration (creation)|V,  S|1 standard action|0 ft.|Effect: whip of earth and stone|1 round/level (D)|none|no}[]
    \DeclareSpellDescription{Rock Whip}{A 15-foot-long lash of crumbling crystal and earth emerges from the palm of your hand. This weapon is treated as a nonmagical whip that deals 1d8 points of bludgeoning damage. You can wield this weapon as a whip as if you were proficient with it, and it isn't subject to the disarm or sunder combat maneuver. The whip passes through natural unworked stone effortlessly, allowing you to ignore cover between you and your target from such sources. Armor and natural armor have no effect on the damage dealt by a rock whip (unlike a normal whip), but the whip deals no damage to outsiders with the earth subtype. Attacks with a rock whip strike with resounding  force; you can make a free bull rush combat maneuver against any creature you strike with a rock whip, using your caster level in place of your base attack bonus and your primary casting ability score modifier (Charisma for sorcerers, Intelligence for wizards, and so on) in place of your Strength modifier.}
        
\DeclareSpell{Shadowfade}{illusion (shadow)|V,  S,  M (a blindfold)|1 standard action|touch|Targets: creature touched|1 minutes/level (D)|Will negates (harmless)|yes (harmless)}[]
    \DeclareSpellDescription{Shadowfade}{In areas of darkness, the target of shadowfade is invisible to creatures using darkvision to see. In areas of dim light, the target gains concealment against creatures using darkvision. This spell has no effect in areas of normal light or brighter, and is automatically dispelled if the target enters an area of bright light or takes a hostile action.}
        
\DeclareSpell{Skyshroud}{divination (scrying)|V,  S,  F (a jar of earth from the surface)|1 minute|long (400 ft. + 40 ft./level); see text|Effect: hemisphere that cannot extend beyond four 10-ft. cubes + one 10-ft. cube/level (S)|1 hour/level|none|no}[]
    \DeclareSpellDescription{Skyshroud}{The caster designates the area of a hemisphere within the spell's range. If a solid object would block the creation of the hemisphere (such as a ceiling or wall), the spell instead conforms to the geometry of the location up to the maximum area of effect for the spell. The hemisphere displays an image of the sky as it appears directly above where the spell was cast. This effect bypasses natural impediments, such as rock or stone, that would obstruct the targeted location from the sky.  The image of the sky created by this spell counts as the actual sky for purposes of effects that require view of the sky, sun, stars, or other features (including for spell preparation or  deific obediences). It does not enable spells or effects that draw effects from the sky (such as call lightning).  Light created by this effect functions normally and harms creatures vulnerable to bright light or direct sunlight.}
        
\DeclareSpell{Concealed Breath}{transmutation|S,  M/DF (a palm-sized stone)|1 standard action|touch|Targets: living creatures touched|1 hour/level; see text|Will negates (harmless)|yes (harmless)}[]
    \DeclareSpellDescription{Concealed Breath}{This spell allows affected creatures to hold their breath freely without negative effects or risk of suffocation. Divide the duration evenly among all creatures touched. A creature that doesn't need to breathe because of this spell is not at risk of drowning and is immune to effects that require breathing, such as inhaled poisons. This does not grant immunity to cloud or gas attacks that don't require breathing. While a creature is holding its breath, it can't speak or cast spells with a verbal component (unless it's using Silent Spell). This spell does not prevent a creature from breathing normally; it just removes the need for the creature to breathe.}
        
\DeclareSpell{Morning Sun}{evocation [light]|V,  S,  M (500 gp worth of gold dust)|1 standard action|close (25 ft. + 5 ft./2 levels)|Area: 60-ft. radius|1 minute/level (D)|Fortitude negates and Reflex half, see text|no}[]
    \DeclareSpellDescription{Morning Sun}{This spell conjures a miniature sphere of sunlight, approximately the size of a human fist, at a desired location within range. The sphere sheds bright light in a 60-foot-radius burst. Creatures that take penalties in bright light do so while within the sphere's area of illumination. Creatures that start their turns within the area of illumination and that are damaged or destroyed by sunlight must succeed at a Fortitude save or become staggered until 1d4+1 rounds after they leave the affected area. Non-creatures, such as hazardous fungi and mold that are destroyed by sunlight, become inert for the duration of the spell.  The sphere cannot be moved from the place it was cast. The sphere deals 10d6 points of fire damage to anything it touches and anything that passes within 5 feet of it. A successful Reflex save reduces this damage by half.}
        
\DeclareSpell{Pale Flame}{evocation [fire]|V,  S,  M (caphorite shard)|1 standard action|0 ft.|Effect: flame in your palm|1 minute/level (D)|none|yes}[]
    \DeclareSpellDescription{Pale Flame}{This spell functions as per produce flame, except that the  flames never glow brighter than dim light, including any fires started by this spell. These flames cast light only half the distance of a torch and cannot be seen from more than 100 feet away. The flames deal 2d6 points of fire damage + 1 point per caster level (maximum +5). Against plants, this damage increases to 2d6 points + 2 points per caster level (maximum +10).}
        
\DeclareSpell{Radiation Ward}{abjuration|V,  S|1 standard action|touch|Targets: creature touched|1 hour/level (D)|Fort negates (harmless)|yes (harmless)}[]
    \DeclareSpellDescription{Radiation Ward}{A creature warded by this spell gains a +4 bonus on saving throws against radiation-based effects. In addition, the warded creature is immediately aware when it enters an area  of radiation, as well as the radiation level (low, medium, high, or severe) suffusing the area.}
        
\DeclareSpell{Entomb}{transmutation [earth]|V,  S,  M (a pristine geode worth at least 1, 000 gp)|1 minute|medium (100 ft. + 10 ft./level)|Effect: up to three 10-ft. cubes/level (S)|permanent|none|no}[]
    \DeclareSpellDescription{Entomb}{You designate an aboveground area and send it deep within the earth. During the casting of this spell, the chosen area is rocked by minor tremors that alert nearby creatures to the impending danger. Once the spell is complete, the chosen area is drawn into the earth and buried in a self-contained vault, with the uppermost point at a depth of up to 10 feet per caster level below the surface. The surrounding atmosphere and all creatures within the selected area at the end of the spell's casting time are entombed in this vault.  The magic of the vault maintains the natural light, temperature, and air quality of the area as it originally existed aboveground, but any magical effects that affected these qualities do not have their duration extended. For example, a sunny field would remain brightly lit indefinitely, despite being buried underground, while a daylight spell would expire at the end of that spell's duration.  When creating the vault, you can leave a tunnel to the surface world (up to 20 feet wide), or you can completely isolate the vault from the outside world. Casting this spell does not destroy or damage anything that may have existed in the ground where you choose to place this vault, but instead pushes it farther down into the earth. Should your vault be dispelled, the entombed area returns to the surface, and anything that may have been displaced by the vault returns to its original place. If there is something already on the surface that blocks the vault's path, the vault pushes it harmlessly aside when returning.}
        
\DeclareSpell{Intensify Psyche}{enchantment [emotion,  mind-affecting]|V,  S|1 standard action|medium (100 ft. + 10 ft. level)|Effect: one creature|1 minute/level|Will negates|yes}[]
    \DeclareSpellDescription{Intensify Psyche}{A sense of euphoria suffuses the target, amplifying all sensations, good or bad. The target creature gains a +2 competence bonus on Diplomacy, Handle Animal, Perform, and Sense Motive checks. In addition, the DC for any spells or effects with the emotion or pain descriptors cast or caused by the target increases by 1 while the target is under the effects of the spell. However, increased sensitivity causes the target to take a -2 penalty on saving throws against spells and effects with the emotion or pain descriptor.}
        
\DeclareSpell{Surface Excursion}{conjuration (teleportation)|V,  S,  M (a handful of soil that has been in the sun for at least 6 hours)|3 rounds|touch|Effect: you plus up to one willing creature per 3 levels|1 hour/level (D)|Will negates (harmless)|no}[]
    \DeclareSpellDescription{Surface Excursion}{You instantly transport yourself and touched allies onto the nearest sky-facing surface directly above you. The destination surface may be solid ground or the surface of a body of water (or other liquid), depending on your position. If reaching the surface would require crossing planar boundaries or no sky-facing surface exists directly above your position, the spell fails and has no effect.  When the spell is cast, a magical beacon appears at both your departure and arrival points, suspended 3 feet in the air. These beacons are invisible to everyone but you and those allies touched when the spell was cast. You and every creature originally targeted by the spell can touch the beacon at your arrival point to teleport back to your original departure point. Each creature that chooses to return in this manner is transported to a free space adjacent to the beacon at your original departure point. If the area around the departure beacon is occupied by a solid body (for instance, rubble from a cave-in), then you and each creature traveling with you take 1d6 points of damage and are shunted to a random open space on a solid surface within 100 feet of the beacon. If there is no free space within 100 feet, the spell effect ends with no return transportation.}
    
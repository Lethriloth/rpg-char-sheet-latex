    
\DeclareSpell{Alpha Instinct}{enchantment (charm) [mind-affecting]|V,  S,  M (a bit of musk from an alpha wolf or another socially dominant animal)|10 minutes|personal|Targetsyou|1 hour/level (D)||}[Gain bonuses when you’re interacting with animals.]
    \DeclareSpellDescription{Alpha Instinct}{Animals perceive you as a social superior. The starting attitude of animals you interact with improves by one step, and you gain a +2 morale bonus on Handle Animal checks. Helpful animals provide simple assistance (such as using the aid another action on skill checks they understand) as long as the spell lasts, but they flee from attacks and do not support you in combat. At one point during the spell's duration, you can issue a nonverbal command to helpful animals within 60 feet to forage for you; the animal or animals immediately head out into the wilderness to do just that. After 1 hour, the animals return to the location where you issued the command to make you an offering of edible food sufficient for one meal for you and a number of others equal to your caster level.}
        
\DeclareSpell{Aquatic Cavalry}{conjuration (summoning) []|V,  S,  DF|1 round|medium (100 ft. + 10 ft./level)|Effect1 hippocampus plus 1 hippocampus/3 caster levels|1 hour/level (D)|none|no}[Summon hippocampi to serve as aquatic mounts.]
    \DeclareSpellDescription{Aquatic Cavalry}{You summon a school of hippocampi (one plus one per 3 caster levels, to a maximum of six at 15th level; Pathfinder RPG Bestiary 2 155) to serve as combat-trained mounts. The hippocampi avoid combat if possible but defend themselves if attacked. If any hippocampus attacks, the remaining duration of the spell changes from 1 hour per level to 1 round per level (so if a full 4 hours remained, the hippocampi last for only 4 more rounds).}
        
\DeclareSpell{Bleed For Your Master}{enchantment (compulsion) [mind-affecting]|V,  S|1 immediate action|touch|Targetsyour animal companion, familiar, or fiendish servant|instantaneous|none|no}[Compel a companion to take damage for you.]
    \DeclareSpellDescription{Bleed For Your Master}{When you would be hit by an attack that requires an attack roll, or when you are within an area of effect that allows you to attempt a Reflex saving throw, with a single touch you compel the target to throw itself in front of the blow or shield you from the effect.
 If an attack roll triggered this spell's casting, the target takes the damage instead of you, even if the attack roll would not normally hit the target's Armor Class.
 If an area of effect that allows you to attempt a Reflex save triggered this spell's casting, the target instead grants you cover against the effect (+2 bonus on Reflex saves) if it is your size or smaller, or improved cover (+4 bonus on Reflex saves and improved evasion against the effect) if it is larger than you. The target automatically fails its Reflex save against the effect but can still benefit from improved evasion and similar mitigating effects. After taking damage, the target is shaken for 1 minute. If the target was already shaken, it becomes frightened instead.}
        
\DeclareSpell{Callback}{conjuration (teleportation) []|V,  S,  M (a crushed grasshopper)|1 standard action|long (400 ft. + 40 ft./level)|Targetsyour familiar or spirit animal|10 minutes/level or until expended (see text)|Fortitude negates (harmless)|yes (harmless)}[Teleport a familiar or spirit animal to your side when it is injured.]
    \DeclareSpellDescription{Callback}{If the target takes hit point damage while within range of this spell, it immediately teleports to your space (or adjacent to your space) after the damage is applied. If the target is killed, its corpse teleports instead. Optionally, you can specify a number of points of damage for your familiar to take before the spell takes effect, but you must do so when the spell is cast. Once the familiar has teleported back to you, the spell ends.}
        
\DeclareSpell{Callback, Greater}{conjuration (teleportation) []|V,  S,  M (a crushed grasshopper)|1 standard action|1 mile/level|Targetsyour familiar or spirit animal|1 hour/level or until expended|Fortitude negates (harmless)|yes (harmless)}[Teleport a familiar or spirit animal from miles away to your side when it is injured.]
    \DeclareSpellDescription{Callback, Greater}{This spell functions as callback, except as noted above.}
        
\DeclareSpell{Cave Fangs}{conjuration (creation) [earth]|V,  S,  M (a sharp gemstone fragment worth 150 gp)|1 standard action|close (25 ft. + 5 ft./2 levels)|Areaone 5-ft. square/level (S); the area must be a stone surface|1 day/level or until triggered (D)|Reflex half (see below)|no}[ Create a trap from stalactites and stalagmites.]
    \DeclareSpellDescription{Cave Fangs}{You create a magical trap in the area that causes deadly stalactites or stalagmites to lurch out of the floor or ceiling of a stone surface to "bite" an intruder. The magical trap is triggered whenever a Small or larger creature (other than you or your animal companion, familiar, or spirit animal) moves through the affected area. The effect of cave fangs depends on whether you create stalactites or stalagmites (see below). You can place these trapped squares anywhere within the spell's range; they need not be adjacent to each other, and you can create any mix of stalactites and stalagmites you wish. Cave fangs can be detected and disarmed as per a magical trap with successful DC 30 Perception and Disable Device checks. All trapped squares created by a single casting are linked, and they are all disabled if one of them is. If you place the traps in an area that is occupied by a creature, the spell effect is triggered at the completion of the casting.
 Stalactites: Shards of rock drop from the ceiling, dealing 3d8 points of bludgeoning and piercing damage (Reflex half) and creating an area of dense rubble that costs 2 squares of movement to enter. Dense rubble adds 5 to the DC of Acrobatics checks and adds 2 to the DCs of Stealth checks. A creature that fails its Reflex save is pinned to the ground under stalactites and rubble, gaining the entangled condition until it can free itself with a successful DC 15 Strength check or DC 20 Escape Artist check. One Small or larger creature can automatically clear the rubble by working for 1 minute.
 Stalagmites: Piercing spires of rock erupt up from the ground, dealing 3d8 points of piercing damage and knocking the creature prone (a creature that succeeds at a Reflex saving throw takes half damage and avoids being knocked prone). Once the stalagmites appear, they function thereafter as spike stones for 1 minute per caster level and then crumble to dust.
 If cave fangs is cast inside of a cave or cavern, each effect of the spell deals 4d8 points of damage instead of 3d8, and creatures that trigger the cave fangs take a -2 penalty on their saving throws against the spell's effect.}
        
\DeclareSpell{Companion Transposition}{conjuration (teleportation) []|V,  S,  F (a feather,  tuft of fur,  or similar memento from your target)|1 standard action|long (400 ft. + 40 ft./level)|Targetsyourself and one animal companion, familiar, or spirit animal within range|instantaneous|none (harmless)|no}[Swap places with your animal companion, familiar, or spirit animal via teleportation.]
    \DeclareSpellDescription{Companion Transposition}{You trade places with the other target, teleporting as if you were both affected by dimension door. Both you and the other target arrive in a square you choose in the other's former space. If the targets are different sizes, they must appear in locations that cover previously occupied squares. As with dimension door, after casting this spell, you can't take any other actions until your next turn, and the other target is staggered until the end of its next turn due to the transposition.}
        
\DeclareSpell{Earth Tremor}{transmutation () [earth]|V,  S|1 standard action|up to 30 ft. (see text)|Area30-ft. line, 20-ft. cone-shaped spread, or 10-ft.-radius spread (see text)|instantaneous|Reflex half (see text)|no}[Unleash a tremor that creates difficult terrain and can knock foes down and damage them.]
    \DeclareSpellDescription{Earth Tremor}{You strike the ground and unleash a tremor of seismic force, hurling up earth, rock, and sand. You choose whether the earth tremor affects a 30-foot line, a 20-foot cone-shaped spread, or a 10-foot-radius spread centered on you. The space you occupy is not affected by earth tremor. The area you choose becomes dense rubble that costs 2 squares of movement to enter. Dense rubble adds 5 to the DC of Acrobatics checks and adds 2 to the DC of Stealth checks. Creatures on the ground in the area take 1d4 points of bludgeoning damage per caster level you have (maximum 10d4) or half damage on a successful save. Medium or smaller creatures that fail their saves are knocked prone.
 This spell can be cast only on a surface of earth, sand, or stone. It has no effect if you are in a wooden or metal structure or if you are not touching the ground.}
        
\DeclareSpell{Echo}{illusion (figment) []|S,  F (a conch shell)|1 standard action|close (25 ft. + 5 ft./2 levels) or long (400 ft. + 40 ft./level); see text|Areaone 10-ft. cube/level (S)|1 round/level (D)|Will disbelief (if interacted with)|yes}[ Cause a sound to repeat itself.]
    \DeclareSpellDescription{Echo}{You cause a sound heard in the target area up to 1 round ago and lasting up to 1 round in duration to repeat at a regular interval. The original sound need not have come from the area, but it echoes from the target area at its original full volume. Any special effects of the sound are not duplicated by this spell. If the area is naturally prone to echoes, such as a space surrounded on at least two sides by cliffs or high river banks, the spell's range is long. Otherwise, the spell's range is close.
 You can concentrate as a standard action to alter the echo. You can change it to be any sound audible in the area within the last 1 round, move the apparent source of the sound within the area, or attempt a Bluff check to create a distraction to hide with a bonus equal to your caster level.}
        
\DeclareSpell{Explosion Of Rot}{necromancy () []|V,  S,  M (a rotting flower)|1 standard action|close (25 ft. + 5 ft./2 levels)|Area10-ft.-radius burst|instantaneous|Reflex half (see text)|yes}[Call forth a burst of decay that damages and can stagger targets.]
    \DeclareSpellDescription{Explosion Of Rot}{You call forth a burst of decay that ravages all creatures in the area. Even nonliving creatures such as constructs and undead crumble or wither in this malignant eruption of rotting energy. Creatures in the area of effect take 1d6 points of damage per caster level (maximum 15d6) and are staggered for 1d4 rounds. A target that succeeds at a Reflex saving throw takes half damage and negates the staggered effect. Plant creatures are particularly susceptible to this rotting effect; a plant creature caught in the burst takes a -2 penalty on the saving throw and takes 1 extra point of damage per die.}
        
\DeclareSpell{Fey Form I}{transmutation (polymorph) []|V,  S,  M (a piece of the creature whose form you plan to assume)|1 standard action|personal|Targetsyou|1 minute/level (D)||}[Assume the form of a Small or Medium fey creature.]
    \DeclareSpellDescription{Fey Form I}{You assume the form of a Small or Medium creature of the fey type. Your base speed changes to match the new form's base speed, with a maximum speed of 60 feet (even if the chosen fey form has a base speed in excess of that speed). If the form you assume has any of the following abilities, you gain those abilities: climb speed 30 feet, fly speed 30 feet (average maneuverability), swim speed 30 feet, darkvision 60 feet, low-light vision, scent, and boot stomp. If the form you assume has the aquatic subtype, you can breathe air and water. If the creature has any weaknesses, you gain those weaknesses. If a listed ability depends on an item (as is the case with boot stomp), this spell transforms the nearest counterpart among your worn gear into that item.
 You can more easily cast spells that the creature has as spell-like abilities, although you must still cast them as normal for your class. When you cast a spell that the creature has as a spell-like ability, it requires no verbal or somatic components and can't be countered.
 Small Fey: If you assume this form, you gain a +2 size bonus to your Dexterity and Constitution scores.
 Medium Fey: If you assume this form, you gain a +2 size bonus to your Strength and Constitution scores.}
        
\DeclareSpell{Fey Form II}{transmutation (polymorph) []|V,  S,  M (a piece of the creature whose form you plan to assume)|1 standard action|personal|Targetsyou|1 minute/level (D)||}[ Assume the form of a Tiny or Large fey creature.]
    \DeclareSpellDescription{Fey Form II}{This spell functions as fey form I, except it also allows you to assume the form of a Tiny or Large creature of the fey type. Your base speed can't increase above 90 feet this way. If the form you assume has any of the following abilities, you gain those abilities: burrow speed 30 feet, climb speed 90 feet, fly speed 60 feet (good maneuverability), swim speed 60 feet, all-around vision, blindsense 30 feet, darkvision 60 feet, low-light vision, scent, see in darkness, abduct, animated hair, bleed, blood rage, boot stomp, burn, compression, constrict, crushing leap, DR 2/cold iron, grab, heavy weapons, icewalking, kneecapper, nasal spray, no shadow, oversized weapons, poison, putrid vomit, rock throwing (50 feet, 1d6 damage), sound mimicry, trackless step, trample, tree meld, undersized weapons, and woodland stride. If the creature has immunity to mind-affecting effects or poison, you gain a +4 resistance bonus on saves against those effects. If the creature has any weaknesses, you gain them.
Tiny Fey: If you assume this form, you gain a +6 size bonus to your Dexterity score and take a -2 penalty to your Strength score.
Large Fey: If you assume this form, you gain a +4 size bonus to your Strength and Constitution scores and take a -2 penalty to your Dexterity score.}
        
\DeclareSpell{Fey Form III}{transmutation (polymorph) []|V,  S,  M (a piece of the creature whose form you plan to assume)|1 standard action|personal|Targetsyou|1 minute/level (D)||}[Assume the form of a Diminutive or Huge fey creature.]
    \DeclareSpellDescription{Fey Form III}{This spell functions as fey form II except it allows you to assume the form of a Diminutive or Huge creature of the fey type. If the form you assume has any of the following abilities, you gain those abilities: burrow speed 60 feet, climb speed 90 feet, fly speed 90 feet (good maneuverability), swim speed 90 feet, all-around vision, blindsense 60 feet, blindsight 30 feet, darkvision 90 feet, low-light vision, scent, see in darkness, tremorsense 60 feet, abduct, animated hair, bleed, blood rage, boot stomp, burn, compression, constrict, crushing leap, DR 5/cold iron, fear aura, frightful presence, grab, heavy weapons, icewalking, kneecapper, luminous, nasal spray, no shadow, oversized weapons, poison, putrid vomit, rend, rock throwing (100 feet, 2d6 damage), sound mimicry, supernatural speed, tear shadow, trackless step, trample, tree meld, undersized weapons, and woodland stride. If the creature has immunity or resistance to any energy types, you gain resistance 20 to those energy types. If the creature has immunity to mind-affecting effects or poison, you gain a +8 resistance bonus on saves against those effects. If the creature has any weaknesses, you gain those weaknesses.
Diminutive Fey: If you assume this form, you gain a +8 size bonus to your Dexterity score and take a -4 penalty to your Strength score.
Huge Fey: If you assume this form, you gain a +6 size bonus to your Strength and Constitution scores and take a -4 penalty to your Dexterity score.}
        
\DeclareSpell{Fey Form IV}{transmutation (polymorph) []|V,  S,  M (a piece of the creature whose form you plan to assume)|1 standard action|personal|Targetsyou|1 minute/level (D)||}[Assume the form of a powerful fey creature.]
    \DeclareSpellDescription{Fey Form IV}{This spell functions as fey form III except it doesn't limit your base speed and also allows you to use more abilities. If the form you assume has any of the following abilities, you gain those abilities: burrow speed 60 feet, climb speed 90 feet, fly speed 120 feet (good maneuverability), swim speed 120 feet, all-around vision, blindsense 60 feet, blindsight 30 feet, darkvision 90 feet, low-light vision, scent, see in darkness, tremorsense 60 feet, abduct, animated hair, beguiling aura, bleed, blood rage, boot stomp, burn, compression, constrict, crushing leap, DR 5/cold iron, fast healing 5, fear aura, frightful presence, grab, heavy weapons, hide in plain sight, icewalking, kneecapper, luminous, nasal spray, no shadow, oversized weapons, poison, putrid vomit, rend, rock throwing (120 feet, 2d10 damage), sound mimicry, supernatural speed, tear shadow, trackless step, trample, transparency, tree meld, undersized weapons, vault, and woodland stride. If the creature has immunity or resistance to any energy types, you gain resistance 30 to those energy types. If the creature has immunity to mindaffecting effects or poison, you gain a +8 resistance bonus on saves against those effects. If the creature has spell resistance, you gain spell resistance 6 + your caster level. If the creature has any weaknesses, you gain those weaknesses.}
        
\DeclareSpell{Flashfire}{evocation () [fire]|V,  S|1 standard action|medium (100 ft. + 10 ft./level)|Areaone 5-ft. square/2 levels|1 round/level|Reflex negates (object) and Fortitude negates|yes (object)}[Cause smoky fires to spring up to burn foes and set them alight.]
    \DeclareSpellDescription{Flashfire}{You cause flames to spring up in the area of effect. These flames deal 1d6 points of fire damage for every 3 caster levels you have (maximum 5d6) to each creature that enters a burning area or begins its turn in the area; these creatures also catch on fire. A creature that succeeds at a Reflex save negates the damage and avoids catching on fire. The area and all adjacent 5-foot squares are smoky, providing concealment within. You can concentrate as a standard action to ignite one 5-foot square adjacent to a currently burning square. Heavy precipitation (including sleet storm) ends the spell. Strong and severe winds spread each square of flashfire downwind by one square each round, but windstorm-force or stronger winds extinguish the fires.}
        
\DeclareSpell{Forest's Sense}{divination (scrying) []|V,  S|1 standard action|1 mile/level|Targetsone creature|1 minute/level (D); see text|Will negates|yes}[Sense the location of a distant target that is near a plant or fungus.]
    \DeclareSpellDescription{Forest's Sense}{You can sense the location of a target creature within range if it is also within 10 feet of a living plant or fungus. You must be able to target the creature by tangible qualities such as its build, clothing texture, size, or tracks, but you need not have line of effect to your target. The fungus or plants near the target serve as a scrying sensor for this spell. Your senses of hearing, smell, and touch extend to all fungus and plants within 10 feet of the target, allowing you to gauge the size and shape of nearby objects and potentially to overhear conversations in which the target is currently participating. Your scent ability and tremorsense extend through this scrying sensor if you have them, but any other special senses you might have do not. For 1 day after casting this spell, you gain a +5 insight bonus on Survival checks to track creatures you sensed via the spell. You can dismiss this bonus on Survival checks.}
        
\DeclareSpell{Greensight}{transmutation () []|V,  S,  M (a leaf)|1 standard action|touch|Targetscreature touched|10 minutes/level|Will negates (harmless)|no}[Grant a target the ability to see through plant matter as if it were transparent.]
    \DeclareSpellDescription{Greensight}{The target of this spell gains the ability to see up to 60 feet through thick plant matter as though it were transparent. Greenery, leaves, and vines-even lichen, moss, and slime-offer no concealment to the recipient's sight, though her vision still can be blocked by solid wood, such as trees or wooden structures. Undergrowth does not grant concealment to a creature against a target of the effects of greensight.}
        
\DeclareSpell{Hidden Spring}{transmutation () [water]|V,  S,  F (a Y-shaped wooden rod)|1 hour|touch|Effecta spring of fresh water|1 hour/level|none|no}[Discover a temporary spring of fresh, flowing water.]
    \DeclareSpellDescription{Hidden Spring}{You spend 1 hour in quiet meditation, holding the focus component in both hands and walking around a natural area. You are simultaneously drawn to areas of natural moisture while drawing that moisture closer. As the spell's casting time concludes, you thrust the focus component into the ground. From that point, a fresh, clean water trickles forth at a rate of 1 gallon every 10 minutes. You cannot cast this spell inside a building or in an area of worked stone, but you can cast it underground. You cannot cast this spell within 1 mile of an existing hidden spring.}
        
\DeclareSpell{Magical Beast Shape}{transmutation (polymorph) []|V,  S,  M (a piece of the creature whose form you plan to assume)|1 standard action|personal|Targetsyou|1 min./level (D)||}[Assume the form of a magical beast.]
    \DeclareSpellDescription{Magical Beast Shape}{This spell functions as beast shape IV except you can assume the form of only a magical beast of a size between Diminutive and Huge. If the form you assume has any of the following abilities, you gain those abilities: burrow speed 60 feet, climb speed 90 feet, fly speed 120 feet (good maneuverability), swim speed 120 feet, blindsense 60 feet, blindsight 30 feet, darkvision 90 feet, low-light vision, scent, see in darkness, tremorsense 60 feet, blood drain, blood frenzy, breath weapon (if damage, up to 12d6 points), constrict, fast healing 5, ferocity, grab, hold breath, jet, no breath, poison, pounce, powerful charge, pull, rake, rend, roar, spikes, trample, trip, and web. If the creature has immunity to poison, you gain a +8 resistance bonus on saves against poison. If the creature has immunity or resistance to any energy types, you gain resistance 20 to those energy types. If the creature has vulnerability to an energy type, you gain that vulnerability.
Diminutive Magical Beast: If you assume this form, you gain a +10 size bonus to your Dexterity score and take a -4 penalty to your Strength score.
Huge Magical Beast: If you assume this form, you gain a +8 size bonus to your Strength score, take a -4 penalty on your Dexterity score, a +2 size bonus to your Constitution score, and gain a +7 natural armor bonus.}
        
\DeclareSpell{Merge With Familiar}{transmutation () []|V,  S|1 standard action|touch|Targetsyour familiar or spirit animal; see text|1 hour/level (D)|Fortitude negates (harmless)|yes (harmless)}[Merge the body of your familiar or spirit animal into your own.]
    \DeclareSpellDescription{Merge With Familiar}{As long as the target is at least one size category smaller than you are, it can merge harmlessly into your body while under the effect of this spell. For the duration of this spell, you and the target can separate or merge at will as a move action. While merged, your familiar or spirit animal can't be targeted or affected by most attacks and effects, though it still suffers from ongoing effects and their durations continue. It can be the recipient of effects that originate from you.}
        
\DeclareSpell{Mirage}{illusion (figment) []|V,  S,  M (a pinch of sand and a drop of water)|10 minutes|long (400 ft. + 40 ft./level)|Areaone 40-ft. cube/level (S)|2 hours/level (D)|Will disbelief (see below)|yes}[Create illusory terrain.]
    \DeclareSpellDescription{Mirage}{You create an image of a pool of water, a group of standing stones, a cove, an island, or another simple land formation over a stretch of flat land or water. The image is purely visual, and structures, equipment, and creatures within the area are not hidden or changed in appearance.
Anyone interacting with the illusion can attempt to disbelieve it, and a creature trained in Survival can make a special attempt to disbelieve the illusion. The creature must be within 120 feet of the illusion's area and can attempt a Survival check instead of a Will saving throw. If the result is equal to or greater than the spell's save DC, the creature realizes the mirage is an illusion as if it disbelieved the spell.}
        
\DeclareSpell{Ooze Form I}{transmutation (polymorph) []|V,  S,  M (a bit of the creature whose form you plan to assume)|1 standard action|personal|Targetsyou|1 minute/level (D)||}[Assume the form of a Small or Medium ooze creature.]
    \DeclareSpellDescription{Ooze Form I}{You assume the form of a Small or Medium ooze. Regardless of the type of ooze you transform into, you gain base speed 10 feet, climb speed 10 feet, swim speed 20 feet, and blindsense 30 feet. You gain a +4 resistance bonus on saving throws against mindaffecting effects and poison. A wood or metal weapon that strikes you takes acid damage as if from your slam unless the wielder succeeds at a Reflex saving throw.
Small Ooze: If you assume this form, you gain a +4 size bonus to your Constitution score, a slam attack (1d3 plus 1d3 acid), and constrict (1d3), and take , a -4 penalty to your Dexterity score.
Medium Ooze: If you assume this form, you gain a +6 size bonus to your Constitution score, a -6 penalty to your Dexterity score, a slam attack (1d4 plus 1d4 acid), and constrict (1d4).}
        
\DeclareSpell{Ooze Form II}{transmutation (polymorph) []|V,  S,  M (a bit of the creature whose form you plan to assume)|1 standard action|personal|Targetsyou|1 minute/level (D)||}[Assume the form of a Large ooze creature.]
    \DeclareSpellDescription{Ooze Form II}{This spell functions as ooze form I except you can also assume the form of a Large ooze. You gain blindsight 30 feet and immunity to critical hits and precision damage as well.
Large Ooze: If you assume this form, you gain a +2 size bonus to your Strength score, a +8 size bonus to your Constitution score, a slam attack (2d4 plus 1d6 acid and grab), and constrict (2d4), and you take a -8 penalty to your Dexterity score.}
        
\DeclareSpell{Ooze Form III}{transmutation (polymorph) []|V,  S,  M (a bit of the creature whose form you plan to assume)|1 standard action|personal|Targetsyou|1 minute/level (D)||}[Assume the form of a Huge ooze creature.]
    \DeclareSpellDescription{Ooze Form III}{This spell functions as ooze form II except you can also assume the form of a Huge ooze. You gain blindsight 60 feet and a +8 resistance bonus on saving throws against mind-affecting effects and poison as well.
Huge Ooze: If you assume this form, you gain a +4 size bonus to your Strength, a +10 size bonus to your Constitution score, a -10 penalty to your Dexterity score, base speed 20 feet, climb speed 20 feet, swim speed 30 feet, a slam attack (2d6 plus 2d6 acid and grab), and constrict (2d6).}
        
\DeclareSpell{Pouncing Fury}{transmutation () []|V,  S|1 standard action|personal|Targetsyou|1 round/level||}[Pouncing Fury]
    \DeclareSpellDescription{Pouncing Fury}{When you charge, you can make a full attack at the end of that charge, but only with claw attacks you have, and you can make only one attack per claw. If you have abilities that grant bonuses on damage rolls or that apply other special effects to charge attacks, only the first claw attack benefits from these bonuses. When you make a claw attack as an attack of opportunity, you can expend one additional use of your attacks of opportunity to make an additional claw attack against the target that provoked the attack.}
        
\DeclareSpell{Replay Tracks}{divination () []|V,  S,  F (a track or other sign of a creature's passing found with Perception or Survival)|3 rounds|personal|Targetsyou|concentration, up to 1 hour/level||}[Reconstruct past events from a set of tracks.]
    \DeclareSpellDescription{Replay Tracks}{This spell allows you to reconstruct past events that occurred in your current location based on the tracks and other signs left behind. Replay tracks reveals events that occurred while the tracks that serve as the focus for the spell were being left, revealing events in the order they happened in real time or in reverse, depending on whether you are following the tracks forward or backward, although the image isn't clear enough to make out details (such as a creature's exact identity). You can attempt Survival checks to follow tracks as part of concentrating on the spell, but only events connected to the tracks used as the focus of the spell are revealed by the spell. Elements that left no trace detectable by you, such as creatures benefitting from pass without trace, are absent from the events you visualize.}
        
\DeclareSpell{Ropeweave}{transmutation () []|V,  S,  F (a rope)|1 minute|touch|Targetsone rope|1 hour/level (D)|none|no}[Create useful tools from a coil of rope.]
    \DeclareSpellDescription{Ropeweave}{You cause the target rope to grow in length and weave itself into one of several forms, each of which can support up to 1,000 pounds, plus 200 pounds per caster level (maximum 3,000 pounds at 10th level). Only one type of construction can be created with each casting of the spell, and the creation remains stationary unless destroyed.
Each 5-foot section of the object created by this spell has a break DC of 23, AC 11, and 1 hit point per caster level (maximum 10 hp), but all sections of the creation are magically supported and need not be anchored to a solid surface or any other portion of the effect. Destroying one part of it does not cause the remainder of the structure to collapse, though each 5-foot section destroyed reduces the maximum weight the creation can support by 200 pounds. During any round in which the rope is overloaded, every remaining section takes 1d4 points of damage.
You can use ropeweave to create any of the following structures.
Rope Bridge: The rope forms a 5-foot-wide bridge that spans up to 10 feet horizontally per caster level you have (maximum 100 feet). Creatures can cross the bridge at half speed with a successful DC 5 Acrobatics check or at full speed with a successful DC 10 Acrobatics check. The DC assumes a creature is using both hands to assist in navigating the rope bridge; the DC increases by 5 if a creature uses only one hand to steady itself and by 10 if the creature does not use its hands. A failed Acrobatics check results in failure to progress across the rope bridge; creatures that fail by 5 or more fall.
Rope Hammock: The rope knits itself into a stationary hammock suspended in midair. The hammock can be suspended at a height of 5 feet plus 5 feet per 2 caster levels you have (maximum 30 feet), with a rope ladder (see below) leading up to a platform of 1 5-foot square per caster level, and at least one such square must be adjacent to the square containing the vertical rope ladder. As a move action, the caster can command the rope ladder to withdraw into the hammock. When the ladder is withdrawn, the rope hammock blends in with its surroundings and muffles sounds and smells from creatures resting on it that are taking no violent actions; noticing the rope hammock requires a successful DC 20 Perception check or Survival check, even for creatures with scent. This does not apply if the creatures in the hammock attack or move more than half speed. The hammock provides a +2 cover bonus to AC against attacks from beneath it.
Rope Ladder: The rope knots itself and hangs suspended in midair, perpendicular to the ground, stretching up to 10 feet vertically per caster level you have (maximum 100 feet). Climbing the rope ladder requires a successful DC 5 Climb check, or DC 0 if there is a wall adjacent to it you can brace it against.
Trip Line: The rope stretches into a tangled mass that fills one 5-foot square per caster level you have (maximum 10 squares), all of which must be contiguous (including diagonally). Any creature entering a square with this trip line must succeed at an Acrobatics check (DC = 10 if moving at half speed, DC = 15 if moving at full speed, and DC = 20 if running or charging). On a failed check, the creature's movement stops when it enters the square; a creature that fails by 5 or more falls prone. A creature larger than Medium gains a +2 bonus on its Acrobatics check for every size category larger than Medium it is.}
        
\DeclareSpell{Scamper}{transmutation () []|V,  S|1 swift action|close (25 ft. + 5 ft./2 levels)|Targetsyour animal companion|1 round; see text|none|no}[Grant your animal companion astonishing agility.]
    \DeclareSpellDescription{Scamper}{Your animal companion moves with astonishing agility and speed until the end of its turn. It can move at full speed while using Acrobatics, and it gains a competence bonus equal to twice your caster level (maximum +20 at 10th level) on Acrobatics checks to avoid attacks of opportunity or move through a square occupied by an enemy.}
        
\DeclareSpell{Sea Of Dust}{transmutation () [water]|V,  S,  DF|1 hour|0 ft.|Area2-mile-radius emanation|permanent|none|no}[Permanently drive water out of a region to create a desert.]
    \DeclareSpellDescription{Sea Of Dust}{You cause all areas of water to which you have line of effect to recede as if you had cast control water. Exposed water that enters the area ebbs away, evaporating or leaching into the ground at a rate of 1 foot of depth per hour. Living creatures in the area when the spell is cast are parched with thirst and take a -4 penalty on Constitution checks to resist the effects of a hot climate, and their daily water requirement to avoid thirst doubles (Pathfinder RPG Core Rulebook 444). Plant creatures and inanimate plants take 1d6 points of nonlethal damage per hour after the first 24 hours, which bypasses hardness and damage reduction. Living creatures with the aquatic or water subtype, unless completely immersed in water, must attempt a DC 20 Constitution check each hour no matter how much water they drink. A creature that fails takes 1d6 points of nonlethal damage and becomes fatigued until it recovers from the nonlethal damage.
After 1 week, the soil in the area of sea of dust begins to break down and blow away. Moderate winds have a 50\% chance each hour to cause sandstorms (Core Rulebook 431). This chance increases to 75\% in strong winds and 100\% in severe or stronger winds. Short-duration wind effects such as gust of wind create sandstorms with the same area and duration as the spell, plus an identical duration after the spell ends.
After 1 month, the soil in an area affected by sea of dust has virtually disintegrated. The area is treated as a shallow bog for the purposes of movement (Core Rulebook 427), with a 25\% chance each hour to encounter an area equivalent to a deep bog 2d6x10 feet across, and a 5\% chance to encounter a collapsing dust drift equivalent to quicksand (Core Rulebook 427) 1d6x5 feet across. The affected region remains desertlike in condition until the magic is dispelled, at which point the region recovers and returns to its original terrain over the course of time (this may take only days or weeks, or it could take months or even years, subject to GM discretion). This spell has no effect if cast on an entirely aquatic region. If the spell is cast on an island, the effects of the spell extend to the shore but not beyond.}
        
\DeclareSpell{Sea Steed}{transmutation (polymorph) []|V,  S,  DF|1 standard action|touch|Targetsa creature you are mounted upon|10 minutes/level (D)|Will negates (harmless)|yes (harmless)}[Your mount adapts to an aquatic environment]
    \DeclareSpellDescription{Sea Steed}{The target adapts to life in the water, gaining piscine scales that cover its body and growing fins in place of feet. It gains the aquatic subtype, the amphibious quality, and a swim speed equal to its base speed before it came under the effect of this spell. While under the effects of this spell, the target's base speed is reduced by 10 feet (minimum 10 feet).}
        
\DeclareSpell{Sea Stallion}{transmutation (polymorph) []|V,  S,  DF|1 standard action|touch|Targetsa creature you are mounted upon|10 minutes/level (D)|Will negates (harmless)|yes (harmless)}[You and your mount adapt to an aquatic environment.]
    \DeclareSpellDescription{Sea Stallion}{This spell functions as sea steed, but as long as you stay mounted on the target, you also gain the amphibious quality and your melee attacks function as if you were under the effect of freedom of movement. The benefits you gain from this spell are not polymorph effects. If you dismount from the target, you retain the amphibious quality for up to 1 minute, but not the other benefits. You regain all of the benefits once you mount the target creature again.}
        
\DeclareSpell{Shapechanger's Gift}{transmutation (polymorph) []|V,  S,  M (a piece of the creature whose form you plan to grant)|1 standard action|touch|Targetswilling living creature touched|10 minutes/level|Fortitude negates (harmless)|yes (harmless)}[Temporarily change a target into an animal, humanoid, or monstrous humanoid.]
    \DeclareSpellDescription{Shapechanger's Gift}{You change the target into an animal (as per beast shape I), humanoid (as per alter self), or monstrous humanoid (as per monstrous physique IUM). The chosen form can't have a fly speed. The subject's statistics change as per the appropriate spell except the creature's ability scores and natural armor do not change. The creature can change between its natural form and the chosen form at will as a standard action.}
        
\DeclareSpell{Shapechanger's Gift, Greater}{transmutation (polymorph) []|V,  S,  M (a piece of the creature whose form you plan to grant)|1 standard action|touch|Targetswilling living creature touched|1 hour/level|Fortitude negates (harmless)|yes (harmless)}[Temporarily change a target into an animal, elemental, fey, humanoid, monstrous humanoid, or vermin.]
    \DeclareSpellDescription{Shapechanger's Gift, Greater}{This spell functions as shapechanger's gift except the chosen form can be an animal (as per beast shape I), elemental (as per elemental body I), fey (as per fey form I*), humanoid (as per alter self), monstrous humanoid (as per monstrous physique IUM), or vermin (as per vermin shape IUM) and the chosen form can have a fly speed.}
        
\DeclareSpell{Signs Of The Land}{divination () []|V,  S|10 minutes|personal|Targetsyou|instantaneous||}[ Learn up to three details about the surrounding territory.]
    \DeclareSpellDescription{Signs Of The Land}{This spell functions as commune with nature except it reveals up to three details about the territory you currently occupy from the following list: bodies of water, features, minerals, or plants. For each feature you learn about, you automatically know the skills you can use to discover that feature and you gain 1d6 Discovery Points toward finding it (see page 124).}
        
\DeclareSpell{Sky Steed}{transmutation (polymorph) []|V,  S,  DF|1 standard action|touch|Targetsa creature you are mounted upon|1 minute/level (D)|Will negates (harmless)|yes (harmless)}[Grant your mount angelic wings.]
    \DeclareSpellDescription{Sky Steed}{The target sprouts angelic wings and gains a fly speed equal to its base speed with average maneuverability. It also gains a bonus on Fly checks equal to your caster level.}
        
\DeclareSpell{Snowball}{evocation () [cold,  water]|V,  S|1 standard action|close (25 ft. + 5 ft./2 levels)|Effectone ball of ice and snow|instantaneous|none|yes}[Throw a conjured ball of snow at a target.]
    \DeclareSpellDescription{Snowball}{You throw a ball of elemental ice and snow at a single target as a ranged touch attack. The snowball deals 1d6 points of cold damage per caster level you have (maximum 5d6).}
        
\DeclareSpell{Soothing Mud}{conjuration (healing) [earth,  water]|V,  S,  DF|1 standard action|medium (100 ft. + 10 ft./level)|Areadust, earth, sand, or water in one 5-ft. square/level|1 round/level (D)|none|no}[Create restorative mud that heals hit point and ability damage.]
    \DeclareSpellDescription{Soothing Mud}{You create an area of healing mud. Water, earth, sand, and dust thicken into a wet mud. The mud functions as difficult terrain and does not sink if created in water unless weighed down by more than 100 pounds per caster level you have. Each round a creature begins its turn in or on the mud, the mud restores 1 hit point to it; this healing is unaffected by effects that increase a creature's healing. A creature that rests partially or completely submerged in this mud for 1 full minute is also healed of 1d4 points of ability damage to an ability score of its choice. A creature can be healed of ability damage this way only once per day.}
        
\DeclareSpell{Sturdy Tree Fort}{transmutation () []|V,  S,  F (a nail,  a rope,  and a short wooden plank)|1 minute|touch|Effectone large tree and a sturdy wooden house|1 hour/level (D)|none|no}[Create a tree with a defensive fort within it.]
    \DeclareSpellDescription{Sturdy Tree Fort}{You cause a large tree to grow in the 5-foot square you touch. The tree has 1 foot of thickness for every 4 caster levels you have, and it can grow to any height you designate, up to 5 feet high per caster level you have. If there is a ceiling or other barrier overhead, the maximum height of the tree cannot exceed the space available. At any point along the tree's height, you can create a sturdy wooden building that fills one 10-foot cube for every 4 caster levels you have. All portions of the structure must be adjacent to the trunk of the tree (or can have the tree trunk extend up through their spaces), but otherwise these cubes need not be contiguous. If you place the cubes so that they are contiguous, the buildings merge together, complete with ladders connecting different vertical levels. The fort is magically supported, but if the tree it is attached to is destroyed, the structure is destroyed as well (the tree has hardness 5 and 20 hp per caster level you have).
Each building you create has shuttered arrow slits on its walls and in the floor, granting those within the structure improved cover against attacks from outside while the arrow slits are open and total cover when they are closed. Each cube within the fort is equivalent to a secure shelter, other than the size and its wooded construction material (hardness 5). When you cast sturdy tree fort, you can designate a number of creatures equal to your caster level. Designated creatures gain a +10 bonus on Climb checks and are not denied their Dexterity bonuses while climbing on the fort's structure, as they find handholds and stable footholds easily on the tree and its buildings.
You cannot cast this spell in an area of worked stone, though you can cast it in natural surroundings that would not normally support the growth of a large tree, such as a cavern, desert, or glacier. If you cast this spell in a forest, jungle, or similar heavily treed terrain, the fort is camouflaged as long as its doors and windows remain closed (or even if they are open, as long as those within remain quiet and take no violent actions), requiring a successful DC 25 Perception check or Survival check to notice its presence.}
        
\DeclareSpell{Tailwind}{transmutation () [air]|V,  S|1 standard action|120 ft.|Targetsone or more creatures, no two of which can be more than 120 feet apart|1 hour/level|Will negates (harmless)|yes (harmless)}[Create a current of wind to enhance or impede flight.]
    \DeclareSpellDescription{Tailwind}{You compel the wind to push the targets in a direction of your choice. The subjects can hustle during local and overland movement without risk of fatigue. If the subjects are flying, the first 20 feet of movement each round in the chosen direction does not count against their movement for the round. Each creature remains affected only while within 120 feet of every other subject. You can change the tailwind's direction once per hour by concentrating as a standard action. This spell has no effect underwater.}
        
\DeclareSpell{Tail Current}{transmutation () [water]|V,  S|1 standard action|120 ft.|Targetsone or more creatures, no two of which can be more than 120 feet apart|1 hour/level|Will negates (harmless)|yes (harmless)}[]
    \DeclareSpellDescription{Tail Current}{This spell functions as tailwind except it creates a helpful current in water instead of in the air. It grants free movement while swimming rather than while flying and functions only underwater.}
        
\DeclareSpell{Tamer's Lash}{evocation () [sonic]|V,  S|1 standard action|0 ft.|Effecta whip of magical sound|1 round/level (D)|Will partial (see text)|yes}[Create a whip made of sound that damages foes and can frighten animals.]
    \DeclareSpellDescription{Tamer's Lash}{A 15-foot-long whip of sonic energy springs from your hand, delivering a loud crack when you strike with it. The tamer's lash acts as a whip that deals 1d4 points of sonic damage on a hit, but it has no physical substance and you cannot use it to perform combat maneuvers, nor can it be sundered or disarmed. If you strike an animal with the tamer's lash, the animal must succeed at a Will saving throw or be unable to attack you for 1d3 rounds. Furthermore, if the whipped animal starts its turn unable to attack you as a result of this spell and it is within 30 feet of you, it must spend its first action moving away from you, if it is able. Combattrained animals, animal companions, and animals with Hit Dice in excess of your caster level + 4 gain a +4 circumstance bonus on their saves to avoid being intimidated in this way. This intimidation effect is a mind-affecting fear effect.}
        
\DeclareSpell{Tidal Surge}{conjuration (creation) [water]|V,  S|1 standard action|30 ft. or 60 ft.; see text|Areacone-shaped burst or line; see text|instantaneous|Reflex half|no}[Create a surge of water to bludgeon foes and extinguish fires.]
    \DeclareSpellDescription{Tidal Surge}{If you cast tidal surge on land, you create an onrushing surge of water 10 feet high in a 30-foot cone that deals 1d6 points of bludgeoning damage for every 2 caster levels you have (maximum 10d6 at 20th level) and extinguishes all nonmagical fires in the area. Magical fire effects in the area of a tidal surge are affected as if you had cast dispel magic. In addition to taking damage, creatures that fail their Reflex saves are pushed 1d4x5 feet away from you, and Medium or smaller creatures are also knocked prone.
If you cast this spell in or on a body of water at least 30 feet across, you can shape the spell either as the cone described above or as a 60-foot line. In either shape, the water deals 1d10 points of bludgeoning damage for every 2 caster levels you have in addition to pushing away creatures that fail their Reflex saves.}
        
\DeclareSpell{Vigilant Rest}{abjuration () []|V,  S,  M (a handful of crushed glass)|1 standard action|touch|Targets1 living creature touched|8 hours (D)|Will (harmless)|yes (harmless)}[Cause a sleeping creature to retain some perception of its surroundings.]
    \DeclareSpellDescription{Vigilant Rest}{While the target of this spell is asleep, she retains some of her normal powers of perception. While she tunes out harmless sounds such as the crackling of the campfire, crickets, or light breezes, she does not take the normal +10 modifier to the DC of Perception checks she attempts while asleep.
If the target detects a threat or unexplained noise while sleeping, she can wake immediately and stand up from her sleeping position as a free action.
If the target is roused but returns to sleep during the spell's duration, she falls asleep immediately and does not count as having her sleep interrupted, even if she engaged in combat while she was awake.}
        
\DeclareSpell{Vine Strike}{conjuration (creation) []|V,  S|1 standard action|personal|Targetsyou|1 minute/level (D)|Reflex negates (see text)|yes}[ Enhance one of your natural or unarmed attacks with thorny vine growth.]
    \DeclareSpellDescription{Vine Strike}{Bristles burst from your body, lodging in your opponent and blossoming into entangling vines as you pummel your target. While this spell is in effect, one of your natural attacks or unarmed strikes deals an additional 1d6 points of damage, and each creature hit with that natural weapon or unarmed strike must succeed at a Reflex save or be entangled for the duration of the spell; on a successful Reflex save, the creature is immune to the entangled effect for 1 round. A creature entangled by this spell can spend a standard action to remove the vines, but can be entangled again by further unarmed strikes.}
        
\DeclareSpell{Wandering Weather}{transmutation () []|V,  S|10 minutes; see text|2 miles|Area2-mile-radius circle, centered on you; see text|4d12 hours; see text|none|no}[Control weather in a large area that moves with you.]
    \DeclareSpellDescription{Wandering Weather}{This spell functions as control weather except you can opt for the weather to remain centered on you as you move.}
        
\DeclareSpell{Winter Grasp}{conjuration (creation) [cold,  water]|V,  S,  M/DF (ground glass)|1 standard action|medium (100 ft. + 10 ft./level)|Area20-ft.-radius spread|1 round/level|none|no}[Create a slippery sheet of ice on the ground.]
    \DeclareSpellDescription{Winter Grasp}{Ice encrusts the ground, radiating supernatural cold and making it hard for creatures to maintain their balance. This icy ground is treated as normal ice, forcing creatures to spend 2 squares of movement to enter an icy square and increasing the DC of Acrobatics checks in the area by 5. A creature that begins its turn in the affected area takes 1d6 points of cold damage and takes a -2 penalty on saving throws against spells with the cold descriptor for 1 round.}
        
\DeclareSpell{With The Wind}{abjuration () [air]|V,  S|1 standard action|close (25 ft. + 5 ft./2 levels)|Targetsone creature|1 minute/level|Will negates (harmless)|yes (harmless)}[Protect a target from being blown away by wind of less than windstorm forc]
    \DeclareSpellDescription{With The Wind}{You create eddies in the air around the subject that protect it from being checked or blown away. The target can't be moved by winds of less than windstorm force unless it wishes to be.}
    
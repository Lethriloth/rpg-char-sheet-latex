    
\DeclareSpell{Ancestral Memory}{divination () []|V,  S|1 standard action|personal|Targetsyou|1 round/level||}[]
    \DeclareSpellDescription{Ancestral Memory}{When you cast this spell, you open your mind to the vast experiences of your ancestors in the hope of learning something pertinent about your current situation. The chance of successfully finding an ancestral memory that is pertinent is equal to 70\% + your caster level. Failure indicates you merely gain a +5 insight bonus on all Intelligence-based skill checks for the duration of the spell.
Success indicates that you not only gain the +5 insight bonus on all Intelligence-based skill checks, but that one of your ancestors came across a situation or problem similar to one you are currently facing. In this case, the GM provides you with some specific information to assist you in overcoming your problem.
For example, a character might encounter a clay golem deep underground, and finds that her magic weapon and spells seem to be useless against the creature. She successfully casts ancestral memory, and "remembers" the proper type of weapons and spells that work against such creatures.}
        
\DeclareSpell{Gorum's Armor}{transmutation () []|V,  S,  M (1 iron spike)|1 standard action|touch|Targets1 suit of metal armor or 1 metal shield|10 minutes/level|Fortitude negates (harmless)|yes (harmless)}[]
    \DeclareSpellDescription{Gorum's Armor}{The targeted suit of armor or shield sprouts thousands of tiny iron spikes like porcupine quills. These do not harm the armor's wearer (though donning or removing armor under the effects of this spell takes twice as long), but they act as armor spikes or shield spikes (as appropriate). Any creature attacking the wearer with natural weapons takes 1 point of piercing damage for each attack that hits. At 5th level, the spikes gain a +1 enhancement bonus on attack and damage rolls; this bonus increases to +2 at 10th level. At 15th level, the spikes also gain the anarchic weapon quality.}
        
\DeclareSpell{Harrowing}{divination () []|V,  S,  F (a Harrow deck)|10 minutes|touch|Targetsone creature|1 day/level or until fulfilled||}[]
    \DeclareSpellDescription{Harrowing}{You use a Harrow deck to tell a fortune for yourself or someone else. If you cast harrowing on another creature, you must remain adjacent to the target for the duration of the casting time. A harrowing must describe one set of events or course of action (for example, "hunting down the pirate king," or "traveling to Viperwall to search for a magic sword") that the target of the spell intends to undertake at some point during the spell's duration.
If you have access to a Harrow deck, draw nine cards when this spell is cast. If you do not have a Harrow deck, you can simulate the draws by rolling a d6 and a d10 for each of the nine cards, as detailed on page 293 of this book. Record the ability score and alignment associated with each card. Each of these cards grants a luck bonus or a penalty on a specific type of d20 check; the magnitude of the penalty or bonus depends upon how closely that particular card's alignment matches the target creature's alignment. If the card and target's alignments are identical, that card provides a +2 luck bonus on the associated suit's check. If the card and target's alignments are of the opposite alignment (see below), the card inflicts a -1 penalty on that associated check. If the card has any other alignment, it provides a +1 luck bonus on the associated suit's check.
While penalties persist on all associated checks for as long as the harrowing persists, the bonuses are one-use bonuses that the harrowed character can "spend" at any time to modify that card's associated check. You can spend a bonus to modify an appropriate roll after the die is rolled, but cannot spend the bonus once you know the result of the roll. Since all of the bonuses granted by a harrowing are luck bonuses, they do not stack with each other. Penalties, on the other hand, do stack.
Once you spend all of the bonuses granted by a harrowing, or once the spell's duration ends, the spell ends and the penalties are removed.
A single creature can only be under the effects of one harrowing at a time. If it is subjected to a second harrowing while a previous harrowing is still in effect, the new harrowing automatically fails.}
        
\DeclareSpell{Infernal Healing}{conjuration (healing) [evil]|V,  S,  M (1 drop of devil blood or 1 dose of unholy water)|1 round|touch|Targetscreature touched|1 minute|Will negates (harmless)|yes (harmless)}[]
    \DeclareSpellDescription{Infernal Healing}{You anoint a wounded creature with devil's blood or unholy water, giving it fast healing 1. This ability cannot repair damage caused by silver weapons, good-aligned weapons, or spells or effects with the good descriptor. The target detects as an evil creature for the duration of the spell and can sense the evil of the magic, though this has no long-term effect on the target's alignment.}
        
\DeclareSpell{Infernal Healing, Greater}{conjuration (healing) [evil]|V,  S,  M (1 drop of devil blood or 1 dose of unholy water)|1 round|touch|Targetscreature touched|1 minute|Will negates (harmless)|yes (harmless)}[]
    \DeclareSpellDescription{Infernal Healing, Greater}{As infernal healing, except the target gains fast healing 4 and the target detects as an evil cleric.}
        
\DeclareSpell{Lover's Vengeance}{enchantment (compulsion) [mind-affecting]|V,  M (a piece of jewelry worth at least 100 gp)|1 minute|touch|Targetscreature touched|up to 1 day/level (D) or until discharged|Will negates (harmless)|Yes (harmless)}[]
    \DeclareSpellDescription{Lover's Vengeance}{You inspire yourself or a lover to a vengeful rage against a chosen enemy, who must be a creature that has wronged you in some way. If cast on you, the next time you are in combat with that enemy, you gain the benefits of a rage spell. If cast on a lover, he or she gains the benefits of a rage spell the next time the lover is in combat against your enemy. This variant of the spell must be cast within 1 hour of an intimate encounter with the target. The rage effect lasts 1 round per level. If the creature that triggers the rage effect is one of your lovers or ex-lovers, the benefits granted by the rage spell double. This spell counts as a contingency spell on the target for the purpose of multiple contingent effects. Worshipers of Calistria are fond of using this spell, and many keep the effect running whenever possible.}
        
\DeclareSpell{Shield Of The Dawnflower}{evocation () [fire,  good,  light]|V,  S,  DF|1 standard action|personal|Targetsyou|1 round/level|see text|no}[]
    \DeclareSpellDescription{Shield Of The Dawnflower}{You create a disk of sunlight on one arm. Any creature that strikes you with a melee attack deals normal damage, but also takes 1d6 points of fire damage + 1 point per caster level (maximum +15). Creatures with reach weapons are not subject to this damage if they attack you. The shield provides illumination as if it were a continual flame spell. You can only have one instance of this spell in effect at a time. It does not stack with similar damaging aura spells such as fire shield.}
        
\DeclareSpell{Teleport Trap}{abjuration () []|V,  S,  M (powdered lodestone and silver worth 100 gp per 40-ft. cube)|10 minutes|medium (100 ft. + 10 ft./level)|Areaone 40-ft. cube/level (S)|1 day/level|Will negates|yes}[]
    \DeclareSpellDescription{Teleport Trap}{Teleport trap wards an area, redirecting all teleportation into or out of the area to a specific point within the area determined by you at the time of casting. The destination must be an open space on a solid surface. The spell's area overlaps walls and other solid and liquid objects (preventing intruders from bypassing the ward by teleporting into a wall or through similar means). A teleporting creature that is affected by a teleport trap can resist the effect with a Will save-if the save is successful, the creature simply doesn't teleport at all (but the use of the teleport effect is still consumed)-either to the intended location or the teleport trap's actual destination. A DC 27 Knowledge (arcana) allows such a creature to recognize the teleport trap's presence, but does not reveal the trap's linked destination.
At your discretion, the teleport trap can exclude a category of creatures, such as an alignment, a type of creature, or creatures that carry a specific item or know a password (though this only works if the creature is teleporting out of the area, not into it). You select this option and the conditions at the time you cast the spell. Overly complicated conditions may cause the spell to fail entirely. Multiple castings of teleport trap can be linked to cover a larger area, allowing teleported creatures to be directed to a single point within the combined area of the spells.
Teleport trap can be made permanent at the cost of 7,000 gp. A single permanency spell can be used on all teleport traps that share a linked destination, but the gold piece cost must be paid for each individual spell.
The Pathfinders of the Grand Lodge make use of permanent teleport traps in several key locations, trapping would-be intruders in a small wing of jail cells. At least one crypt of the Whispering Tyrant makes use of the spell as well, trapping grave robbers in coffin-sized stone cysts, there to die a slow and agonizing death from thirst and starvation.}
        
\DeclareSpell{Unbreakable Heart}{enchantment (compulsion) [mind-affecting]|V,  S|1 standard action|close (25 ft. + 5 ft./2 levels)|Effect1 creature|1 round/level|Will negates (harmless)|yes (harmless)}[]
    \DeclareSpellDescription{Unbreakable Heart}{The target creature gains a +4 morale bonus on saving throws against mind-affecting effects that rely on negative emotions (such as crushing despair, rage, or fear effects) or that would force him to harm an ally (such as confusion). If the target is already under such an effect when receiving this spell, that effect is suppressed for the duration of this spell. It does not affect mind-affecting effects based on positive emotions (such as good hope or the inspire courage bard ability). A creature can still be charmed or otherwise magically controlled while under this spell's effects, but if such a creature ever receives a new saving throw against that effect as a result of being ordered to attempt to harm or otherwise oppose a true ally, he can roll that saving throw twice and take the better result as his actual roll. Calm emotions counters and dispels unbreakable heart.}
        
\DeclareSpell{Vision of Lamashtu}{illusion (phantasm) [mind-affecting,  evil]|V,  S|10 minutes (see text)|unlimited|Targetsone living creature|instantaneous|Will negates (see text)|yes}[]
    \DeclareSpellDescription{Vision of Lamashtu}{This spell functions exactly as the spell nightmare. In addition to the effects of that spell, you can cause a second spell to be delivered when the target wakes at the nightmare's conclusion.
You must have this second spell prepared, and it must be cast immediately after vision of Lamashtu (effectively adding the two spells' casting times). This second spell "rides along" with the nightmare, affecting the target as soon as it wakes from its fitful sleep. Any spell can be sent along with the nightmare, so long as it is of 6th level or lower, affects one target (which is always the nightmare's recipient), and does not deal hit point damage. The second spell's range is irrelevant for the purposes of vision of Lamashtu, and even touch attacks can be delivered in this manner (you must still make a successful touch attack in order to affect the target, though, with the act of touching occurring within the context of the victim's nightmare). The target is allowed to save against the second spell if a save is allowed. For example, a cleric of Lamashtu could send bestow curse along as part of a vision of Lamashtu, but not blade barrier (affects an area), destruction (too high level), or inflict moderate wounds (deals hit point damage).}
    
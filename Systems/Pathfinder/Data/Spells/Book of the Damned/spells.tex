    
\DeclareSpell{Awaken The Devoured}{divination () [pain]|V,  S|1 standard action|medium (100 ft. + 10 ft./level)|Targetsone daemon per 4 caster levels|instantaneous and 1 round/level|Will partial (see text)|yes}[]
    \DeclareSpellDescription{Awaken The Devoured}{This spell is often used by conjurers as a method to torment daemons and force compliance, for it awakens the broken, anguished memories of the countless souls that the target daemons have consumed. All daemons targeted by awaken the devoured must be within 30 feet of each other. The fragmented memories haunt and afflict the daemons, dealing 1d8 points of damage per caster level (maximum 15d8) and making the daemons confused for 1 round per level. A daemon that succeeds at a Will save halves the damage and negates the confusion effect.}
        
\DeclareSpell{Charon's Dispensation}{abjuration () []|V,  S,  M (2 silver coins)|1 standard action|close (25 ft. + 5 ft./2 levels)|Targetsone creature/level|1 minute/level|Will negates (harmless)|yes (harmless)}[]
    \DeclareSpellDescription{Charon's Dispensation}{The target of this spell gains immunity to the harmful effects of touching or drinking from the River Styx and a +4 profane bonus on saves against effects that alter or suppress memory (such as modify memory or mindwipeOA). This does not grant the target the ability to breathe water, nor does it grant any protection against creatures or mundane hazards such as rapids. The spell has no power to restore memory to a creature already suffering from an existing fugue or amnesia state.}
        
\DeclareSpell{Create Drug}{conjuration (creation) []|V,  S|1 round|0 ft.|Effect1 dose of a drug/3 levels|1 minute|none (see below)|no}[]
    \DeclareSpellDescription{Create Drug}{The caster conjures into being one of the following drugs: aether, flayleaf, opium, pesh, scour, shiver, or zerk. (At the GM's discretion, other drugs of similar power can be included on this list.) The drug doses the caster creates with this spell must be used within 1 minute of being conjured, or they dissolve into worthless dust or evaporate into noxious but fleeting vapors, though the effects of the drug may last far longer if taken before it decays.  The drugs created by this spell cannot be sold, but they can be given to other creatures. A creature that takes a dose of one of these drug typically must be either willing or helpless, though some drugs might be inhaled, applied to injuries, or secretly slipped into food if the caster acts swiftly (see each drug's description). The DC to resist a drug created by this spell is based on the conjurer's caster level, not the DC listed in the common versions of the drug. For more information, see Drugs and Addiction on page 236 of the Pathfinder RPG GameMastery Guide.  As a special use of this spell, a lawful evil worshiper of Mahathallah, the Dowager of Illusions, can create doses of the drug adyton (see the sidebar above). A Mahathallah worshiper can create adyton only once per week, regardless of her level or how many times she casts this spell. The spell otherwise functions- and creates as many doses-as normal.}
        
\DeclareSpell{Create Soul Gem}{necromancy () [death,  evil]|V,  S,  F (crystal lens worth 500 gp)|1 round|close (25 ft. + 5 ft./2 levels)|Targetsone dying or recently dead creature|1 day/level|Will negates|yes}[]
    \DeclareSpellDescription{Create Soul Gem}{You draw forth the ebbing life force of a dying creature or one that has died in the past round, focusing it through a crystal lens focus to transform into a soul gem. If the creature is alive and fails its saving throw, it dies and you capture its soul in the gem. If the creature is dead, it can still resist the spell effect by attempting a Will save as if it were still alive. The value of the soul gem created depends on the nature of the creature it is made from (see page 191). Soul gems created by this spell crumble to dust once the spell's duration expires, releasing the trapped soul to travel on to judgment in the Great Beyond.  Only one soul gem can be created from a dying creature. Any attempt to resurrect a body whose soul is trapped in a soul gem requires a caster check against create soul gem's save DC. Failure results in the resurrection spell having no effect, while success shatters the target's soul gem and returns the creature to life as normal. If the soul gem resides in an unholy location, such as that created by the unhallow spell, the DC of this check increases by 2.  If you are a souldrinker (see page 212), you can cast this spell and expend 5 soul points to fill the gem with the equivalent of one basic soul.}
        
\DeclareSpell{Hellfire Ray}{evocation () [evil]|V,  S,  F/DF (any unholy symbol or heretical tome)|1 standard action|close (25 ft. + 5 ft./2 levels)|Effectray|instantaneous|none (see text)|yes}[]
    \DeclareSpellDescription{Hellfire Ray}{A blast of hellfire blazes from your hands. You can fire one ray, plus one additional ray for every 4 caster levels beyond 11th (to a maximum of three rays at 19th level). Each ray requires a ranged touch attack to hit and deals 1d6 points of damage per caster level (maximum 15d6). Half the damage is fire damage, but the other half results directly from unholy power and is therefore not subject to being reduced by fire resistance. The rays can be fired at the same target or at different targets, but all rays must be fired simultaneously and aimed at targets within 30 feet of each other.  Any creature killed by this spell must attempt a Will saving throw; failure means the creature's soul is damned to Hell as a burst of brimstone appears around its corpse. A nonevil spellcaster attempting to bring the creature back from the dead  must attempt a caster level check (DC = 10 + the slain creature's level) to succeed; failure means the spellcaster cannot try again for 1 day. Evil spellcasters can raise the slain character normally, without requiring a check. A raised c}
        
\DeclareSpell{Malediction}{necromancy () [curse,  evil]|V,  S|1 standard action|touch|Targets1 creature touched|1 minute and instantaneous (see text)|Will negates|yes}[]
    \DeclareSpellDescription{Malediction}{Channeling the blasphemy of fiends into your hand, you mark your target with a brief but fundamental corruption, causing its soul to be irretrievably damned should it die within the next minute. If you are lawful evil, souls are sent to Hell. If you are neutral evil, souls are sent to Abaddon. If you are chaotic evil, souls are sent to the Abyss.  A target killed while under the effect of this spell cannot be resurrected by normal means. Only a worshiper of a deity or demigod of your alignment can return a soul damned by malediction to life without difficulty. Other spellcasters must succeed at a caster level check (DC = 10 + your caster level) to restore to life a creature slain while under the effects of  malediction. Miracle or wish can return the victim of a malediction to life without requiring a caster level check.  A soul can also be freed by the efforts of someone bodily going to the appropriate plane, locating the affected soul, and leading it out of the plane, which allows it to go to its intended destination in the afterlife and be resurrected as normal. You can end the effects of your own malediction by casting the spell again and concentrating on a past target. Doing so only frees the past target to go to its rightful place in death; it does not return the target to life.  Spells such as break enchantment, dispel magic, and remove curse negate this spell if successfully cast before the target dies.}
        
\DeclareSpell{Parasitic Soul}{necromancy () [death,  evil]|V,  S,  F (a gem or crystal worth at least 100 gp)|1 standard action|medium (100 ft. + 10 ft./level)|Targetsone creature|permanent (D)|Will negates|yes}[]
    \DeclareSpellDescription{Parasitic Soul}{This spell functions like magic jar except as noted above, and instead of your own soul, you can transfer a trapped soul (such as one trapped in a soul gem or trapped with soul bind or trap the soul) from the receptacle into an unwilling target's body. If the target creature fails its saving throw, it dies and the trapped soul in the receptacle permanently inhabits the body as if using magic jar. The trapped soul does not get a saving throw to resist this transfer. To dismiss the spell, you must be within range of the possessed body.}
        
\DeclareSpell{Rift Of Ruin}{conjuration (calling) [chaotic,  evil]|V,  S|1 standard action|long (400 ft. + 40 ft./level)|Effect5-ft.-wide, 60-ft.-deep extradimensional hole, up to 5 ft. long per level (S)|1 round/level (see text)|Reflex partial|no}[]
    \DeclareSpellDescription{Rift Of Ruin}{This spell tears a rift in reality, creating an extradimensional hole with a depth of 60 feet. You must create the rift on a horizontal surface of sufficient size. Since the rift extends into the Abyss, it does not displace the original underlying material or allow access to areas below the surface-you can create the rift on the deck of a ship as easily as in a dungeon floor or the ground of a forest. Any Large or smaller creature standing in the area where you conjure the rift must succeed at a Reflex save to avoid falling into the hole. If successful, the creature picks which side of the rift it remains on once the rift opens. Unattended objects and structures that can be fully engulfed by the rift automatically fall into it.  The walls of the rift are covered with razor-sharp blades, while the floor seethes with boiling pools of acid, strange chewing  vermin, writhing shards of ice, and all manner of other chaotic and deadly manifestations of the Abyss. A creature that falls into the rift takes 6d6 points of falling damage. Any creature in the rift (starting on the round it enters) takes an additional 6d6 points of damage from the rift's environs, even if the creature is merely climbing or flying within the rift rather than standing at the bottom. This additional damage changes from round to round and is randomly selected from acid, bludgeoning, cold, electricity, fire, piercing, slashing, or sonic. Each round, a creature in the rift can attempt a Reflex save to take half damage that round. The rift's walls have a Climb DC of 25.  When this spell's duration ends, the rift snaps shut, violently expelling all creatures still within. These creatures take double damage from the rift's environs in that round and are knocked prone as they are returned to the surface above.  At any time during the spell's duration, you can use it to conjure a number of Abyssal denizens into the surrounding region as a standard action. Doing so causes the rift to snap shut, ending the spell's duration and returning any creatures that had fallen into it to the ground as detailed above. As the rift snaps shut, choose one of the following creatures or groups of creatures to appear in the area; these Abyssal denizens are treated as if you had summoned them via summon monster VII and remain for a number of rounds equal to the remaining duration of the rift of ruin spell. You can choose to summon one of the following: one bebilith, one vrock, 1d3 shadow demons, 1d3 succubi, 1d4+1 babaus, or 1d4+1 brimoraksB6.}
        
\DeclareSpell{Sacrifice}{enchantment (charm) [mind-affecting]|V,  S,  M (see text)|1 minute|close (25 ft. + 5 ft./2 levels)|Targetsone summoned elemental or outsider (see text)|instantaneous, 1 hour, or 1 day (see text)|none|no}[]
    \DeclareSpellDescription{Sacrifice}{You make a sacrifice to aid in conjuring and commanding a creature called with planar ally, planar binding, or a similar spell. A sacrifice can be used in a variety of ways.  Bargain: Making a sacrifice directly to the conjured being grants you a bonus on opposed Charisma checks made to compel the creature into service for the next hour.  Enticement: Making a sacrifice the round before conjuring increases the DC of the Will save an outsider must attempt to resist being conjured.  Payment: Making a sacrifice directly to the conjured being allows you to pay for one service from the creature in commodities other than gold.  Reinforcement: Making a sacrifice the round before creating a magic circle and preparing a summoning diagram amplifies the power of its warding magic, increasing the DC of Charisma checks the creature might attempt to escape. This lasts 1 day.  Multiple sacrifices can be made to affect a single conjuring, but the bonuses provided by this spell do not stack. Therefore, while you can make sacrifices to aid in conjuring and bargaining with a creature, you cannot make multiple sacrifices (even of varying types) to enhance the same effect for a particular conjuration.  A sacrifice can consist of any kind of commodity the target creature favors, including living creatures, treasures, or more ephemeral offerings. While this spell is not fundamentally evil, good-aligned creatures are more selective in what offerings they accept, typically scoffing at blood sacrifices. Many sacrifices are fundamentally evil acts, such as murdering a pious innocent to conjure a fiend. Any creature might reject certain types of sacrifices, thus denying you the benefits of this spell, as the offering must appeal to the target-few outsiders would care for 2,000 gp worth of parchment, while 2,000 gp of diamonds would be widely coveted. The GM determines what sacrifices creatures find appealing.  The table below lists a number of likely offerings, along with the bonus such gifts provide and the offering's equivalent value in gold pieces for the purposes of planar ally. Several of these sacrifices involve the loss of ability scores, levels, or lives, and some can cause changes in alignment. Any change wrought by such sacrifices (loss of ability score or level, or change in alignment) cannot be recovered, cured, or undone by any spell or effect short of miracle or wish. The same is true of creatures killed as a sacrifice; such creatures cannot be resurrected by any magic less powerful than these spells. Any object sacrificed with this spell is effectively destroyed or removed to an extraplanar holding of the conjured creature's choice. The bonuses and values noted on the sacrifice effects table are guidelines for offerings; certain types of treasures or lives might prove especially valuable to specific creatures, with extraordinary sacrifices (such as a potent artifact or the life of a high-level paladin) garnering increased bonuses.  You cannot make greater sacrifices than those noted on the table to gain increased bonuses or gold values. For example, you could not gain 2 permanent negative levels to gain a +16 bonus, nor gain increased benefit from slaying 20 Hit Dice worth of creatures to pay for a 10-HD creature's service.  Granted Type Sacrifice Bonus GP Value  Treasures 100 gp/HD of target +1 Equal  Lives1 One living creature +2 200 gp/HD with HD equal to target  Body/Mind1 Reduction of ability score +4 500 gp/point by 1 reduced  Morals2 Alignment shifts one step +6 1,000 gp/step toward target's  Soul1 One permanent negative +8 2,500 gp level  1 When used to sacrifice a life, body, mind, or soul other than the caster's own, this is an evil act.  2 A character can sacrifice only her own morals, and can do so only once per lifetime.}
        
\DeclareSpell{Soul Transfer}{conjuration (summoning) []|V,  S,  M (a gem worth 1, 000 gp per HD of the targeted creature or soul)|1 standard action|close (25 ft. + 5 ft./2 levels)|Targetsone petitioner, incorporeal soul, or similar creature|permanent; see text|Will negates|yes (see text)}[]
    \DeclareSpellDescription{Soul Transfer}{This spell functions like the spell completion option of trap the soul, except it works only on bodiless souls (such as incorporeal undead or a soul trapped in a gem) and creatures whose substance is a physical incarnation of a soul (such as a petitioner). It does not work on creatures formed from souls or planar material (such as most outsiders). Soul transfer is mainly used to transfer souls from one receptacle to another, but it can also be used to capture vulnerable souls that aren't bound to mortal flesh (such as incorporeal creatures and petitioners). When used to capture a petitioner, the petitioner's physical body vanishes, reappearing only when its soul is released from the receptacle.}
        
\DeclareSpell{Waters Of Lamashtu}{conjuration (creation) []|V,  S,  M (250 gp of powdered amber)|1 standard action|close (25 ft. + 5 ft./2 levels)|Effectup to 1 flask of the waters of Lamashtu per 2 levels|instantaneous|Fortitude partial|no}[]
    \DeclareSpellDescription{Waters Of Lamashtu}{This spell generates what appears to be clear, pure water, but it is in fact a foul secretion known as the waters of Lamashtu. The liquid functions in all the same ways as unholy water (see curse water). In addition, any creature that is anointed with or drinks this fluid must attempt a Fortitude save (drinking the waters of Lamashtu is particularly effective-a creature that drinks the liquid takes a -4 penalty on its save to resist its effects). Success causes the creature to become violently ill, vomit up the fluid, and become sickened for 1d4 rounds. Failure indicates the water takes root and wreaks havoc on the victim's mind (dealing 1d6 points of Intelligence damage) and twists and deforms its body (dealing 1d6 points of Dexterity damage). The subject's Dexterity and Intelligence cannot drop below 1 as a result of this effect. Casting this spell creates approximately 2 ounces of the waters of Lamashtu-enough for one dose or, if bottled, one use as a thrown weapon.  The fluid can be created and stored indefinitely, though it cannot be created inside a creature. Extensive exposure to the waters of Lamashtu (such as drinking nothing else for months at a time) can have other long-term effects on the target, including the development of monstrous deformities or even total transformation into a beast, depending on the GM's discretion (these mutations are rarely, if ever, beneficial to the victim).}
    
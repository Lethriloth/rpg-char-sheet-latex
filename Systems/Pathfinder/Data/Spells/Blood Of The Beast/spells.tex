    
\DeclareSpell{Bit Of Luck}{evocation|V,  S,  M (a four-leaf clover)|1 standard action|personal|Targets: you|10 minutes/level||}[]
    \DeclareSpellDescription{Bit Of Luck}{For the spell's duration, the caster gains a reservoir of luck with a total number of points equal to 1 point per 2 caster levels. During the spell's duration, the caster can spend 1 point from this reservoir when making an attack roll or skill check to add a +1d8 luck bonus to the d20 roll's result. This bonus can be added before or after the roll or check's result is revealed, and if this bonus is large enough to turn a failure into a success, the roll succeeds. The spell instantly ends when all points are expended or when it is cast on the target again.  An 8th-level caster can instead spend 4 points to add a +2d8 luck bonus, and a 16th-level caster can spend 8 luck points to instead add a +3d8 luck bonus.}
        
\DeclareSpell{Curse Of Befouled Fortune}{necromancy [curse]|V,  S|1 standard action|touch|Targets: creature touched|permanent|Will negates|yes}[]
    \DeclareSpellDescription{Curse Of Befouled Fortune}{You curse the target, making it incredibly unlucky. While affected by curse of befouled fortune, the target can't receive luck bonuses or benefit from effects that grant it the ability to roll multiple times and take the higher result (like the swashbuckler's charmed life ability). The target also can't choose a die result, such as taking 10 or 20 in lieu of rolling (as per the bard's lore master ability). Finally, the first time each turn the target would succeed at an attack, saving throw, or skill check, it must roll twice and use the worse of the two results.  This curse cannot be dispelled, but it can be removed with a break enchantment, limited wish, miracle, remove curse, or wish spell.}
        
\DeclareSpell{Batrachian Surge}{transmutation|V,  S|1 swift action|personal|Targets: you|1 round + 1 round/3 levels (D)|none|yes (harmless)}[]
    \DeclareSpellDescription{Batrachian Surge}{You tap into your latent amphibian strengths, unlocking a short-lived physical talent. Armor or gear you are wearing adjusts to your new shape for the duration of the spell. When you cast batrachian surge, choose one of the following features to gain its associated benefits. You can have only one batrachian surge spell active on you at a time.  Gills: Your throat expands, and gill slits appear along your neck. You can breathe underwater.  Leaping: Your legs elongate and become especially muscular. You are always treated as having a running start when attempting Acrobatics checks to jump, and you gain a competence bonus on Acrobatics checks to jump equal to your caster level.  Swimming: You grow a large tadpole tail, and your other limbs shrink slightly. You gain a swim speed equal to your base land speed.  Tongue: Your tongue extends to the length of your body. Increase your reach by 5 feet when delivering touch spells. This increased reach doesn't stack with any other spells or abilities that affect your reach.}
        
\DeclareSpell{Sweat Poison}{necromancy [poison]|V,  S|1 standard action|personal|Targets: you|1 minute/level (D)|none|no}[]
    \DeclareSpellDescription{Sweat Poison}{Glands along your neck, back, or wrists swell and exude a viscous injury poison (save Fort DC 14; frequency 1/round for 4 rounds; effect 1d2 Str; cure 1 save). You are not immune to this poison, and unless you have the poison use class feature or a similar ability, you are at risk of poisoning yourself. You can apply this poison to a weapon as a move action, and each dose you apply reduces the remaining duration of this spell by 1 minute. If doing so would reduce the remaining duration to 0 minutes or less, the spell ends, and any applied poisons retain their potency only until the end of your turn.  When you apply the poison, you can choose to reduce the spell's remaining duration by 2 or more additional minutes (maximum = your caster level) in order to enhance that dose of poison. For every 2 minutes of duration expended, the poison's save DC increases by 1, the number of rounds it lasts increases by 1/2 (round down), and the number of saves required to cure it increases by 1/4 (round down). If you have the toxic skin alternate racial trait (Advanced Race Guide 190), you can expend one daily use as a free action to enhance this spell's poison's damage to 1d3 Strength and its starting save DC to 15.}
        
\DeclareSpell{Contagious Suggestion}{enchantment (compulsion) [language-dependent,  mind-affecting]|V,  S|1 standard action|close (25 ft. + 5 ft./level)|Targets: one living creature|1 hour/level or until completed (D)||}[]
    \DeclareSpellDescription{Contagious Suggestion}{This spell functions as per suggestion, except the target can pass on the enchantment to other targets. The target is compelled to communicate your suggestion to another creature, forcing the new target to attempt a saving throw as if it were the initial target. If a secondary target successfully saves, the suggestion effect on the initial target isn't negated. If a secondary target fails, it is placed under the same compulsion as the initial target and can further spread the suggestion. This spell can affect a total number of Hit Dice of creatures equal to your caster level. Creatures that save against this spell cannot be affected by that particular casting of contagious suggestion for 24 hours.}
        
\DeclareSpell{Gullibility}{enchantment (charm) [mind-affecting]|V,  S|1 standard action|close (25 ft. + 5 ft./level)|Targets: one creature|10 minutes/level (D)|Will negates|yes}[]
    \DeclareSpellDescription{Gullibility}{You befuddle the target's mind, making it more willing to believe even the most outlandish tales. The target takes a -10 penalty on Sense Motive checks for the spell's duration. In addition, creatures that attempt to lie to or deceive the target gain a +10 bonus on their Bluff checks, as if the target wanted to believe them and was drunk or impaired simultaneously. Furthermore, the believability of any lie told to a creature under the effects of gullibility increases by one step; an impossible lie seems far-fetched, a far-fetched lie seems unlikely, and an unlikely lie seems believable.  A creature gains no benefits from glibness on B luff checks to lie to or deceive a creature that is under the effects of gullibility; in effect, gullibility r enders a t arget s o w illing t o believe what others say that glibness cannot make their words any more believable.}
        
\DeclareSpell{Greater Hypnotism}{enchantment (compulsion) [mind-affecting]|V,  S|1 round|close (25 ft. + 5 ft./2 levels)|Area: several living creatures, no two of which may be more than 30 ft. apart|10 minutes/level (D)|Will negates|yes}[]
    \DeclareSpellDescription{Greater Hypnotism}{This functions as hypnotism, except it affects 2 Hit Die of creatures per caster level you have. You can make up to five requests (instead of a single request) to each creature affected by the spell, and the requests can be as long and as complicated as you desire (though they still must be reasonable).}
        
\DeclareSpell{Metabolic Molting}{transmutation|V,  S,  M (a valuable gemstone worth at least 250 gp)|10 minutes|touch|Targets: willing creature or dead body touched|see text|Will negates (harmless)|no}[]
    \DeclareSpellDescription{Metabolic Molting}{You encase the willing subject in a jeweled shell that has the same hardness and hit points as 5 inches of iron (hardness 10, hp 150). For 7 days, the subject enters a state of suspended animation (as per temporal stasis), during which it heals from even the most grievous wounds. Each day, the subject regains 5 hit points per Hit Die it has as well as recovering from 2 points of ability damage or ability drain from any ability score of its choice. After 7 days, the shell crumbles away and the subject emerges, its body's severed limbs, broken bones, and ruined organs regrown (as per regenerate).  If the gemstone you use for the component is worth at least 5,000 gp, metabolic molting can bring a target back to life, as long as it has been dead for no more than 1 hour when the spell is cast. When the spell is used on such a creature, the creature comes back to life after 7 days and stabilizes at 0 hit points. A creature brought back to life through metabolic molting gains 1 permanent negative level, or 2 points of Constitution drain if it is 1st level. Like raise dead, this spell can't save creatures slain by death effects.}
        
\DeclareSpell{Naga Shape I}{transmutation (polymorph)|V,  S,  M (a piece of the creature whose form you want to assume)|1 standard action|personal|Targets: you|1 minute/level (D)||}[]
    \DeclareSpellDescription{Naga Shape I}{Whenever you cast this spell, you can assume the form of any type of Large naga (including most nagas, but not royal nagas), though not the form of a specific individual. In effect, you transform into a Large serpent, gaining a +4 size bonus to your Strength, a -2 penalty to your Dexterity, and a +4 natural armor bonus. Unlike with beast shape II, however, you keep your own head when using this spell and can cast spells with verbal and somatic components, even though the naga form doesn't have hands.  If the form you assume has any of the following abilities, you gain the listed ability: climb 60 feet, fly 60 feet (good maneuverability), swim 60 feet, darkvision 60 feet, low-light vision, and scent.}
        
\DeclareSpell{Naga Shape Ii}{transmutation (polymorph)|V,  S,  M (a piece of the creature whose form you want to assume)|1 standard action|personal|Targets: you|1 minute/level (D)||}[]
    \DeclareSpellDescription{Naga Shape Ii}{This spell functions as naga shape I. Additionally, if the form you assume has any of the following abilities, you gain the listed ability: burrow 30 feet, climb 90 feet, fly 90 feet (good maneuverability), swim 90 feet, blindsense 30 feet, darkvision 60 feet, low-light vision, compression, constrict, detect thoughts, dreamsight, grab, hypnosis, poison, sneak attack +2d6, and spit.}
        
\DeclareSpell{Naga Shape Iii}{transmutation (polymorph)|V,  S,  M (a piece of the creature whose form you want to assume)|1 standard action|personal|Targets: you|1 minute/level (D)||}[]
    \DeclareSpellDescription{Naga Shape Iii}{This spell functions as naga shape I. Additionally, if the form you assume has any of the following abilities, you gain the listed ability: burrow 60 feet, climb 90 feet, fly 120 feet (good maneuverability), swim 120 feet, blindsense 60 feet, darkvision 90 feet, low-light vision, tremorsense 60 feet, bleed, compression, constrict, detect thoughts, dreamsight, grab, hypnosis, poison, sneak attack +3d6, and spit.}
        
\DeclareSpell{Depilate}{necromancy|S,  M (a sliver of cow's tongue)|1 round|medium (100 ft. + 10 ft./level)|Targets: one creature with hair or fur|instantaneous|Will negates|yes}[]
    \DeclareSpellDescription{Depilate}{This simple jinx causes a target's hair or fur to fall out in patchy clumps, leaving the creature disheveled and less commanding. A jinxed creature takes a -2 penalty on all Diplomacy, Intimidate, and Perform checks until the hair begins to regrow 1 week later, or until the damage can be concealed with a successful Disguise or Heal check (the DC is equal to the original save DC of the spell). A successful break enchantment, remove curse, or similar effect instantly regrows the target's hair or fur and removes the spell's effects.}
        
\DeclareSpell{Fumblestep}{conjuration|V,  S,  M (a sharp pebble)|1 standard action|close (25 ft. + 5 ft./2 levels)|Targets: one creature|1 minute/level or until discharged (see text)|none (see text)|yes}[]
    \DeclareSpellDescription{Fumblestep}{Fumblestep coats a target's feet in slick ectoplasm that retains some psychic connection to your mind. While the spell remains in effect, the target creature takes a -1 penalty on Reflex saving throws and Acrobatics checks.  At any point during the spell's duration while the target remains in range, you can discharge the spell as a standard action, mentally tugging at the ectoplasm to perform a trip combat maneuver, using your caster level instead of your base attack bonus, and using your Charisma, Intelligence, or Wisdom modifier, whichever is highest, instead of your Strength modifier. This trip attempt does not provoke an attack of opportunity and cannot be affected by feats or other means of altering a trip combat maneuver's effects.}
        
\DeclareSpell{Lightfingers}{transmutation|V,  S|1 standard action|close (25 ft. + 5 ft./2 levels)|Targets: one creature|instantaneous|none|yes}[]
    \DeclareSpellDescription{Lightfingers}{A flutter of subtle, telekinetic force causes your target to drop a single nonmagical carried item weighing no more than 1 pound per caster level, such as a coin purse, a key ring, or a loose object in its pocket. The target must succeed at a Perception check against the spell's save DC to notice the dropped item. You select what item the target drops as long as you can see both your target and the item you wish to affect. This use of the spell can't cause a target to lose a held item or anything it is wearing, including jewelry.  You can instead cast lightfingers to attempt a single disarm or stealAPG combat maneuver, using your caster level instead of your base attack bonus, and using your Charisma, Intelligence, or Wisdom modifier, whichever is highest, instead of your Strength modifier. This disarm or steal attempt does not provoke attacks of opportunity and cannot be affected by feats or other means of altering a combat maneuver's effects. This use of the spell is immediately obvious.}
    
    
\DeclareSpell{Amnesia}{enchantment (compulsion) [mind-affecting]|V,  S|1 round|close (25 ft. + 5 ft./2 levels)|Targets: one living creature|instantaneous|Will negates|yes}[]
    \DeclareSpellDescription{Amnesia}{You cause the target to lose most of its memories; its skills, its past, and even its name become mysteries to it. While the target can build new memories, it has trouble accessing those gained before falling victim to the spell. The target can still speak and read any languages it knows and perform basic tasks, but it loses all class abilities, feats, and skill ranks gained before being affected by amnesia. It retains its base attack bonus, saving throws bonuses, Combat Maneuver Bonus, Combat Maneuver Defense, total experience points, Hit Dice, and hit points. If the target gains a character level while suffering from amnesia, it can use any abilities gained by that class level normally. If the class level it gained was from a class in which it already has levels, it gains the abilities of a 1st-level character of that class, even though it is technically of a higher level in that class. If the amnesia is cured, the target regains the full abilities of the class, including those gained from any levels taken while suffering from this condition. Amnesia can be removed only by heal, limited wish, miracle, psychic surgeryOA, or wish.}
        
\DeclareSpell{Greater Appearance Of Life}{illusion (glamer)|V,  S,  M (one Tiny or larger living creature)|1 round|touch|Targets: one willing undead creature touched|1 hour/level|Will disbelief or Will negates (see text)|no}[]
    \DeclareSpellDescription{Greater Appearance Of Life}{This spell functions as appearance of life (Pathfinder RPG Horror Adventures 108), except that you can disguise a single undead creature regardless of its number of Hit Dice, and the illusion also creates relevant smells, sounds, textures, and temperatures  to match the appearance. If you disguise the target as a living version of itself, creatures that interact with the target take a -5 penalty on their Will saving throws to disbelieve the illusion, though if the target attacks a creature, the attacked creature no longer takes this penalty.}
        
\DeclareSpell{Covetous Urge}{enchantment (compulsion) [curse,  language-dependent,  mind-affecting]|V,  S,  M (a golden coin)|1 standard action|close (25 ft. + 5 ft./2 levels)|Targets: one living creature with 7 HD or fewer|1 minute/level (D)|Will negates|yes}[]
    \DeclareSpellDescription{Covetous Urge}{You curse the target with insatiable greed, causing it to attempt to steal any item worth 100 gp or more that it can see, each time it enters a new room or area. If it can see multiple items of value in a single room or area, it attempts to steal the item it thinks is worth the most. This covetous compulsion might cause the target to act recklessly where it normally would not (unless acting in this manner would clearly cause it to die, suffer great harm, or get caught).  In combat, if the target sees a valuable item on an opponent, it must, for example, attempt to disarm its foe to take a valuable weapon or use the steal combat maneuver (Pathfinder RPG Advanced Player's Guide 322) to grab a precious item kept on the opponent's belt. Once the target has either attempted to steal an item in combat or else taken damage, this magical compulsion subsides until the threat of the current combat ends and the target enters a new room or area. Affected targets in combat that have neither attempted to steal an item nor taken damage (perhaps because the opponent with the most valuable item is far away) can attempt a saving throw each round to ignore the spell's effect until the threat ends and the target enters a new room.  Break enchantment, limited wish, miracle, remove curse, or wish can each end covetous urge before the duration expires. Dispel magic does not affect covetous urge.}
        
\DeclareSpell{Hide Bruises}{illusion (glamer)|V,  S,  M (drop of blood)|1 standard action|touch|Targets: one creature touched|10 minutes/level (D)|none and Will disbelief (see text)|yes (harmless)}[]
    \DeclareSpellDescription{Hide Bruises}{The target's wounds to seem to disappear. A creature attempting a Heal check to tell the severity of the target's injuries takes a -10 penalty on that check. A creature that physically touches the glamered areas can attempt a Will save to recognize it as an illusion.}
        
\DeclareSpell{Resist Starvation}{transmutation|V,  S,  M (crumb of food)|1 standard action|touch|Targets: one living creature touched|1 day/level|Will negates (harmless)|yes (harmless)}[]
    \DeclareSpellDescription{Resist Starvation}{If the target doesn't eat on the day this spell is cast on it, the DC to avoid taking nonlethal damage from starvation on the following day doesn't increase by 1. This spell fails if the target hasn't already attempted at least one Constitution check to avoid starvation.}
        
\DeclareSpell{Beacon Of Guilt}{evocation [curse]|V,  S|1 standard action|touch|Targets: one object|24 hours or until discharged, then instantaneous|Will negates (object); see text|yes}[]
    \DeclareSpellDescription{Beacon Of Guilt}{You place an invisible ward upon an object that is triggered the first time a creature tries to move the object from its current location. The next creature to touch the object is cursed to become obvious  to everyone around it. The creature must succeed at a Will save or be surrounded in an aura of twinkling red light that functions as faerie fire (spell resistance applies). The curse bestowed by this spell cannot be dispelled, but a break enchantment, limited wish, miracle, remove curse, or wish spell can remove it.}
        
\DeclareSpell{Escape Alarm}{abjuration|V,  S,  F (a short segment of a chain)|1 standard action|medium (100 ft. plus 10 ft./level)|Area: ten 10-foot cubes/level|24 hours|none|no}[]
    \DeclareSpellDescription{Escape Alarm}{You place a ward on an area that notifies you when a creature exits it. This functions as alarm, except as noted. It alerts you when a creature leaves, rather than enters, the area, and you can't select a password to bypass its effects. Instead, when you place an escape alarm, you can attune up to one additional creature per caster level to the spell. You are automatically attuned to your own escape alarm and don't count against the limit. Attuned creatures can enter and exit the spell's area without triggering the alarm. If you select a mental alarm rather than an audible one, all attuned creatures receive the mental alert when someone exits the warded area.}
        
\DeclareSpell{Outbreak}{necromancy|V,  S|1 standard action|close (25 ft. + 5 ft./2 levels)|Area: 20-ft. burst|instantaneous|Fortitude partial|yes}[]
    \DeclareSpellDescription{Outbreak}{You cause any diseased creatures in the area to become extremely contagious. Any affected creature is overcome by a fit of wracking coughs, causing it to become fatigued and exposing any creatures within 10 feet of the diseased creature to the diseases it carries. On a successful save, a creature is fatigued for 1 round and doesn't have a chance to infect nearby creatures.}
        
\DeclareSpell{Virulent Miasma}{necromancy|V,  S,  M (a dried scab from a diseased creature)|1 standard action|medium (100 ft. + 10 ft./level)|Effect: cloud spreads in 20-ft. radius, 20 ft. high|1 round/level|Fortitude negates (see text)|see text}[]
    \DeclareSpellDescription{Virulent Miasma}{Virulent miasma creates fog like fog cloud, except that the vapors carry the taint of disease. Living creatures take a -4 penalty on saving throws against disease effects as long as they remain within the cloud and for 1d4+1 rounds after leaving. This effect of the spell allows a Fortitude save to negate it, and it is subject to spell resistance. A creature that succeeds at its saving throw against the fog (or ignores the fog due to spell resistance) is not affected and need not make further saves even if it remains in the fog.}
        
\DeclareSpell{Cloak Of Shadows}{illusion (shadow) [evil,  shadow]|V,  S|1 standard action|personal|Targets: you|1 minute/level||}[]
    \DeclareSpellDescription{Cloak Of Shadows}{You wrap yourself in a protective mantle of semi-real shadow. This grants a number of advantages: you gain concealment (20\% miss chance), a +5 competence bonus on Stealth checks, and DR 5/good. This shroud of shadows also protects you from direct  sunlight, negating sunlight vulnerability, sunlight powerlessness, and similar effects.  When you are in dim light or darkness, the first two benefits increase against foes that don't have darkvision or the see in darkness ability (Pathfinder RPG Bestiary 2 301). Such creatures suffer a 50\% miss chance (though you don't gain total concealment) and your competence bonus on Stealth checks increases to +10 with respect to such creatures.}
        
\DeclareSpell{Shadow Claws}{illusion (shadow) [shadow]|V,  S|1 standard action|personal|Targets: you|1 minute/level||}[]
    \DeclareSpellDescription{Shadow Claws}{You summon a pair of claws over your hands formed from semi-real material. This grants you two primary claw attacks dealing 1d4 points of damage if you are Medium (1d3 if Small) plus 1 point of Strength damage. A successful Fortitude saving throw negates the Strength damage (DC = this spell's DC).}
        
\DeclareSpell{Shadow Jaunt}{illusion (shadow) [shadow]|V,  S|1 standard action|close (25 ft. + 5 ft./2 levels)|Targets: you|instantaneous and 1 round; see text||}[]
    \DeclareSpellDescription{Shadow Jaunt}{You instantly travel between shadows to a point within range, though line of sight is not necessary. You leave a shadowy image of yourself in your former location and are wrapped in shadow at your destination; you can attempt a Stealth check as a free action to hide in your new location. Creatures that fail opposed Perception checks typically aren't aware that you are no longer at your former location unless they are familiar with this spell and identify the effects. Any attack on your former location causes the shadows to disperse, allowing any creature that can see your former location to immediately notice your disappearance. In addition, for 1 round, the envelope of shadow around you grants you concealment (20\% miss chance). This spell doesn't function if there are no shadows within 5 feet of your starting location, and you can't travel through your own shadow. If you choose a destination that has no shadows, you appear from the shadow closest to your destination that's within the spell's range (this could be your starting location if there are no other shadows within the spell's range).}
        
\DeclareSpell{Fool's Gold}{illusion|V,  S,  M (a copper piece or silver piece)|1 standard action|touch|Targets: objects touched|1 hour/level|none or Will disbelief (see text)|no}[]
    \DeclareSpellDescription{Fool's Gold}{You can temporarily make copper or silver seem to be an equivalent amount of gold. The spell affects 1 gp/level worth of material (thus, at 3rd level, the spell affects up to 300 copper pieces, 30 silver pieces, or a copper or silver item worth no more than 3 gp). Coins increase in value as normal for the new type of coinage. For items other than coins, some of the value of the item comes from its craftsmanship, regardless of the metal used, so the value of the item seems to be 5 times (for silver items) or 50 times (for copper items) its true value. Thus, a copper candlestick originally worth 5 cp transformed by this spell appears to be worth 250 cp, or 2 gp and 5 sp. A successful DC 25 Appraise check automatically detects the true nature of the coins or items. Creatures inspecting or interacting with the coins or items can attempt a saving throw to disbelieve the illusion.}
        
\DeclareSpell{Burning Entanglement}{evocation|V,  S,  DF|1 standard action|long (400 ft. + 40 ft./level)|Area: 40-ft.-radius spread|1 round/level|Reflex negates or partial (see text)|no}[]
    \DeclareSpellDescription{Burning Entanglement}{This spell functions as per entangle, except it sets the foliage on fire. A creature that begins its turn entangled by the spell takes 4d6 points of fire damage (Reflex half), and a creature that begins its turn in the area but is not entangled takes 2d6 points of fire damage (Reflex negates). Smoke rising from the vines partially obscures visibility. Creatures can see things in the smoke within 5 feet clearly, but attacks against anything farther away in the smoke must contend with concealment (20\% miss chance). When the spell's duration expires, the vines burn away entirely.}
        
\DeclareSpell{Nature's Paths}{divination|V,  S,  M/DF (a smooth stone)|1 standard action|touch|Targets: one creature|8 hours (D)|Will negates (harmless)|yes (harmless)}[]
    \DeclareSpellDescription{Nature's Paths}{The target instinctively knows the shortest, easiest, and fastest way through the wilderness. For the purpose of determining overland speed, the target treats any trackless terrain as though there were a trail or road, and any terrain with a road or trail as though there were a highway (Pathfinder RPG Core Rulebook 171-172). Up to one creature per caster level traveling with the target can also benefit from the effect. The spell functions only outdoors and does not function in magically altered terrain.}
        
\DeclareSpell{Dousing Rain}{evocation [water]|V,  S,  M/DF (a drop of water)|1 standard action|medium (100 ft. + 10 ft./level)|Area: cylinder (10-ft. radius, 40 ft. high)|1 round/level (D)|none|no}[]
    \DeclareSpellDescription{Dousing Rain}{With a beckoning gesture, you call forth a downpour of rain. For the duration of the spell, the following effects apply within the affected area. Nonmagical fires are automatically extinguished, and all creatures and objects in the area gain fire resistance 5. The conjured water is conductive, and whenever a doused creature takes electricity damage, it takes 1 additional point of electricity damage. At 6th, 12th, and 18th levels, the fire resistance increases by 5 and the additional electricity damage increases by 1 point.}
        
\DeclareSpell{Reinvigorating Wind}{enchantment (compulsion) [air,  mind-affecting]|V,  S,  M/DF (a flower petal)|1 standard action|30 ft.|Area: cone-shaped burst|instantaneous|Will negates (harmless)|yes (harmless)}[]
    \DeclareSpellDescription{Reinvigorating Wind}{You exhale deeply, creating a gentle magical wind that invigorates any allies in the affected area, as follows. Any sleeping allies immediately wake up. Fascinated allies are shaken free of the fascinate effect. Flat-footed allies no longer count as flat-footed even if they have not acted yet. The duration of effects that cause any allies to be confused, frightened, paralyzed, slowed, or stunned is decreased by 1d4 rounds (roll separately for each target). If the duration of any such effect is reduced to 0 rounds or fewer, the effect ends for that ally. Finally, any allies lying prone may stand up as an immediate action, provoking attacks of opportunity as normal.}
        
\DeclareSpell{Steady Saddle}{transmutation|V,  S,  DF|1 standard action|touch|Targets: saddle touched|1 minute/level (D)|Will negates (harmless, object)|yes (harmless, object)}[]
    \DeclareSpellDescription{Steady Saddle}{A saddle affected by this spell becomes more comfortable to sit in, and the magic stabilizes any shaking motion caused by riding at a high speed. For the duration of the spell, the penalty to use ranged weapons while mounted in the target saddle decreases by 2. This stacks with the benefit of the Mounted Archery feat and similar effects. Furthermore, the DC for any concentration check required as a result of the mount's movement decreases by 2.}
        
\DeclareSpell{Wicker Horse}{conjuration (creation)|V,  S,  M/DF (a reed or twig)|10 minutes; see text|touch|Effect: a horselike wicker construct|1 hour/level (D)|none|no}[]
    \DeclareSpellDescription{Wicker Horse}{Reeds, grasses, creepers, or thin tree branches (your choice) that you touch animate, twisting and bending to form a wicker horse complete with a riding saddle. The object is animated, but retains a wickerlike appearance.  A wicker horse has the statistics of a light horse or a pony, except it gains construct traits and counts as both an animal and a construct for the purposes of spells and effects. It gains a number of extra hit points equal to your caster level. It has no skills or feats except as noted below.  A horse made of reeds gains a number of Swim ranks equal to your caster level. A horse made of grasses gains the Run feat and a number of Acrobatics ranks equal to your caster level. A horse made of creepers gains a number of ranks in Climb and Stealth equal to your caster level. A horse made of tree branches is continually affected by barkskin, as cast by a magic-user with a caster level equal to yours.  The wicker horse does not follow any commands given with the Handle Animal skill, and only you can ride it. You can cast this spell only in an environment where suitable plant material is available. When the spell ends or the horse loses all its hit points, the horse falls apart.}
        
\DeclareSpell{Cursed Treasure}{necromancy [curse]|V,  S,  M (a platinum piece)|1 minute|touch|Targets: unattended object touched|permanent until discharged, then permanent (see text)|Fortitude negates (object), then Will negates (see text)|yes (object), then yes (see text)}[]
    \DeclareSpellDescription{Cursed Treasure}{You touch a piece of treasure or container filled with treasure and place a terrible curse upon it, choosing from any option available with bestow curse. The next creature to take the treasure or remove items from the container is affected by the curse, unless it succeeds at a Will save (spell resistance applies). If you are the next creature to take the treasure or get objects from the container, though, the spell discharges harmlessly instead.}
        
\DeclareSpell{Rotgut}{transmutation|V,  S,  M (a pinch of hops)|1 round|close (25 ft. + 5 ft./level)|Targets: 1 gallon of water/level|instantaneous|Fortitude negates (object)|yes (object)}[]
    \DeclareSpellDescription{Rotgut}{You transform the target into a cheap alcohol of your choice, such as beer, grog, mead, rum, or wine. The alcohol doesn't taste good, but it's drinkable and just as effective as normal alcohol in making creatures inebriated (Pathfinder RPG GameMastery Guide 237). This spell doesn't work on holy water, potions, magical liquids, or water that is part of a creature.}
        
\DeclareSpell{Walk The Plank}{conjuration (creation)|V,  S,  M (a shark tooth and a splinter of a ship's hull)|1 standard action|close (25 ft. + 5 ft./level)|Effect: 20-ft.-by-20-ft. hole, 10 ft. deep/4 levels|1 round plus 1 round/level|Reflex negates (see text)|no}[]
    \DeclareSpellDescription{Walk The Plank}{This spell functions as per create pitAPG, except as noted here. All but the top 10 feet of the hole is filled with saltwater, reducing the fall damage for anyone falling into the pit to 1d3 points of nonlethal damage and allowing Huge or larger creatures at the water's surface to simply pull themselves out of the pit as part of a normal move action, without a Climb check. Additionally, the pit contains a single advanced shark (Pathfinder RPG Bestiary 294, 247) for every 4 caster levels you have (maximum four advanced sharks). If desired, when you cast the spell, you can substitute two of the summoned sharks for one great white shark (Pathfinder RPG Bestiary 4 241). These sharks immediately attack any creature that falls into the pit, even you. If multiple creatures fall in the pit, the sharks attack randomly. When the spell ends, creatures inside the hole rise up with the bottom of the pit, as normal for create pit, but the conjured water and sharks vanish.}
        
\DeclareSpell{Hobble}{transmutation|V,  S,  M (a drop of glue)|1 standard action|close (25 ft. + 5 ft./2 levels)|Targets: one creature/level, no two of which can be more than 20 feet apart|1 round/level|Fortitude negates|yes}[]
    \DeclareSpellDescription{Hobble}{This spell disrupts the method a target uses for movement. Creatures under the effects of this spell move at half their base speed (rounded down to the nearest 5-foot increment) but can still take 5-foot steps as normal. Incorporeal creatures and creatures flying with perfect maneuverability are immune to the effects of this spell. Each round, a target receives another save to end the effect.}
        
\DeclareSpell{Hoodwink}{enchantment (compulsion) [mind-affecting]|V,  S,  M (small piece of black cloth)|1 standard action|close (25 ft. +5 ft./level)|Targets: up to one creature/level, no two of which can be more than 30 ft. apart|1 hour/level (D)|Will negates|yes}[]
    \DeclareSpellDescription{Hoodwink}{You befuddle the targets' senses, preventing the targets from clearly perceiving their surroundings.  Instead, hoodwinked creatures see only the roughest shapes and details and hear only muffled noises. However, targets can clearly hear verbal communication from the spell's caster. This spell does not impart understanding of the caster's language if the creature can't already understand it. Other than to hear the caster speak, targets automatically fail Perception checks with DCs greater than 5, and they are too disoriented by the spell to accurately remember the path they took while under its effects.  While under the effects of this spell, targets without the help of a guide move at half speed, and any attacks they make are treated as though the creature they are attacking had concealment. Moving faster than half speed requires targets to succeed at a DC 10 Acrobatics check. Those that fail this check fall prone. If the target is attacked or physically harmed by any creature other than another target of hoodwink, the spell's effect ends immediately for that target only.}
        
\DeclareSpell{Geomessage}{illusion (figment)|V,  S,  M (a scrap of vellum)|1 minute|touch|Targets: surface touched|1 day/level|none|no}[]
    \DeclareSpellDescription{Geomessage}{You create a written message of 50 words or fewer, or else an image, a drawing, or a similar diagram. The figment hangs invisible and obscured upon the target surface for the duration of the spell. Another casting of geomessage is needed to cause the figment to arrange itself into the intended message.  When you cast the spell, you can choose to incorporate a passphrase into the spell. If you do so, the message is protected as if nondetection had been cast on it for the full duration of the spell, though creatures that cast geomessage on the surface and speak the passphrase ignore the nondetection effect and cause the image to become visible until the caster speaks the passphrase again.  Without a passphrase, the message simply remains invisible to those who cast geomessage. Detection methods like detect magic and see invisibility indicate the message's presence but do not reveal its contents.}
    
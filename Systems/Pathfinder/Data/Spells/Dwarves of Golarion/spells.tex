    
\DeclareSpell{Ancestral Communion}{divination|V,  S,  F/DF (stone or metal image of your ancestor)|1 minute|personal|Targets: you|1 minute/level||}[]
    \DeclareSpellDescription{Ancestral Communion}{You contact the spirits of your ancestors and use their great wisdom to bolster your own knowledge. Consulting with the spirits is a full-round action. If you consult with the spirits before making a Knowledge check, you gain a +4 insight bonus on the check. If you have already failed at a Knowledge check, you may consult with your ancestors and make another attempt. The insight bonus on these checks increases to +6 at caster level 7th and +8 at caster level 11th. You may consult with the spirits for this purpose as often as you like while the spell remains in effect. Only you can hear the spirits speak to you.}
        
\DeclareSpell{Ancestral Gift}{conjuration (summoning)|V,  S,  F/DF (stone or metal image of your ancestor)|1 standard action|personal|Effect: magical weapon|10 minutes/level||}[]
    \DeclareSpellDescription{Ancestral Gift}{A ghostly manifestation of one of your ancestors appears before you bearing a weapon of your choice in its hands. The weapon may be any simple, martial, or dwarven weapon. It has a +1 enhancement bonus and one weapon special ability (your choice) from the Pathfinder RPG Core Rulebook with a price equivalent to a +1 bonus (if the weapon is a double weapon, the ability and the enhancement bonus only apply to one end, or the weapon can have a +1 enhancement bonus on both ends but no other magical abilities).

You may use the weapon as if you were proficient in it. The weapon may not be wielded by anyone else, and if removed from your grasp, it vanishes and the spell ends immediately.

If you conjure a weapon with the flaming, frost, shock, or thundering property, this spell has the fire, cold, electricity, or sonic descriptor (respectively).}
        
\DeclareSpell{Summon Ancestral Guardian}{conjuration (summoning)|V,  S,  F/DF (stone or metal image of your ancestor)|1 standard action|medium (100 ft. + 10 ft./level)|Effect: two summoned ancestor spirits|1 round/level|none|yes}[]
    \DeclareSpellDescription{Summon Ancestral Guardian}{You call the spirits of two ancestors to manifest in the mortal world and attack your enemies. Each appears as a transparent image of a powerful, wise dwarf armed with a traditional dwarven weapon of your choice. These spirits move and attack at your direction, each having the abilities of a spiritual weapon, except they can attack different targets and deal physical damage (bludgeoning, piercing, or slashing, according to the weapon the spirit wields) instead of force damage. Like creatures conjured with a summon monster spell, your ancestors are not harmed if these manifestations are destroyed.}
        
\DeclareSpell{See Through Stone}{divination|V,  S,  DF|1 standard action|touch|Targets: creature touched|concentration, up to 1 round/level|Will negates (harmless)|yes (harmless)}[]
    \DeclareSpellDescription{See Through Stone}{You gain the ability to see through solid rock as if it were transparent glass. You may see through 1 foot of stone per caster level. You see within the stone as if you were looking at the area in normal light, even if there is no illumination, though low-light vision and darkvision have no effect on your ability to see through stone. Metal at least 1 inch thick or wood or dirt at least 3 feet thick blocks your vision.

The spell does not negate concealment for those creatures hiding behind stone objects (the stone is still an obstacle to your attacks).}
        
\DeclareSpell{Rune Of Durability}{transmutation|V,  S,  M (iron filings)|1 minute|touch|Targets: weapon touched|permanent|none|no}[]
    \DeclareSpellDescription{Rune Of Durability}{You inscribe an angular rune upon the surface of a weapon, increasing its hit points. A weapon that bears this rune multiplies its hit points by 2, as if it were one size category larger than it actually is. Placing more than one rune of this type on a weapon has no effect.}
        
\DeclareSpell{Rune Of Warding}{abjuration|V,  S,  M (powdered adamantine,  diamond,  or mithral worth 200 gp)|1 hour|touch|Targets: doorway or portal touched|permanent until discharged|Reflex half|no (object) and yes (see text)}[]
    \DeclareSpellDescription{Rune Of Warding}{You inscribe a series of runes upon the surface of a door or around the border of an entryway. They function as a glyph of warding (blast glyph), though unlike a glyph of warding, these runes are always visible. The runes count as a glyph of warding for the purpose of what spells can defeat it, placing multiple glyphs in the same area, and so on.}
        
\DeclareSpell{Oath Of Justice}{necromancy|V,  S,  DF|1 standard action|touch|Targets: two creatures touched|permanent (see text)|none|no}[]
    \DeclareSpellDescription{Oath Of Justice}{This spell seals a solemn vow between two creatures. When this spell is cast, the targets must clasp hands and swear their oath in Kols's name. The spell functions like mark of justice, except as noted above and rather than being cursed, the oath-breaker gains a mark on the face indicating to all dwarves who see it that the target has broken a sacred oath, which gives the oath-breaker a -4 penalty to influence dwarves. The mark can be removed as described in the mark of justicespell, or the other target can forgive the oath- breaker, which causes the mark to vanish.}
        
\DeclareSpell{Tactical Formation}{abjuration|V,  S|1 standard action|close (25 ft. + 5 ft./2 levels)|Area: up to one creature/level, no two of which may be more than 30 ft. apart|10 minutes/level|Will negates (harmless)|yes (harmless)}[]
    \DeclareSpellDescription{Tactical Formation}{This spell increases the effectiveness of a group's formation in battle. When cast, all creatures under the effect of the spell must be adjacent to one another, forming an unbroken chain of squares (which may include creatures sharing the same square). This chain does not need to be a straight line. Each target in the chain receives a +2 defection bonus to AC as long as the targets stay adjacent to at least one other creature affected by the spell; moving more than 5 feet from another target ends the spell with respect to that creature only. For example, a cleric could cast it on himself and four dwarves blocking a 20-foot-wide corridor; the cleric can move freely from the left side of the formation to the right side (whether in front of or behind the other targets) and not break the spell as long as he stays within 5 feet of at least one of them.}
    
    
\DeclareSpell{Heavy Water}{transmutation [water]|V,  S,  DF|1 standard action|medium (100 ft. + 10 ft./level)|Area: cylinder of water (5-ft. radius/level, 30 ft. deep)|1 minute/level (D)|none (see text)|no}[]
    \DeclareSpellDescription{Heavy Water}{You cause a volume of water to become heavier than normal.

Swimming in or through such water requires a Swim check with a DC equal to the saving throw DC of this spell; even creatures with a swim speed must attempt this check. Success allows a creature to swim at up to half its speed as a full-round action; a creature cannot swim as a move action while in an area of heavy water. If a creature fails its Swim check by 4 or less, it makes no progress.

If it fails by 5 or more, it goes underwater.

All Perception checks to see through the affected water take a -10 penalty. All ships sailing through an area of heavy water move at half speed.}
        
\DeclareSpell{Hydrophobia}{necromancy [emotion,  fear,  mind-affecting]|V,  S|1 standard action|close (25 ft. + 5 ft./2 levels)|Area: 30-ft.-radius burst|1 round/level|Will negates|yes}[]
    \DeclareSpellDescription{Hydrophobia}{Targets in the area must succeed at a Will save or become deathly afraid of drowning. If the target is swimming or otherwise submerged in water, it must spend all of its efforts attempting to escape from the water. As long as an affected target remains in water, it takes 1d6 points of nonlethal damage each round as it thrashes about and swallows water. Even out of water, targets cannot imbibe potions or willingly interact with any fluids for the duration of this spell.}
        
\DeclareSpell{Blood Boil}{necromancy|V,  S,  M (a drop of mercury)|1 standard action|touch|Targets: one living creature|3 rounds|Fortitude negates (see text)|yes}[]
    \DeclareSpellDescription{Blood Boil}{The temperature of the target creature's blood (or other similar body fluid) begins to rise over the next 3 rounds during the caster's turn. Each round starting with the first, the target creature attempts a new saving throw at the start of the caster's turn to resist that round's effect. A successful save does not end the spell effect, but does prevent that round's effect. On the round that this spell is cast, the target becomes fatigued. On the next round, as the blood temperature begins to rise, the target's capillaries burst, dealing 1d6 points of Constitution damage to the target. On the third and final round, the target's blood begins to boil; the spell deals 1d6 points of damage per caster level (to a maximum of 15d6), and-if the target is still alive-the target becomes exhausted rather than fatigued.}
    
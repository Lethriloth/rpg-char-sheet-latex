    
\DeclareSpell{Blade Of Light}{transmutation () [good]|V,  S|1 standard action|touch|Targetsone melee weapon|1 round/level|none|no}[]
    \DeclareSpellDescription{Blade Of Light}{This spell infuses a weapon with pure sunlight (the weapon need not be a bladed weapon). A weapon enhanced by this spell sheds light as if daylight had been cast on it. It functions as a magic and good weapon for the purposes of overcoming damage reduction. The weapon grants a +2 sacred bonus on attack rolls against undead and deals +1d6 points of damage on a successful hit against such foes; against incorporeal undead, it functions as if it had the ghost touch weapon special ability. The weapon automatically confirms critical hits against foes that are vulnerable to sunlight.}
        
\DeclareSpell{Burning Sands}{conjuration (creation) [earth,  fire]|V,  S,  M (a handful of sand)/DF|1 standard action|medium (100 ft. + 10 ft./level)|Area20-ft. radius|1 round/level|none|no}[]
    \DeclareSpellDescription{Burning Sands}{You cause sheets of hot sand to spread over the ground in the area of effect. This layer of sand is 1 foot deep and constantly shifts and churns, transforming the ground in the area into difficult terrain. The sand itself burns, and periodic flames rise from the grit. While these flames cannot ignite objects, they deal 1d4 points of fire damage to any creature that ends its turn in contact with the ground within the area of effect. At the end of the duration, the sand vanishes, leaving no aftereffects (other than damage dealt).}
        
\DeclareSpell{Cleansing Fire}{evocation () [good,  fire]|V,  S,  DF|1 standard action|touch|Area30-ft. radius emanating from the touched point|concentration, up to 1 round/level|none|yes}[]
    \DeclareSpellDescription{Cleansing Fire}{A ring of fire surrounds the area affected by the spell. The flames radiate heat, but not enough to damage adjacent creatures. The flames deal 2d6 points of fire damage to any creatures that pass through them (or 4d6 points of damage to undead creatures). If you manifest a portion of the ring in a square that contains a creature, that creature takes damage as if it passed through the ring. The ring of flames attempts to dispel all ongoing spell effects with the evil descriptor within it, targeting each spell effect once per round as per dispel magic. Any attempt to cast a spell with the evil descriptor inside the ring of flames is targeted by a}
        
\DeclareSpell{Sun's Disdain}{transmutation () [curseUM]|V,  S,  M (a piece of glass)/DF|1 standard action|touch|Targetscreature touched|permanent|Will negates|yes}[]
    \DeclareSpellDescription{Sun's Disdain}{You alter a creature so the touch of the sun is hostile to it. The cursed creature gains light blindness and is blinded for 1 round if exposed to bright light, such as sunlight or the daylight spell. The cursed creature is dazzled as long as it remains in areas of bright light, and it is treated as being particularly susceptible to sunlight for the purposes of resolving spells like searing light, sunbeam, and sunburst.}
        
\DeclareSpell{Sun's Disdain, Mass}{transmutation () [curseUM]|V,  S,  M (a piece of glass)/DF|1 standard action|long (400 ft. + 40 ft./level)|Targetsone creature/level, no two of which can be more than 30 ft. apart|permanent|Will negates|yes}[]
    \DeclareSpellDescription{Sun's Disdain, Mass}{This spell functions like sun's disdain, except that it can affect multiple creatures.}
        
\DeclareSpell{Sunstalker}{illusion (glamer) []|V,  S,  M/DF|1 standard action|touch|Targetscreature touched|1 round/level|Will negates (harmless)|yes}[]
    \DeclareSpellDescription{Sunstalker}{A touched creature becomes invisible (as per invisibility) as long as it is in direct sunlight (or bright light shed by a daylight spell). If an action the target creature takes causes the invisible creature to become visible but does not end the sunstalker spell's duration, the creature can use this spell's magic to become invisible again as a standard action as long as it stands in an area of sunlight (as defined above). The creature gains only concealment (20\% miss chance) in normal light.}
        
\DeclareSpell{Aldori Alacrity}{transmutation () []|V,  S,  M (ginger root shavings)|1 standard action|personal|Targetsyou|1 minute/level||}[]
    \DeclareSpellDescription{Aldori Alacrity}{This spell's energy quickens your steps, allowing fancy footwork. You gain a +10-foot enhancement bonus to your speed and a +1 dodge bonus to your AC. These bonuses do not stack with those granted by haste or similar effects. While under the effects of Aldori alacrity, you can take 5-foot steps in difficult terrain.}
        
\DeclareSpell{Contest Of Skill}{transmutation () [curseUM]|V,  S,  M (a tiny gong)|1 standard action|close (25 ft. plus 5 ft./2 levels)|Targetsone creature|1 minute/level|Will negates|yes}[]
    \DeclareSpellDescription{Contest Of Skill}{The randomness of critical hits has long vexed certain duelists in the Aldori school who consider luck to be an unwelcome addition to duels, which they view as contests of skill rather than chance. The creature targeted by this spell is infused with magical power, altering the odds of combat to weaken lucky strikes. Critical threats made by the affected creature automatically fail to confirm. Critical threats that automatically confirm due to class features or feats, such as the fighter's weapon mastery, are unaffected by this spell. In addition, when the initial target of this spell fails its Will save to resist contest of skill, it can select one additional creature within range of the spell to suffer the effects of the spell as well (Will negates). If this secondary target negates this spell with a successful save, the duration of contest of skill on the primary target is reduced to 1 round.}
        
\DeclareSpell{Deivon's Parry}{transmutation () []|V,  S|1 immediate action|personal|Targetsyou|instantaneous||}[]
    \DeclareSpellDescription{Deivon's Parry}{Originally designed by an Aldori magus who had long admired swashbucklers for their ability to deflect blows, Deivon's parry has swiftly gained popularity among bards and magi alike. Your reflexes snap into action, allowing you to deflect a blow. This spell can be cast only when an opponent makes a melee attack against you, and only before the attack roll has been made. Make a parry attempt with a light or one-handed weapon you're wielding, as if using the swashbuckler's opportune parry and riposte deed (except that this does not require panache). You cannot riposte as part of this spell's effect.}
        
\DeclareSpell{Tactical Adaptation}{transmutation () []|V,  S,  F (a gold and sapphire ring worth 500 gp,  which the caster must wear for the spell's duration)|1 standard action|personal|Targetsyou|1 minute/level||}[]
    \DeclareSpellDescription{Tactical Adaptation}{This spell was developed by an Aldori Academy student after several weeks of observing (and occasionally joining) Restov barroom brawls. You draw on your mind's hidden reserves, instantaneously mastering advanced combat techniques. When you cast this spell, choose a combat feat. You must meet all prerequisites for this feat, treating your magus level as your base attack bonus for this purpose. For the duration of this spell, you are treated as if you had the chosen feat. Once you cast this spell, you cannot change the chosen feat (or any decisions related to that feat) for additional castings of this spell on the same day. A creature can benefit from only one tactical adaptation spell at a time.}
        
\DeclareSpell{Tieldlara's Feint}{enchantment (charm) [mind-affecting]|V,  S|1 standard action|20 ft.|Targetsone creature|1 round/level|Will negates|yes}[]
    \DeclareSpellDescription{Tieldlara's Feint}{With a lascivious wink and a sensuous sway, you flood a foe's mind with lustful thoughts of you, distracting your target and rendering that creature vulnerable to your attacks. An affected creature takes a penalty equal to half your caster level on concentration checks and to its CMD against dirty trickAPG, grapple, and stealAPG combat maneuvers you perform. The creature is also denied its Dexterity bonus to AC for the first melee attack you make against it each round. If the target would not normally be sexually attracted to you, it receives a +4 bonus on its saving throw. If you end your turn more than 20 feet from the target or end your turn where your target cannot see you, this spell immediately ends.}
        
\DeclareSpell{Haunting Reminder}{enchantment () [fear,  mind-affecting]|V,  S|1 standard action|close (25 ft. + 5 ft./2 levels)|Targetsone creature|12 hours + 2 hours/level or until triggered, then 1 day/level|Will negates|yes}[]
    \DeclareSpellDescription{Haunting Reminder}{As part of casting this spell, you can attempt a single Intimidate check to make the target act friendly to you. If you succeed, the target assists you normally, but it remains fearful of your retribution even after its attitude shifts to unfriendly. If the creature reports your coercion to authorities, attacks you, or otherwise acts in a purposeful way that threatens you or your objectives (at the GM's discretion), the spell triggers, inflicting the creature with the unshakable certainty that you will find and punish it. This imposes the shaken condition on the creature for 1 day per caster level; during this time, the creature takes a -2 penalty on saving throws against spells you cast with the fear descriptor.}
        
\DeclareSpell{Obscured Script}{illusion (phantasm) []|V,  S,  M (a scrap of paper bearing a simple sentence written in at least five languages)|1 standard action|touch|Targetsone touched page, scroll, inscription, book, or other document no more than 3 feet in any dimension|24 hours (D)|none; see text|no}[]
    \DeclareSpellDescription{Obscured Script}{You create a latent illusion that clouds the judgment and reading ability of any creature that examines a targeted text. While casting the spell, you can designate up to one creature other than yourself per level to be immune to the effects. All other creatures that read the text must succeed at a Will save or find it difficult to properly parse and decipher the contents. This increases the DC of Linguistics checks to decipher the text by an amount equal to your caster level (maximum +10), and it increases the DC of the Wisdom check to avoid drawing a false conclusion by an equal amount. Unaffected creatures gain a circumstance bonus equal to your caster level (maximum +10) on Bluff checks to deceive affected creatures about the text's contents. The spell obscures any magical glyphs, runes, or symbols within the text, increasing the Perception DC to find such traps by an amount equal to half your caster level (maximum +5). Obscured script can be made permanent with a permanency spell by a caster of 10th level or higher for the cost of 5,000 gp.}
        
\DeclareSpell{Sympathetic Aura}{illusion (glamer) []|V,  S,  F (a small square of silk that must be passed over the object that receives the aura)|1 standard action|touch|Targetsone touched object weighing up to 5 lbs./level|1 day/level (D)|none (see text)|no}[]
    \DeclareSpellDescription{Sympathetic Aura}{This spell functions like magic aura, except it extends its altered aura to similar items in a 5-foot radius. For this purpose, similar items must be alike in approximate shape, composition, and function, so a potion of cure light wounds targeted by this spell to appear nonmagical would also obscure the auras of nearby potions, elixirs, and oils, as well as flasks. Items other than the targeted item retain their altered aura for 1 round per caster level after being removed from the spell's area.}
        
\DeclareSpell{Absorb Rune I}{abjuration () []|V,  S,  M (a white silk glove worth 25 gp)|1 standard action|touch|Targetsone spell effect|instantaneous plus 1 minute/ level (see text)|Will negates (harmless)|yes (harmless)}[]
    \DeclareSpellDescription{Absorb Rune I}{This spell allows you to lay your hand upon a magical glyph, symbol, or other magical spell effect (referred to in this spell description as a "rune") and attempt to absorb the essence of its effect. To absorb a rune, you must be aware of the rune's existence (but need not know the details of what it actually does) and then succeed at a caster level check (DC = 10 + the caster level of the spell affect being absorbed) as you touch the rune in question. If you fail this caster level check, the magical rune is not triggered unless you fail the roll by 5 or more. If you succeed at the caster level check, the rune is removed from the surface it was originally placed upon and duplicated on the cloth of a silk glove worn on your hand. The rune remains located on the glove's palm in an inert state for up to 1 minute per caster level. As a standard action taken at any time during that duration, you can transfer the rune to another surface similar to the one it was originally placed upon, at which point the rune's function either returns to normal or dissipates harmlessly as if successfully dispelled (your choice). If the spell's duration expires before you place the rune on a new surface, the absorbed rune dissipates harmlessly. Absorb rune I affects only runes whose effects are equivalent to a spell of 3rd level or lower. An attempt to use absorb rune I on a more powerful effect automatically triggers the rune when you touch it.}
        
\DeclareSpell{Absorb Rune II}{abjuration () []|V,  S,  M (a white silk glove worth 25 gp)|1 standard action|touch|Targetsone spell effect|instantaneous plus 1 minute/ level (see text)|Will negates (harmless)|yes (harmless)}[]
    \DeclareSpellDescription{Absorb Rune II}{This spell functions like absorb rune I, but it can affect runes equivalent to a spell of 5th level or lower.}
        
\DeclareSpell{Absorb Rune III}{abjuration () []|V,  S,  M (a white silk glove worth 25 gp)|1 standard action|touch|Targetsone spell effect|instantaneous plus 1 minute/ level (see text)|Will negates (harmless)|yes (harmless)}[]
    \DeclareSpellDescription{Absorb Rune III}{This spell functions like absorb rune I, but it can affect runes equivalent to a spell of 8th level or lower.}
        
\DeclareSpell{Hidden Knowledge}{transmutation () []|V,  S,  M (a drop of ink)|1 round|personal|Targetsyou|up to 1 day/level (see text)||}[]
    \DeclareSpellDescription{Hidden Knowledge}{This subtle but useful spell allows you to safeguard important knowledge-even from yourself. While casting this spell, you recite one piece of knowledge (up to a maximum of 50 words). Upon completion of the spell's casting, you transfer the knowledge from your mind to your skin in the form of an intricate, runic tattoo placed anywhere you choose on your body. The knowledge disappears utterly from your mind, and you might not realize you forgot something. The magic of the spell patches over gaps in your memory with recollections from the past. Until the spell's duration ends, the knowledge is lost to you. When you cast this spell, you decide how long you wish the spell's duration to be, up to a maximum duration of 1 day per level. Many Cyphermages commission nonmagical tattoos to disguise the effects of this spell. A detect magic spell or a successful Linguistics or Spellcraft check (DC 20 + your Intelligence modifier) reveals an enchanted tattoo, but not its contents. The effects of hidden knowledge can be dispelled normally, in which case the knowledge is completely lost.}
        
\DeclareSpell{Rune Trace}{divination () []|V,  S,  M (pinch of powdered gemstones worth 25 gp)|1 minute|touch|Targetsrune touched|instantaneous|none|no}[]
    \DeclareSpellDescription{Rune Trace}{By immersing yourself fully in the intricacies of a carved or written rune of any kind, you can divine the elements of that rune's nature. While casting rune trace, you must run your fingers (which cannot be gloved at the time) over the rune, glyph, symbol, or other marking you want to examine (hereafter referred to as the "rune"). This does not trigger any effects that touching the rune would normally trigger. Runes, symbols, and other effects that trigger when read still trigger as normal if you do so- but note that this spell does not require you to view and read the rune that you're targeting. When the spell's casting time ends, you instantly receive flashes of insight regarding the rune's nature, history, and purpose, including the following information. Age: You learn if the rune was placed within the last 24 hours, within the last month, within the last year, within the last decade, within the last century, or prior to the last century.  Insight: If you have cast rune trace on a magical rune, you gain a +5 bonus on all skill rolls and checks to dispel, disable, or otherwise tamper with the rune.  Language: You learn what language the character of the rune is taken from, or in the case of a rune that has no language, you learn that it is a unique image.  Purpose: You learn the general purpose of the rune (such as whether it's a decoration, information, a magical defense, or a warning).}
        
\DeclareSpell{Rune Of Rule}{transmutation () []|V,  S,  M (vial of paint worth 25 gp)|1 minute|close (25 ft. + 5 ft./2 levels)|Targetsone living creature|1 day or until activated (see description)|Will (harmless)|yes (harmless)}[]
    \DeclareSpellDescription{Rune Of Rule}{Investigations into the ancient and mysterious traditions of the seven virtues of rule of the lost human empire of Azlant, which were later corrupted by the runelords into the more familiar seven deadly sins, inspired the Cyphermages to develop the runes of rule. This spell allows you to place a rune upon another creature that can then be used to aid it at a later time. You determine the spell's effect at the time of casting by using your finger to paint a specific rune on the recipient's body with a dose of specially prepared paint worth 25 gp. The inscribed rune lasts for 24 hours or until the spell is activated. Unless otherwise noted, the creature upon which the rune of rule has been inscribed can activate it at any time as a swift action. If the spell effect isn't used, all markings associated with the rune disappear and the effect fades. You can never place the spell upon yourself-it must be bestowed on someone else. The seven runes of rule, along with their specific effects when the user activates them, are listed below. A creature can bear only one rune of rule at a time. Charity: A creature bearing this mark gains a +5 insight bonus on a single attack roll or skill check attempted at the request of another creature, provided the creature bearing this mark does not gain any immediate benefit or reward for making the roll or attempting the check. Generosity: A creature bearing the rune of generosity can activate it as an immediate action, but it must do so as another creature within 30 feet activates a consumable magic item such as a potion or scroll that was given to it at some point within the last 24 hours by the creature bearing the rune. When the rune is activated, the effects of the consumable magic item resolve at a caster level that is 2 higher than the item's actual caster level. Humility: When a creature activates the rune of humility, it does not provoke attacks of opportunity for 1 round. Kindness: A creature must be using the aid another action or casting a healing spell in order to activate a rune of kindness. If the creature activates the rune while using the aid another action, the bonus imparted on a success increases to +5. If the creature instead activates the rune while casting a healing spell, the effective caster level of the spell increases by 2. Love: When a creature activates the rune of love as he casts a spell with the charm descriptor, the save DC of that spell increases by 1. Alternatively, a creature can activate the rune of love after he rolls damage for a weapon or spell attack, causing the damage dealt to become nonlethal damage. Temperance: A creature bearing the rune of temperance can activate it as an immediate action immediately upon failing a saving throw against a poison, disease, drug, or similar effect. The creature can immediately attempt a second saving throw against the effect and can use the result of that second saving throw as the actual result. Zeal: A creature that activates the rune of zeal gains a +3 bonus on Will saves, and the save DCs of all language-dependent effects created by the creature increase by 1; these effects last for 1 round.}
        
\DeclareSpell{Slave To Sin}{enchantment (compulsion) [emotionUM,  mind-affecting]|V,  S,  M/DF (a scrap of paper inscribed with the Sihedron rune)|1 standard action|close (25 ft. + 5 ft./2 levels)|Targetsone living creature|1 round/level|Will negates (see below)|yes}[]
    \DeclareSpellDescription{Slave To Sin}{This spell allows you to reach into an evil creature's mind to expose and exploit its susceptibility to whichever of the seven so-called "deadly sins" it is most susceptible to. The targeted creature must succeed at a Will save or it is overwhelmed with an inability to repress urges to indulge in whatever sin most closely mirrors its personality, and a glowing rune appears upon the creature's body (usually the brow), identifying the sin. (The GM adjudicates which rune manifests on the creature's body, as appropriate.) Each round, at the start of its turn, the target must attempt a new Will saving throw against the spell's DC to resist the sin's enslavement. If the target succeeds at this saving throw, it is sickened for that round by the distractions of its sin. If it fails, it is sickened and staggered as it spends part of its action wallowing in its targeted sin (a wrathful victim might waste time spouting threats and profanity, while a slothful victim might merely be slow and hesitant, and a lustful victim could well spend precious moments ogling or fawning over something of beauty).}
        
\DeclareSpell{Detect Charm}{divination () []|V,  S|1 standard action|60 ft.|Areacone-shaped emanation|concentration, up to 1 minute/level (D)|none|no}[]
    \DeclareSpellDescription{Detect Charm}{This spell functions as per detect magic, except that it detects only charm, compulsion, and possession effects. You immediately detect the strength and location of each such aura on all creatures in the area. You can attempt to identify the properties of each aura (see Spellcraft on page 106 the Pathfinder RPG Core Rulebook). In addition to noticing the targets of these effects, you can recognize when creatures in the area are using these effects on others by attempting a Sense Motive check as a standard action (DC = 20 + caster level). If you succeed, you can attempt a Spellcraft check to identify what magic it is using (even if the target is not in the area).}
        
\DeclareSpell{Summon Flight Of Eagles}{conjuration (summoning) []|V,  S,  F (a gold feather worth 100 gp)|1 round|close (25 ft. + 5 ft./2 levels)|Effect1d4+1 summoned creatures|10 minutes/level|none|no}[]
    \DeclareSpellDescription{Summon Flight Of Eagles}{You summon 1d4+1 giant eagles to serve as you designate. The summoned birds can fight if you wish, but can also serve as mounts.}
        
\DeclareSpell{Suppress Charms And Compulsions}{abjuration () []|V,  S|1 standard action|close (25 ft. + 5 ft./2 levels)|Targetsone creature plus one additional creature per 4 levels, no two of which can be more than 30 ft. apart|10 minutes or concentration (up to 1 round/level); see text|Will negates (harmless)|yes (harmless)}[]
    \DeclareSpellDescription{Suppress Charms And Compulsions}{You bolster the subject's sense of willpower and self-worth when you cast this spell. As you cast it, you must decide if you want to grant a bonus to saving throws against charms and compulsions or suppress charms and compulsions. If you grant a bonus to saving throws, you grant all affected creatures a +4 morale bonus on saving throws against charm and compulsion effects for 10 minutes. If instead you suppress charms and compulsions, the spell's duration drops to concentration, to a maximum duration of 1 round per level. As long as you continue to concentrate, the spell suppresses all existing charm and compulsion effects affecting the targets, regardless of whether the effect is beneficial or harmful. New charm or compulsion effects that successfully target such a protected creature are automatically suppressed as long as you continue concentrating. If you cease concentrating, the spell effect immediately ends, and remaining charm or compulsion effects resume for the rest of their remaining durations as normal.}
        
\DeclareSpell{Brand Of Conformity}{transmutation () [curseUM]|V,  S,  DF|1 round|touch|Targetscreature touched|1 day/level|Fortitude negates|yes}[]
    \DeclareSpellDescription{Brand Of Conformity}{This spell etches the symbol of a nation, organization, or order on the target, dealing 1 point of damage. The mark can be placed on any exposed portion of the creature, typically the head or forearm. While branded in this way, the recipient loses the ability to speak or understand its racial language and the language of its homeland or primary culture (if these are different). For example, an elf raised among the Shoanti would lose Elven and Shoanti (if she knew both those languages), while a halfling raised among dwarves would lose both Halfling and Dwarven (if she knew both those languages). A gnome raised among gnomes would lose Gnome, but no other language. This spell never suppresses Common or Undercommon, nor does it prevent spellcasting, though it may prevent a target from speaking intelligibly if it suppresses all of the target's languages. A brand of conformity can be hidden beneath clothing or removed by scraping it away (the latter deals 1d6 points of damage, though the brand returns if that damage is healed). In any event, the language-suppression element of a brand of conformity continues to function even when hidden or scraped away.}
        
\DeclareSpell{Brand Of Hobbling}{transmutation () [curseUM]|V,  S,  DF|1 round|touch|Targetscreature touched|1 day/level|Fortitude negates|yes}[]
    \DeclareSpellDescription{Brand Of Hobbling}{This spell etches a symbol of chains, a particular prison, or a lawful institution on the target, dealing 1d6 points of damage. The mark can be placed on any exposed portion of the creature, typically on the head or forearm. While the recipient is branded in this way, all of its movement speeds are reduced by half (rounded down to the next 5-foot increment). A brand of hobbling can be hidden by clothing or removed by scraping the brand away (the latter deals 1d6 points of damage, though the brand returns if that damage is healed), but in either case the effects of a brand of hobbling continue to function.}
        
\DeclareSpell{Brand Of Tracking}{transmutation () [curseUM]|V,  S,  DF|1 round|touch|Targetscreature touched|permanent|Fortitude negates|yes}[]
    \DeclareSpellDescription{Brand Of Tracking}{This spell etches the symbol of an eye or a lawful institution on the target, dealing 1d6 points of damage. The mark can be placed on any exposed portion of the creature, typically the head or forearm. While the target is branded in this way, twice per day as a standard action, you can instantly determine in what direction and how far away the target is, as long as it is within a range equal to 1 mile per your caster level. If the target travels beyond this range or to another plane, you cannot gain information about the recipient's location. Similarly, if the target becomes warded by a spell or effect that thwarts divination spells (such as nondetection) or moves into such an area (such as one created by antimagic field), you cannot gain information about the target's location. This effect otherwise works as a locate creature spell that lasts for 1 minute per caster level. It can be fooled by mislead but not by polymorph. A brand of tracking can be hidden by clothing or temporarily removed by scraping it away (the latter deals 1d6 points of damage, though the brand returns if that damage is healed, or 1 day later otherwise). The effects of a brand of tracking continue to function even when hidden or scraped away. A brand of tracking cannot be dispelled, but it can be permanently removed by any means that removes a mark of justice (see page 312 of the Pathfinder RPG Core Rulebook).}
        
\DeclareSpell{Dirge Of The Victorious Knights}{illusion (shadow) []|V,  S,  F (a medal from a dead Hellknight or a copy of a Chelish opera script,  either worth at least 100 gp)|1 round|120 ft.|Effect120-ft. line, 10 ft. wide|instantaneous|Reflex half|yes}[]
    \DeclareSpellDescription{Dirge Of The Victorious Knights}{By performing part of the Chelish opera Victory of the Hellknights, you call forth spectral illusions of mounted Hellknights to trample your foes under the hooves of their glorious steeds. The shadowy knights appear in an adjacent 10-foot square and ride forward in the direction you indicate, dealing 1d6 points of damage per caster level (maximum 20d6) to all creatures in their path. Half of this damage is cold damage, while half results directly from arcane power and is not subject to cold resistance or immunity. The knights cannot pass through force effects or barriers that block incorporeal creatures or undead. I}
        
\DeclareSpell{Infernal Challenger}{conjuration (calling) [lawful; see text]|V,  S,  F/DF (an iron badge or medallion)|1 round|close (25 ft. + 5 ft./2 levels)|Effectone called bearded devil and testing ground; see text|1 minute/level|none|no}[]
    \DeclareSpellDescription{Infernal Challenger}{This spell summons a bearded devil, causing it to appear where you designate. The area within a 50-foot radius of where the devil appears is ringed with a smoldering, red glow. This area is the testing ground. To successfully cast this spell, no creature other than the devil's challenger (see below) can be within this area while the spell is being cast. On the round it appears, the devil identifies itself (typically by name and with a brief recitation of honors) and states it is prepared to do battle with a sole mortal champion. In the same round, either you or a creature you designate while casting the spell must identify the devil's challenger. For the duration of the spell, the devil does everything in its power to kill the challenger. You do not control the devil or have any influence over how it conducts itself in battle, but the devil cannot leave the testing ground. Unlike with summon monster and similar spells, you cannot dismiss the devil. Rather, the conjured devil remains until any one of the following criteria is met: it is reduced below 0 hit points or otherwise defeated, its challenger is slain, its challenger leaves the testing ground, or it takes damage from any source other than its challenger. Upon any of these occurrences, the devil vanishes. Conjuring a devil is typically an evil act. If cast for any purpose besides the administering of a Hellknight test, this spell has the evil descriptor.}
        
\DeclareSpell{Shackle}{conjuration (creation) []|V,  S|1 standard action|touch|Effectone set of restraints; see text|1 hour/level|Reflex negates; see text|no}[]
    \DeclareSpellDescription{Shackle}{You summon a set of Small or Medium masterwork restraints into being. When you cast this spell, you can choose whether the restraints are manacles or fetters (manacles specifically designed to fit around the ankles). You also summon the restraints' key to your person; the Disable Device DC to open the locked restraints is equal to 15 + your caster level + the modifier of your primary casting ability score. Typically, the restraints appear in your hands. However, as part of the spell's casting, you can make a melee touch attack against a creature; if you succeed and the creature then fails a Reflex saving throw against the spell's DC, the locked fetters appear clasped on the creature or locked manacles clasp you and the creature together. You can instead have manacles restrain the target's limbs in front of or behind it. If you cast the spell as a full-round action, you can make melee touch attacks against two creatures you can reach. If you succeed at both attacks and both creatures fail their Reflex saving throws, you can cause the set of manacles to bind the targets together. As you increase in level, the restraints become stronger and you gain more control over them. If you are 6th level or higher, you can make the restraints mithral, or you can summon Tiny or Large restraints. If you are 12th level or higher, you can summon Diminutive or Huge restraints. If you use an emerald worth at least 1,000 gp as a material component while casting this spell, as a standard action once during its duration, you can affect any creature bound by these restraints as per dimensional anchor with a duration of 1 minute per caster level. If you are 18th level or higher and use an emerald worth at least 1,000 gp as a material component while casting this spell, the restraints' hardness increases to 30, they have 60 hit points, and they can't be broken with a successful Strength check. (The emerald must be worth at least 2,000 gp if you also wish to use the dimensional anchor effect described earlier.) If you cast this spell before the duration of a previous casting has lapsed, you can create a new set of restraints or reset the previous spell's duration. If you used an expensive material component during a previous casting, you must again use a component when creating new restraints or resetting the duration if you wish to maintain the special effect. You do not need to touch these previously created shackles to renew their duration, though you must be on the same plane as the shackles. When this spell ends, the restraints disappear, and any ongoing effects created by the spell end.}
        
\DeclareSpell{Brightest Light}{evocation () [light]|V,  S|1 standard action|touch|Targetsobject touched|1 hour/level (D)|none|no}[]
    \DeclareSpellDescription{Brightest Light}{This trademark spell of the Lantern Bearers functions as daylight, except it lasts longer. In addition, as a swift action once during the spell's duration, you can will the light to try to end a magical darkness effect located within 60 feet of the light this spell emits. Attempt a dispel check (1d20 + your caster level), with a DC equal to 11 + the caster level of the darkness effect. If you succeed, the darkness effect ends. Regardless of whether you are successful, the light from this spell dims to the brightness of a torch for the spell's remaining duration.}
        
\DeclareSpell{Detoxify}{transmutation () []|V,  S,  M (a dandelion stem)|1 standard action|close (25 ft. + 5 ft./2 levels)|Targetsone creature|10 minutes/level|Fortitude negates|yes}[]
    \DeclareSpellDescription{Detoxify}{You remove a creature's ability to poison others, whether inherent or via poisoned weapons. Whenever an affected creature would inflict poison with an attack, spell, or other method (including auras and other constant methods), that poison is automatically neutralized. This does not grant the subject of the spell any resistance to poison itself.}
        
\DeclareSpell{Preserve Grace}{divination () [good]|V,  S,  F (a silver hand mirror worth 50 gp)|1 minute|touch|Targetsone good-aligned creature|1 day|Will negates (harmless)|no}[]
    \DeclareSpellDescription{Preserve Grace}{The Lantern Bearers must often make difficult decisions during their missions, such as determining what to do with captured enemies-should they be executed, imprisoned, set free, or given a chance to mend their ways? With this spell, you grant a creature the ability to foresee moral consequences, as if the creature were wearing a phylactery of faithfulness. The spell's guidance focuses on the protection of beauty and encourages peaceful resolution and mercy.}
        
\DeclareSpell{Mask From Divination}{divination () []|V,  S,  F (an eyeless mask),  M (diamond dust worth 200 gp)|1 standard action|touch|Targetscreature touched|24 hours|Will negates (harmless, object)|yes (harmless, object)}[]
    \DeclareSpellDescription{Mask From Divination}{As part of the action used to cast this spell, you place the mask used as the focus component on the target's face, after which it adheres tightly to the target for the spell's duration and cannot be removed by physical force. Despite lacking eyes, this mask does not impair its wearer's vision in any way. While in effect, this spell functions like nondetection, except it also foils divination spells that attempt to gather information about the creature, even if they don't target it specifically. In the case of divination spells that would normally reveal the wearer's presence, such as see invisibility, the spell works, but the wearer is detected only if the caster succeeds at a caster level check. Likewise, scrying attempts that specifically target the wearer do not work at all unless the caster succeeds at a caster level check. In addition, the DCs of skill checks to learn about the wearer or identify the wearer and its strengths and weaknesses- including Diplomacy checks to gather information, Perception checks opposed by the target's Disguise checks, and Knowledge checks-increase by 4. The mask cannot be removed for the duration of the spell, and the fact that the target is wearing the focus mask cannot be hidden in any way from creatures that observe the target.}
        
\DeclareSpell{Planar Inquiry}{conjuration (calling) []|V,  S,  M (offerings worth 100 gp per HD of creature called)|10 minutes|close (25 ft. + 5 ft./2 levels)|Effectone called outsider who answers questions|instantaneous; see text|none|no}[]
    \DeclareSpellDescription{Planar Inquiry}{Although he was hardly the first to turn to the Outer Planes for answers, Jatembe's dealings with outsiders in his pursuit of enlightenment are legendary, and the Magaambya credits the Old-Mage with the creation of this spell. This spell calls a creature from another plane to your precise location, functioning like lesser planar ally except as noted. When you call a creature using planar inquiry, the only task that you can ask of the creature is for it to answer questions or gather information regarding a specific topic (a person, a place, or a thing). After hearing your request, if the creature has an appropriate Knowledge skill, it can attempt a check to provide the information it has. If it lacks such a skill, the called creature leaves for 1d4 hours to gather this information. Upon its return, you roll 1d20 + your caster level, and use the result to determine what information the creature has gathered about the subject (as if using Diplomacy). The called creature stays for up to 10 minutes as it relays this information to you, after which it departs to its home plane. If the creature is attacked or damaged at any time during the spell's duration, the spell ends and the creature returns to the plane from which you summoned it. When you cast this spell, you can choose a specific kind of outsider to call, even calling an individual creature by name. The kind of outsider called doesn't alter the effects of the spell, but when you use planar inquiry to summon a creature with an alignment or elemental subtype, the spell gains that descriptor. You cannot call an outsider whose Hit Dice exceed your caster level (maximum 18 HD) and you cannot use this spell to contact a unique outsider (such as a deity's herald) or an outsider with mythic ranks.}
        
\DeclareSpell{Frost Mammoth}{conjuration (creation) [cold]|V,  S,  M (a fragment of mammoth tusk)|1 round|close (25 ft. + 5 ft./2 levels)|Effectone frost mammoth|1 round/level (D)|none|no}[]
    \DeclareSpellDescription{Frost Mammoth}{A blast of snow suddenly fills an area with a space of 15 feet, immediately taking the shape of a woolly mammoth made of snow with tusks of solid ice. The mammoth has statistics identical to those of a mastodon (Pathfinder RPG Bestiary 128), except it also has the cold subtype (and thus gains immunity to cold and vulnerability to fire). The frost mammoth obeys your telepathic commands. It allows you or anyone you designate to ride it, and it is treated as if combat trained. At 17th level, a frost mammoth you conjure deals an additional 1d6 points of cold damage with each physical attack.}
        
\DeclareSpell{Invoke Primal Power}{transmutation () []|V,  S|1 swift action|personal|Targetsyou|1 round/level|none|no}[]
    \DeclareSpellDescription{Invoke Primal Power}{This spell must be cast as you activate wild shape to assume the form of an animal of at least Large size. When you assume the animal form, it takes on many of the primal characteristics of the savage, prehistoric creatures that dwell in the Realm of the Mammoth Lords. When this spell's duration expires, you return to your natural form automatically-in effect, you shorten the duration of that use of wild shape to the duration of this spell when you cast invoke primal power. In addition to the normal effects you gain upon assuming the wild shape, you also gain a +4 size bonus to your Constitution score and increase the natural armor bonus granted by the effect by 2 (thus, if you assume the size of a Large animal as if via beast shape II, your natural armor bonus is +6 instead of +4). Furthermore, you gain one of the following additional abilities while you are in your enhanced wild shape form. At caster level 13th, you gain two of the following abilities, and at caster level 17th, you gain three of the abilities.  Cold Resistance: Your new body is covered with thick layers of fur and insulating blubber. You gain cold resistance 30.  Ferocity: You remain conscious and can continue to fight even if your hit point total is below 0. You are still staggered and lose 1 hit point each round, and still die when your hit point total reaches a negative amount equal to your Constitution score.  Giant Slayer: Your new form is particularly suited for slaying giants. Against creatures with the giant subtype, you gain a +2 insight bonus on attack rolls and on damage rolls with your new form's natural attacks.  Powerful Charge: Choose one of the natural attacks you gain from your new form. When you make a charge attack, this natural attack deals 2d6 additional points of damage in addition to the normal benefits and hazards of a charge. This damage applies only to the first attack you make when you perform your charge.  Rend: Once per round, if you hit the same target with two or more natural attacks in a round, you rend the target of those attacks, dealing 2d6 additional points of damage.  Swift: Your new form's base speed increases by 20 feet.  Trample: As a full-round action, you can attempt to overrun any creature that is at least one size category smaller than you. This works just like the overrun combat maneuver, but you do not need to attempt a check; you merely have to move over your opponents. Targets take damage equal to 2d6 plus 1-1/2 times your Strength modifier from your trample, and can make attacks of opportunity against you, but at a -4 penalty. If a targets forgoes an attack of opportunity, it can attempt a Reflex save to avoid the trample and take half damage. The save DC against this effect is equal to 10 + half your caster level + your Strength modifier. You can deal only trampling damage to each target once per round, no matter how many times your movement takes you over a target creature.  Trip: Choose one of the natural attacks granted to you by your new form. Up to once per round, you can attempt to trip an opponent as a free action without provoking an attack of opportunity if you hit with the specified attack. If the attempt fails, you are not tripped in return.}
        
\DeclareSpell{Bone Flense}{transmutation () []|M/F (a jagged shard of bone from a humanoid creature)|1 standard action|close (25 ft. + 5 ft./2 levels)|Targetsone living creature|1 round/level|Fortitude negates|yes}[]
    \DeclareSpellDescription{Bone Flense}{Choose a creature that you can see. When that creature is struck by a weapon wielded by a member of the Red Mantis, a sawtooth sabre, or by the claw of a giant mantis, the bone nearest to the wound instantly sprouts jagged, razor-sharp spurs that flense the muscle and flesh from the inside out unless the target succeeds at a Fortitude save. The creature takes 1d6 points of piercing damage per caster level. While the bone returns to normal immediately after dealing damage, the creature takes 1d4 points of bleed damage for 1 round per 2 caster levels. The victim is sickened by the pain for the duration of the bleeding. If the target creature has no bones, this spell has no effect.}
        
\DeclareSpell{Crimson Breath}{transmutation () []|S|1 standard action|close (25 ft. + 5 ft./2 levels)|Targetsone creature|instantaneous|Fortitude negates|no}[]
    \DeclareSpellDescription{Crimson Breath}{When you cast this spell, your salivary glands transform and instantly fill with potent venom. Make a ranged touch attack against a single target in range; if you hit, the target is affected by a dose of breath of the mantis god (see page 158). The Fortitude save DC for this poison is the same as the spell's DC, not the standard save DC for the poison.}
        
\DeclareSpell{Mark Of Blood}{transmutation () [curseUM]|V,  S,  M (a drop of your blood)|1 standard action|touch|Targetsone weapon touched|1 minute and permanent (see text)|Will negates|yes}[]
    \DeclareSpellDescription{Mark Of Blood}{You place a drop of your blood on a weapon and charge it with magic so that you transfer a small amount of your life essence to the next living creature you strike with the weapon. Thereafter, you can spend a move action to know the direction and general distance of that creature. The target can negate this effect with a successful Will save. You must strike a creature within 1 minute of casting this spell or the magic is wasted, but once the mark of blood takes effect it is permanent until dispelled or removed via an effect like remove curse.}
        
\DeclareSpell{Mirror Mantis}{illusion (phantasm) [mind-affecting]|S,  F (a hand mirror stained with a dried drop of blood)|1 standard action|medium (100 ft. + 10 ft./level)|Targetsone living creature|1 hour/level|Will negates|yes}[]
    \DeclareSpellDescription{Mirror Mantis}{Whenever the target of this spell looks into a mirror or other reflective surface where it can see its own visage clearly, it sees a Red Mantis assassin (in full armor, face hidden by a mantis mask) looking back at it. The reflection is harmless and mimics the target's motions perfectly; no one else can see this illusion. The first time the target sees the Red Mantis reflection, it must succeed at a Will save be or shaken and take a -2 penalty on saves against fear effects for as long as it can see the altered image and for 3 rounds thereafter. A new save must be attempted each time the creature views its reflection during the spell's duration. Once a creature successfully saves, it becomes immune to being shaken by the caster's mirror mantis spell for 24 hours, though it will still see the Red Mantis reflection until the spell ends.}
        
\DeclareSpell{Sarzari Shadow Memory}{divination () []|V,  S,  M (a work of art bearing your target's likeness worth at least 1, 500 gp)|1 hour|personal|Targetsyou|1 month|Will negates (harmless)|no (harmless)}[]
    \DeclareSpellDescription{Sarzari Shadow Memory}{While casting this spell, you slowly destroy a piece of art bearing your victim's likeness and beseech Achaekek to grant you access to the knowledge and lore contained within the innermost sanctum of the Crimson Citadel, the Sarzari Library. This endows you with information that could facilitate the target's assassination by your hands. This information must be chosen from one of the three following categories-back doors, character, or weakness. "Back doors" informs you of hidden passages and other means of navigation within your target's home. While inside your target's home, you gain a +10 insight bonus on Perception checks and gain the constant benefits of detect secret doors and find traps. You automatically see through any illusory walls in the target's home. "Character" lets you know of a compromising detail about your target's personal life, giving you a +10 bonus on Bluff, Diplomacy, and Intimidate checks when attempting to gather information about your contracted victim from other people. Your target takes a -2 penalty on all saving throws against mind-affecting effects you generate, and you gain a +4 bonus on caster level checks you attempt to overcome the target's spell resistance. "Weakness" grants you knowledge of physical vulnerabilities your target has, as well as information about the target's damage reduction, immunities, and other defensive abilities. If these vulnerabilities and defenses change during the spell's duration, you immediately know. You automatically confirm all critical threats against the target. Once your victim is slain or a month has passed (whichever comes first), the information you gained from the spell fades. While you can recall certain elements of these memories, you no longer gain any of the benefits associated with the spell. You can only have one target affected by Sarzari shadow memory at any one time; if you cast this spell a second time while a previous casting is still active, the effects of the new spell replace the effects of the old one.}
        
\DeclareSpell{Sawtooth Terrain}{transmutation () [earth]|V,  S,  DF|1 round|close (25 ft. + 5 ft./2 levels)|Effectone 10-foot square of difficult and damaging terrain|1 round/level|Reflex negates|yes}[]
    \DeclareSpellDescription{Sawtooth Terrain}{This spell targets an area of earth, metal, stone, or wood, causing long, serrated blades to shoot out from random points within the area of effect. Any creature in the area when the spell is first cast must attempt a Reflex save to avoid taking 3d8 points of piercing damage from the blades and an additional 2d6 points of bleed damage. A creature that fails the save also has its speed reduced by half for 24 hours or until the injured creature benefits from a "cure" spell (which restores lost hit points as normal). Another creature can remove the penalty by succeeding on a Heal check against the spell's save DC as a standard action. The lashing limbs remain active in the area for the remainder of the spell's duration, during which time the area is treated as difficult terrain, and any creature that moves into or through the area takes 1d8 points of piercing damage. The transformed area can be located on any visible surface within range, including floors, walls, doors, ceilings, or other generally flat surfaces.}
        
\DeclareSpell{Instant Portrait}{conjuration (creation) []|V,  S|1 standard action|touch|Effecta monochromatic illustration up to 1 square foot in area|instantaneous|none|no}[]
    \DeclareSpellDescription{Instant Portrait}{You touch a surface and produce a painted portrait of either yourself or a creature you can see without needing to attempt a Craft (painting) check. The surface to be affected must be relatively flat, such as a piece of paper or a wall. The image can be any color of your choice, but is monochromatic. While it is of too poor a quality to sell, it is otherwise accurate enough to recognize the subject or to serve as a target for the enter image spell. The portrait reflects your perception of the creature depicted, including any disguises, magical or mundane, that the subject is wearing at the time of the painting's creation. The portrait is only as durable as normal paint and can be removed by mundane means.}
        
\DeclareSpell{Wall Of Silver}{abjuration () [good]|V,  S|1 standard action|medium (100 ft. + 10 ft./level)|Effecttransparent wall 20 ft. high by up to 20 ft. long/level|1 round/level|see text|yes}[]
    \DeclareSpellDescription{Wall Of Silver}{You create a wall of translucent silver energy that hums and vibrates at the slightest touch. Objects and nonevil creatures can pass through this wall without difficulty. However, spells and effects with the evil descriptor treat this barrier as a wall of force, which blocks line of effect. Evil creatures that pass through the wall take 3d6 points of damage + 1 point of damage per caster level (maximum 3d6+20) and are blinded for 1 round. Creatures of any alignment that are particularly vulnerable to silver (such as those with damage reduction bypassed by silver, like devils or lycanthropes) instead take 1d6 points of damage per caster level (maximum 15d6), are staggered for 1 round, and are permanently blinded. A creature that succeeds at a Will save reduces the damage by half and negates the blinding and staggering effects. If you create a wall of silver so that it appears where creatures are, each creature takes damage as if passing through the wall. Each such creature can avoid the wall (ending up on the side of its choice) and thus take no damage by succeeding at a Reflex save.}
        
\DeclareSpell{Storm Sight}{divination (scrying) []|V,  S,  F (a storm,  either natural or magical)|1 round|personal|Targetsyou|1 minute/level (D)|Will negates (harmless)|yes (harmless)}[]
    \DeclareSpellDescription{Storm Sight}{You step into any magical or naturally occurring storm and commune directly with the power of the wind and rain. This spell allows the storm's power to fill you, granting you knowledge and images of what else lies within its reach. As long as you remain within the storm, you can concentrate on it for 1 minute to determine whether or not other living creatures with an Intelligence score of 3 or higher are caught in the storm within 400 feet of you. Due to the nature of the spell, you do not need to attempt concentration checks against violent wind or storms when casting the spell, but other elements that might disrupt spellcasting require checks as normal. While the spell is in effect, you no longer need to attempt concentration checks as a result of wind or storms for spells you cast or concentrate on. Once you've sensed creatures with this spell by concentrating for 1 minute, you can continue concentrating to learn more. After you concentrate for 1 additional round, the spell reveals to you via winds and rain the presence of any Small or larger living creature with an Intelligence of 3 or higher within 400 feet of you. After 2 consecutive rounds of concentration, you learn the approximate distance to the detected life forms from your current location. After 3 rounds of concentration, the rain and winds allow you to determine the approximate direction of each life form. Once you've concentrated in this way for 3 rounds, you gain a bonus equal to half your caster level on your next initiative check during the spell's duration.}
    
    
\DeclareSpell{Daemon Ward}{necromancy|V,  S,  M (50 gp worth of powdered silver)|1 standard action|touch|Targets: living creature touched|1 min./level|Will negates (harmless)|yes (harmless)}[]
    \DeclareSpellDescription{Daemon Ward}{This spell functions like death ward, except as noted above and it only protects against these attacks from daemons.\\\\

{\centering\bf Death Ward\hrule}

The subject gains a +4 morale bonus on saves against all death spells and magical death effects. The subject is granted a save to negate such effects even if one is not normally allowed. The subject is immune to energy drain and any negative energy effects, including channeled negative energy.

This spell does not remove negative levels that the subject has already gained, but it does remove the penalties from negative levels for the duration of its effect.

Death ward does not protect against other sorts of attacks, even if those attacks might be lethal.}
        
\DeclareSpell{Death Knell Aura}{necromancy [death,  evil]|V,  S|1 standard action|20 ft.|Area: 20-ft.-radius emanation, centered on you|1 round/level (D)|Will negates|yes}[]
    \DeclareSpellDescription{Death Knell Aura}{You gain a shimmering gray aura that feeds on the souls of creatures who die within it. The aura sheds light as a candle. If a creature at -1 or fewer hit points is within the aura at the start of its turn, it must save or die, granting you the benefits of death knell.}
        
\DeclareSpell{Greater Death Knell Aura}{necromancy [death,  evil]|V,  S|1 standard action|20 ft.|Area: 20-ft.-radius emanation, centered on you|1 round/level (D)|Will negates|yes}[]
    \DeclareSpellDescription{Greater Death Knell Aura}{This spell functions as death knell aura, except a dying creature in the area cannot stabilize on its own and automatically takes 1 point of bleed damage on its turn each round. An incorporeal undead or living spirit traveling outside the body (such as a person using astral projection or magic jar) within the aura at the start of its turn takes 1d8 points of damage.}
        
\DeclareSpell{Scourge Of The Horsemen}{necromancy [acid,  evil]|V,  S|1 standard action|medium (100 ft. + 10 ft./level)|Area: 30-ft. burst|instantaneous|Fortitude half|yes}[]
    \DeclareSpellDescription{Scourge Of The Horsemen}{This spell blasts the area with a horrific combination of soul-rending energy and physical corrosion. Creatures in the area of effect gain 1d4 negative levels, and take 1d6 points of acid damage per caster level (maximum 20d6).}
        
\DeclareSpell{Summon Cacodaemon}{conjuration (summoning) [evil]|V,  S,  F (a silver hook)|1 round|close (25 ft. + 5 ft./2 levels)|Effect: one summoned creature|1 round/level (D)|none|no}[]
    \DeclareSpellDescription{Summon Cacodaemon}{This spell functions like summon monster, except it summons a single cacodaemon.\\\\

{\centering\bf Summon Monster I\hrule}

This spell summons an extraplanar creature (typically an outsider, elemental, or magical beast native to another plane). It appears where you designate and acts immediately, on your turn. It attacks your opponents to the best of its ability. If you can communicate with the creature, you can direct it not to attack, to attack particular enemies, or to perform other actions. The spell conjures one of the creatures from the 1st Level list on Table 10-1. You choose which kind of creature to summon, and you can choose a different one each time you cast the spell.

A summoned monster cannot summon or otherwise conjure another creature, nor can it use any teleportation or planar travel abilities. Creatures cannot be summoned into an environment that cannot support them. Creatures summoned using this spell cannot use spells or spell-like abilities that duplicate spells with expensive material components (such as wish).

When you use a summoning spell to summon a creature with an alignment or elemental subtype, it is a spell of that type. Creatures on Table 10-1 marked with an "*" are summoned with the celestial template, if you are good, and the fiendish template, if you are evil. If you are neutral, you may choose which template to apply to the creature. Creatures marked with an "*" always have an alignment that matches yours, regardless of their usual alignment.

Summoning these creatures makes the summoning spell's type match your alignment.}
        
\DeclareSpell{Greater Summon Cacodaemon}{conjuration (summoning) [evil]|V,  S,  F (a silver hook)|1 round|close (25 ft. + 5 ft./2 levels)|Effect: one summoned creature|1 round/level (D)|none|no}[]
    \DeclareSpellDescription{Greater Summon Cacodaemon}{This spell functions like summon cacodaemon, except it summons 1d4+1 cacodaemons.\\\\

{\centering\bf Summon Cacodaemon\hrule}

This spell functions like summon monster, except it summons a single cacodaemon.\\\\

{\centering\bf Summon Monster I\hrule}

This spell summons an extraplanar creature (typically an outsider, elemental, or magical beast native to another plane). It appears where you designate and acts immediately, on your turn. It attacks your opponents to the best of its ability. If you can communicate with the creature, you can direct it not to attack, to attack particular enemies, or to perform other actions. The spell conjures one of the creatures from the 1st Level list on Table 10-1. You choose which kind of creature to summon, and you can choose a different one each time you cast the spell.

A summoned monster cannot summon or otherwise conjure another creature, nor can it use any teleportation or planar travel abilities. Creatures cannot be summoned into an environment that cannot support them. Creatures summoned using this spell cannot use spells or spell-like abilities that duplicate spells with expensive material components (such as wish).

When you use a summoning spell to summon a creature with an alignment or elemental subtype, it is a spell of that type. Creatures on Table 10-1 marked with an "*" are summoned with the celestial template, if you are good, and the fiendish template, if you are evil. If you are neutral, you may choose which template to apply to the creature. Creatures marked with an "*" always have an alignment that matches yours, regardless of their usual alignment.

Summoning these creatures makes the summoning spell's type match your alignment.}
        
\DeclareSpell{Summon Ceustodaemon}{conjuration (summoning) [evil]|V,  S,  F (ashes of a dead animal)|1 round|close (25 ft. + 5 ft./2 levels)|Effect: one summoned creature|1 round/level (D)|none|no}[]
    \DeclareSpellDescription{Summon Ceustodaemon}{This spell functions like summon monster, except it summons a single ceustodaemon.\\\\

{\centering\bf Summon Monster I\hrule}

This spell summons an extraplanar creature (typically an outsider, elemental, or magical beast native to another plane). It appears where you designate and acts immediately, on your turn. It attacks your opponents to the best of its ability. If you can communicate with the creature, you can direct it not to attack, to attack particular enemies, or to perform other actions. The spell conjures one of the creatures from the 1st Level list on Table 10-1. You choose which kind of creature to summon, and you can choose a different one each time you cast the spell.

A summoned monster cannot summon or otherwise conjure another creature, nor can it use any teleportation or planar travel abilities. Creatures cannot be summoned into an environment that cannot support them. Creatures summoned using this spell cannot use spells or spell-like abilities that duplicate spells with expensive material components (such as wish).

When you use a summoning spell to summon a creature with an alignment or elemental subtype, it is a spell of that type. Creatures on Table 10-1 marked with an "*" are summoned with the celestial template, if you are good, and the fiendish template, if you are evil. If you are neutral, you may choose which template to apply to the creature. Creatures marked with an "*" always have an alignment that matches yours, regardless of their usual alignment.

Summoning these creatures makes the summoning spell's type match your alignment.}
        
\DeclareSpell{Summon Derghodaemon}{conjuration (summoning) [evil]|V,  S,  F (a handful of bug carapaces)|1 round|close (25 ft. + 5 ft./2 levels)|Effect: one summoned creature|1 round/level (D)|none|no}[]
    \DeclareSpellDescription{Summon Derghodaemon}{This spell functions like summon monster, except it summons a single derghodaemon.\\\\

{\centering\bf Summon Monster I\hrule}

This spell summons an extraplanar creature (typically an outsider, elemental, or magical beast native to another plane). It appears where you designate and acts immediately, on your turn. It attacks your opponents to the best of its ability. If you can communicate with the creature, you can direct it not to attack, to attack particular enemies, or to perform other actions. The spell conjures one of the creatures from the 1st Level list on Table 10-1. You choose which kind of creature to summon, and you can choose a different one each time you cast the spell.

A summoned monster cannot summon or otherwise conjure another creature, nor can it use any teleportation or planar travel abilities. Creatures cannot be summoned into an environment that cannot support them. Creatures summoned using this spell cannot use spells or spell-like abilities that duplicate spells with expensive material components (such as wish).

When you use a summoning spell to summon a creature with an alignment or elemental subtype, it is a spell of that type. Creatures on Table 10-1 marked with an "*" are summoned with the celestial template, if you are good, and the fiendish template, if you are evil. If you are neutral, you may choose which template to apply to the creature. Creatures marked with an "*" always have an alignment that matches yours, regardless of their usual alignment.

Summoning these creatures makes the summoning spell's type match your alignment.}
        
\DeclareSpell{Summon Erodaemon}{conjuration (summoning) [evil]|V,  S,  F (a bent or tarnished wedding band)|1 round|close (25 ft. + 5 ft./2 levels)|Effect: one summoned creature|1 round/level (D)|none|no}[]
    \DeclareSpellDescription{Summon Erodaemon}{This spell functions like summon monster, except it summons a single erodaemon.\\\\

{\centering\bf Summon Monster I\hrule}

This spell summons an extraplanar creature (typically an outsider, elemental, or magical beast native to another plane). It appears where you designate and acts immediately, on your turn. It attacks your opponents to the best of its ability. If you can communicate with the creature, you can direct it not to attack, to attack particular enemies, or to perform other actions. The spell conjures one of the creatures from the 1st Level list on Table 10-1. You choose which kind of creature to summon, and you can choose a different one each time you cast the spell.

A summoned monster cannot summon or otherwise conjure another creature, nor can it use any teleportation or planar travel abilities. Creatures cannot be summoned into an environment that cannot support them. Creatures summoned using this spell cannot use spells or spell-like abilities that duplicate spells with expensive material components (such as wish).

When you use a summoning spell to summon a creature with an alignment or elemental subtype, it is a spell of that type. Creatures on Table 10-1 marked with an "*" are summoned with the celestial template, if you are good, and the fiendish template, if you are evil. If you are neutral, you may choose which template to apply to the creature. Creatures marked with an "*" always have an alignment that matches yours, regardless of their usual alignment.

Summoning these creatures makes the summoning spell's type match your alignment.}
        
\DeclareSpell{Summon Meladaemon}{conjuration (summoning) [evil]|V,  S,  F (an empty wooden bowl)|1 round|close (25 ft. + 5 ft./2 levels)|Effect: one summoned creature|1 round/level (D)|none|no}[]
    \DeclareSpellDescription{Summon Meladaemon}{This spell functions like summon monster, except it summons a single meladaemon.\\\\

{\centering\bf Summon Monster I\hrule}

This spell summons an extraplanar creature (typically an outsider, elemental, or magical beast native to another plane). It appears where you designate and acts immediately, on your turn. It attacks your opponents to the best of its ability. If you can communicate with the creature, you can direct it not to attack, to attack particular enemies, or to perform other actions. The spell conjures one of the creatures from the 1st Level list on Table 10-1. You choose which kind of creature to summon, and you can choose a different one each time you cast the spell.

A summoned monster cannot summon or otherwise conjure another creature, nor can it use any teleportation or planar travel abilities. Creatures cannot be summoned into an environment that cannot support them. Creatures summoned using this spell cannot use spells or spell-like abilities that duplicate spells with expensive material components (such as wish).

When you use a summoning spell to summon a creature with an alignment or elemental subtype, it is a spell of that type. Creatures on Table 10-1 marked with an "*" are summoned with the celestial template, if you are good, and the fiendish template, if you are evil. If you are neutral, you may choose which template to apply to the creature. Creatures marked with an "*" always have an alignment that matches yours, regardless of their usual alignment.

Summoning these creatures makes the summoning spell's type match your alignment.}
        
\DeclareSpell{Summon Thanadaemon}{conjuration (summoning) [evil]|V,  S,  F (two silver coins)|1 round|close (25 ft. + 5 ft./2 levels)|Effect: one summoned creature|1 round/level (D)|none|no}[]
    \DeclareSpellDescription{Summon Thanadaemon}{This spell functions like summon monster, except it summons a single thanadaemon. You can only use this spell in an area with enough open water to accommodate the daemon's skiff, or when on the Astral Plane or Ethereal Plane.\\\\

{\centering\bf Summon Monster I\hrule}

This spell summons an extraplanar creature (typically an outsider, elemental, or magical beast native to another plane). It appears where you designate and acts immediately, on your turn. It attacks your opponents to the best of its ability. If you can communicate with the creature, you can direct it not to attack, to attack particular enemies, or to perform other actions. The spell conjures one of the creatures from the 1st Level list on Table 10-1. You choose which kind of creature to summon, and you can choose a different one each time you cast the spell.

A summoned monster cannot summon or otherwise conjure another creature, nor can it use any teleportation or planar travel abilities. Creatures cannot be summoned into an environment that cannot support them. Creatures summoned using this spell cannot use spells or spell-like abilities that duplicate spells with expensive material components (such as wish).

When you use a summoning spell to summon a creature with an alignment or elemental subtype, it is a spell of that type. Creatures on Table 10-1 marked with an "*" are summoned with the celestial template, if you are good, and the fiendish template, if you are evil. If you are neutral, you may choose which template to apply to the creature. Creatures marked with an "*" always have an alignment that matches yours, regardless of their usual alignment.

Summoning these creatures makes the summoning spell's type match your alignment.}
    
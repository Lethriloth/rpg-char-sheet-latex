    
\DeclareSpell{Aroden's Spellbane}{abjuration|V,  S,  F (cold iron scepter worth at least 1, 000 gp)|1 standard action|10 ft.|Area: 10-ft.-radius emanation, centered on you|1 hour/level (D)|none|see text}[]
    \DeclareSpellDescription{Aroden's Spellbane}{One of many spells originally created by the Last Azlanti before he became a god, Aroden's spellbane creates an area within which spells selected by you simply do not function.

Select one spell per five caster levels at the time of casting.

The spells selected cannot be changed after the spell is cast. Aroden's spellbane otherwise functions like antimagic field, except its emanation only prevents the functioning of the selected spells. Only the exact spells mentioned are affected-a spellbane set to prevent the casting of summon nature's ally II would not prevent castings of summon nature's ally I or summon nature's ally III. If you move into an area where a previously cast spell you have selected as a banned spell is in effect, that spell is affected as if by antimagic field. If the spell affects a summoned creature that has spell resistance, you must make a caster level check against the creature's spell resistance to make it wink out.

Aroden's spellbane can even negate an antimagic field, another Aroden's spellbane, or any spell that specifies immunity to antimagic field (such as wall of force, prismatic sphere, and prismatic wall). Multiple spellbane effects can overlap.

Their effects stack, preventing the functioning of every spell targeted by any of the multiple spellbane emanations. Spell effects created by artifacts or deities cannot be suppressed by Aroden's spellbane.}
        
\DeclareSpell{Bladed Dash}{transmutation|V|1 standard action|personal|Targets: you|instantaneous||}[]
    \DeclareSpellDescription{Bladed Dash}{Both Quantium and Jalmeray claim that this spell was born in their arcane universities. Regardless of the spell's origin, it quickly spread throughout the Inner Sea and beyond as spellcasting sword-fighters learned of its existence.

When you cast this spell, you immediately move up to 30 feet in a straight line any direction, momentarily leaving a multi-hued cascade of images behind you. This movement does not provoke attacks of opportunity. You may make a single melee attack at your highest base attack bonus against any one creature you are adjacent to at any point along this 30 feet. You gain a circumstance bonus on your attack roll equal to your Intelligence or Charisma modifier, whichever is higher.

You must end the bonus movement granted by this spell in an unoccupied square. If no such space is available along the trajectory, the spell fails. Despite the name, the spell works with any melee weapon.}
        
\DeclareSpell{Greater Bladed Dash}{transmutation|V|1 standard action|personal|Targets: you|instantaneous||}[]
    \DeclareSpellDescription{Greater Bladed Dash}{This spell functions like bladed dash, save that you can make a single melee attack against every creature you pass during the 30 feet of your dash. You cannot attack an individual creature more than once with spell.}
        
\DeclareSpell{Blast Barrier}{transmutation [sonic]|V,  S,  M (handful of snow,  earth,  or gravel)|1 standard action|close (25 ft. + 5ft./2 levels)|Effect: 1-ft.-thick wall up to 10 ft. high by 20 ft. wide|concentration, up to 1 round/2 levels|Reflex half (see below)|yes (see below)}[]
    \DeclareSpellDescription{Blast Barrier}{Originally used by the winter witches on the field of battle in the early days of Irrisen, blast barrier has entered the oral traditions of many northern barbarian tribes as legends of winter witches possessing powers to bend the very ground of a battlefield to their will. Blast barrier, however, has proven to be an exceptionally versatile spell as far as terrains are concerned, for it works equally well in swamps, deserts, or any region where the ground is soft or easy to shape.

 When you cast blast barrier, you cause a rippling wall of loose earth, mud, snow, sand, or gravel to spring up in a designated space within the spell's range. This wall provides total cover to all Large or smaller creatures and objects. The barrier can only spring up in an area of natural, unworked ground. The energy that forms the wall's matrix is unstable, and you must concentrate to maintain the wall's shape. A blast barrier has an AC of 9, hardness 0, and 5 hit points per caster level. When a blast barrier reaches 0 hit points, or when you cease concentrating on maintaining it, the energies that maintain the barrier's shape fail with explosive results, sending sharp chunks of the materials comprising the wall and magical energy out along both sides. Any creature that is adjacent to a blast barrier when it explodes takes 2d6 points of slashing damage and 1d6 points of sonic damage per 3 caster levels (maximum 6d6). A successful Reflex save halves the total damage done. Spell resistance applies as well.

 This instability can make using a blast barrier risky, but many of the spellcasters that pioneered the spell became experts at its tactical applications, often using the barriers to cover an escape while lobbing spells and parting shots, hoping to trigger the barrier's destruction just as their would-be pursuers approached.}
        
\DeclareSpell{Call Weapon}{transmutation|V,  S|1 swift action|30 feet|Targets: one melee weapon wielded by an ally|instantaneous||}[]
    \DeclareSpellDescription{Call Weapon}{This spell first rose to prominence among the elves of Kyonin in the war to retake their ancestral land from the demon lord Treerazer and his minions. Elite units of elven magi entered battle with this spell prepared to shield fallen comrades or stand firm against fell foes. As the elven presence returned to the world, this spell spread throughout the Inner Sea and beyond.

When you cast this spell, you cause a weapon wielded by an ally within 30 feet to telekinetically fly across the space between you and into your open hand. This extra energy persists in the weapon for the rest of the round, granting you a +2 circumstance bonus on attack rolls and weapon damage rolls made during the same round you cast this spell.

If the ally targeted for this spell is unwilling to give up her weapon, the spell fails. An unconscious or dying ally is considered a "willing" target so long as the weapon to be called is still in contact with the ally's body.}
        
\DeclareSpell{Crusader's Edge}{transmutation [good]|V,  S,  M (dried blood from an evil outsider,  sprinkled on the weapon)|1 standard action|touch|Targets: melee weapon touched|1 minute/level|Fortitude negates|no}[]
    \DeclareSpellDescription{Crusader's Edge}{This spell was created by the paladins of the Mendevian Crusades, and co-opted by inquisitors and rangers dedicated to tracking and fighting demons, devils, and other evil extraplanar creatures.

When you cast this spell on a melee weapon you imbue it with a powerful holy energy, granting the weapon the bane weapon quality against evil outsiders. Furthermore, whenever you score a successful critical hit against an outsider with the evil subtype, you not only deal normal critical damage with the weapon but also nauseate the outsider for 1d3 rounds-the outsider can reduce this nauseated condition to sickened for 1 round with a successful Fortitude save.}
        
\DeclareSpell{Eaglesoul}{conjuration (summoning) [good]|V,  S,  M (vellum inscribed with good outsider's name)|1 standard action|personal|Targets: you|1 hour/level (see below)||}[]
    \DeclareSpellDescription{Eaglesoul}{As the Inner Sea's bravest men and women answered the call of what became the Second Mendevian Crusade, constructing the wardstones that eventually kept the brutal chaos of the Worldwound at bay, they realized that they needed help beyond traditional magics.

Legends say that the first eaglesoul spell was created when an agathion avoral joined his own spirit with that of a courageous paladin who was about to be overwhelmed on the field of battle. The holy knight used the combined might of the agathion's great strength and his own to win the day for the crusaders. Now, although the spell is still in use chiefly among those that patrol the borders of the Worldwound, other champions of good have carried it with them to all corners of the Inner Sea region.

When you cast this spell, you reach into the great beyond and beseech a good-aligned outsider for their aid against evil.

The outsider infuses a small portion of its own power into you, making you a powerful force for good. You gain a +2 morale bonus on all Perception checks made against evil creatures, a +2 bonus on Initiative checks, and detect evil as a constant spell-like ability.

In addition, once during the spell's duration you can call forth a surge of holy power when fighting an evil creature.

Doing so is a swift action that shortens the spell's remaining duration so that its remaining hours of duration become rounds of duration. For the rest of this duration, the surge of power grants you the following benefits:  A +2 sacred bonus to AC  A +4 sacred bonus to Strength  Resistance 5 to acid and fire  A +5 sacred bonus on all Intimidate checks made against evil creatures  Fast healing 2  Any critical threat roll made against an evil creature with a weapon you wield is automatically confirmed.

Although this surge of power can be activated against any evil opponent, this ability activates automatically as soon as you attack any evil outsider, regardless of whether you hit or not, and regardless of whether you actually recognize that the target is in fact an evil outsider. In such cases the activation is a free action.

Nongood spellcasters can cast this spell, but doing so causes them to be sickened (for spellcasters who are neither good nor evil) or staggered (for spellcasters who are evil) for the spell's duration.}
        
\DeclareSpell{Eldritch Conduit}{transmutation|V,  S,  M (a small mirror)|1 standard action|close (25 ft. + 5 ft./2 levels)|Targets: one creature|1 round/level|Will negates|yes}[]
    \DeclareSpellDescription{Eldritch Conduit}{Originally created by a disciple of the archwizard Nex, this spell was used to turn enemy soldiers in Geb's undead armies into conduits capable of blasting spell energy back into the far reaches of the Gebite lines. Since then, it has become a favorite among arcane casters with an understanding of battlefield tactics.

If the target of this spell fails to resist its effects with a Will save, he becomes outlined in faint radiance, as if via faerie fire. At any time before the eldritch conduit expires, you may cast another spell with an area effect of cone, cylinder, line, or sphere and use the subject of the eldritch conduit as the point of origin for that spell. Doing so ends the spell immediately. The target must be within close range (25 feet + 5 feet/2 levels) in order for you to use the conduit-if the target moves out of range, the eldritch conduit effect persists but cannot be utilized by you until you get back within range.}
        
\DeclareSpell{Greater Eldritch Conduit}{transmutation|V,  S,  M (a small mirror)|1 standard action|medium (100 ft. + 10 ft./level)|Area: one creature/level, no two of which can be more than 30 ft. apart|1 minute/level|Will negates|yes}[]
    \DeclareSpellDescription{Greater Eldritch Conduit}{This spell functions like eldritch conduit save for the differences listed above and that the target must be within medium range (100 ft. + 10 ft./level) in order for you to use the conduit. Using a creature as a conduit for a spell ends the greater eldritch conduit effect on that creature, but does not end the effect for other eldritch conduits.}
        
\DeclareSpell{Fleshcurdle}{transmutation (polymorph)|V,  S,  M (scrap of pickled flesh)|1 standard action|close (25 ft. + 5 ft./2 levels)|Targets: one living or undead creature|1 round/level|Fortitude negates|yes}[]
    \DeclareSpellDescription{Fleshcurdle}{You warp the target creature's flesh, discoloring it and causing it to become misshapen and impairing its function. When you cast this spell, you must choose one of three types of effects to inflict on the target-movement, attacks, or defense.

Attacks: One of the creature's natural attacks takes a -2 penalty on attack and damage rolls, only scores a critical hit on a natural 20, and only deals x2 damage on a confirmed critical hit.

Defense: The creature's natural armor bonus decreases by -4, to a minimum bonus of +0.

Movement: One of the creature's movement speeds (chosen by you) is halved.

Most undead are susceptible to fleshcurdle, but amorphous creatures and creatures without flesh are immune (such as elementals, oozes, plants, gaseous or incorporeal creatures, and skeletons).}
        
\DeclareSpell{Forceful Strike}{evocation [force]|V,  S|1 swift action|touch or reach of melee weapon|Targets: 1 creature|instantaneous|Fortitude partial|yes}[]
    \DeclareSpellDescription{Forceful Strike}{You cast this spell as you strike a creature with a melee weapon, unarmed strike, or natural attack to unleash a concussive blast of force. You deal normal weapon damage from the blow, but also deal an additional amount of force damage equal to 1d4 points per caster level (maximum of 10d4). The force of the blow may be enough to knock the target backward as well. To determine if the target is pushed back, make a combat maneuver check with a bonus equal to your caster level to resolve a bull rush attempt against the creature struck. You do not move as a result of this free bull rush, but it can push the target back if it defeats the target's CMD. A successful Fortitude save halves the force damage and negates the bull rush effect.}
        
\DeclareSpell{Geb's Hammer}{necromancy|V,  S,  M (a leather glove coated in dried embalming herbs)|1 standard action|close (25 ft. + 5 ft./2 levels)|Effect: sphere of undead remains composed of 3 or more destroyed undead|1 round/level|none|yes}[]
    \DeclareSpellDescription{Geb's Hammer}{Centuries of war with Nex trained the necromancers of Geb to extract every last resource from the mindless undead that make up most of Geb's rank-and-file troops.

When you cast this spell, you draw the remains of nearby destroyed undead together and fuse them into a mass of flesh and bone you can then hurl at any foes within range. Three corpses within range of the spell are required for the spell to function. Geb's hammer can be directed to attack one foe within range per round as a move action. It uses your caster level as its base attack bonus, modified by your Intelligence, Wisdom, or Charisma modifier (whichever one is highest). On a hit, the corpse hammer deals 1d6 points of damage per three caster levels (to a maximum of 6d6 points of damage).

Geb's hammer also has secondary effects based on the nature of the three bodies you use to create it. If the majority used to create Geb's hammer (at least two) were skeletal, the jagged bits of bone cause the corpse hammer to deal slashing damage and increase Geb's hammer's critical threat range to 19-20. On the other hand, if the majority were fleshy (at least two), the increased mass causes Geb's hammer to deal bludgeoning damage and increase its critical hit damage to x3.

Undead that have been destroyed by positive energy or a similar effect that does not leave a corpse, like a disintegrate spell, cannot be used to form Geb's hammer.}
        
\DeclareSpell{Geniekind}{transmutation (polymorph)|V,  S,  M (a pinch of dust,  embers,  wind-blown sand or drops of water,  depending on the genie type)|1 standard action|personal|Targets: you|1 round/level||}[]
    \DeclareSpellDescription{Geniekind}{Keleshite wizards and clerics have always sought to emulate the genies that they bind. The geniekind spell is the result of attempts to gain the power and influence of these proud outsiders without entirely forsaking the caster's own form.

Upon casting this spell, you must choose one type of genie to transform into, selecting from djinni, efreeti, marid, or shaitan. You retain your basic physical appearance but shift in some way to become more akin to the genie type you chose. While under the effects of geniekind, you gain a +2 racial bonus on all saving throws against paralysis, poison, sleep, and stunning effects, and a +4 enhancement bonus to your natural armor bonus. You also gain a +2 enhancement bonus to Constitution and a +5 bonus on all Diplomacy checks made when interacting with creatures of the same elemental subtype as your chosen genie. In addition, you gain other abilities depending upon the type of genie you choose to assume the form of, as detailed below.

Djinni: You gain the ability to fly at a speed of 60 feet with perfect maneuverability. When flying, your lower torso trails away into a vortex of wind and smoke. You gain resist electricity 10.

Efreeti: Your flesh turns a deep red and you grow large horns on your head. Your unarmed strikes and any melee weapon you wield deal +1d6 points of fire damage. You gain resist fire 10.

Marid: Your flesh turns blue and you gain webbed fingers and toes. You gain a swim speed of 60 feet and can breathe water. You gain resist cold 10.

Shaitan: Your flesh gains the coloration of stone and your hair appears to be sculpted from fine crystals. You gain a burrow speed of 60 feet and resist acid 10.}
        
\DeclareSpell{Hungry Darkness}{evocation [darkness,  force]|V,  S,  M (a bat's tooth)|1 standard action|close (25 ft. + 5 ft./2 levels)|Area: 60-ft.-radius spread|1 round/level (D)|none|yes (see text)}[]
    \DeclareSpellDescription{Hungry Darkness}{This spell creates an area of intense blackness, as deeper darkness, but filled with unseen chewing teeth and ravenous maws. Any creatures beginning its turn within the hungry darkness is gnawed and slashed by these unseen fangs of force, dealing 3d6 points of force damage and 2 points of Constitution damage per round. Once a creature leaves the cloud, it continues to take 1d6 points of bleed damage each round until it receives magical healing or enters an area of bright light. Spell resistance can prevent damage from the hungry darkness but not against the darkness it creates.}
        
\DeclareSpell{Hunter's Lore}{divination|V,  S,  M (a scrap of paper torn from a book about monsters)|1 standard action|personal|Targets: you|1 minute/level||}[]
    \DeclareSpellDescription{Hunter's Lore}{Inquisitors and rangers alike learned long ago that knowledge of an enemy grants power over that enemy. This spell was created so that almost any enemy could have its vulnerabilities known and exploited.

For the duration of this spell you may spend a full-round action to take 20 on any Knowledge skill checks made to identify a foe's weaknesses, provided the foe in question is within line of sight and provided that you have at least one rank in that particular Knowledge skill. This spell allows you to make untrained Knowledge checks to determine a monster's weaknesses as if you had ranks in that skill, but you cannot take 20 on these checks.}
        
\DeclareSpell{Impart Mind}{transmutation|V,  S,  M (diamond dust worth at least 1, 000 gp)|2 rounds|touch|Targets: permanent nonintelligent magic item touched|1 hour/level|none|no}[]
    \DeclareSpellDescription{Impart Mind}{This spell grants the target magic item temporary intelligence by siphoning off a tiny portion of your own mind to infuse the object. Only permanent magic items may be enhanced by this spell-one-use items and charged items cannot be affected by impart mind. Intelligent magic items cannot be affected by impart mind.

When you cast impart mind on a magic item, the item gains an Intelligence, Wisdom, and Charisma score of 10 and gains your alignment. You have no special control over the item once it becomes intelligent, although since it has your alignment, personality conflicts with items you wield won't be a problem. Determine the item's ego normally, as per the rules on intelligent items in the Core Rulebook.

An item made intelligent via impart mind communicates via speech and has normal senses to a range of 60 feet. It speaks and reads one language known by you-if you know multiple languages, you may choose which language the item knows.

When you cast impart mind on an item, roll once on Table 15-24 on page 534 of the Core Rulebook to randomly determine the item's power-add your caster level to this roll. By expending additional diamond dust in excess of the 1,000 gp worth of material components required to cast this spell, you can gain further bonuses on the d\% roll made to determine the item's power. Every additional 100 gp in diamond dust you use in excess grants a cumulative +1 bonus to the roll, up to a maximum additional bonus equal to your caster level. If the item gains the ability to cast a spell, determine the spell it can cast randomly from spells you know of the appropriate level. If the item gains ranks in a skill, it gains ranks in a random skill in which you have at least 1 rank.

If you roll above 100, then you can choose one item power on Table 15-24 in the Core Rulebook to grant the item, and the item gains a special purpose. Roll once on Tables 15-25 and 15-26 on page 534 of the Core Rulebook to determine what the item's special purpose and dedicated powers are. These d\% rolls are not modified by your caster level or any additional powdered gems you used to cast the spell.

Once this spell ends, the item reverts to its previous nature-if you cast this spell on the item again, it gains entirely different powers as determined by a new set of rolls.}
        
\DeclareSpell{Khain's Army}{necromancy [evil]|V,  S,  M/DF (a handful of ghoul's teeth)|1 standard action|5 feet|Effect: 1d4+1 ghouls and 1 ghast|1 round/level|Fortitude half (see text)|no}[]
    \DeclareSpellDescription{Khain's Army}{Originally created by the priest-king of Nemret Noktoria, the ghoul Kortash Khain, for use by his minions to bolster their forces in battles against their enemies, Khain's army has become a favorite of many necromancers throughout Osirion and beyond. By scattering a handful of ghoul's teeth across the ground, you cause 1d4+1 ghouls led by a single ghast to rise up from the ground around you. The ghouls and their ghast leader must appear in squares adjacent to you, but after that they follow your spoken commands unerringly.

If one of the ghouls is destroyed while the spell's duration is still in effect, it bursts into a spray of rotten flesh and necromantic energy that deals 1d6 points of negative energy damage to all adjacent targets-this energy heals undead targets as typical for negative energy damage. If the ghast is destroyed in this manner, it deals twice as much negative energy damage as a ghoul. A successful Fortitude save halves the negative energy damage dealt. When this spell's duration expires, any remaining undead created by this spell crumble apart into dust and blow away without dealing any additional negative energy damage.}
        
\DeclareSpell{Kiss Of The First World}{transmutation|V,  S,  M (diamond dust worth 100 gp)|1 standard action|touch|Area: living or undead creature touched|1 round/level|Will negates|yes}[]
    \DeclareSpellDescription{Kiss Of The First World}{This spell-a favorite of fey spellcasters-is traditionally bestowed with a kiss, though all that's really required to gift someone with a kiss of the First World is a touch. This spell infuses a living creature with a surge of positive energy from the First World, filling the target with the raw energies of life. The exact effects of this spell vary, depending on the nature of the creature touched. Constructs are immune to the effects of this spell.

Living Creature: A living creature gains a 20-foot increase to his base land speed and a +2 insight bonus on all Charismabased skill checks. In addition, the creature gains fast healing 2. Fire, acid, and negative energy cause this fast healing to stop functioning on the round following the attack.

Undead Creature: An undead creature targeted by this spell is staggered for the duration of this spell. It does not gain the benefits of any channel resistance it might normally enjoy, and the save DCs for any of its special attacks (but not spell-like abilities or spells) are reduced by 2.}
        
\DeclareSpell{Light Of Iomedae}{conjuration [good,  light]|V,  S,  DF|1 minute|medium (100 ft. +5 ft./level)|Targets: all undead in a 10-foot-radius spread|1 minute/level|Will partial|yes}[]
    \DeclareSpellDescription{Light Of Iomedae}{With this spell, you create shafts of blue light that illuminate all undead creatures in the area. Affected undead take a -20 penalty on all Stealth checks. Invisible undead are not made visible by this effect, but the light does make it easy to pinpoint the exact squares in which such undead are located (they still retain the 50\% miss chance granted by invisibility).

The light of Iomedae increases light levels by one step in a 5-foot radius around an affected undead creature. Once an undead is affected, it remains illuminated as long as remains within the spell's range, even if it leaves the spell's original radius, until the spell's duration ends.

Affected undead must also make a Will save when they are first illuminated by the light of Iomedae. Those who fail this save lose all benefits of channel resistance and take a -2 penalty on all saving throws made against positive energy effects.}
        
\DeclareSpell{Martial Marionette}{enchantment (compulsion) [mind-affecting]|V,  S,  M (a marionette's crossbar)|1 standard action|close (25 ft. + 5 ft./2 levels)|Targets: 1 creature|1 round/level|Will negates|yes}[]
    \DeclareSpellDescription{Martial Marionette}{When you cast this spell, you take partial control of an opponent's limbs, making it difficult for him to attack you.

Any attacks made against you by the target of the spell take a -2 penalty due to the erratic and random motions the spell forces onto any efforts to strike you. In addition, any creature suffering the effects of this spell cannot flank you and cannot aid other opponents in flanking you.

As long as the affected creature is adjacent to you, you can cause the creature's limbs to flail into the path of other attacks against you as an immediate action. This provides partial cover against that attack, granting you a +2 bonus to AC and a +1 bonus on Reflex saves.}
        
\DeclareSpell{Martyr's Bargain}{transmutation [good]|V|1 immediate action|personal|Targets: you|1 round/level|none|no}[]
    \DeclareSpellDescription{Martyr's Bargain}{Among the faithful followers of the gods of purity-whether they be the servants of Desna in Nidal, zealous followers of Milani struggling against Cheliax's government, paladins of Iomedae fighting against the horrors of the Worldwound, or simply those that fight evil the world over-martyr's bargain represents true faith and true sacrifice.

 You cast this spell as an immediate action when you are subject to a spell or spell-like ability that deals hit point damage, after attack rolls and saving throws have been rolled but before the damage itself is determined. The damage dealt by the spell and any related effects are then delayed for you (and you only) for a number of rounds equal to your caster level.

 At the end of that time (or immediately if martyr's bargain is dispelled), the delayed damage takes effect on you as it would have at the time it was cast, but is maximized as if affected by the Maximize Spell metamagic feat. Spells and spell-like abilities that were already maximized gain no additional benefit from this spell. Nothing can prevent this delayed damage from affecting you.

 You can be affected by only one martyr's bargain spell at a time. If you cast this spell while you are already under the effects of a previous martyr's bargain, the previous spell effect ends and you immediately take the damage it had delayed.}
        
\DeclareSpell{Music Of The Spheres}{conjuration (healing) [sonic]|V,  S,  M (a stick of incense treated with special balms)|1 standard action|20 ft.|Area: 20-ft.-radius spherical emanation, centered on you|concentration, up to 1 round per level|none|yes (harmless)}[]
    \DeclareSpellDescription{Music Of The Spheres}{As any scholar of Desnan lore or astrologer can tell you, the music of the spheres is the harmonic constant that plays under and through all of reality. It is this constant song, this otherworldly music, that keeps the laws of reality constant and the connections between the planes of existence strong.

With this spell, one can amplify the underlying music of the spheres in the spell's area of effect to infuse yourself and all creatures within 20 feet of you, friend and foe alike. All creatures that begin their turn within the area of this spell's effect gain fast healing 5, resistance 10 to all energy types, and a +3 sacred bonus on all saving throws against poison and disease. Any creature that enters the area of effect does not gain the benefits of the music of the spheres until it begins its turn in that area. You must maintain concentration on the amplification of the music or the effects immediately end, but you can move around to prevent enemies from gaining the benefits of this spell.}
        
\DeclareSpell{Orchid's Drop}{transmutation|V,  S,  M (a much-diluted drop of sun orchid nectar worth 500 gp)|1 standard action|personal|Targets: you|1 hour/level||}[]
    \DeclareSpellDescription{Orchid's Drop}{Alchemists have tried for centuries to recreate Artokus Kirran's feat of genius that created the sun orchid elixir.

Although they have yet to unlock the elixir's exact formula, their efforts have not been entirely wasted. One by-product of their experiments was the orchid's drop formula. This extract, distilled from a much-diluted drop of the nectar of a sun orchid flower, can transform an alchemist's mutagen into a potent healing tonic.

As long as you're under the effects of orchid's drop, drinking a dose of your mutagen heals you of 2d10 points of damage. For the spell's duration, you gain a +2 alchemical bonus on all saving throws.}
        
\DeclareSpell{Pugwampi's Grace}{enchantment (compulsion) [mind-affecting]|V,  S,  M (a pugwampi's hair)|1 standard action|short (25 ft. + 5 ft./2 levels)|Targets: one creature (see below)|1 round/level|Will negates|yes}[]
    \DeclareSpellDescription{Pugwampi's Grace}{Hated by adventurers throughout the Inner Sea region, the gremlins known as pugwampis infect those around them with a malignant form of unluck-an effect that this spell emulates. If the target fails its Will save, it becomes infused with the so-called "grace of the pugwampi." A creature affected by this spell must roll two d20s whenever a situation calls for a d20 roll (such as an attack roll, a skill check, or a saving throw) and must use the lower of the two results generated. As with the aura shed by actual pugwampies, this spell has no effect on animals, gremlins, or gnolls. The effects of this spell are negated as long as a target gains any sort of luck bonus to a d20 roll (such as those granted by a luckstone or divine favor), but the spell's duration is not impacted by these effects. If the luck bonus goes away before the pugwampi's grace effect ends, the unluck returns and remains until the spell's normal duration runs out.}
        
\DeclareSpell{Shadow Barbs}{illusion (shadow) [darkness]|V,  S,  M (a single link from a spiked chain)|1 standard action|0 ft.|Effect: spiked chain-like shadowy weapon|1 round/level|Will negates (see text)|no}[]
    \DeclareSpellDescription{Shadow Barbs}{This spell, developed originally by priests of Zon-Kuthon in Nidal's early years, has recently crossed the boundary between divine and arcane magic. Although it still bears some of the stigma of being associated with the Midnight Lord, it's rapidly becoming a favorite spell of magi and other martially minded arcane spellcasters.

When you cast this spell, you create a shadowy spiked chain that shimmers and pulses with darkness. The chain exists as long as you carry it; if you ever drop the chain, give it to another, or are disarmed, it immediately vanishes and the spell's duration ends.

The chain radiates darkness in a 10-foot-radius spread around you, reducing the illumination level in this area by one step, but not below the level of dim light.

You can wield the shadow barbs as a spiked chain as if you were fully proficient with spiked chains. Any additional abilities or feats that you possess that apply to spiked chains apply to the shadow barbs as well. The weapon functions as a +2 vicious spiked chain. Its enhancement bonus increases to +3 at caster level 11th, to +4 at caster level 15th, and finally to +5 at caster level 19th. When the spell effect ends, you can make a Will save against the spell-if successful, all of the damage caused to you by the shadow barbs' vicious weapon quality vanishes, unless you are dead or unconscious at the time the spell ends, in which case you automatically fail this Will save and the vicious weapon damage remains.}
        
\DeclareSpell{Shining Cord}{evocation [force]|V,  S,  M (a small length fine of silver chain worth 100 gp)|1 standard action|30 ft.|Targets: 1 creature|1 round/level or instantaneous (see below)|Fortitude partial (see below)|yes}[]
    \DeclareSpellDescription{Shining Cord}{When you cast this spell, you make a ranged touch attack against a single opponent within 30 feet. If you hit, a thin silver strand extends from your body to its, forming a connection that allows you to anticipate its actions. For the duration of the spell, you receive a +5 insight bonus on all Perception and Sense Motive checks opposed by the target.

You gain a +5 insight bonus on all Spellcraft checks made to identify your opponent's spell as part of a counterspelling attempt. Finally, you gain a +2 dodge bonus to your AC against attacks made against you by the target.

Lastly, if either you or the target moves more than 30 feet away from the other, the cord crackles with a surge of light and sends a blast of force along its length to the other end.

This blast deals 1d6 points of damage per two caster levels (maximum 10d6) and stuns the one who moved out of range for 1 round-a successful Fortitude save halves the damage and negates the stun effect. This blast ends the spell and severs the connection between you and your target.

Although the shining cord makes a visible connection between you and your target, creatures can move through it without ill effect. The cord even passes through solid objects as necessary to maintain the connection between you and the target.}
        
\DeclareSpell{Siphon Magic}{abjuration|V,  S,  M (a coiled length of copper wire wrapped around the palm)|1 standard action|touch|Targets: creature touched|instantaneous|none|no}[]
    \DeclareSpellDescription{Siphon Magic}{This spell attempts to transfer a magical effect from a creature you touch to yourself. When you touch the creature, siphon magic attempts to end one ongoing spell that has been cast on that creature, as if via a targeted dispel magic. If you know the specific spell effect you wish to target, you can name that spell effect to target that specific spell; otherwise siphon magic begins with the highest-level spell in effect and works its way down through all spells affecting the target until it dispels one or runs out of effects, as per dispel magic.

If siphon magic successfully ends a spell effect on the target, the remaining duration of that spell effect is transferred to you. That spell effect plays out for the rest of its duration as if you had been the original target. If the spell allows a saving throw to resist the effect, you gain a saving throw as if the spell were just being cast upon you, although this does not "reset" the spell's duration.}
        
\DeclareSpell{Song Of Kyonin}{conjuration (healing)|V,  S|1 standard action|close (25 ft. + 5 ft./2 levels)|Targets: up to 3 creatures, no two of which can be more than 30 ft. apart|1 round/level or until performance ends or changes (see text)||}[]
    \DeclareSpellDescription{Song Of Kyonin}{Certain elven bards of Kyonin are known for the restorative power of their performances-mostly as a result of this spell. You must have a bardic performance in effect to cast this spell (although this spell is called song of Kyonin, the bardic performance need not be singing). As long as that performance continues, up to 3 creatures affected by the performance gain fast healing 2. When this bardic performance ends or you change to a different bardic performance, the fast healing granted by this spell ends as well, but all creatures affected by this spell heal 1d8 points of damage + 1 point per caster level (maximum +15) and are cured of any of the following conditions: exhausted, fatigued, nauseated, paralyzed, sickened, or stunned.}
        
\DeclareSpell{Spell Absorption}{abjuration|V,  S,  M (a prism)|1 round|personal|Targets: you|1 round/level||}[]
    \DeclareSpellDescription{Spell Absorption}{If you successfully counterspell a 3rd-level or lower level spell (through either dispel magic or normal means) while spell absorption is in effect, you absorb the countered spell and use it to regain spells you have already cast. If you're a wizard, you regain the use of any single spell that you have cast since the last time you prepared spells. If you're a sorcerer, you regain a single spell slot. The spell recovered or spell slot regained must be of an equal level or lower than the spell you counterspelled.}
        
\DeclareSpell{Greater Spell Absorption}{abjuration|V,  S,  M (a prism)|1 round|personal|Targets: you|1 round/level||}[]
    \DeclareSpellDescription{Greater Spell Absorption}{This spell functions as spell absorption, save that you can absorb countered spells of 6th level or lower.}
        
\DeclareSpell{Spellscar}{abjuration|V,  S,  M (a pinch of sand from the Spellscar Desert)|1 standard action|medium (100 ft. + 10 ft./level)|Area: two 10-ft. cubes per level (S)|10 minutes/level (D)||}[]
    \DeclareSpellDescription{Spellscar}{This potent spell invokes the same sort of magical devastation that created the Mana Wastes so long ago- albeit on a much more localized (and thankfully temporary) scale. Within the area you choose to affect with spellscar, the terrain takes on a strange pale hue, as if colors were muted. Periodically, ripples of vibrant color wriggle through the terrain. Within this area, any spell, spell-like ability, or magic item activation automatically triggers a primal magic event-a spellcaster can avoid triggering such an event by making a concentration check (DC 15 + twice the spell's level), but non-spellcasters who activate magic items have no such option.

You gain a +4 insight bonus on concentration checks made to avoid triggering primal magic events while within a spellscar you have created, and if you do trigger a primal magic event, you may roll d\% twice and pick which of the two results you wish to have occur.}
        
\DeclareSpell{Suppress Primal Magic}{abjuration|V,  S,  M (a pinch of sand from the Spellscar Desert)|1 standard action|10 ft.|Area: 10-ft.-radius emanation centered on you|1 round/level||}[]
    \DeclareSpellDescription{Suppress Primal Magic}{Nexian wizards first created this spell while researching the nature of the Mana Wastes, hoping to create small zones within the magic-starved region where their own spells could still function reliably. The best the Nexians could manage was this spell-a method to temporarily stabilize magic so that spellcasting within a small area can be accomplished without fear of triggering primal magic events. In the area of effect of this spell, primal magic events cannot be triggered.

The emanation grants a +4 circumstance bonus on all saving throws against effects generated by primal magic outside of the spell's effect that expand into the area.

Unfortunately, suppressed primal magic tends to build up around the emanation created by this spell. When suppress primal magic's duration ends (or when the spell is dispelled), a primal magic event is immediately triggered at the center of the emanation if that point is still in an area where primal magic is active. The CR of this event is equal to the caster level of the recently ended suppress primal magic spell. The original caster of the suppress primal magic spell can attempt to negate this triggered primal magic event by making a DC 20 Will save-most spellcasters instead relocate to an area not affected by primal magic if they can so that they can end the spell's effect safely.}
        
\DeclareSpell{Tattoo Potion}{transmutation|V,  S,  M (a potion to be tattooed,  special inks worth 500 gp)|1 minute|one potion|Effect: one spell tattoo|instantaneous|none|no}[]
    \DeclareSpellDescription{Tattoo Potion}{When you cast this spell, you mix special tattoo inks into a potion of your choice. Once you finish casting tattoo potion, the potion begins bubbling and fizzing-if no one drinks the potion within 1 minute of the spell being cast, the potion bubbles away into vapor and is destroyed.

When a potion under the effects of tattoo potion is imbibed, the effects of the potion do not occur. Instead, the potion transforms into a spell tattoo (see page 16) on the drinker's chest tattoo slot-if the drinker already has a magic tattoo in this location, the tattoo potion is wasted. Once the potion transforms into a spell tattoo, it remains in place permanently until it is used as a spell tattoo.}
        
\DeclareSpell{Transfer Tattoo}{transmutation|V,  S,  M (tattooing needle)|1 standard action|touch|Targets: one magic tattoo|instantaneous|Fortitude negates|yes}[]
    \DeclareSpellDescription{Transfer Tattoo}{With this spell, you can transfer one magic tattoo from one creature to another. A target that isn't willing to have his tattoo removed or to receive the transferred tattoo can resist this spell with a Fortitude save-if successful, the transfer fails and the caster of this spell is staggered for 1 round by the backlash of magical energy. A tattoo can be transferred from a dead creature in this manner to a living host, provided the body has been dead no longer than one hour per caster level.}
        
\DeclareSpell{Vengeful Comets}{evocation [cold]|V,  S|1 standard action|long (400 ft. + 40 ft./level)|Effect: 1 comet per 4 levels|1 round/level or until completely discharged (see below)|none|yes}[]
    \DeclareSpellDescription{Vengeful Comets}{This spell causes a number of miniature comets (up to one per four caster levels) to orbit in the air above your head. Bits of snow and cold wind drift down from the orbiting comets, granting you a circumstance bonus equal to the number of comets on all saving throws against fire effects.

The actual use for the comets, though, is to make vengeful strikes against foes who dare to target you with offensive spells. As an immediate action whenever you are affected by a spell cast by another creature, you can fire one of your vengeful comets as a bolt of icy retribution (provided the source of the offensive spell is within range of your vengeful comet, of course). The comet requires a ranged touch attack to hit. If it hits, a comet deals 1d6 points of bludgeoning damage and 3d6 points of cold damage to the target, plus an additional amount of cold damage equal to the level of the spell you are retaliating against. If the spell you're retaliating against had the fire descriptor, you may opt to fire two comets instead of one.}
        
\DeclareSpell{Vex Giant}{transmutation|V,  S,  M (a fragment from a Large or larger weapon)|1 standard action|personal|Targets: you|1 round per level||}[]
    \DeclareSpellDescription{Vex Giant}{The giants and their kin have long plagued the peoples of Varisia, particularly the Shoanti of the Storval Plateau. Among the various tactics and methods the Shoanti have developed to fight against their enemies, this spell is one of the most widespread today.

When you cast vex giant, your senses and reflexes become particularly honed against a single target within 60 feet, provided the target is at least one size category larger than you. You may select your focused foe as a free action when you cast this spell-switching your focus to a different foe within 60 feet is a move action. If a foe moves beyond 60 feet from you, you lose your focus on that foe, although you may regain it by moving within 60 feet and spending a move action. Although the name of the spell is vex giant, it works equally well on any foe that's at least one size category larger than you.

Against a foe you are focused on, you do not provoke attacks of opportunity by moving through their threatened area. Additionally, the first successful melee attack you make against the foe in a round deals an additional 1d6 points of damage. Finally, you gain a +4 insight bonus on all combat maneuver checks made against your focused foe.}
        
\DeclareSpell{Weaponwand}{transmutation|V,  S,  F (a magic wand)|1 round|touch|Targets: one weapon|1 minute/level|Will negates (harmless, object)|yes (harmless, object)}[]
    \DeclareSpellDescription{Weaponwand}{When you cast this spell on a weapon, you cause a portion of the weapon to open like the skin of a partially peeled apple, revealing a space large enough to insert a single wand within.

As part of the spell's casting, you can insert a single wand into the weapon, at which point the weapon returns to its original form with the wand held inside of it without negatively impacting the weapon's integrity. For the spell's duration, a character who wields the transmuted weapon is also considered to be wielding the wand as well. You can attack normally with the weapon or use the weapon as if it were the encased wand. If the effect created by the wand requires an attack roll to successfully strike a foe, you may make the attack roll as if you were making an attack with the weapon at its highest bonus (including any bonuses the weapon would normally receive) rather than just a normal attack with the wand-doing so does not allow you to add the weapon's damage to the wand's attack roll, but instead allows you to use your skill with the weapon to boost your chance of hitting with the spell.

At the end of the spell's duration, the encased wand is ejected from the weapon. If you have a free hand, you may catch the weapon as a free action; otherwise, the wand drops to the ground. If the weapon housing the wand is broken or destroyed during the duration of weaponwand, the encased wand is similarly broken or destroyed.}
        
\DeclareSpell{Zone Of Foul Flames}{transmutation|V,  S,  M (a sliver of tree bark from a burnt tree from the Uskwood)|1 standard action|close (25 ft. + 5 ft./2 levels)|Area: 20-ft.-radius spread|1 minute/level|Will negates|yes}[]
    \DeclareSpellDescription{Zone Of Foul Flames}{The twisted druids of the Uskwood forsook fire in ages past for the glory of Zon-Kuthon. Despite this aversion, these servants of Nidal know well the sting of flame, and have learned to turn it back on those that wield it. This spell creates a zone where magical fire effects twist and lash out against those who create the effects. A zone of foul flames looks unremarkable to the casual observer, but a Perception check (DC = 20 + the caster's level) reveals a faint rippling effect in the area, as if of heat distortions in the air.

Whenever a creature casts a spell with the fire descriptor or activates a magical fire effect (as from a magic item or a special attack) while that creature is located in a zone of foul flames, that creature takes full fire damage from the effect.

If the effect allows a saving throw to reduce the damage, the victim may attempt the same saving throw to reduce the damage. All creatures in a zone of foul flames gain a +4 circumstance bonus on all saving throws against fire effects, except for those made by a creature attempting to save against fire damage from his own magic. Nonmagical fire in a zone of foul flames burns half as brightly but is otherwise not affected.}
        
\DeclareSpell{Corpse Hammer}{necromancy|V,  S,  M (a leather glove coated in dried embalming herbs)|1 standard action|close (25 ft. + 5 ft./2 levels)|Effect: sphere of undead remains composed of 3 or more destroyed undead|1 round/level|none|yes}[]
    \DeclareSpellDescription{Corpse Hammer}{When you cast this spell, you draw the remains of nearby destroyed undead together and fuse them into a mass of flesh and bone you can then hurl at any foes within range.

 Three corpses within range of the spell are required for the spell to function. The hammer can be directed to attack one foe within range per round as a move action. It uses your caster level as its base attack bonus, modified by your Intelligence, Wisdom, or Charisma modifier (whichever one is highest). On a hit, the corpse hammer deals 1d6 points of damage per three caster levels (to a maximum of 6d6 points of damage).

 Corpse hammer also has secondary effects based on the nature of the three bodies you use to create it. If the majority used to create it (at least two) were skeletal, the jagged bits of bone cause the hammer to deal slashing damage and increase its critical threat range to 19-20. On the other hand, if the majority were fleshy (at least two), the increased mass causes the spell to deal bludgeoning damage and increase its critical hit damage to x3.

 Undead that have been destroyed by positive energy or a similar effect that does not leave a corpse, like a disintegrate spell, cannot be used to form a corpse hammer.}
    
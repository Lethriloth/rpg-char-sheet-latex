    
\DeclareSpell{Absurdity}{illusion (phantasm) [emotionUM,  mind-affecting]|V,  S|1 standard action|medium (100 ft. + 10 ft./level)|Targets: up to one creature/level, no two of which can be more than 30 ft. apart|1 minute/level (D)|Will negates|no}[ Threats seem like a ridiculous farce.]
    \DeclareSpellDescription{Absurdity}{Your targets perceive intimidation and threats as laughably absurd. A character attempting to intimidate your targets is instead perceived as having exaggerated facial features or babbling and awkward speech. Effects that impose fear conditions become objects of ridicule, with the targets making fun of the source of the object as it attempts to scare them.  Absurdity protects your targets from gaining lesser fear conditions (spooked, shaken, and scared), granting them immunity to spooked and a 50\% chance to negate shaken or scared conditions instead of gaining them from any effect, including uses of the Intimidate skill to demoralize. Any other effect related to a spell or ability that generates fear (such as dying from a phantasmal killer) affects them normally, and effects that ignore immunity to fear also ignore absurdity.  However, target creatures also have serious difficulty noticing potential threats; they take a -10 penalty on Sense Motive checks to avoid surprise or to notice that a creature is actively threatening or malicious and a -2 penalty on initiative checks. The spell doesn't prevent spells or effects that provide early warning or a form of danger sense (like anticipate perilUM or find traps) from alerting the affected character to danger.}
        
\DeclareSpell{Alleviate Corruption}{abjuration|V,  S,  M (rare herbs,  incense,  and jewels worth 500 gp/the target's character level or HD)|1 minute|touch|Targets: one creature other than yourself|instantaneous|Will negates (harmless)|yes (harmless)}[ Attempt to weaken a corruption’s effect at risk to yourself.]
    \DeclareSpellDescription{Alleviate Corruption}{You combat the rising corruption (see page 14) in another creature or advance the long-term process of exorcising it entirely. This spell has two purposes, and you choose which application to use at the time of casting.  Combat Corruption: Each time a creature fails a saving throw to prevent its corruption from progressing (as described in that  corruption's Progression section), it advances to the next corruption stage-at stage 3, it loses the battle against its corruption. Using this spell can lower the target's corruption stage by 1. This has no effect if the target isn't at progression stage of 1 or higher.  Remove Corruption: You can remove 1 manifestation level from the target creature, as well as the most recently acquired manifestation and its corresponding gift and stain. Doing this also reduces the power of gifts and stains that vary based upon manifestation level. This application of alleviate corruption doesn't reduce the target's corruption stage. If the target creature loses all manifestation levels, it is cured of the corruption.  Either application of the spell requires a caster level check. The DC is equal to 10 + double the target's manifestation level + 3 times the target's corruption stage (effectively 0, 3, or 6, since stage 3 removes the character from player control).  Tampering with corruptions is dangerous, as their nature is contagious. If you fail this caster level check by 5 or more, you contract the corruption and gain a manifestation (the rules for the useful and vile corruption variants still apply). A roll of natural 1 on the caster level check is an automatic failure. If you already have the same corruption, you must instead attempt an immediate saving throw against it. Casting alleviate corruption on yourself automatically fails, as does casting it on a target who has fully succumbed to a corruption (failed three saving throws against it). A creature can be the beneficiary of this spell only once per week, whether it succeeds or not.}
        
\DeclareSpell{Appearance Of Life}{illusion (glamer) [evil]|V,  S,  M (one Tiny or larger living creature)|1 round|long (400 ft. + 40 ft./caster level)|Targets: one or more undead creatures|10 minutes/level (D)|Will disbelief or Will negates (see text)|no}[ Undead appear to be alive.]
    \DeclareSpellDescription{Appearance Of Life}{The illusion makes undead creatures of Medium size or smaller appear as if they were living humanoid creatures. You can target a number of undead creatures whose total number of Hit Dice is no greater than twice your caster level. When you create the illusion, you choose the races, genders, and attire for the undead creatures. Additionally, the illusion makes the undead creatures' movements appear lifelike (two shuffling zombies could be made to appear as two strolling lovers). The illusion doesn't create smell, sound, texture, or temperature. Undead with Intelligence scores can attempt a saving throw to negate the effect of the spell, but mindless undead do not. Any creatures interacting with the illusion receive a saving throw to disbelieve the illusion.  The illusion interferes with detect undead, requiring the caster to succeed at a caster check (DC = 11 + your caster level) for the spell to determine the creatures are undead.  Appearance of life can be made permanent with a permanency spell by a caster of 12th level or higher for the cost of 10,000 gp.}
        
\DeclareSpell{Assume Appearance}{transmutation (polymorph)|V,  S,  F (corpse of the deceased creature whose form you plan to assume)|1 minute|personal|Targets: you|1 day/level (D)||}[ Use a creature’s corpse to adopt its form.]
    \DeclareSpellDescription{Assume Appearance}{This spell functions similarly to alter self, except for the following differences. You assume the exact form of a deceased Small or Medium creature of the humanoid type. Your voice changes to match that of the form you assume. The creature whose form you assume must be dead and you must have access to its fresh corpse (either dead less than 24 hours, or preserved via gentle repose or similar effect). Any attempt to copy the form of a living creature causes the spell to fail. You do not have access to the assumed form's abilities, memories, mannerisms, or speech patterns. The spell grants a +10 bonus on Disguise checks to appear as the imitated creature.  If the assumed creature is returned to life while this spell is active, assume appearance immediately ends.}
        
\DeclareSpell{Greater Assume Appearance}{transmutation (polymorph)|V,  S,  F (corpse or likeness of the deceased creature whose form you plan to assume)|1 minute|personal|Targets: you|1 day/level (D)||}[ Use a likeness to adopt a dead creature’s form.]
    \DeclareSpellDescription{Greater Assume Appearance}{This spell functions similar to assume appearance, except that you can imitate a creature even if you have only a likeness (such as a sketch or painting) that is fairly accurate. If there is an age difference between the likeness and the actual creature, you take on the appearance of the creature near the end of its life. If you have heard the creature whose form you plan to assume speak aloud, you can also copy its voice as vocal alterationUM (speak with dead can also be used to fulfill this requirement). The spell doesn't grant insight into the dead creature's mannerisms.  Additionally, if a likeness used as a focus for the spell is kept on your person or within 30 feet at all times, the aura of greater assume appearance is redirected to the likeness instead of to you, similar to a reverse application of the misdirection spell.}
        
\DeclareSpell{Ban Corruption}{abjuration [good]|V,  S,  F (a ceremonial dagger)|1 standard action||Area: 30-ft.-radius emanation centered on you|concentration plus 1 round/level, up to 1 hour/level|Will negates|yes}[ Eliminate the gifts of nearby corrupted creatures.]
    \DeclareSpellDescription{Ban Corruption}{You strip all creatures within the emanation of any gifts associated with corruptions (see page 14). Abilities associated with corruptions cease to function, as if they were magically dispelled. The corruption stains still fully affect creatures. A creature that enters the emanation and fails its saving throw loses its gifts until the spell's duration ends, even if it leaves the emanation, but a successful saving throw renders a creature immune to the spell, even if it enters and exits the emanation several times.}
        
\DeclareSpell{Barbed Chains}{conjuration (summoning) [emotionUM,  fear,  mind-affecting]|V,  S,  M (a length of chain doused with fresh blood)|1 standard action|close (25 ft. + 5 ft./level)|Effect: a chain sharpened at one end|instantaneous|Will partial (see text)|no}[ Hellish chains attack and cause a target to become shaken.]
    \DeclareSpellDescription{Barbed Chains}{You summon a chain from another realm, causing it to burst out from the ground and strike a target within the spell's range. You can have the chain either make a melee attack (dealing 1d6 points of slashing damage) or attempt a trip combat maneuver against the target. The chain uses your base attack bonus plus your key spellcasting ability score modifier as its attack bonus and combat maneuver bonus. On a successful hit or combat maneuver check, the target must attempt a Will save. If it fails its save, the creature is shaken for 1d4 rounds. You summon one additional chain every 3 levels after 1st, for a total of two chains at 4th level, three at 7th level, and a maximum of four at 10th level. Multiple chains can attack the same target but the shaken effect doesn't stack.}
        
\DeclareSpell{Blood Ties}{necromancy [evil]|V,  S,  F (a small length of crimson string)|1 standard action|unlimited (see text)|Targets: two living creatures|1 day/level (D)|Will negates (see text)|no}[ When a target is harmed, so is the target’s relative.]
    \DeclareSpellDescription{Blood Ties}{One of the targets of this spell must be a hostage-a bound, pinned, or helpless creature within your reach (though the spell doesn't end if the hostage later moves beyond your reach). The second target must be a member of the first target's family. If neither creature successfully saves against the spell, the second target also takes any damage conferred to the hostage, provided both are still alive, to a maximum of 10 points of damage per caster level per day (damage beyond that necessary to kill the first target does not transfer). This transferred damage is typeless and ignores the second creature's damage reduction and resistances. If either target succeeds at the saving throw, you can never attempt to exploit the connection between these particular two creatures again. The connection between your targets weakens when the two are less familiar with one another. Each target gains a +2 bonus on its save if the two are distant relations or related by marriage. Each target takes a -2 penalty on its save if the two targets are parent and child or siblings.}
        
\DeclareSpell{Bloodbath}{necromancy|V,  S,  F (a ceremonial dagger)|1 standard action|close (25 ft. + 5 ft./level)|Targets: up to one living creature/level, no two of which can be more than 30 feet apart|1 round/level (D)|Fortitude negates|yes}[ Cause yourself and enemies to bleed.]
    \DeclareSpellDescription{Bloodbath}{You cut yourself with the dagger, dealing yourself 1d6 points of bleed damage. So long as you keep taking this bleed damage, your targets each bleed for 1d6 points of damage at the beginning of their turns. The bleed damage on any target ends if that target receives magical healing, or if your bleed damage ends for any reason. This spell has no effect if you're immune to bleed damage or can otherwise reduce or ignore the bleed damage to yourself.}
        
\DeclareSpell{Boneshaker}{necromancy|V,  S,  F (human-shaped fetish made of bones)|1 standard action|medium (100 ft. + 10 ft./level)|Targets: one living creature or undead creature with a skeleton|instantaneous|Fortitude partial or negates (see text)|yes}[ Momentarily control a living or undead creature’s skeleton.]
    \DeclareSpellDescription{Boneshaker}{By using a bone fetish like a marionette, you take control of a target creature's skeleton. This has a variety of effects depending on whether the target is living or undead.  A living creature has its skeleton rattle within its flesh, causing it grievous harm. The target takes 3d6 points of damage, plus 1d6 additional points of damage per 2 caster levels you have.  In addition, you can move the target 5 feet. This movement doesn't provoke attacks of opportunity. A successful saving throw halves the damage and negates the movement.  An undead creature takes no damage. Instead, you manipulate the undead, forcing it to take an immediate action to either move up to its speed (provoking attacks of opportunity as normal) or make a single attack against a creature of your choice in its reach. Either of these is the most basic version of the action the creature can take (it doesn't activate any special abilities that it could apply to the movement or attack, such as grab). A successful saving throw negates this effect. A mindless undead creature doesn't receive a save against this effect.}
        
\DeclareSpell{Borrow Corruption}{transmutation [evil]|V,  S,  M (a dirty scrap of a good priest's raiment)|1 standard action|touch|Targets: creature touched|1 minute/level (D)|none|no}[ Temporarily gain the effects of a corrupted creature’s manifestations.]
    \DeclareSpellDescription{Borrow Corruption}{You touch a creature with at least one manifestation from a corruption (see page 14). You temporarily gain any stains and gifts that corruption grants to the creature, and the creature retains them. If the gifts have limited uses, you count as having no uses remaining. You take 1d3 points of Wisdom drain whenever you cast this spell.  If your campaign uses the sanity system (see page 12), your sanity score decreases by 2 (and thus your sanity edge decreases by 1) each time you cast this spell, instead of you taking Wisdom drain. Only greater restoration, miracle, or wish can remove this decrease. A single casting of greater restoration removes one decrease of sanity from casting borrow corruption, while a casting of miracle or wish removes them all.}
        
\DeclareSpell{Charnel House}{illusion (shadow) [evil,  shadowUM]|V,  S,  M (one Tiny or larger living creature)|10 minutes|close (25 ft. + 5 ft./2 levels)|Area: 20-ft. cube (S)|10 minutes/level|Will partial|no}[ Create an area of semi-real gore.]
    \DeclareSpellDescription{Charnel House}{By sacrificing a living creature, you create the grisly illusion of viscera and gore splattered about the nearby area. When the casting time is complete, the walls drip with blood and the floor is slick with unidentifiable lumps of meat and other signs of a massacre. Anyone entering the area must attempt a Will save or be sickened for 1d6 rounds. This is a mind-affecting effect. If a creature succeeds at its save, it can see the semi-real nature of the illusion.  The room itself is covered in a layer of slippery, semi-real gore. Any creature attempting to walk within or through the area of gore can move at half normal speed with a DC 10 Acrobatics check. Failure means it can't move that round, and failure by 5 or more means it falls prone (see the Acrobatics skill for details). Creatures that don't move on their turns don't need to attempt this check and are not considered flat-footed. A creature that succeeded at its Will save gains a +5 bonus on the Acrobatics check.}
        
\DeclareSpell{Compelling Rant}{enchantment (charm)|V,  S,  M/DF (a handwritten sheet of notes)|1 minute|medium (100 ft. + 10 ft./level)|Targets: any number of creatures|concentration plus 1 round/level, up to 1 hour/level|Will negates (see text)|no}[ People believe your ridiculous speech as long as you keep talking.]
    \DeclareSpellDescription{Compelling Rant}{You deliver a confusing but fascinating monologue, relaying conspiracies or metaphysical revelations that confound your audience, throughout the spell's casting time and as long as you  concentrate. You take 1d4 points of Wisdom drain when you begin the speech and can't reduce or prevent this damage in any way. Each intelligent creature in the spell's area takes 1d6 points of Wisdom damage as their perceptions of reality realign with yours for the spell's duration. Listeners with at least 5 HD can attempt a Will save to negate the effects. Your targets view you with a friendly attitude and respond to criticism of you with irritation. Bluff, Diplomacy, or Intimidate checks to create doubt about your purpose in those affected by the spell take a -4 penalty. The attitude of your targets regarding any creature that criticizes you or your vision automatically changes one step toward hostile; a Diplomacy check that fails by 5 or more reduces their attitude further. Your targets retain the same alignment and their prior beliefs, in addition to the ones you force on them. You can't force  beliefs on a target if such beliefs would be necessarily against the nature of its alignment and prior beliefs, but targets are otherwise able to hold contradictory beliefs.  If your campaign uses the sanity system (see page 12), your sanity score decreases by 2 (and thus your sanity edge decreases by 1) instead of taking Wisdom drain each time you cast this spell. Only greater restoration, miracle, or wish can remove this decrease. A single casting of greater restoration removes one decrease in sanity from casting compelling rant, while a casting of miracle or wish removes them all. Affected targets take sanity damage equal to half your caster level (maximum 10) instead of Wisdom damage. The targets might actually change alignment and beliefs if their loss of sanity afflicts them with an appropriate madness, in which case those effects last even after the spell's duration has ended.}
        
\DeclareSpell{Contact Entity I}{evocation|V,  S,  M (see text)|1 minute|100 miles|Targets: up to 20 entities of 6 HD or fewer; see text|instantaneous|none|no}[ Ask eldritch entities to find and converse with you.]
    \DeclareSpellDescription{Contact Entity I}{You send out a magical message to any eldritch entities of a particular kind within a 100-mile radius, which can be delivered to up to 20 such creatures, starting with the nearest creatures until the limit has been met. This spell can't contact creatures with more than 6 Hit Dice. You can't send a specific message, but this spell (and all similar contact entity spells) can be characterized as an open invitation to make contact and establish communication. If there is an appropriate entity within range, the spell succeeds automatically. You don't know whether the message was received, nor any specific details about what creatures received it or how many. Creatures that receive the message know the location and distance from where the spell was cast. Because this spell doesn't  call or summon the target, the target must have its own way to reach the place where the spell was cast.  How creatures respond to a contact spell is circumstantial and it is possible the creatures will simply ignore the spell. Creatures that come and investigate do so in their own time. They usually arrive cautiously, aware of the potential for ambush. Targets of the spell might inform their organization or community if they have one. There are no restrictions on how the creatures react to being contacted, and they might respond with hostility, parley, entertain an alliance, or subjugate the caster and their related community. Using this spell counts as mentally contacting the creature for the purpose of any of its special abilities (such as the star-spawn's overwhelming mind). For the purpose of spells like scrying, the creature has firsthand knowledge of you and a connection similar to if it possessed a likeness of you.  Each type of creature requires a different material component that must be included when casting the spell, as shown on Table 4-1: Contact Entity on page 112. Some of these components are expensive or might require quests to acquire. Contacting certain types of creatures makes the spell chaotic, evil, or both, as indicated on the table.}
        
\DeclareSpell{Contact Entity Ii}{evocation|V,  S,  M (see text)|1 minute|200 miles|Targets: up to 20 entities of 12 HD or fewer|instantaneous|none|no}[ Ask more powerful eldritch entities to find and converse with you.]
    \DeclareSpellDescription{Contact Entity Ii}{This spell functions like contact entity I, except that you can contact creatures from the contact entity II list, and as noted above.\\\\

{\centering\bf Contact Entity I\hrule}

You send out a magical message to any eldritch entities of a particular kind within a 100-mile radius, which can be delivered to up to 20 such creatures, starting with the nearest creatures until the limit has been met. This spell can't contact creatures with more than 6 Hit Dice. You can't send a specific message, but this spell (and all similar contact entity spells) can be characterized as an open invitation to make contact and establish communication. If there is an appropriate entity within range, the spell succeeds automatically. You don't know whether the message was received, nor any specific details about what creatures received it or how many. Creatures that receive the message know the location and distance from where the spell was cast. Because this spell doesn't  call or summon the target, the target must have its own way to reach the place where the spell was cast.  How creatures respond to a contact spell is circumstantial and it is possible the creatures will simply ignore the spell. Creatures that come and investigate do so in their own time. They usually arrive cautiously, aware of the potential for ambush. Targets of the spell might inform their organization or community if they have one. There are no restrictions on how the creatures react to being contacted, and they might respond with hostility, parley, entertain an alliance, or subjugate the caster and their related community. Using this spell counts as mentally contacting the creature for the purpose of any of its special abilities (such as the star-spawn's overwhelming mind). For the purpose of spells like scrying, the creature has firsthand knowledge of you and a connection similar to if it possessed a likeness of you.  Each type of creature requires a different material component that must be included when casting the spell, as shown on Table 4-1: Contact Entity on page 112. Some of these components are expensive or might require quests to acquire. Contacting certain types of creatures makes the spell chaotic, evil, or both, as indicated on the table.}
        
\DeclareSpell{Contact Entity Iii}{evocation|V,  S,  M (see text)|1 minute|200 miles|Targets: up to 20 entities of 18 HD or fewer|instantaneous|none|no}[ Ask very powerful eldritch entities to find and converse with you, or they may reply telepathically.]
    \DeclareSpellDescription{Contact Entity Iii}{This spell functions like contact entity II, except that you can contact creatures from the contact entity III list, and as noted above. Additionally, if a contacted creature has telepathy, it can send a telepathic message to you of up to 10 words.\\\\

{\centering\bf Contact Entity Ii\hrule}

This spell functions like contact entity I, except that you can contact creatures from the contact entity II list, and as noted above.\\\\

{\centering\bf Contact Entity I\hrule}

You send out a magical message to any eldritch entities of a particular kind within a 100-mile radius, which can be delivered to up to 20 such creatures, starting with the nearest creatures until the limit has been met. This spell can't contact creatures with more than 6 Hit Dice. You can't send a specific message, but this spell (and all similar contact entity spells) can be characterized as an open invitation to make contact and establish communication. If there is an appropriate entity within range, the spell succeeds automatically. You don't know whether the message was received, nor any specific details about what creatures received it or how many. Creatures that receive the message know the location and distance from where the spell was cast. Because this spell doesn't  call or summon the target, the target must have its own way to reach the place where the spell was cast.  How creatures respond to a contact spell is circumstantial and it is possible the creatures will simply ignore the spell. Creatures that come and investigate do so in their own time. They usually arrive cautiously, aware of the potential for ambush. Targets of the spell might inform their organization or community if they have one. There are no restrictions on how the creatures react to being contacted, and they might respond with hostility, parley, entertain an alliance, or subjugate the caster and their related community. Using this spell counts as mentally contacting the creature for the purpose of any of its special abilities (such as the star-spawn's overwhelming mind). For the purpose of spells like scrying, the creature has firsthand knowledge of you and a connection similar to if it possessed a likeness of you.  Each type of creature requires a different material component that must be included when casting the spell, as shown on Table 4-1: Contact Entity on page 112. Some of these components are expensive or might require quests to acquire. Contacting certain types of creatures makes the spell chaotic, evil, or both, as indicated on the table.}
        
\DeclareSpell{Contact Entity Iv}{evocation|V,  S,  M (see text)|1 minute|500 miles|Targets: up to 20 entities of 24 HD or fewer|instantaneous|none|no}[ Ask extraordinarily powerful eldritch entities to find and converse with you, or they may reply telepathically.]
    \DeclareSpellDescription{Contact Entity Iv}{This spell functions like contact entity III, except that you can contact creatures from the contact entity IV list, and as noted above.\\\\

{\centering\bf Contact Entity Iii\hrule}

This spell functions like contact entity II, except that you can contact creatures from the contact entity III list, and as noted above. Additionally, if a contacted creature has telepathy, it can send a telepathic message to you of up to 10 words.\\\\

{\centering\bf Contact Entity Ii\hrule}

This spell functions like contact entity I, except that you can contact creatures from the contact entity II list, and as noted above.\\\\

{\centering\bf Contact Entity I\hrule}

You send out a magical message to any eldritch entities of a particular kind within a 100-mile radius, which can be delivered to up to 20 such creatures, starting with the nearest creatures until the limit has been met. This spell can't contact creatures with more than 6 Hit Dice. You can't send a specific message, but this spell (and all similar contact entity spells) can be characterized as an open invitation to make contact and establish communication. If there is an appropriate entity within range, the spell succeeds automatically. You don't know whether the message was received, nor any specific details about what creatures received it or how many. Creatures that receive the message know the location and distance from where the spell was cast. Because this spell doesn't  call or summon the target, the target must have its own way to reach the place where the spell was cast.  How creatures respond to a contact spell is circumstantial and it is possible the creatures will simply ignore the spell. Creatures that come and investigate do so in their own time. They usually arrive cautiously, aware of the potential for ambush. Targets of the spell might inform their organization or community if they have one. There are no restrictions on how the creatures react to being contacted, and they might respond with hostility, parley, entertain an alliance, or subjugate the caster and their related community. Using this spell counts as mentally contacting the creature for the purpose of any of its special abilities (such as the star-spawn's overwhelming mind). For the purpose of spells like scrying, the creature has firsthand knowledge of you and a connection similar to if it possessed a likeness of you.  Each type of creature requires a different material component that must be included when casting the spell, as shown on Table 4-1: Contact Entity on page 112. Some of these components are expensive or might require quests to acquire. Contacting certain types of creatures makes the spell chaotic, evil, or both, as indicated on the table.\\\\

{\centering\bf Contact Entity I\hrule}

You send out a magical message to any eldritch entities of a particular kind within a 100-mile radius, which can be delivered to up to 20 such creatures, starting with the nearest creatures until the limit has been met. This spell can't contact creatures with more than 6 Hit Dice. You can't send a specific message, but this spell (and all similar contact entity spells) can be characterized as an open invitation to make contact and establish communication. If there is an appropriate entity within range, the spell succeeds automatically. You don't know whether the message was received, nor any specific details about what creatures received it or how many. Creatures that receive the message know the location and distance from where the spell was cast. Because this spell doesn't  call or summon the target, the target must have its own way to reach the place where the spell was cast.  How creatures respond to a contact spell is circumstantial and it is possible the creatures will simply ignore the spell. Creatures that come and investigate do so in their own time. They usually arrive cautiously, aware of the potential for ambush. Targets of the spell might inform their organization or community if they have one. There are no restrictions on how the creatures react to being contacted, and they might respond with hostility, parley, entertain an alliance, or subjugate the caster and their related community. Using this spell counts as mentally contacting the creature for the purpose of any of its special abilities (such as the star-spawn's overwhelming mind). For the purpose of spells like scrying, the creature has firsthand knowledge of you and a connection similar to if it possessed a likeness of you.  Each type of creature requires a different material component that must be included when casting the spell, as shown on Table 4-1: Contact Entity on page 112. Some of these components are expensive or might require quests to acquire. Contacting certain types of creatures makes the spell chaotic, evil, or both, as indicated on the table.}
        
\DeclareSpell{Cruel Jaunt}{conjuration (teleportation) [evil,  fear,  mind-affecting]|V,  S,  M/DF (a dilated human eyeball)|1 standard action|medium (100 feet + 10 ft./level)|Targets: you|1 round/level (D)|none|no}[ Sense creatures suffering from fear, then teleport close to them.]
    \DeclareSpellDescription{Cruel Jaunt}{You gain the ability to detect fear, as per sense fear (see page 127), but you sense creatures within medium range.  Once per round as a standard action, you can teleport to a creature suffering from a fear effect within the spell's range if you are aware of the creature and its rough location. You can carry objects with you so long as you don't transport more than your maximum load. You arrive in a random open space within 20 feet of the creature and immediately begin sensing the location of creatures with a fear condition from your new location.  Once you teleport to a new location with this spell, you can take no additional actions for the round, as if casting dimension door. You can't teleport if there are no creatures suffering from a fear effect in range.}
        
\DeclareSpell{Curse Of Fell Seasons}{transmutation [curseUM,  darkness]|V,  S,  F (darkwood carving of a tree worth 15, 000 gp)|10 minutes|touch|Area: 2-mile radius emanating from the touched point|permanent (D)|none|no}[ Curse an area’s weather.]
    \DeclareSpellDescription{Curse Of Fell Seasons}{By touching the ground, you drastically change the weather in the area, as the unseasonable weather curse (see page 145). The focus merges into the cursed area as part of the spell and can only be retrieved if the curse is lifted.}
        
\DeclareSpell{Curse Of Night}{evocation [curseUM,  darkness]|V,  S,  F (jet gemstones worth a total of 10, 000 gp)|10 minutes|touch|Area: 1-mile radius emanating from the touched point|permanent (D)|none (see curse text)|no}[ Curse an area with eternal night.]
    \DeclareSpellDescription{Curse Of Night}{By touching the ground, you curse an area to remain in bleak darkness, as the endless night curse (see page 143). The focus merges into the cursed area as part of the spell and can only be retrieved if the curse is lifted.}
        
\DeclareSpell{Lesser Curse Terrain}{necromancy [curseUM,  evil]|V,  S,  M (the heart of a creature that dwelled in the area and powdered onyx worth 350 gp)|10 minutes|touch|Area: 300-ft. radius emanating from the touched point|1 day (D)|none|no}[ Curse an area with three mild hazards.]
    \DeclareSpellDescription{Lesser Curse Terrain}{By touching the ground, you curse the land with three unnatural hazards, as the minor perilous demesne curse (see page 145). Though this makes the hazards appear frequently, their manifestations are still unpredictable-you can't control when the hazards begin or end, nor where they appear within the cursed terrain.  Curse terrain spells can be made permanent with a permanency spell. The minimum caster level and gp cost are shown on the table. The spell is still dismissable if made permanent.     SpellMinimum Caster LevelGP CostLesser curse terrain9th2,500 gpCurse terrain11th7,500 gpGreater curse terrain15th17,500 gpSupreme curse terrain19th27,500 gp }
        
\DeclareSpell{Curse Terrain}{necromancy [curseUM,  evil]|V,  S,  M (the heart of a creature that dwelled in the area and powdered onyx worth 700 gp)|10 minutes|touch|Area: 1-mile radius emanating from the touched point|1 day (D)|none|no}[ Curse an area with four hazards.]
    \DeclareSpellDescription{Curse Terrain}{By touching the ground, you curse the land with four unnatural hazards. This functions as lesser curse terrain, but with the effects of the major perilous demesne curse (see page 145).}
        
\DeclareSpell{Greater Curse Terrain}{necromancy [curseUM,  evil]|V,  S,  M (the heart of a creature that dwelled in the area and powdered onyx worth 1, 500 gp)|10 minutes|touch|Area: 5-mile radius emanating from the touched point|1 day (D)|none|no}[ Curse an area with six dangerous hazards.]
    \DeclareSpellDescription{Greater Curse Terrain}{By touching the ground, you curse the land with six unnatural hazards. This functions as lesser curse terrain, but with the effects of the greater perilous demesne curse (see page 145).}
        
\DeclareSpell{Supreme Curse Terrain}{necromancy [curseUM,  evil]|V,  S,  M (the heart of a creature that dwelled in the area and powdered onyx worth 4, 000 gp)|10 minutes|touch|Area: 5-mile radius emanating from the touched point|1 day (D)|none|no}[ Curse an area with seven deadly hazards.]
    \DeclareSpellDescription{Supreme Curse Terrain}{By touching the ground, you curse the land with seven unnatural hazards. This functions as lesser curse terrain, but with the effects of the grand perilous demesne curse (see page 144).}
        
\DeclareSpell{Damnation}{evocation [good]|V,  S,  M/DF (a drop of holy water)|1 standard action||Area: 30-ft.-radius burst centered on you|instantaneous|Will half|yes}[ Punish creatures for evil spells they know or that affect them.]
    \DeclareSpellDescription{Damnation}{You pass judgment on your enemies' intentions and punish them with holy power. Each creature in the spell's area takes 1d8 points of damage per spell level, determined by the most powerful effect with the evil descriptor either affecting that creature or in its spell repertoire. A caster who prepares spells takes this damage based on the highest-level evil spell he prepared that day, even if he already cast that spell. A spontaneous spellcaster takes this damage based on the highest-level evil spell she knows, even if she has no remaining spell slots available to cast that spell. A creature with evil spell-like abilities takes this damage based on the highest-level evil spell-like ability it can use. A creature who doesn't cast spells but is the willing beneficiary of evil spells takes this damage based on the highest-level evil spell currently affecting it, including spells that affect an area such as desecrate. A creature subject  to the harmful effects of an evil spell are unaffected unless it voluntarily accepted the spell's effects.}
        
\DeclareSpell{Death Clutch}{necromancy [death,  evil]|V,  S|1 standard action|close (25 ft. + 5 ft./2 levels)|Targets: one living creature|instantaneous|Fortitude partial (see text)|yes}[ Rip out someone’s heart.]
    \DeclareSpellDescription{Death Clutch}{Chanting an unholy litany, you reach out with a grasping motion toward your target and cause its heart to leap out of its chest and into your hand. A target with 200 or fewer hit points remaining that fails its saving throw is instantly reduced to a number of negative hit points equal to your caster level or its Constitution score - 1, whichever is less negative. The creature is staggered until the beginning of your next turn, at which point it dies. If the affected creature receives a regenerate spell before the beginning of your next turn, the creature gains the normal benefits of that spell and, thanks to its heart's regeneration, it doesn't immediately die when your next turn begins. If a creature that dies from death clutch is brought back from the dead by a breath of life or raise dead spell, it must also be targeted with regenerate on the following round to restore its missing heart or be unable to return to life.  A target with 201 or more hit points that fails its saving throw manages to keep its heart from leaping out of its chest, but it is still staggered for 1 minute and takes 1d4 points of Constitution drain and 1d4 points of Constitution bleed.  Regardless of its current hit points, if the target succeeds at its Fortitude save, it is still staggered until the beginning of your next turn as it feels its heart wrenching within its chest.}
        
\DeclareSpell{Decapitate}{evocation|V,  S,  F (a sliver from a guillotine blade)|1 immediate action|close (25 ft. + 5 ft./2 levels)|Targets: one creature (see text)|instantaneous|Fortitude partial (see text)|yes}[ Turn a critical hit into a decapitation.]
    \DeclareSpellDescription{Decapitate}{You can cast this spell only as a response to a confirmed critical hit against the target that would deal slashing damage. If the target fails the saving throw and has a discernible head, the attack deals an extra 4d6 points of damage and the critical multiplier of the critical hit increases by 1. If the critical hit then brings the target to 0 hit points or fewer, the target is instantly decapitated and dies unless it can survive decapitation. Even on a successful saving throw, the critical hit deals an extra 4d6 points of damage.}
        
\DeclareSpell{Decollate}{necromancy|V,  S,  F (a red wax pencil)|1 standard action|touch|Targets: one willing humanoid or monstrous humanoid creature|24 hours|none|yes (harmless)}[ A target can safely remove its head.]
    \DeclareSpellDescription{Decollate}{A thin red line circles the target's neck. The target's head becomes detachable so long as she removes it willingly. While the target's head is detached, she gains DR 2/- and immunity to decapitation effects and other effects that require their target to have a head or a particular facial feature. The target is blind so long as she has no head, but she gains blindsense to a distance of 15 feet. The target hears normally even without its head.  While detached, the target's head appears to be dead. The target can't see through its eyes or hear events around the head. The target's body knows the direction and distance to its head. Without additional protection, the severed head has AC 7, hardness 5, and 10 hit points. Destroying the head while the  spell is in effect kills the target of the spell. If the spell's duration expires normally without the head being reattached, the target dies. If either the target or the head are removed to a different plane, or if the spell is dispelled, the head teleports back to its owner and reattaches without further harm.}
        
\DeclareSpell{Dreadscape}{illusion (phantasm) [emotionUM,  fear,  mind-affecting]|V,  S,  M (a pinch of black sand)|1 standard action|close (25 ft. + 5 ft./2 levels)|Targets: up to one creature/level, no two of which can be more than 30 feet apart|10 minutes/level|Will negates|yes}[ Surroundings and unfamiliar creatures seem like something out of a nightmare.]
    \DeclareSpellDescription{Dreadscape}{Your targets see their surroundings as a nightmarish reflection of the world around them. Buildings and furnishings take on a dirty, ruined appearance. Even allies appear foreign and hostile, with friendly speech turning into garbled mockery and threats.  Each target gains the scared condition and has a hostile attitude toward any new creature it encounters (though not toward creatures that were already present at the time of the casting). Being hostile doesn't necessarily mean the target will attack, and creatures can attempt Diplomacy checks to gain a target's trust at the normal DC. If a creature becomes frightened or panicked while under the influence of dreadscape, that creature takes 1d6 points of Wisdom damage, though only once per casting of dreadscape.  If your campaign uses the sanity system (see page 12), creatures that become frightened or panicked take 2d6 sanity damage instead of Wisdom damage.}
        
\DeclareSpell{Flesh Puppet}{necromancy [evil]|V,  S,  M (an onyx worth 25 gp and a silken string)|1 round|touch|Targets: one corpse touched|permanent (D)|none|no}[ Control a zombie in human guise.]
    \DeclareSpellDescription{Flesh Puppet}{You animate one corpse that has been dead no more than 48 hours. It rises as a zombie (Pathfinder RPG Bestiary 288) that is magically tethered to you and obeys your commands. As noted in animate dead, you can't control more than 4 HD per caster level worth of undead in total, nor can a single casting create more than 2 HD per caster level.  This spell disguises the zombie's appearance and allows you to control it. The zombie's outward appearance, movement, and voice appear the same as if it were still alive. The zombie's normal staggered condition doesn't apply (though it can still be staggered by other means). Successfully detecting the flesh puppet as a zombie without magic requires an opposed Perception check against your Disguise check, and you add your caster level as a bonus on this Disguise check.  An ephemeral string connects you to the zombie. Through this string, you have a mental link to the zombie and can command it as a swift action. The zombie uses its own actions to complete your commands. The zombie can speak up to 25 words in 1 round, but you must mentally impart what you intend it to say as a swift action. It is incapable of articulating speech on its own. The zombie can be ordered to perform very simple tasks it knew in life but can't make attacks, cast spells, or perform complex or difficult tasks requiring constant concentration.  The string connecting you and the zombie is nearly invisible. A DC 30 Perception check is required to detect it. It has hardness 0 and 1 hp. The length of string you can create is 100 feet + 10 feet per caster level you have. The string snaps if you and the zombie move farther apart than this length, though the zombie won't move out of range unless forced to do so or unless you command it to do so. If the string to the zombie is severed, the spell immediately ends. The ephemeral string can pass through  physical barriers, but not barriers of magical force, and it can be damaged as though it were a physical object.  When this spell ends, the zombie immediately reverts back to a normal corpse. The spell ends automatically if you cast flesh puppet or flesh puppet horde on a new corpse.}
        
\DeclareSpell{Flesh Puppet Horde}{necromancy [evil]|V,  S,  M (an onyx worth 50 gp for each zombie and a silken string)|10 minutes|touch|Targets: one or more corpses touched|permanent (D)|none|no}[ Control multiple zombies in human guise.]
    \DeclareSpellDescription{Flesh Puppet Horde}{This spell functions as flesh puppet, but can animate multiple zombies. As noted in animate dead, you can't control more than 4 HD per caster level worth of undead in total, nor can a single casting create more than 2 HD per caster level. A separate string attaches to each zombie in your horde. Severing a zombie's string reverts that zombie to a corpse, but doesn't end the spell for other zombies. Because commanding a flesh puppet requires a swift action, you can issue commands to only one zombie per round, though zombies you previously commanded continue to follow their orders. Likewise, you can command only one zombie to speak per round.  Unlike with flesh puppet, you can command a zombie to attack. If you do, all your zombies immediately gain the staggered quality and no longer appear to be alive.  This spell ends automatically if you cast flesh puppet or flesh puppet horde on a new corpse.}
        
\DeclareSpell{Flesh Wall}{necromancy [evil]|V,  S,  M (one corpse for every 5-ft. square of the wall),  DF|1 standard action|medium (100 ft. + 10 ft./level)|Effect: a wall of corpses with an area of up to one 5-ft. square/ level (S)|concentration + 1 round/level (D)|none|no}[ Create a wall of zombies.]
    \DeclareSpellDescription{Flesh Wall}{You animate corpses, forming them into a wall of joined flesh and limbs. The wall inserts itself into any surrounding nonliving material if its area is sufficient to do so. The wall can't be created so that it occupies the same space as a creature or another object. The wall must be vertical, but can be shaped as you see fit.  The wall is considered to be undead. It uses your Will saving throw to resist channel energy.  A flesh wall is 2 feet thick. Each 5-foot square of the wall has 12 hit points and DR 5/slashing. A section of wall whose hit points drop to 0 is breached. As a move action, you can cause the fleshwall to constrict, shrinking it by a 5-foot square to fill the hole. Additionally, as a standard action, you can cause a 5-foot square of  the wall to permanently detach, forming a human zombie (Bestiary 288) under your verbal control (this zombie doesn't count against your normal limit of commanded undead). The zombie reverts back into a normal corpse when the spell's effect ends. Each 5-foot square of the wall makes a single slam attack against an adjacent enemy on your turn, as a human zombie. The squares of the wall threaten their adjacent squares and can even provide flanking.  Creatures can force their way slowly through the wall by making a Strength check as a full-round action. The DC to move through the wall is equal to 15 + your caster level. A creature that fails the check is trapped in the wall, takes 3d6 points of crushing damage, and is denied its Dexterity bonus to AC against the wall's slam attack. The creature can make an attempt to escape the wall on its next turn.  You can use zombies already under your control as the material components for a flesh wall. However, they and any other corpses in the wall revert back to inanimate corpses when the spell ends.}
        
\DeclareSpell{Flickering Lights}{evocation [darkness,  light]|V,  S,  M (a patch of white cloth and a patch of black cloth)|1 round|medium (100 ft. + 10 ft./level)|Area: contiguous area consisting of one 10-foot cube/level (S)|1 round/level|Will negates|yes}[ Create an area of inconsistent lighting.]
    \DeclareSpellDescription{Flickering Lights}{You cause the illumination in the area to seem to flicker erratically, fluctuating between absolute darkness and blinding brightness. The level of light in the area changes at the start of each creature's turn, as determined by rolling a percentile die and consulting the following table.     d\%Illumination level1-10Supernatural darkness11-25Darkness26-50Dim light51-90Normal light91-00Bright light     Even darkvision can't see through supernatural darkness (as deeper darkness). Bright light affects creatures with light blindness or light sensitivity. For the purpose of superseding its effects with higher-level light or darkness spells, flickering lights counts as a light spell when it increases the ambient light level and a darkness spell when it decreases the ambient light level.}
        
\DeclareSpell{Grasping Corpse}{necromancy [evil]|V,  S,  M (pinch of powdered onyx worth 1 gp)|1 standard action|close (25 ft. + 5 ft./2 levels)|Targets: one corpse|instantaneous|none|no}[ Cause a corpse to grab or trip a foe.]
    \DeclareSpellDescription{Grasping Corpse}{You can cause one nearby corpse to animate for a brief moment. Choose a creature within 30 feet of the corpse (even if the creature is outside the spell's range). The corpse shambles toward the creature and then attempts to trip or grapple it (your choice). The corpse does not provoke attacks of opportunity.  Attempt a special combat maneuver check against the chosen creature. Your CMB for this combat maneuver is equal to your caster level plus your Intelligence, Wisdom, or Charisma modifier, whichever is highest. If you chose to trip the creature, it falls prone if you equal or exceed its CMD. If you chose to grapple, the  creature gains the grappled condition until it breaks free from the corpse. Treat your save DC for a spell of this level as the CMD of the grasping corpse. Alternatively, destroying the corpse with damage ends the grapple. The corpse has 12 hit points and DR 5/slashing.}
        
\DeclareSpell{Green Caress}{transmutation|V,  S,  M (pinch of moss or a kudzu leaf)|1 standard action|touch|Targets: one living creature|7 days (see text)|Fortitude partial (see text)|yes}[ Slowly transform a creature into an inanimate plant.]
    \DeclareSpellDescription{Green Caress}{You cause the target to transform into a plant over time. You must succeed at a melee touch attack to infect the target. If the target succeeds at its Fortitude saving throw, it takes 1d4 points of ability damage to each physical ability score (Strength, Dexterity, and Constitution) and the spell ends. If the target fails its saving throw, it takes 1d4 points of ability damage to each physical ability score immediately and continues to take 1d4 points of ability damage to each of its physical ability scores every day, until the spell expires. It can't recover this ability damage as long as the spell lasts, even with magic. While the spell continues, the target takes on physical plant characteristics as appropriate to the environment. It begins to diminish in height and its skin turns plantlike. For example, the target's skin might turn green and its hair is slowly replaced with grass or leaves or the target's skin might become smooth, pale, and flabby as the creature transforms into a large mushroom in an underground environment.  If the damage to any single ability score equals or exceeds that score, the target fully transforms into a normal small tree or shrub. This final transformation is instantaneous, ending green caress. The target remains alive but is considered the same as a regular tree, shrub, or other vegetation. Any ability damage from other sources, like poison or disease, also applies toward transforming the creature. If the target retains at least 1 point in all of its physical ability scores at the end of 7 days, the final transformation doesn't occur and the spell ends. Any changes in appearance gradually reverse themselves as the ability damage heals.  Break enchantment, dispel magic, and remove curse can end the spell before the duration expires, but the spell is contagious. If a caster level check attempt to remove green caress fails by 5 or more, the creature who attempted to remove the effect must attempt a Fortitude saving throw as if it had just been targeted with the spell. If the spell's target attempts to remove the effect from itself and fails, it causes the spell to behave as if affected by plant growth (as described below).  Polymorph any object ends the spell and totally restores the target without any risk to the caster, even after the target has been finally transformed, as do limited wish, miracle, and wish (treat polymorph any object as if it had an instantaneous duration if it is used in this way). If the target is in the area of a plant growth spell as it is cast, it must attempt a Fortitude save at green caress's  DC or immediately take another 1d4 points of ability damage to each physical ability score; this stacks with multiple castings of plant growth. If the target is in the area of diminish plants as it is cast, it ignores the next ability damage from green caress to one of its three physical ability scores (chosen randomly); this doesn't stack with multiple castings of diminish plants.}
        
\DeclareSpell{Hedging Weapons}{abjuration [force]|V,  S,  DF|1 standard action|personal|Targets: you|1 minute/level (D)||}[ Floating weapons protect you and make ranged attacks.]
    \DeclareSpellDescription{Hedging Weapons}{A weapon made from divine force appears and floats near you. This weapon takes the shape of your deity's favored weapon (if you have no deity, the weapon appears as a simple weapon with special significance to you). You gain one additional weapon at 6th level and every 4 caster levels thereafter-two at 6th, three at 10th, four at 14th, and a maximum of five weapons at 18th level. The weapon averts and deflects attacks, granting you a +1 deflection bonus to AC for each weapon summoned (maximum +5 at 18th level). As a standard action, you can grasp a weapon and throw it as a ranged attack at any target you can see within 30 feet of you (even if it's a type of weapon that can't normally be thrown). On a successful hit, the weapon deals 2d6 points of force damage to the target. This force weapon has the same threat range and critical multiplier as a standard weapon of its type, but no other special abilities. Because it deals force damage, DR doesn't apply. Each weapon thrown lowers the total deflection bonus to your AC by 1 as it disappears immediately after the attack action. The spell immediately ends once you throw all the weapons.}
        
\DeclareSpell{Holy Javelin}{conjuration (creation) [good]|V,  S,  DF|1 standard action|medium (100 ft. + 10 ft./level)|Effect: javelin of divine energy|1 round + 1 round/4 levels|none|yes}[ Deal ongoing damage to evil creatures and apply penalties.]
    \DeclareSpellDescription{Holy Javelin}{You create a shimmering javelin of holy energy to hurl at an enemy as a ranged touched attack. The javelin deals 1d6 points of damage to an evil creature on a successful attack; it dissipates harmlessly against creatures of any other alignment. Each time a creature starts its turn while impaled by the javelin, it takes another 1d6 points of damage. For every 4 caster levels you have, the javelin remains in the creature for an additional round (to a maximum of 5 rounds at 18th level). While the creature remains impaled, it takes a -2 penalty on attack rolls and skill checks. As a move action, the creature (or another adjacent creature)  can attempt to pull the javelin out (causing it to immediately disappear) with a DC 12 Strength check.  The holy javelin glows like a torch, and this light clearly indicates the impaled creature's location, even if it turns invisible. Since this light is not a light effect, just the glow of a conjured javelin, darkness spells always suppress it, even if they are lower level.}
        
\DeclareSpell{Horrific Doubles}{illusion (figment)|V,  S|1 standard action|personal|Targets: you|1 minute/level||}[ Call forth disturbing mirror images.]
    \DeclareSpellDescription{Horrific Doubles}{You create several illusory doubles of yourself, where you and each image all seem slightly off or wrong in appearance. Treat this spell as mirror image, except as noted.  Each creature that can see the doubles must succeed at a Will save or become shaken for as long as it can see any of the doubles. A successful saving throw negates the shaken condition  and renders the creature immune to the further effects of this spell (beyond the usual effects of mirror image). In addition, the first time a creature that failed its initial saving throw destroys one of the images, it must succeed at a Will save or its perception of the double shifts at the last second. The double takes on the face of the attacker, the face of a loved one, or some other equally disturbing image, causing the attacker to become frightened for 1 round and take 1d3 points of Wisdom damage from the traumatic shock. Both additional effects are mind-affecting fear effects, and spell resistance applies against them.  If your campaign uses the sanity system (see page 12), a creature takes 1d8 points of sanity damage instead of Wisdom damage.}
        
\DeclareSpell{Hunger For Flesh}{necromancy [evil,  mind-affecting]|V,  S,  M/DF (a ghoul fang)|1 standard action|close (25 ft. + 5 ft./2 levels)|Targets: one humanoid, magical beast, or monstrous humanoid|1 round/level|Will negates|yes}[ Give a creature a bite attack and a hunger for its own kind’s flesh.]
    \DeclareSpellDescription{Hunger For Flesh}{Your target's belly distends and its front teeth grow longer and sharper. The creature ravenously craves the flesh of its own kind, gaining the staggered condition as hunger pangs rack its altered body. The target gains a bite attack as a primary natural attack that deals damage appropriate for its size (1d6 if Medium, 1d4 if Small). There's a 25\% chance on each of the creature's turns that it can't overcome its hunger. If so, it must move directly toward the nearest corporeal creature of its type and subtype (if applicable) and make a bite attack against it. If the target lacks enough actions to attack on that turn, it moves as close to the creature as it can, but on its next turn, if the percentile dice indicate it overcomes its hunger, the target is not forced to pursue or attack further.  On any round after the target deals damage to another creature of its own type and subtype (if applicable) with its bite attack, the target loses the staggered condition. If it doesn't continue dealing bite damage to applicable creatures, the target regains the staggered condition at the beginning of its next turn.}
        
\DeclareSpell{Mass Hunger For Flesh}{necromancy [evil,  mind-affecting]|V,  S,  M/DF (a ghoul fang)|1 standard action|close (25 ft. + 5 ft./2 levels)|Targets: one humanoid, magical beast, or monstrous humanoid/ level, no two of which can be more than 30 feet apart|1 round/level|Will negates|yes}[ Give creatures bite attacks and a hunger for their own kind’s flesh.]
    \DeclareSpellDescription{Mass Hunger For Flesh}{This spell functions like hunger for flesh, except that it affects multiple targets. When forced to feed by the spell, affected creatures attack creatures not affected by this spell if there are any such appropriate creatures nearby, but otherwise they attack other affected creatures.\\\\

{\centering\bf Hunger For Flesh\hrule}

Your target's belly distends and its front teeth grow longer and sharper. The creature ravenously craves the flesh of its own kind, gaining the staggered condition as hunger pangs rack its altered body. The target gains a bite attack as a primary natural attack that deals damage appropriate for its size (1d6 if Medium, 1d4 if Small). There's a 25\% chance on each of the creature's turns that it can't overcome its hunger. If so, it must move directly toward the nearest corporeal creature of its type and subtype (if applicable) and make a bite attack against it. If the target lacks enough actions to attack on that turn, it moves as close to the creature as it can, but on its next turn, if the percentile dice indicate it overcomes its hunger, the target is not forced to pursue or attack further.  On any round after the target deals damage to another creature of its own type and subtype (if applicable) with its bite attack, the target loses the staggered condition. If it doesn't continue dealing bite damage to applicable creatures, the target regains the staggered condition at the beginning of its next turn.}
        
\DeclareSpell{Impossible Angles}{illusion (figment)|V,  S,  M (a melted prism)|1 standard action|medium (100 ft. + 10 ft./level)|Area: contiguous area up to one 5-foot cube/caster level (S)|1 minute/level|Will negates|yes}[ Distort geometry in an area.]
    \DeclareSpellDescription{Impossible Angles}{You cause the surrounding area to appear to distort. The angles and corners of the area subtly twist and contort, creating unnatural and impossible shapes. Any creature entering the area must succeed at a Will save or become disoriented. Disoriented characters treat the area as difficult terrain and are sickened. In addition, whenever a disoriented creature uses an action to move (including taking a 5-foot step if it can do so in difficult terrain), roll 1d8 to see which direction it moves, in a similar manner to determining where a splash weapon lands on a miss (Pathfinder RPG Core Rulebook 202). On a 1, the creature moves in its intended direction, with 2 through 8 rotating around the creature's starting square in a clockwise direction. Only the creature's first  5 feet of movement each round are affected in this way-it can move normally for any remaining movement, either from the same action or from later actions, as the creature acclimatizes to the distortion. An affected creature can attempt a new Will save each round to end the disoriented effect. A creature that leaves the area and re-enters must attempt the saving throw again, even if it succeeded at its initial save.}
        
\DeclareSpell{Life Blast}{necromancy|V,  S,  M (a dead leaf)|1 standard action|150 ft.|Area: 150-ft. line|instantaneous|Will half|yes}[ Drain life from local vegetation to launch a blast of positive energy.]
    \DeclareSpellDescription{Life Blast}{This spell must be cast in an area with vegetation or it has no effect. When you cast this spell, you draw the life force from the surrounding land and hurl it at your enemies, dealing 1d6 points of positive energy damage per caster level (to a maximum of 12d6 at 12th level) to any undead creatures in the spell's area. However, doing so blights the land around you in a spread with a radius of 5 feet per caster level you have (to a maximum of 60 feet at 12th level). All vegetation in that area immediately withers and dies. Plant creatures aren't affected.  The blast starts from your palm and is able to travel through solid objects and obstacles.}
        
\DeclareSpell{Locate Gate}{divination|V,  S,  F/DF (a small lodestone sphere)|1 standard action|long (400 ft. + 40 ft./level)|Area: circle centered on you with a radius of 400 ft. + 40 ft./level|1 minute/level|none|no}[ Find a nearby magical portal.]
    \DeclareSpellDescription{Locate Gate}{You sense the direction of the nearest teleportation circle (permanent or with a remaining duration), gate spell, or other effect which magically connects two different locations (for example, an active magic item, a creature's special ability, or unique adventure location). Locate gate detects only spells or effects with a permanent or ongoing duration, not instantaneous effects like dimension door or teleport.  Locate gate can be blocked by spells like nondetection, if the effect originates from a specific object or creature. However, spell effects can't likewise be warded. For example, ring gates are a specific object, but a permanent teleportation circle is not (the surface in which it's inscribed doesn't count). Locate gate isn't blocked by lead, water, or other physical environmental conditions, but it is blocked by any intervening area that is dimensionally warded (such as by dimensional lock or forbiddance).}
        
\DeclareSpell{Mad Sultan's Melody}{enchantment (compulsion) [mind-affecting,  sonic]|V,  S,  F (masterwork flute,  pipe,  or string instrument)|1 standard action|close (25 ft. + 5 ft./2 levels)|Targets: one creature/2 levels, no two of which may be more than 30 ft. apart (see text)|1 round/level (D)|Will negates|yes}[ Bizarre cacophony fascinates eldritch creatures.]
    \DeclareSpellDescription{Mad Sultan's Melody}{You imitate the mad cacophony created by the awful beings associated with the Outer God, Azathoth. This spell targets only creatures with the ooze type, creatures with the amorphous special ability, and non-bipedal creatures with a special association with the Outer Gods. This music draws the targets' attention to the caster, as per the fascinate bardic performance. It affects mindless creatures despite the mindless quality typically granting immunity to mind-affecting effects, though it doesn't ignore any other immunity to mind-affecting effects the creature might have. The caster doesn't have to maintain the effect each round-the music continues for the duration of the spell. Creatures fascinated by mad sultan's melody become immune to any other casting of the spell for 24 hours after the spell ends, and the spell ends if the fascination breaks on any of its targets for any reason (such as an attack).  If you have the bardic performance class feature and the fascinate bardic performance, you can choose to use the spell's saving throw DC or a DC equal to 10 + 1/2 your levels in the class that grants you bardic performance + your Charisma modifier, whichever is higher. If you choose to use the latter DC, each round of the melody costs 1 round of bardic performance and it counts as an active performance for determining how many performances you can have active.  This spell takes a toll on the caster. Each time you cast it, you take 1d4 points of Wisdom damage. If your campaign uses the sanity system (see page 12), you instead take 2d6 points of sanity damage.}
        
\DeclareSpell{Massacre}{necromancy [death]|V,  S,  M (a flask of ectoplasmic residue)|1 standard action||Area: 60-ft. line|instantaneous|Fortitude negates|yes}[ Slaughter creatures in a line.]
    \DeclareSpellDescription{Massacre}{You unleash a wave of necromantic energy that snuffs out the life force of those in its path. This wave pulses out from you in a line 5 feet wide and 30 feet long. The wave visibly rips the souls from the bodies of those it passes through, which manifest as screaming, transparent versions of the affected creatures. The wave kills every living creature of 17 or fewer HD in the line, starting with the creature closest to you, to a maximum of 1d4 HD of creatures  per caster level. No creature of 18 or more HD can be affected. If a creature succeeds at its saving throw or has too many HD, it doesn't count against the HD the spell can kill The wave continues to affect creatures as it rolls away from you until you either run out of HD to affect or reach the limit of the spell's area. If the spell does not kill any creatures, the unreleased necromantic energy violently explodes in the final square of the 60-foot line, dealing 10d6 points of damage + 1 point per caster level to any creature in that square with no saving throw. If several creatures occupy the same square, roll randomly to determine which is affected.}
        
\DeclareSpell{Maze Of Madness And Suffering}{conjuration (teleportation) [evil,  mind-affecting]|V,  S,  M (an ornate puzzle box worth 1, 000 gp and soaked in fresh blood)|1 standard action|close (25 ft. + 5 ft./2 levels)|Targets: one creature|see text|Will partial, see below|yes}[ Send a target into a dangerous extradimensional maze.]
    \DeclareSpellDescription{Maze Of Madness And Suffering}{This spell works like maze, except the DC of the Intelligence check to escape is 22 and each round a creature remains in the maze a different effect occurs depending on which section of the maze it is in. Roll on the following table each time the creature attempts an Intelligence check to escape the maze to see which part of the maze it wanders through; if a creature doesn't attempt an Intelligence check, it stays in the same section and suffers that section's effects. Any conditions or damage taken persist for the listed duration even if the creature exits the maze before then, but conditions don't stack with themselves.     d\%Maze Section1-20Circus21-40Haunted forest41-60Hellscape61-80Oasis of respite81-100Strange city Circus: The creature finds itself in a nightmarish circus of giant beasts, garishly painted faces, and hideous, mocking laughter. The creature must succeed at a Will save or gain a lesser madness (see page 182). The save uses the madness's normal DC.  Haunted Forest: The creature travels through a dark forest of grasping trees while shadowy beasts prowl at the edge of its vision. The creature must succeed at a Will save or become frightened for 2d4 rounds. This is a fear effect. A frightened creature can still attempt Intelligence checks to escape the maze, but it takes 3d6 points of slashing and piercing damage from the trees' branches.  Hellscape: The creature navigates a landscape of fire and stone walls carved with diabolical faces while the cries of the damned echo in the air. The creature must succeed at a Will save or become paralyzed by fear and potential torment for 1 round. This is a  fear effect. The creature can't attempt to escape the maze while paralyzed, and if it fails three consecutive saving throws against this paralysis, the hellscape delivers a coup de grace, dealing the creature 4d6 points of fire damage. The creature must succeed at a Fortitude save (DC = 10 + the damage dealt) or die. A creature that survives the coup de grace escapes from the paralysis as well.  Oasis of Respite: Whether it appears as a beautiful glade, a perfumed palace, or a literal desert oasis, this section of the maze is supernaturally peaceful, especially in comparison to the rest of the maze. The creature must succeed at a Will save or become fascinated for 1 round. The creature can't attempt to escape the maze while fascinated. A creature that fails three consecutive saving throws to this fascination enters a state of lethargy and gives up ever escaping the maze, remaining in this area until the spell ends on its own in 10 minutes (as per maze).  Strange City: The creature journeys through a city of cyclopean architecture under a sky of indescribable color and numerous stars. The creature must succeed at a Will save or take 2d4 points of Wisdom damage. If your campaign uses the sanity system (see page 12), the creature instead takes 2d10 points of sanity damage.}
        
\DeclareSpell{Night Terrors}{illusion (phantasm) [emotionUM,  evil,  mind-affecting]|V,  S,  M/DF (a drop of black ink)|1 standard action|touch|Targets: intelligent creature touched|1 day/level (D)|Will negates|yes}[ Disturb a creature’s rest with dark dreams.]
    \DeclareSpellDescription{Night Terrors}{The target of this spell gains no benefit from normal or magical sleep, writhing in a series of nightmares that torture its psyche and diminish its ability to perform strenuous tasks. The target doesn't heal ability or hit point damage naturally and can't prepare spells or regain spell slots. After one night of poor sleep, the target is fatigued (or exhausted if it was fatigued before trying to rest). A creature affected by this spell doesn't recover from the fatigued or exhausted condition inflicted by this spell after resting, nor do spells such as lesser restoration provide any respite. Each restless night, the target takes 1d4 points of Wisdom damage, which also can't be recovered by magic while night terrors is active.  The images from previous nightmares continue to haunt the target's mind while awake. If the target has rested at least once while affected by night terrors and then becomes subjected to a fear condition, the target experiences the next higher level of fear than it would normally. However, per the alternate rules for fear (see page 10), this spell cannot cause a lesser state of fear to become a greater one. If the effect causing the fear condition doesn't usually stack with other fear effects, the target's level of fear does not increase. This aspect of the spell is a fear effect.  The affected creature can attempt a new saving throw once per day to end night terrors, but multiple attempts to rest in a given day do not afford the target multiple saves. A creature that  successfully saves against night terrors ends the spell and rests normally that night but gains the benefits of only that night's rest, not any benefits missed on previous nights.  If your campaign uses the sanity system (see page 12), the target takes 1d8 points of sanity damage instead of Wisdom damage.}
        
\DeclareSpell{Pessimism}{enchantment (compulsion) [emotionUM,  mind-affecting]|V,  S,  M/DF (a shard of a broken mirror)|1 standard action|close (25 ft. + 5 ft./2 levels)|Targets: one creature|1 hour/level|Will negates|yes}[ Force a creature to see the negative side of things.]
    \DeclareSpellDescription{Pessimism}{You erode the target's confidence and instill a sense of despair. This fear of failure manifests as a -2 penalty on attack rolls, saving throws, ability checks, skill checks, and weapon damage rolls. Additionally, the target can't gain morale bonuses of any kind while the spell remains in effect. No amount of achievement counters the spell's effects-the target simply explains away positive events with a self-critical perspective.  Certain events can solidify the character's belief that it is doomed. Whenever the target rolls a natural 1 on an attack roll or saving throw, fails an ability or skill check by more than 5, or takes additional damage as a result of a foe's confirmed critical hit, the penalty to roll imposed by pessimism becomes -3 for 1 round. This increased penalty doesn't stack, even if multiple catastrophes occur on the same round.}
        
\DeclareSpell{Phantasmal Asphyxiation}{illusion (phantasm) [mind-affecting]|V,  S,  M (an empty vial)|1 standard action|medium (100 ft. + 10 ft./level)|Targets: one living creature|1 round/level (D)|Will disbelief, then Fortitude partial (see text)|yes}[ Trick a creature into thinking it can’t breathe.]
    \DeclareSpellDescription{Phantasmal Asphyxiation}{Your target must succeed at a Will save or believe it can no longer breathe. An affected target must attempt a Fortitude save each round at the beginning of its turn. The first time it fails the Fortitude save, it is staggered until the next time it attempts a save against the spell. If the target fails a second Fortitude save in a row, it falls unconscious for the spell's remaining duration. The target breathes normally while unconscious, but is shaken for 1 minute upon awakening. If the target succeeds at two Fortitude saves in a row, it shakes off the spell's effects entirely. Otherwise, the target continues to attempt a save each round until it falls unconscious or the spell ends. This spell has no effect on creatures that don't need to breathe.}
        
\DeclareSpell{Phantasmal Putrefaction}{illusion (phantasm) [fear,  mind-affecting]|V,  S|1 standard action|medium (100 ft. + 10 ft./level)|Targets: one creature/level, no two of which can be more than 30 ft. apart|1 round/level (D)|Will disbelief, then Fortitude partial (see text)|yes}[ Trick creatures into thinking their flesh is rotting.]
    \DeclareSpellDescription{Phantasmal Putrefaction}{You implant within the minds of your targets the illusion that their skin is rotting away, large rents are appearing all over their bodies, and their internal organs are spilling out into a putrid half-liquid mass at their feet. Those who fail to disbelieve phantasmal putrefaction immediately take 1d4 points of Wisdom damage. This damage occurs only once. Each round at the beginning of its turn, an affected target receives another Will save to disbelieve the effect, and targets that fail must succeed at a Fortitude save or faint, falling asleep as per sleep (except that it isn't a magical sleep effect). Waking up doesn't end the spell for a target; it must continue to attempt Will saves to disbelieve and Fortitude saves to avoid fainting each round until the spell ends or the target successfully disbelieves.  Targets of the spell perceive everyone else around them to be rotting away, but other creatures see no visible effect of the spell, so they, in addition to those who disbelieve, can communicate the nature of the illusion to allies, providing those allies with a +4 bonus on the saving throw to disbelieve.  If your campaign uses the sanity system (see page 12), a creature takes 1d8 points of sanity damage instead of Wisdom damage.}
        
\DeclareSpell{Phobia}{enchantment (compulsion) [emotionUM,  fear,  mind-affecting]|V,  S,  M (a single white hair)|1 standard action|close (25 ft. + 5 ft./2 levels)|Targets: one intelligent creature|instantaneous|Will negates|yes}[ Induce an irrational fear in a creature to the point of madness.]
    \DeclareSpellDescription{Phobia}{You instill the target with an intense, instinctual fear of a condition or circumstance, more powerful than the phobia lesser madness (see page 182). You can name an energy type (acid, cold, electricity, fire, or sonic), a hazard (such as an avalanche or earthquake), or a single creature of the animal type or all vermin (applying when the target sees a swarm or a single Small or larger creature). Alternatively, you can name the following specific environments: darkness (darker than dim light, and you can't apply this phobia to a creature with natural racial darkvision, see in darkness, blindsight, or similar senses), enclosed spaces (places that require the creature to squeeze),heights (10 times the  target's height, and you can't apply this phobia to creatures with a natural fly speed), or water (you can't apply this phobia to aquatic creatures or creatures with a natural swim speed).  When the target takes damage of the energy type (for an energy type phobia), or perceives the presence of the creature, environment, or hazard, it must attempt a DC 20 Will save. If it fails, it becomes panicked, but even if it succeeds, it becomes shaken and feels intensely uncomfortable; a creature shaken in this way does not need to roll further saving throws against its phobia until its shaken condition ends, even if it continues to be exposed to its phobia.  A creature panicked by phobia can begin to act normally 1 minute after it ceases being able to perceive its phobia or after taking the energy damage if the phobia is an energy type, though the shaken condition ends immediately after the creature can no longer perceive its phobia. Break enchantment, heal, limited wish, miracle, or wish can remove a phobia spell.}
        
\DeclareSpell{Plundered Power}{necromancy [evil]|V,  S,  M (crushed rubies worth 2, 500 gp and a major organ from the sacrificed target,  see text)|10 minutes|close (25 ft. + 5 ft./2 levels)|Targets: one creature|1 day/level|Will negates|no}[ Kill a creature and steal its strongest spell-like ability.]
    \DeclareSpellDescription{Plundered Power}{You must sacrifice a creature just as you finish casting this spell. When the sacrifice dies, its blood or ichor pools and hardens into a single red bloodstone. This bloodstone contains the essence and spirit of the sacrificed creature, allowing anyone holding the stone to activate the creature's highest-level activated racial spell-like ability, taking the actions necessary to activate the stored spell-like ability. Any creature who holds the stone can activate it, but the stone can be used only once per day (or the frequency of the creature's spell-like ability, whichever is less). If a creature has several racial spell-like abilities of the same spell level, choose randomly from among them to determine which one is stored in the bloodstone. The bloodstone can only store racial spell-like abilities that emulate a spell, including altered spells, like invisibility (self only), but not unique spell-like abilities possessed by the creature. Abilities with altered effects are still altered, so a creature activating the bloodstone for greater teleport (self only) would affect only itself and a bloodstone storing summon monster II (Small fire elemental only) would still only be able to summon a Small fire elemental. Additionally, the bloodstone can't store spell-like abilities that duplicate spells with expensive components costing more than 250 gp.  The save DC, caster level, and other attributes of this ability remain the same as when the creature was still alive, though the creature who activates the bloodstone can choose the targets, area, or shape of the spell, and make other decisions, such as  controlling the target of dominate person. The bloodstone can take any actions necessary to manipulate aspects of the spell, concentrating on the spell, redirecting the spell, and so on, up to a full normal round's worth of actions.  If the sacrificed creature doesn't die during the casting of the spell (for instance if it was merely a summoned creature) or is brought back to life before the spell's duration expires, the bloodstone loses its magical power. The fragment of the creature's spirit trapped in the bloodstone does not hinder attempts to restore the creature to life.}
        
\DeclareSpell{Profane Nimbus}{evocation [evil]|V,  S,  DF|1 standard action|personal|Targets: you|1 round/level (D)||}[ Unholy energy damages good creatures that attack you and protects you from good attacks.]
    \DeclareSpellDescription{Profane Nimbus}{You are surrounded by a nimbus of shadow shaped like your god's unholy symbol or a symbol of your faith. Any good creature striking you with unarmed strikes, natural weapons, or a handheld weapon deals normal damage, but at the same time, the attacker takes 1d6 points of damage + 1 point per caster level (maximum +15). Creatures wielding melee weapons with reach are not subject to this damage if they attack you. Spell resistance applies against this damage. You also take half damage from magical attacks with the good descriptor. If such an attack allows a Reflex save for half damage, you take no damage on a successful saving throw.}
        
\DeclareSpell{Pyrotechnic Eruption}{evocation [fire]|V,  S,  M (a pinch of coal)|1 standard action|medium (100 ft. + 10 ft./level)|Targets: one creature|see text|Reflex half (see text)|yes}[ Erupting flames burn a target several times.]
    \DeclareSpellDescription{Pyrotechnic Eruption}{The caster causes jets of flame to erupt from the ground and surround the target. The target takes 1d6 points of damage per caster level (maximum 15d6) unless it succeeds at a Reflex save for half damage. The blaze surrounds the target for the duration of the spell, forcing the creature to attempt a new save each round. Each round, the damage dealt is reduced to half as many d6; the spell expires when it would deal no damage. If the target moves, the pyrotechnic eruption follows, even if the target teleports.  Anyone attempting to touch the target takes damage, using the same amount of dice as the last time the target attempted a save (Reflex half). A creature can take the place of the target by bull rushing or grappling it and switching places. The new creature then automatically takes the current round's damage with no saving throw and can begin to attempt Reflex saves starting on its next turn.}
        
\DeclareSpell{Quick Change}{transmutation (polymorph)|V,  S|1 standard action|personal|Targets: you|1 hour/level||}[ Use change shape as a swift action and surprise foes.]
    \DeclareSpellDescription{Quick Change}{If you have the change shape special quality, you can revert to your true form as a swift action. If a creature is not aware of your true form, when you use this spell to revert to your true form and attack that creature in the same round, the creature is denied its Dexterity bonus to AC against your first attack.}
        
\DeclareSpell{Rigor Mortis}{transmutation [painUM]|V,  S,  M/DF (a knucklebone)|1 standard action|medium (100 ft. + 10 ft./level)|Targets: one living creature|instantaneous; see text|Fortitude partial (see text)|yes}[ Painfully swell a target’s joints.]
    \DeclareSpellDescription{Rigor Mortis}{The joints of a creature affected by this spell stiffen and swell, making movement painful and slow. The target takes 1d6 points of nonlethal damage per caster level. Additionally, the target takes a -4 penalty to Dexterity and its movement speed decreases by 10 feet; these additional effects last for 1 minute per caster level, though another creature can spend 1 minute and attempt a DC 25 Heal check to end them early. A successful save halves the nonlethal damage and negates the penalty to Dexterity and movement.}
        
\DeclareSpell{Sacramental Seal}{necromancy|V,  S,  F (an object worth at least 2, 000 gp)|1 round|touch|Targets: creature touched|instantaneous|Will negates|yes}[ Seal a creature away inside the spell’s focus.]
    \DeclareSpellDescription{Sacramental Seal}{You trap the target in an object decorated with the holy symbols of your god or faith. While trapped in the object, the creature can't take any actions and is immune to spells and spell-like abilities. The creature remains permanently trapped in the object as long the object remains in your possession. Only a freedom, miracle, or wish spell can dispel the enchantment, though destroying the object frees the creature.  If you relinquish your stewardship of the object (such as giving it away or leaving it in a remote location or extradimensional space), the trapped creature begins to gain control over the object. It immediately gains the ability to communicate telepathically with any creature now in possession of the object. It still can't take any actions besides communicating but can use feats and skills related to speaking (such as Bluff and Diplomacy).  After 1 week of the object being out of your presence, the creature can create a number of haunts with a total CR (that is, the CR of the encounter with all of the haunts at once) equaling 1/4 the creature's Hit Dice. These haunts are centered on the object. The creature can also communicate telepathically up to a range of 100 feet at this point.  After 1 month of the object being out of your presence, the CR total of the haunts the trapped creature can create increases to 1/2 its Hit Dice. In addition to telepathy, it can also impart mental images of its choosing into the mind of any creature holding or carrying the object.  After 1 year of being out of your presence, in addition to the above abilities, the creature can attempt to possess any living  creature with an Intelligence score of 3 or higher that touches the object, as per possessionOA. However, the creature can't personally destroy the object, even while possessing another creature.  Because the binding magic irrevocably weakens the longer you're away from the item, returning it to your ownership doesn't reverse any of the effects. You must free the creature and impose another sacramental seal if you want to restrict its abilities again.  If the object is placed in the stewardship of creatures or a location belonging to your faith, it still counts as being out of your presence but it takes ten times longer for the creature to manifest the above abilities (it would take 10 weeks for it to manifest the ability to create haunts, for example).}
        
\DeclareSpell{Sacred Nimbus}{evocation [good]|V,  S,  DF|1 standard action|personal|Targets: you|1 round/level (D)||}[ Holy energy damages evil creatures that attack you and protects you from evil attacks.]
    \DeclareSpellDescription{Sacred Nimbus}{You are surrounded by a nimbus of golden light shaped like your god's holy symbol or a symbol of your faith. Any evil creature striking you with unarmed strikes, natural weapons, or a handheld weapon deals normal damage, but at the same time, the attacker takes 1d6 points of damage + 1 point per caster level (maximum +15). Creatures wielding melee weapons with reach are not subject to this damage if they attack you. Spell resistance applies against this damage. You also take half damage from magical attacks with the evil descriptor. If such an attack allows a Reflex save for half damage, you take no damage on a successful saving throw.}
        
\DeclareSpell{Screaming Flames}{evocation [evil,  fire,  mind-affecting]|V,  S,  M/DF (a charred animal or humanoid skull fragment)|1 standard action|close (25 ft. + 5 ft./2 levels)|Effect: a sheet of flame up to 15 ft. long and 10 ft. high that moves 15 ft. in a straight line.|instantaneous|Reflex half and Will negates (see text)|yes}[ Send forth a wave of flames screaming with the agony of the damned.]
    \DeclareSpellDescription{Screaming Flames}{A sheer wall of flame appears and rushes away from you. Tendrils of fire reach out of it, shaped into skulls screaming in agony. When the wall passes through a creature's space, that creature must succeed at a Reflex save or take 1d8 points of fire damage for every 2 caster levels you have (maximum 5d8). Any creature taking fire damage must also succeed at a Will save or take 1d3 points of Wisdom damage. Deaf creatures receive a +4 circumstance bonus on their Will saves.  If your campaign uses the sanity system (see page 12), a creature takes 1d6 points of sanity damage instead of Wisdom damage.}
        
\DeclareSpell{Sense Fear}{divination|V,  S,  M/DF (a patch of fur)|1 standard action|personal|Targets: you|10 minutes/level (D)||}[ Perceive nearby creatures that are experiencing fear.]
    \DeclareSpellDescription{Sense Fear}{You sense the fear of those nearby, feeling it in the air around you. You can detect spooked, shaken, scared, frightened, panicked, terrified, and horrified creatures within 30 feet, and you immediately know what level of fear they are experiencing. You determine the creatures' approximate direction from you, though you do not sense a given creature's identity or know which squares the creatures occupy. If you're within 5 feet of a creature whose fear you sense, you pinpoint that creature's location, as if using blindsense.}
        
\DeclareSpell{Sense Madness}{divination|V,  S|1 standard action|close (25 ft. + 5 ft./2 levels)|Targets: creatures in range (see text)|concentration, up to 1 round/level (D)|none|no}[ Determine mental disturbances in nearby creatures.]
    \DeclareSpellDescription{Sense Madness}{You can sense the presence of mental disturbance in creatures within range, focusing your detection on one creature each round. If the target is currently suffering from any form of madness (see page 182) or addiction (GameMastery Guide 236), has total sanity damage greater than or equal to its sanity edge (see page 12), or is under a magical compulsion that could be detected by detect magic, you can detect the presence of such a disturbance on the first round of concentration.  If you concentrate on the same creature for an additional round, you learn if the creature's total sanity damage is greater than or equal to its sanity edge, and you can attempt a Sense Motive check against a DC equal to 10 + the save DC of the target's madness or addiction to determine the exact nature of a madness or addiction. Additionally, you can attempt a Spellcraft check with a DC equal to 11 + the caster level of the compulsion against each magical compulsion currently affecting the target that could be detected by detect magic. If successful, you can identify the specific spell affecting the target. You gain a +4 bonus on your Spellcraft check if the effect is confusion, insanity, or a similar effect that explicitly causes madness or insanity. If you successfully identify such an effect with sense madness, the insights provided  by this spell grant you a +1 circumstance bonus on your next caster level check in the following 1 minute made to counter, dispel, or remove that specific effect. After attempting a check to identify a madness, addiction, or compulsion, you can't attempt that check against the same creature again, even if you concentrate on the creature again.}
        
\DeclareSpell{Sleepwalking Suggestion}{enchantment (compulsion) [mind-affecting]|V,  S,  M (two doses of oil of taggit worth 180 gp total)|1 standard action|close (25 ft. + 5 ft./2 levels)|Targets: one creature|24 hours|Will negates or none (see text)|yes}[ Cause a creature to perform a suggested action while asleep.]
    \DeclareSpellDescription{Sleepwalking Suggestion}{You compel the target creature to rise from its sleep as if sleepwalking and perform a course of activity (limited to a sentence or two). The spell takes effect immediately if the target is already asleep, or as soon as the target falls asleep if it is conscious. You can specify that the creature should wait before taking action when you give your instructions if you so choose. A sleeping creature doesn't get a save against this spell. The suggested activity must not cause the target to directly harm itself or others, but it doesn't need to be reasonable. For example, you could instruct the target to unlock doors and windows or poison its ally's rations, but not jump off a roof or perform a coup de grace action against a sleeping ally. Instructions that cause direct harm prompt the target to awaken just prior to performing the action and the target experiences a vague memory of what it was supposed to do.  The target moves at its speed, but isn't capable of running or moving at a higher rate of speed. It moves in dark conditions as if it had darkvision and can perform most simple and skill-based actions, but it can't engage in combat, spellcasting, or actions that require significant cognitive awareness (like making complex decisions, solving puzzles, or using complicated magic items). If the creature takes any damage while sleepwalking, it must attempt a new saving throw. If it succeeds, the spell ends and the creature awakens. When the target completes the suggested course of activity, or when the spell ends or is dismissed, the target remains unconscious and returns to where it was sleeping and must be awoken normally. It retains no memory of what it did while unconscious (with the only exception described above).  Unlike the sleepwalkAPG spell, attempts to use sleepwalking suggestion on a creature that is unconscious for any reason other than sleep automatically fail. A creature affected by sleepwalking suggestion is unaware it has been programmed to act in its sleep (unless awoken because it was about to cause harm).}
        
\DeclareSpell{Slough}{transmutation [evil]|V,  S,  M/DF (a pinch of dried skin flakes)|1 standard action|medium (100 ft. + 10 ft./levels)|Targets: one living creature|instantaneous|Fortitude negates|yes}[ Slough off a target’s skin.]
    \DeclareSpellDescription{Slough}{The target takes 1d4 points of Constitution damage each round as its skin loosens and splits. Once it has taken 5 or more points of Constitution damage in this way, its skin falls off to reveal its musculature. The creature ceases taking Constitution damage, but takes a -4 penalty on saving throws against disease, pain effects, or poison, and on Charisma-based skill checks with the exception of Intimidate and Use Magic Device. The creature's skin regrows rapidly-once its Constitution damage is fully healed, its skin becomes intact once again and the penalties end.}
        
\DeclareSpell{Stave Off Corruption}{abjuration|V,  S,  M (see text)|1 standard action|touch|Targets: one creature|1 day/2 caster levels|Will negates (harmless)|yes (harmless)}[ Protect against a corruption’s progression.]
    \DeclareSpellDescription{Stave Off Corruption}{You ward the target to slow the advancement of its corruption. The target receives a +2 circumstance bonus on saving throws against the advancement of its corruption, as described in the catalyst section of each corruption. Multiple applications of this spell do not stack, nor does the bonus stack with morale, profane, or sacred bonuses on saving throws against the advancement of the target's corruption.  The material components for this spell vary with the type of corruption. For example, garlic is used for a vampirism corruption, a sprig of wolfsbane is used for a lycanthropy corruption, and so on. The GM can determine appropriate material components, but the components should cost at least 25 gp.}
        
\DeclareSpell{Straitjacket}{conjuration (creation)|V,  S,  M (leather strap)|1 standard action|close (25 ft. + 5 ft./2 caster levels)|Targets: one creature|1 round/level (D)|none|yes}[ Restrain a creature’s arms and grant it a second saving throw against certain magic.]
    \DeclareSpellDescription{Straitjacket}{You cause a straitjacket to spring into existence to restrain the target creature. You attempt a combat maneuver against the target's CMD. Your CMB for this check is equal to your caster level + 5 due to the straightjacket's Strength. This combat maneuver doesn't provoke an attack of opportunity, and if it succeeds, the target is restrained. This is similar to the grappled condition, except the target is able to move at its normal speed, the restraint doesn't cause the target to require concentration checks to use abilities that require concentration, and the target can't take actions that require one or more hands or arms to perform (including casting spells that require a somatic or material component).  Once it has been restrained for 1 round, the target can attempt an additional saving throw against a magical effect that makes it confused, a curse, a fear effect, or a madness. The caster chooses which effect the target gets an extra save against. This extra save applies to only effects with durations of 10 minutes per level or less, and it doesn't apply to instantaneous effects. This extra save occurs only once per casting of straitjacket.  A creature can attempt to escape the straitjacket with a combat maneuver check or Escape Artist check, with a DC equal to 10 + your caster level + your Intelligence, Wisdom, or Charisma modifier, whichever is highest.}
        
\DeclareSpell{Symbol Of Exsanguination}{necromancy [evil]|V,  S,  M (powdered garnet and bloodstone worth a total of 500 gp)|10 minutes|0 ft.; see text|Effect: one symbol|10 minutes/level|Fortitude negates|yes}[ Triggered rune causes nearby creatures to bleed.]
    \DeclareSpellDescription{Symbol Of Exsanguination}{This spell functions like symbol of death, except that each creature within the radius of symbol of exsanguination begins to bleed uncontrollably and violently from their eyes, ears, nose, and mouth, as well as any existing open wounds. Creatures that failed their saving throws immediately take 1d6 points of bleed damage and are sickened as long as they continue to take bleed damage from the symbol.  While a DC 15 Heal check or magical healing can end the bleed damage as normal, the bleed damage and sickened condition begin again each round so long as the creature begins its turn still within 60 feet of the symbol, including if the creature leaves the area and returns. Creatures that succeed at their saving throw suffer no ill effects from the symbol, even if they leave its area and return.  Unlike symbol of death, symbol of exsanguination has no hit point limit on the creatures it can affect. Once triggered, a symbol of exsanguination remains active for 10 minutes per caster level.  Magic traps such as symbol of exsanguination are hard to detect and disable. While any character can use Perception to find a symbol, only a character who has the trapfinding class feature can use Disable Device to disarm it. The DC in each case is equal to 25 + spell level, or 28 (27 for a bloodrager) for symbol of exsanguination.\\\\

{\centering\bf Symbol Of Death\hrule}

This spell allows you to scribe a potent rune of power upon a surface.

When triggered, a symbol of death kills one or more creatures within 60 feet of the symbol (treat as a burst) whose combined total current hit points do not exceed 150. The symbol of death affects the closest creatures first, skipping creatures with too many hit points to affect.

Once triggered, the symbol becomes active and glows, lasting for 10 minutes per caster level or until it has affected 150 hit points' worth of creatures, whichever comes first. A creature that enters the area while the symbol of death is active is subject to its effect, whether or not that creature was in the area when it was triggered. A creature need save against the symbol only once as long as it remains within the area, though if it leaves the area and returns while the symbol is still active, it must save again.

Until it is triggered, the symbol of death is inactive (though visible and legible at a distance of 60 feet). To be effective, a symbol of death must always be placed in plain sight and in a prominent location. Covering or hiding the rune renders the symbol of death ineffective, unless a creature removes the covering, in which case the symbol of death works normally.

As a default, a symbol of death is triggered whenever a creature does one or more of the following, as you select: looks at the rune; reads the rune; touches the rune; passes over the rune; or passes through a portal bearing the rune. Regardless of the trigger method or methods chosen, a creature more than 60 feet from a symbol of death can't trigger it (even if it meets one or more of the triggering conditions, such as reading the rune). Once the spell is cast, a symbol of death's triggering conditions cannot be changed.

In this case, "reading" the rune means any attempt to study it, identify it, or fathom its meaning. Throwing a cover over a symbol of death to render it inoperative triggers it if the symbol reacts to touch. You can't use a symbol of death offensively; for instance, a touch-triggered symbol of death remains untriggered if an item bearing the symbol of death is used to touch a creature. Likewise, a symbol of death cannot be placed on a weapon and set to activate when the weapon strikes a foe.

You can also set special triggering limitations of your own. These can be as simple or elaborate as you desire. Special conditions for triggering a symbol of death can be based on a creature's name, identity, or alignment, but otherwise must be based on observable actions or qualities. Intangibles such as level, class, HD, and hit points don't qualify.

When scribing a symbol of death, you can specify a password or phrase that prevents a creature using it from triggering the symbol's effect. Anyone using the password remains immune to that particular rune's effects so long as the creature remains within 60 feet of the rune. If the creature leaves the radius and returns later, it must use the password again.

You also can attune any number of creatures to the symbol of death, but doing this can extend the casting time. Attuning one or two creatures takes negligible time, and attuning a small group (as many as 10 creatures) extends the casting time to 1 hour. Attuning a large group (as many as 25 creatures) takes 24 hours. Attuning larger groups takes an additional 24 hours per 25 creatures. Any creature attuned to a symbol of death cannot trigger it and is immune to its effects, even if within its radius when it is triggered. You are automatically considered attuned to your own symbols of death, and thus always ignore the effects and cannot inadvertently trigger them.

Read magic allows you to identify a symbol with a Spellcraft check (DC 10 + the symbol's spell level). Of course, if the symbol is set to be triggered by reading it, this will trigger the symbol.

A symbol of death can be removed by a successful dispel magic targeted solely on the rune. An erase spell has no effect on a symbol of death. Destruction of the surface where a symbol of death is inscribed destroys the symbol but also triggers it.

Symbol of death can be made permanent with a permanency spell.

A permanent symbol of death that is disabled or has affected its maximum number of hit points becomes inactive for 10 minutes, but then can be triggered again as normal.

Note: Magic traps such as symbol of death are hard to detect and disable. A rogue (only) can use the Perception skill to find a symbol of death and Disable Device to thwart it. The DC in each case is 25 + spell level, or 33 for symbol of death.}
        
\DeclareSpell{Temporary Graft}{transmutation|V,  S,  F (a dismembered body part)|1 standard action|personal|Targets: you|1 minute/level (D)||}[ Graft a body part onto yourself to gain one of several benefits.]
    \DeclareSpellDescription{Temporary Graft}{You temporarily graft a dismembered body part onto yourself. You must have the body part in your possession or the spell has no effect. The graft must come from a creature the same size category as you. You can have only one graft active at a time. Grafting on a new body part immediately ends the effects of the older casting and causes that body part to fall off. The effect of the graft depends on the type of body part you choose to graft to yourself, as follows.  Fins: The fins grant you a 40 foot swim speed.  Head: The head grants you all-around vision. If the head came from a creature that had darkvision, low-light vision, or scent, you gain the appropriate senses at half the normal range. You can speak through either head, but not both simultaneously. You don't receive any additional special abilities the head might have had (like a medusa's petrifying gaze ability).  Leg: The leg increases your base movement speed by 5 feet and grants you a +2 bonus to CMD against overrun and trip combat maneuvers (the usual bonus for having an extra leg).  Wings: The wings grant you a 40-foot fly speed (poor maneuverability).}
        
\DeclareSpell{Torpid Reanimation}{necromancy [evil]|V,  S,  M (an onyx gem worth at least 25 gp per HD of the undead)|1 standard action|touch|Targets: one or more corpses touched|1 day/caster level and instantaneous (see text)|none|no}[ Animate dead when a specific trigger condition occurs.]
    \DeclareSpellDescription{Torpid Reanimation}{This spell works like animate dead except you set a specific condition for when the animation occurs and you do not immediately have control of the creatures. The condition can be simple, such as being touched, a certain word spoken, or on hearing a specific sound. You can also set a condition based on time, but it must occur within a number of days equal to your caster level. At any time, you can cause the animation to occur as a standard action.  You can gain control of any uncontrolled undead you created with this spell as a free action if you are within at least 60 feet of the undead creature and have line of sight to it; if you have control of an undead you created with this spell and later lose control of that undead, you can't use this ability to regain control of the same undead. The limit on the number of Hit Dice you can control with animate dead still applies.}
        
\DeclareSpell{Verminous Transformation}{transmutation (polymorph)|V,  S,  M (a handful of bat wings,  insects,  and rat tails)|1 standard action|personal|Targets: you|1 round/level (D)||}[ Partially transform into a swarm.]
    \DeclareSpellDescription{Verminous Transformation}{You partially transform your body into a swarm of bats, insects, rats, and spiders. As a standard action, you can send out a piece of your body to attack up to four Medium or smaller creatures (or one larger creature) within 10 feet of you. The creatures take 4d6 points of damage and must succeed at a Fortitude save or take 1d3 points of Constitution damage and 1d3 points of Strength damage; the ability damage is a poison effect. Abilities that protect against swarm attacks apply to this damage and spell resistance applies.  Additionally, since your body is partially a swarm, you take only half damage from piercing and slashing attacks, unless those attacks would deal full damage or greater to a swarm (for instance, an area piercing attack would deal full damage to you). You can still be targeted by single-target spells but you gain a +2 circumstance bonus on any saving throws against them.  Casting spells in this form is difficult and any spells with somatic components require a concentration check as if casting defensively; if you are also casting the spell defensively, roll a single concentration check at a -5 penalty.}
        
\DeclareSpell{Vile Dog Transformation}{transmutation (polymorph) [evil]|V,  S,  M (a strip of leather)|10 minutes|touch|Targets: one or more dogs touched|1 hour/level|Will negates (see text)|no}[ Transform ordinary dogs into fiendish minions.]
    \DeclareSpellDescription{Vile Dog Transformation}{You transform one or more dogs into evil, monstrous creatures. Each dog can attempt a Will save to negate the transformation, but if the dog trusts you (it has been trained by you for a purpose, or has been in your care for at least 30 days and generally treated well), it takes a -4 penalty on this saving throw. Each transformed dog has the same stats as a hell hound (Bestiary 173) except that it doesn't have fire immunity and cold vulnerability and instead has acid, cold, and fire resistance 5. Additionally, each transformed dog deals an extra 1d6 points of acid damage with its bite attack (instead of fire damage), and its breath weapon is replaced with a vaporous cloud that spreads out in front of the creature in a 10-foot cone, dealing 2d4 points of acid damage. The creature can't understand Infernal but understands any languages you know. The dog is normally neutral evil, but is chaotic evil if you are chaotic or lawful evil if you are lawful.  For every 3 caster levels you have, you can transform one dog (to a maximum of five dogs at 15th level). At the end of the spell's duration, the dogs immediately dissolve into a stinking pile of gore and bones.  You can command the creature in the same way you would a creature you summoned via a summon monster spell. If you buy dogs to use with this spell, they cost 15 gp for a lap dog, 25 gp for a guard dog, or 150 gp for a riding dog (Pathfinder RPG Ultimate Equipment 82). Any kind of dog can be transformed by this spell and has the same statistics.}
        
\DeclareSpell{Waves Of Blood}{conjuration (creation)|V,  S,  M (a drop of the caster's blood)|1 standard action|30 ft.|Area: cone-shaped burst|instantaneous and 1 round; see text|Reflex negates, Fortitude negates (see text)|no}[ A cone of blood pushes creatures, sickens them, and makes the ground slick.]
    \DeclareSpellDescription{Waves Of Blood}{You cause torrents of roiling blood to push your opponents away from you. This wave attempts a bull rush combat maneuver against all creatures within its area of effect, and you bull rush creatures of any size in this way. Attempt a single combat maneuver check and apply the result to each creature within the area. Your CMB for this bull rush is equal to your caster level plus your Intelligence, Wisdom, or Charisma modifier, whichever is highest. This bull rush doesn't provoke attacks of opportunity. Any creature in the area  must also succeed at a Fortitude saving throw or become sickened for 1d6 rounds by the tide of blood.  The area covered by the cone remains slick for 1 round, requiring a successful DC 10 Acrobatics check from any creature attempting to move within it (as if moving on uneven ground). Any creature that falls prone due to failing the check must succeed at a Fortitude save or become sickened until it stands back up.}
        
\DeclareSpell{Wither Limb}{necromancy|V,  S|1 standard action|touch|Targets: living creature touched|instantaneous|Fortitude negates|yes}[ Make one of the target’s limbs useless.]
    \DeclareSpellDescription{Wither Limb}{You cause one limb of the target to shrivel and weaken. The target takes 2d6 points of damage. The points are permanently lost until the target's limb is restored by heal, limited wish, miracle,  regenerate, or wish. Wither limb affects only living creatures of the humanoid or monstrous humanoid type, or similar bipedal creatures at the GM's discretion. You choose the limb affected. This spell's effects stack, until all a target's arms and legs (or equivalent limbs) are withered.  Arm: The target looses the use of one arm, which might affect what weapons and shields it can use. The creature must immediately drop all objects held in the withered limb, though it can shift a two-handed weapon it is holding in both hands to a remaining arm as an immediate action. Worn objects-gauntlets, rings, and magic items in the wrist slot-remain. This prevents the creature from using two-weapon fighting, claw attacks from that arm, and so on. If multiple castings of this spell wither all of a creature's arms, it can't manipulate objects or cast spells requiring somatic components.  Leg: The creature's movement speeds are halved, except for flying or other speeds that don't involve its legs. If multiple castings of this spell wither all of a creature's legs, it can only crawl at a speed of 5 feet each round.  Wing: The creature loses access to any fly speed that depends on its wings if even a single wing is withered.}
    
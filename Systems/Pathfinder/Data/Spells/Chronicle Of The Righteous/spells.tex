    
\DeclareSpell{Blood Of The Martyr}{necromancy|V,  S|1 standard action|medium (100 ft. + 10 ft./level)|Targets: one living creature|1 round/level|Fortitude negates|yes}[]
    \DeclareSpellDescription{Blood Of The Martyr}{You cause the target to bleed from every orifice, and her organs and blood become suffused with positive energy. If the subject fails her Fortitude save, she takes 1d6 points of bleed damage per 4 caster levels (to a maximum of 4d6 at 16th level) when you cast this spell. Any creature that takes a full-round action to sup the blood of the bleeding subject heals a number of hit points equal to twice as many as the subject lost that round due to the bleed effect. The subject must be willing or helpless to sup her blood, which provokes attacks of opportunity. Only one creature can be healed in this way per round. The subject can lick her own wounds in this way to regain half as many hit points as she lost that round. If the bleeding effect is stopped or the spell's duration ends, the subject's blood no longer heals those who drink it, though in the latter case the subject continues to bleed until the bleeding is stopped via magical healing or a successful DC 15 Heal check.}
        
\DeclareSpell{Charitable Impulse}{enchantment (compulsion) [mind-affecting]|V,  S,  F/DF (a miniature collection plate)|1 standard action|close (25 ft. + 5 ft./2 levels)|Targets: one humanoid creature|1 round/level|Will negates|yes}[]
    \DeclareSpellDescription{Charitable Impulse}{This spell makes a creature more charitable, compelling it to aid others rather than use violence. An affected creature practices nonviolent combat behaviors according to the following list of priorities, beginning with the first priority. The subject continues to perform a priority until he can no longer fulfill its demands (at which point he moves to the next priority) or until the spell ends, whichever comes first.  1st Priority: Heal injured creatures within 30 feet, beginning with the closest creatures and using whatever methods the subject has at hand (including potions, spells, and so on).  2nd Priority: The subject gives his weapon away to the nearest creature within 30 feet who will accept it. If no creature accepts the weapon, the subject drops the weapon on the ground.  3rd Priority: Cast beneficial spells and/or use beneficial magic items (including potions, wands, and so on) on creatures within 30 feet, starting with the closest creatures.  4th Priority: The subject gives away his non-worn possessions-the contents of a backpack or similar item count as one item each, as does the container itself-to creatures within 30 feet. If no creature accepts the items, the subject drops the items on the ground.  5th Priority: The subject gives away his remaining possessions (including his armor, boots, cloak, and so on) to creatures within 30 feet. If no creature accepts the items, the subject drops them on the ground.  If the subject fulfills all five priorities, the spell effect ends. The subject cannot attack or take attacks of opportunity, but can defend himself as normal. If the subject is attacked, the spell's effect immediately ends.}
        
\DeclareSpell{Elemental Assessor}{evocation [acid,  cold,  electricity,  fire]|V,  S,  M/DF (four needles)|1 standard action|close (25 ft. + 5 ft./2 levels)|Effect: one elemental ray|1d4+1 rounds (see text)|none|yes}[]
    \DeclareSpellDescription{Elemental Assessor}{Azata champions developed this spell to deal with fiends with unknown resistances. A ray of spiraling colors springs from your hand and streaks to its target. You must make a successful ranged touch attack to hit your target with the ray, which deals 2d6 points of acid damage, 2d6 points of cold damage, 2d6 points of electricity damage, and 2d6 points of fire damage. The type of energy that does the most points of damage to the target then persists, dealing another 4d6 points of that type of damage per round for 1d4 rounds.}
        
\DeclareSpell{Sanctify Weapons}{transmutation|V,  S,  DF|1 standard action|close (25 ft. + 5 ft./2 levels)|Area: 20-ft.-radius spread|1 round/level|Will negates (harmless, object)|yes (harmless, object)}[]
    \DeclareSpellDescription{Sanctify Weapons}{This spell originated among the armies of Heaven. Choose a specific subtype of evil outsider when you cast this spell, such as daemon, demon, devil, or div. All manufactured weapons in the area of effect bypass the DR of that type of outsider. The weapons do not become aligned or change composition.}
        
\DeclareSpell{Summon Stampede}{conjuration (summoning)|V,  S,  M (piece of fur from a herd animal)|1 round|medium (100 ft. + 10 ft./level)|Effect: 20-ft.-radius herd of animals|1 round/level|Reflex halves (see text)|no}[]
    \DeclareSpellDescription{Summon Stampede}{You conjure a herd of aurochs or similar herd animal that immediately stampedes in the direction you indicate. The herd takes up a 20-foot-radius space and moves at a rate of 120 feet per round in a straight line. Any creatures caught in the herd's path take 4d6+9 points of damage that round as they are trampled beneath dozens of animals' hooves. A successful Reflex save halves this damage.  If the stampede's path would put it in an obviously dangerous area (such as over a cliff or through a fire) or force it to move through a solid barrier, the herd stops at the obstacle and moves in a new randomly determined direction until it reaches another obstacle or the spell's duration ends.}
        
\DeclareSpell{Vinetrap}{conjuration|V,  S,  DF|10 minutes|long (400 ft. + 40 ft./level)|Area: radius spread of up to 10 ft./level, 90 ft. high|1 hour/level (D)|Reflex negates (see text)|yes}[]
    \DeclareSpellDescription{Vinetrap}{Vines choked with thorns, blossoms, leaflets, and other floral debris burst to life on and around the subject of this spell, winding around limbs and armor and making it progressively more difficult for the subject to maneuver. When this spell is cast, the subject may attempt a Reflex save. Success indicates that the vines fail to take root and the spell has no effect. On a failed save, the subject's base speed is immediately reduced by 5 feet. Each round thereafter, the subject must make another successful Reflex save or his speed is reduced by another 5 feet. This occurs each round until he is reduced to a speed equal to half of what it was before the spell was cast.  As a full-round action, the subject or an adjacent creature can tear the vines off the subject's body, resetting the speed penalty to just 5 feet, though the vines continue to grow each round thereafter for the spell's duration, requiring additional checks, unless it is actually dispelled. The spell's effects can also be prematurely ended by dealing at least 20 points of fire damage to the subject. When the spell's duration ends or the effect is terminated, the vines immediately wilt and the subject's speed returns to normal. Regardless of the subject's base speed, the vines cannot reduce a creature's speed below 5 feet.  This spell has additional effects depending on which version you cast, chosen when you cast the spell.  Blessed Thorns: The vines sprout vicious thorns made of celestial steel. Each round at the beginning of your turn, the subject takes 2d4 points of damage, plus 1 point of damage for each round since the creature last took a full-round action to tear the vines off. When you cast this version of the spell, you can also choose to make the thorns either cold iron or silver for the purposes of overcoming damage reduction.  Noxious Vines: The vines emit noxious fumes that act as a lung and eye irritant. The subject and any creatures adjacent to the subject must attempt a DC 15 Fortitude check each round at the beginning of their turn. On a failed save, the creatures becomes blinded for that round and can't cast spells with verbal components.  Swift Vines: The vines act twice as quickly as normal, and slow the creature by 10 feet per round instead of 5 feet (though they still can't decrease the creature's speed to less than half). In addition, once the creature is slowed to half-speed, it becomes staggered until the vines are torn off or the spell ends.}
    
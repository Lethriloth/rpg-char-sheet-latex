    
\DeclareSpell{Ablative Sphere}{abjuration () []|V,  S,  M (a crystalline sphere worth 10 gp)|1 standard action|personal|Targetsyou|1 minute per level (D)||}[]
    \DeclareSpellDescription{Ablative Sphere}{The Garundi tenaciously protect their homes, and through the years they have perfected magic to aid them in their defense.
An immobile, crystalline, weblike globe surrounds you.
When the ablative sphere winks into existence, it provides you with improved cover (Pathfinder RPG Core Rulebook 196). The barrier does not impede a spell's line of sight or effect.
The sphere is 1 inch thick per caster level, has hardness 5, and 3 hit points per inch of thickness. When an ablative sphere loses hit points, the level of cover it provides is reduced. When the ablative sphere has lost one-third of its hit points, it provides cover instead of improved cover. Once it has lost two-thirds of its hit points, it provides only partial cover. Finally, when the ablative sphere's hit points reach 0, the globe is destroyed. When an attack reduces an ablative sphere's hit points to 0, you take any remaining damage.}
        
\DeclareSpell{Burning Arc}{evocation () [fire]|V,  S|1 standard action|close (25 ft. + 5 ft./2 levels)|Targetsone primary target plus one additional target/3 levels  (each of which must be within 15 ft. of the primary target)|instantaneous|Reflex half|yes}[]
    \DeclareSpellDescription{Burning Arc}{Keleshites brag that they stole this spell from genie-kind thousands of years ago while other civilizations struggled without fire.
This spell causes an arc of flame to leap from your fingers, burning a number of enemies nearby. It deals 1d6 points of fire damage per caster level (maximum 10d6). For every additional target the discharge arcs to, reduce the number of damage dice by half (rounded down). Therefore, at 9th level, your burning arc deals 9d6 points of fire damage to the primary target, then 4d6 points of fire damage to a secondary target, then 2d6 points of fire damage to an additional target.
Each target can attempt a Reflex saving throw for half damage. The Reflex DC to halve the damage of the secondary bolts is 2 lower than the DC to halve the damage of the primary bolt. You may choose secondary targets as you like, but they must all be within 15 feet of the primary target, and no target can be struck more than once. You can choose to affect fewer secondary targets than the maximum.}
        
\DeclareSpell{Snow Shape}{transmutation () [water]|V,  S,  M/DF (a miniature shovel)|1 standard action|touch|Targetssnow or snow-sculpted object touched, up to 5 cubic ft.  + 1 cubic ft./level|instantaneous|none|no}[]
    \DeclareSpellDescription{Snow Shape}{In frozen northern lands, where the earth may be hidden beneath heavy drifts of snow, Ulfen druids developed a variation of stone shape that other spellcasters have since learned.
You can form a mass of snow into any shape that suits your purpose, as per the spell stone shape. While it's possible to make crude objects with snow shape, most fine details aren't possible.
However, a successful Craft (weapons) check allows you to create a bladed weapon from the snow. The DC of this check is equal to the DC listed with the Craft (weapons) skill (Core Rulebook 93).
You must be the one to make the Craft check and must do so at the time of casting this spell. A failed check means that the spell is cast normally but the weapon created is malformed and useless. This spell can only be used to craft weapons and not more precise tools or elaborate armors.
Once you create the item with this spell, it solidifies into super-hard ice, gaining a hardness of 5 and 10 hit points per inch of thickness. This weapon takes double damage from fire.
Anyone using an ice weapon takes a -2 penalty on attacks due to the slippery, unwieldy nature of the weapon, but the weapon deals 1 point of cold damage in addition to its normal damage. A weapon created by this spell lasts for 24 hours before melting into uselessness.}
        
\DeclareSpell{Summon Totem Creature}{conjuration (summoning) []|V,  S,  M/DF (a piece of bone from any one of your totem animals)|10 minutes|close (25 ft. + 5 ft./2 levels)|Effectone summoned creature|1 hour (D)|none|no}[]
    \DeclareSpellDescription{Summon Totem Creature}{The Shoanti revere more than the deities of distant planes, but also the animals and natural forces of the rugged lands they call home. Shoanti spellcasters have learned ways to call upon the might of their quahs' totem creatures in times of need. A character must have been raised by the Shoanti and be considered part of a quah to be able to cast this spell. Characters with access to this spell can only summon creatures revered by the quah they are a part of, as noted on the following lists. Except as noted above, this spell functions as summon nature's ally III.
Lyrune-Quah (Moon Clan): air elemental (small), 1d3 bats, wolf, 1d3 owls.
Shadde-Quah (Axe Clan): 1d3 eagles, earth elemental (small), water elemental (small).
Shriikirri-Quah (Hawk Clan): air elemental (small), 1d3+1 hawks (familiars), horse.
Shundar-Quah (Spire Clan): air elemental (small), earth elemental (small), 1d3 eagles.
Sklar-Quah (Sun Clan): 1d3 fire beetles, fire elemental (small), horse.
Skoan-Quah (Skull Clan): 1d3+1 eagles (vultures), 1d3 fire beetles, 1d3 giant centipedes.
Tamiir-Quah (Wind Clan): air elemental (small), 1d3 eagles, earth elemental (small).}
    
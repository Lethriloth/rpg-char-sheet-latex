    
\DeclareSpell{Cosmic Ray}{evocation () []|V,  S,  M (a piece of meteorite)|1 standard action|close (25 ft. + 5 ft./2 levels)|Effectray|instantaneous and 1 round/level (see text)|Fortitude partial|yes}[]
    \DeclareSpellDescription{Cosmic Ray}{You fling a ray of deadly cosmic energy at the target creature. If you succeed at a ranged touch attack with the ray, the target takes 1d6 points of damage per caster level (maximum 20d6) and must make a Fortitude save. On a failed save, the target becomes sickened for 1 round per caster level and emits toxic cosmic energy for as long as it is sickened; any creature that comes within 5 feet of the affected target must succeed at a Fortitude save (DC = spell's DC) or become sickened for 1 round per 2 caster levels.}
        
\DeclareSpell{Gravity Sphere}{transmutation () []|V,  S,  M (a marble)|1 standard action|medium (100 ft. + 10 ft./level)|Effect30-ft.-radius sphere of altered gravity|1 round/level (D)|none (see text)|no}[]
    \DeclareSpellDescription{Gravity Sphere}{You affect the local gravity field in a 30-foot-radius sphere around the spell's designated point of origin. Creatures in the affected area can be weighed down by high gravity, lightened by low gravity, or left to float in an area with no gravity, depending on the effect you choose for your gravity sphere. See page 18 for rules on the effects of high, low, and no gravity.
Creatures that fall within the area of a gravity sphere spell take more damage from the fall if the gravity within is higher and less damage if the gravity within is lower. However, if the creature falls through a gravity sphere and continues to fall in a non-affected area, the rest of the fall damage is calculated normally. For instance, if a creature falls through 20 feet of a lowgravity sphere and an additional 30 feet outside the sphere, it would take 1d6 points of damage for the 20 feet of low gravity plus 3d6 points of damage as normal for the 30 feet outside the sphere (for a total of 4d6 points of damage).
This spell doesn't counteract or negate the effects of other spells that affect gravity; both effects occur simultaneously. For instance, a no-gravity gravity sphere spell cast within the area of a reverse gravity spell would simply mean that creatures float about until the gravity sphere's duration expires or they exit the gravity sphere, at which point they return to the top of the reverse gravity effect's area. If both gravity-altering magical effects could not feasibly take place simultaneously (at the GM's discretion), the gravity sphere spell supersedes the previous effect (if it's higherlevel than the previous effect) or simply fails (if it's lower-level).}
        
\DeclareSpell{Planetarium}{illusion (figment) []|V,  S|1 standard action|close (25 ft. + 5 ft./2 levels)|Effect15-ft.-radius spherical projection of night sky|concentration + 3 rounds|Will disbelief (harmless)|no}[]
    \DeclareSpellDescription{Planetarium}{You project an image of the night sky based on your current location and the local time, allowing you to observe the heavens and all of its celestial bodies and features even during the daytime, indoors, or underground. Anyone within the planetarium's sphere can see the projection, though outside of the sphere the image becomes grainy and indistinct.}
        
\DeclareSpell{Planetary Adaptation}{transmutation () []|V|1 standard action|personal|Targetsyou|1 hour/level||}[]
    \DeclareSpellDescription{Planetary Adaptation}{This spell functions as planar adaptation (Pathfinder RPG Advanced Player's Guide 236), except that it works only on worlds of the Material Plane. The cold void of space is considered a single world for the purpose of this spell, allowing you to survive in vacuum.}
        
\DeclareSpell{Planetary Adaptation, Mass}{transmutation () []|V,  S|1 standard action|close (25 ft. + 5 ft./2 levels)|Targetsone creature/level, no two of which can be more than 30 ft. apart|1 hour/level|Will negates (harmless)|yes (harmless)}[]
    \DeclareSpellDescription{Planetary Adaptation, Mass}{This spell functions as planetary adaptation, except as noted above.}
        
\DeclareSpell{Reboot}{transmutation () []|V,  S,  F (a ruby worth at least 25 gp per HD of the target construct)|1 round|close (25 ft. + 5 ft./2 levels)|Targetsone destroyed construct of up to 2 HD/level|1 round/level (D)|none|no}[]
    \DeclareSpellDescription{Reboot}{Whispering in the dense, information-rich machine language of the First Ones, you bring a destroyed construct back to operational status for a short time, restoring it to 1 hit point. The construct can be further healed with spells like make whole, but it returns to its destroyed state as soon as this spell's duration expires or it is brought to 0 hit points, whichever comes first. As long as the construct is active, it obeys your commands to the best of its ability, fighting on your behalf and carrying out tasks that it is capable of performing.
Constructs with more than twice as many Hit Dice as your caster level cannot be targeted by this spell.}
        
\DeclareSpell{Starsight}{divination () []|V,  S|1 standard action|personal|Targetsyou|10 minutes/level||}[]
    \DeclareSpellDescription{Starsight}{You can observe the night sky and all of its celestial bodies as if it were a clear night, regardless of weather conditions that would otherwise block your view. Your vision penetrates any light pollution from nonmagical sources, though this spell doesn't function in daylight, indoors, or underground. You see through forest canopies and similar natural obstructions, but only for the purpose of stargazing. For the spell's duration, you gain a +2 insight bonus on Knowledge (geography) checks relating to the stars and planets and Survival checks to avoid getting lost.}
    
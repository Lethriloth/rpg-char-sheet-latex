    
\DeclareSpell{Advanced Scurvy}{necromancy [disease,  evil]|V,  S|1 standard action|touch|Targets: living creature touched|instantaneous|Fortitude negates|yes}[]
    \DeclareSpellDescription{Advanced Scurvy}{The subject contracts an advanced form of scurvy. He becomes constantly fatigued, suffers from bone pain (-1 penalty on Strength-and Dexterity-based checks), wounds easily (add +1 point of damage to any bleed effects affecting the target), experiences loose teeth, and is slow to heal (natural healing occurs at half the normal rate). Scurvy can be treated magically or can be overcome with proper nutrition; eating the right foods ends the fatigue and bone pain within 1-2 days and provides a full cure 2d6 days after that.}
        
\DeclareSpell{Cloud Of Seasickness}{conjuration (creation) [poison]|V,  S,  M (a piece of seaweed)|1 standard action|close (25 ft. + 5 ft./2 levels)|Effect: cloud spreads in 20-ft. radius, 20 ft. high|1 round/level|Fortitude negates; see text|no}[]
    \DeclareSpellDescription{Cloud Of Seasickness}{This spell functions like stinking cloud, except as noted above and that the vapors make creatures sickened instead of nauseated.  Cloud of seasickness can be made permanent with a permanency spell (requiring a 9th-level caster and costing 2,500 gp). A permanent cloud of seasickness dispersed by wind reforms in 10 minutes.\\\\

{\centering\bf Stinking Cloud\hrule}

Stinking cloud creates a bank of fog like that created by fog cloud, except that the vapors are nauseating. Living creatures in the cloud become nauseated. This condition lasts as long as the creature is in the cloud and for 1d4+1 rounds after it leaves.

(Roll separately for each nauseated character.) Any creature that succeeds on its save but remains in the cloud must continue to save each round on your turn. This is a poison effect.

Stinking cloud can be made permanent with a permanency spell. A permanent stinking cloud dispersed by wind reforms in 10 minutes.}
    
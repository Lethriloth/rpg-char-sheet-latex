    
\DeclareSpell{Aram Zey's Focus}{divination|V,  S,  F (masterwork thieves' tools worth 100 gp)|1 standard action|personal|Targets: you|1 minute/level (D)||}[]
    \DeclareSpellDescription{Aram Zey's Focus}{Aram Zey created this spell for use by his students, both to increase their confidence in their skills and to ensure more of them survived encounters with deadly traps. If you don't have the trapfinding class ability, this spell grants you the trapfinding ability of a rogue of half your character level.

If you have the trapfinding ability granted by class levels, however, this spell grants you a +5 competence bonus on all Disable Device checks made to disarm mechanical (but not magical) traps. While under the effects of Aram Zey's focus, whenever you trigger a trap by rolling poorly on a Disable Device check, you may roll a second Disable Device check. This new roll uses the same modifiers as the first roll. If your second roll is high enough to avoid accidentally springing the trap, you avoid setting it off, but still fail to disarm it. Each time you take advantage of this feature, the remaining duration of the spell is reduced by 1 minute-if less than a minute's worth of duration remains, the spell ends as soon as you reroll your Disable Device check.}
        
\DeclareSpell{Aram Zey's Trap Ward}{abjuration|V,  S,  M (masterwork thieves' tools worth 100 gp)|1 standard action|personal|Targets: you|10 minutes/level or until discharged||}[]
    \DeclareSpellDescription{Aram Zey's Trap Ward}{While he was researching the underlying causes of the resonance created by overlapping abjurations, Aram Zey discovered a way to manipulate that resonance to the caster's advantage when encountering magical traps.

The protection provided by Aram Zey's trap ward comes into play whenever the caster is subjected to the effects of a magical trap. The spell immediately discharges and interferes with the trap's function in an attempt to counter the trap's magic. When this occurs, make a caster level check as an immediate action. The DC of this check is equal to the trap's Disable Device DC. If you're successful, the trap ward dispels the magical effect of the trap before the effect actually manifests, effectively preventing the trap from triggering for the next 1d4 rounds and ending Aram Zey's trap ward immediately.}
        
\DeclareSpell{Bite The Hand}{enchantment (compulsion)|V,  S,  DF|1 standard action|close (25 ft. + 5 ft./2 levels)|Targets: one creature summoned by a spell or spell-like ability|1 round/level (D)|Will negates|yes}[]
    \DeclareSpellDescription{Bite The Hand}{With a short command and a wave of the hand, you compel the target creature to attack the being who summoned it, to the best of its ability. If the being who summoned it is not present, the creature acts normally according to its last task or instructions. This spell has no effect on called creatures, summoned creatures not brought forth by spells or spell-like abilities (such as a summoner's eidolon), or bonded creatures not explicitly summoned, such as a paladin's mount or wizard's familiar.}
        
\DeclareSpell{Mass Bite The Hand}{enchantment (compulsion)|V,  S,  DF|1 standard action|medium (100 ft. + 10 ft./level)|Targets: one creature summoned by a spell or spell-like ability/level, no two of which can be more than 30 ft. apart|1 round/level (D)|Will negates|yes}[]
    \DeclareSpellDescription{Mass Bite The Hand}{This spell functions like bite the hand, except as noted above.

The target creatures do not need to have all been summoned by the same being.\\\\

{\centering\bf Bite The Hand\hrule}

With a short command and a wave of the hand, you compel the target creature to attack the being who summoned it, to the best of its ability. If the being who summoned it is not present, the creature acts normally according to its last task or instructions. This spell has no effect on called creatures, summoned creatures not brought forth by spells or spell-like abilities (such as a summoner's eidolon), or bonded creatures not explicitly summoned, such as a paladin's mount or wizard's familiar.}
        
\DeclareSpell{Corpse Lanterns}{necromancy [light]|V,  S|1 standard action|medium (100 ft. + 10 ft./level)|Effect: up to 4 lights, all within a 10-ft.-radius area|1 minute/level (D)|none|no}[]
    \DeclareSpellDescription{Corpse Lanterns}{This spell functions as dancing lights, except it summons up to four spheres of light, each of which glows a sickly pale green. These corpse lanterns shed dim light in a 20-foot radius, and do not increase the light level in areas of normal light or bright light. In dim or normal light, the radiance of corpse lanterns provides a strange contrast, giving all creatures in the area a -5 penalty on Stealth checks. In addition, the hue interferes with illusion (pattern) spells, giving all creatures in the illuminated area a +2 bonus on any saving throws against such spells. Unlike dancing lights, you may have more than one corpse lanterns spell active at a time, but you may only move one set in any given round.

Moving the corpse lanterns does not require concentration.

Corpse lanterns can be made permanent on an area with a permanency spell by a caster of at least 11th level for the cost of 7,500 gp.}
        
\DeclareSpell{Gilded Whispers}{divination|V,  S,  M (100 gp of powdered gemstones)|1 round|touch|Targets: a gold or platinum coin|1 day/level (D)|Will negates (object)|yes (object)}[]
    \DeclareSpellDescription{Gilded Whispers}{Developed by priests of Abadar to catch thieves and skimmers, gilded whispers later spread to other faiths and was adapted to the arcane arts through the combined efforts of Aram Zey and Kreighton Shaine. Pathfinders most commonly use this spell to track bribes and illicit purchases back to their ultimate source, especially when they suspect the influence of Aspis Consortium agents.

Gilded whispers allows you to use a single coin as a conduit for an eavesdropping spell. When you use a divination (scrying) spell or item, such as clairvoyance/ clairaudience, scrying, or a crystal ball, you can choose to target a coin you have affected with gilded whispers instead of a creature or location (even if you would not otherwise be able to target an object), though any range limits on the scrying effect still apply. If the coin is held or carried by a creature, its owner receives any applicable saving throw against the effect. The caster of gilded whispers treats the coin as a familiar subject. The residual psychic impressions left upon the coin by other handlers help mask this dweomer from detection, protecting gilded whispers against location by detect magic, arcane sight, and similar effects unless the latter spell's caster succeeds on a caster level check (1d20 + caster level) against a DC of 11 + the caster level of the spellcaster who cast gilded whispers.

The scrying sensor created by using a divination (scrying) spell to observe or listen to the coin's surroundings can be detected as normal.}
        
\DeclareSpell{Lipstitch}{necromancy|S,  M (a bone needle and sinew thread)|1 standard action|close (25 ft. + 5 ft./2 levels)|Targets: one creature|instantaneous|Fortitude negates|yes}[]
    \DeclareSpellDescription{Lipstitch}{A rare spell without verbal components, lipstitch sews the target's lips tightly together if it fails a saving throw, such that no clear speech, bite attacks, spellcasting, or use of command words is possible. The target takes 1d6 points of damage as the stitches weave through flesh. The victim can still make enough noise to be heard at a distance with a DC 10 Perception check.

The thread created by lipstitch can be burst with a DC 20 Strength check as a standard action or can be sliced open with a piercing or slashing weapon (wielded by the target or an ally) as a full-round action. Cutting the thread provokes attacks of opportunity, while making a Strength check does not. Either option causes 1d6 points of damage and 1 point of bleed damage. The target has a 20\% chance of failing to cast spells with verbal components until the bleeding is stopped. The effects of multiple castings of this spell do not stack. Optionally, the thread can be removed more carefully over the course of a minute with a DC 20 Heal check. If the check fails, the target takes damage and bleeds as described above. If the check succeeds, the stitches are removed with no harm. Creatures with no mouths are unaffected by lipstitch. Creatures with multiple mouths lose the use of only one mouth per casting-the particular mouth is chosen by the caster.}
        
\DeclareSpell{Petulengro's Validation}{divination|V,  S,  M (a bit of hair,  a fingernail,  or a similar portion of a creature)|1 standard action|touch|Targets: creature touched|instantaneous|none|yes (harmless)}[]
    \DeclareSpellDescription{Petulengro's Validation}{After a particularly harrowing brush with death at the hands of doppelgangers, Venture-Captain Eliza Petulengro devised a means of being sure her companions were actually who they appeared to be. To cast this spell, you must have a bit of hair, a fingernail clipping, or some other portion of a creature. The sample must be no more than 1 week old per caster level. As part of casting, you touch the target creature, and instantly know whether the target is the same creature the sample is from. Note that if you wish to be discrete, you can cast the spell away from the target and hold the charge before touching the creature, so that the casting is not noticed. You can also use this spell to divine whether a dead body, or even partial remains from a body, belonged to the same person whose fingernail clipping or bit of hair you used when casting the spell.}
        
\DeclareSpell{Sequester Thoughts}{enchantment (compulsion) [mind-affecting]|V,  S,  M (a gemstone worth at least 500 gp)|10 minutes|personal|Targets: one willing creature|permanent until discharged (see text)||}[]
    \DeclareSpellDescription{Sequester Thoughts}{Sequester thoughts allows you to erase a creature's memory of either an event lasting not more than 1 minute per caster level or all of its knowledge about a single topic (using the GM's discretion as to what constitutes a single topic). For example, you could erase a single battle from a creature's memory, or all knowledge of a plot to assassinate a king.

The memories you remove are stored within the gem used at the time of casting. If the gem is shattered, the memories return to the creature as long as the two are within 30 feet of each other. Once sequester thoughts has been cast, the spell remains active on the gem and can be dispelled (which shatters it). No portion of the spell remains active on the target creature, and the target does not radiate magic as a consequence of the spell, nor can its memories be returned by dispelling the creature or subjecting it to antimagic. If the gem is shattered or dispelled out of range from the creature, the thoughts sequestered within are forever lost save by the use of wish, miracle, or the like.

Sequester thoughts protects against detect thoughts, zone of truth, discern lies, and similar spells where the memories removed are concerned, though careful questioning may reveal the gaps in the creature's memory, or that it has been affected by the spell. Note that the creature itself does not remember any details of what memories were removed until the gem is broken.}
        
\DeclareSpell{Sharesister}{necromancy|V,  S,  M (a drop of your own blood)|1 standard action|touch|Targets: you and one creature of your gender|1 minute/level|Will negates (harmless)|yes}[]
    \DeclareSpellDescription{Sharesister}{(harmless) Ithuna Vardsdottir claims to have unearthed this ancient prayer in a ruined temple of Desna, though Pathfinders have reported the use of similar magic in Irrisen among the White Witches. While the name of this spell is sharesister, it works equally well on male or female creatures-both targets of the spell must simply be of the same gender.

When you deliver the spell, you receive a negative level for the duration of the spell, and the other target receives a +1 insight bonus to her caster level and a +1 insight bonus to the save DCs of all of her spells. At 11th level, you can opt to take four negative levels to grant a +2 insight bonus to the other target's caster level and spell save DCs if you wish, while at 17th level you can take 6 negative levels to increase the insight bonus to +3. Any effect that removes or prevents the negative level immediately ends the sharesister spell. Negative levels received from the spell vanish as soon as this spell effect ends.

Negative levels from multiple castings of this spell stack.}
        
\DeclareSpell{Stalwart Resolve}{enchantment (compulsion) [mind-affecting]|V,  S,  DF|1 standard action|touch|Targets: creature touched|1 round/level|Will negates (harmless)|yes}[]
    \DeclareSpellDescription{Stalwart Resolve}{(harmless) Stalwart resolve was originally created to temporarily aid those suffering from certain afflictions. The recipient of stalwart resolve ignores the effects of ability damage and penalties to a single ability score of your choice, except that damage equal to or greater than the ability score still causes unconsciousness or death. This applies whether or not the ability damage or penalty happened before or during the spell's duration, and whether or not multiple sources are involved. This spell has no effect on ability drain.}
        
\DeclareSpell{Stolen Light}{illusion (figment)|V,  S,  F (a gem worth at least 500 gp)|1 full round|touch|Targets: transparent gem touched|permanent or 1 minute/level (see text)|Will negates (object)|yes (object)}[]
    \DeclareSpellDescription{Stolen Light}{Kreighton Shaine researched this spell from the fragmentary notes of a Vudrani ascetic recorded in a strange tome, and rumors credit him with no fewer than a dozen permanent stolen light gems hidden in compartments and drawers in his study. Stolen light stores images within a gem. To store an image, as part of casting you must touch a gem worth not less than 500 gp. You trap within the gem an image of everything visible within a 30-foot cone measured from the gem, in a direction of your choice. Alternatively, you can capture a less detailed image of a single object within sight.

Once the casting is complete, the gem turns opaque, and the image inside cannot be seen. The stolen image remains within the gem until released or dispelled.

To release an image, you touch a gem holding stolen light as a standard action and speak a command word chosen at the time of casting. For 1 minute per caster level, the image stored within the gem becomes visible. Details can be made out as clearly as they could be perceived at the time of casting. Darkvision is of no use for making out details in a stolen image, though low-light vision or other exceptional visual talents may reveal information the caster did not see.

Light sources brighter than bright light are reduced to bright light in the stolen image. Once the image has been released, it cannot again be recovered from the gem.

Stolen light can be made permanent with the permanency spell by a caster of 10th level for a cost of 5,000 gp. This leaves the gem capable of projecting the image indefinitely, activated and deactivated by its command word, until it is destroyed or dispelled.}
        
\DeclareSpell{Twisted Innards}{transmutation|V,  S,  M (a cocoon tied with string)|1 standard action|personal|Targets: you|1 minute/level||}[]
    \DeclareSpellDescription{Twisted Innards}{For the duration of this spell, your vital organs writhe, shift, and move about, making it difficult to strike you in a vulnerable area. While this spell is in effect, critical hits and sneak attacks against you have a 25\% chance of failing to inflict any additional damage-though you still take the normal damage from the attack. At 7th level, the chance to ignore additional damage increases to 50\%, while at 13th level the chance increases to 75\%.}
    
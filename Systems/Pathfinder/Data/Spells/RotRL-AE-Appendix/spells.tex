    
\DeclareSpell{Blood Money}{transmutation|V,  S|1 swift action|0 ft.|Effect: 1 material component|Instantaneous||}[]
    \DeclareSpellDescription{Blood Money}{You cast blood money just before casting another spell. As part of this spell's casting, you must cut one of your hands, releasing a stream of blood that causes you to take 1d6 points of damage. When you cast another spell in that same round, your blood transforms into one material component of your choice required by that second spell. Even valuable components worth more than 1 gp can be created, but creating such material components requires an additional cost of 1 point of Strength damage, plus a further point of damage for every full 500 gp of the component's value (so a component worth 500-999 gp costs a total of 2 points, 1,000-1,500 costs 3, etc.). You cannot create magic items with blood money.  For example, a sorcerer with the spell stoneskin prepared could cast blood money to create the 250 gp worth of diamond dust required by that spell, taking 1d6 points of damage and 1 point of Strength damage in the process.  Material components created by blood money transform back into blood at the end of the round if they have not been used as a material component. Spellcasters who do not have blood cannot cast blood money, and those who are immune to Strength damage (such as undead spellcasters) cannot use blood money to create valuable material components.}
        
\DeclareSpell{Covetous Aura}{abjuration|V,  S|1 standard action|personal|Area: 25-ft.-radius emanation centered on you|1 round/level or until discharged|none|no}[]
    \DeclareSpellDescription{Covetous Aura}{Anytime a harmless (so noted by a spell's saving throw description) spell of 3rd level or lower is cast within a covetous aura's area of effect, you may choose to immediately gain the benefit of that spell as if it had also targeted you. The intended target still gains the effect of the spell. You gain the benefits of this duplicated spell only if the caster is in range of the covetous aura. You are considered the caster of the additional spell effect. If the effect allows for multiple targets other than yourself, you cannot use the stolen spell effect to target other creatures-a covetous aura only aids you. Once you choose to gain the benefit of another spell, the covetous aura immediately ends. Rumors hold that this unusual spell was invented thousands of years ago by Runelord Belimarius, who was constantly jealous of other spellcasters' abilities.}
        
\DeclareSpell{Deathwine}{necromancy|V,  S|1 minute|touch|Targets: 1 potion touched/level|1 hour/level|none (object)|no (object)}[]
    \DeclareSpellDescription{Deathwine}{This spell allows you to turn a potion into a temporary pool of necromantic energy. Only a potion created using a conjuration (healing) spell can be affected by this spell. An affected potion turns dark red and reveals a necromantic aura if detect magic is cast on it while it remains under this spell's effects.  When you drink a potion affected by this spell, you do not gain the potion's normal effect. Instead, the first necromancy spell you cast within the next minute is cast at a higher caster level. The bonus to caster level is equal to the spell level of the spell used to create the potion that deathwine affects. For example, a 5th-level wizard who drinks deathwine made from a potion of cure serious wounds would cast his next necromancy spell as an 8th-level caster, as cure serious wounds i s a 3 rd-level spell.  In addition, any undead creature (or other creature healed by negative energy) that drinks a potion affected by deathwine is healed of 1d8 points of damage. Any potion not imbibed before this spell's duration expires is destroyed at the end of the deathwine's duration.}
        
\DeclareSpell{Sign Of Wrath}{evocation [force]|V,  S,  F (a gem worth 1, 000 gp inscribed with the Thassilonian symbol of wrath)|1 standard action|personal|Area: 25-ft.-radius burst centered on you|instantaneous|Reflex half|yes}[]
    \DeclareSpellDescription{Sign Of Wrath}{A giant, glowing symbol of wrath appears below you, forcibly repulsing all nearby creatures. All creatures within the area of effect take 1d6 points of force damage per caster level (maximum 15d6) and are subjected to a bull rush that attempts to push them directly away from you. The blast's bull rush effect has a CMB bonus equal to your caster level + your Intelligence, Wisdom, or Charisma modifier (whichever is highest). You are unaffected by both the spell's damage and its bull rush effect, and may select up to one creature per 4 caster levels to also be ignored by the spells effects.}
        
\DeclareSpell{Swipe}{conjuration (teleportation)|V,  S|1 standard action|close (25 ft. + 5 ft./2 levels)|Targets: one held item|instant|none|no}[]
    \DeclareSpellDescription{Swipe}{By flicking a finger in the appropriate direction and proclaiming ownership, you attempt to magically wrest an item from the target's grip and summon it to your hand. To claim an object held by an opponent, you must make a CMB check-this check has a bonus equal to your caster level + your Intelligence, Wisdom, or Charisma modifier (whichever is highest). If you fail this check, the target retains the item and the spell fails. If you succeed, the item teleports into one of your free hands or comes to rest at your feet.}
        
\DeclareSpell{Unconscious Agenda}{enchantment (compulsion) [language-dependent,  mind-affecting]|V|10 minutes|Close (25 ft. + 5 ft./2 levels)|Targets: One humanoid|One week/level or until discharged (D)|Will negates|yes}[]
    \DeclareSpellDescription{Unconscious Agenda}{This spell plants a subconscious directive in the target's mind that forces him to act as you dictate when specific circumstances arise. The target humanoid can be either conscious or unconscious, but must understand your language. Upon casting this spell, you must state a course of action you wish the target to take. This course of action must be described in 20 words or fewer. You must then state the condition under which you wish the target to take this action, also describing it in 20 or fewer words. Actions or conditions more elaborate than 20 words cause the spell to fail. Unconscious agenda cannot compel a target to kill himself, though it can compel him to perform exceedingly dangerous acts, face impossible odds, or undertake almost any other course of activity. You cannot issue new commands to the target after the spell is cast.  If the target fails his save against this spell, he is not compelled to act in any way until the specified trigger circumstances are encountered. He also has no knowledge of the details of the spell affecting him, and has no memory of the last 10 minutes (although he might come to notice the missing time or the presence of the caster). He can function as he wishes until the events you detailed as the condition take place. Upon experiencing the prerequisite condition, the target is forced to perform the course of action you described as per the spell dominate person. (If the compelled action is against the victim's nature, he immediately gains a new saving throw at a +5 bonus against the spell to end its effects.) For the next hour, the target acts as you dictated, doing all he can to fulfill your command. If, at the end of the hour, the target still has not completed your command, the target is released from the enchantment and the spell ends. Once the course of action is completed, the spell ends. The target has full memory of acts performed during this hour.  It's difficult to detect an unconscious agenda before the spell is triggered. Casting detect magic on one affected by it only reveals an aura of enchantment if the caster of detect magic has a higher caster level then the caster of unconscious agenda. Even if the spell is detected, it can only be removed by break enchantment, limited wish, remove curse, miracle, or wish. Dispel magic does not affect unconscious agenda.}
    
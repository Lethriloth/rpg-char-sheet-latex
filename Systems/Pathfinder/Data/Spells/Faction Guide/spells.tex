    
\DeclareSpell{Arcane Reinforcement}{transmutation|V,  S|1 round|personal|Targets: you|concentration (up to 8 hours)||}[]
    \DeclareSpellDescription{Arcane Reinforcement}{You cast this spell as you begin crafting an item using a Craft skill, and add your Spellcraft ranks to Craft checks made to create that item. As part of concentrating on the spell, you must continue chanting the verbal components, pausing at most for a few seconds at a time to speak, chew, and so on. If the crafting takes more than 1 day to complete, you must cast the spell each day to gain its benefit.

This spell originated among wizards of the Arcanamirium, though the Pathfinder Society also uses it.}
        
\DeclareSpell{Canopic Conversion}{necromancy [death,  evil]|V,  S,  F (four alabaster canopic jars worth 100 gp each),  M (black onyx worth 100 gp per hit die of the target)|1 round|close (25 ft. + 5 ft./2 levels)|Targets: one living humanoid|instantaneous and see text|Fortitude half|yes}[]
    \DeclareSpellDescription{Canopic Conversion}{This spell eviscerates the target, drawing forth its life essence as well as its internal organs. The target takes 1d6 hit points of damage per caster level (maximum 20d6). If this damage kills the target, the spell pulls the creature's organs into a set of 4 canopic jars and seals them; 1d4 rounds later, the corpse revives as a mummy (if 8 HD or fewer) or an advanced mummy (if 9 HD or more).

The mummy is not under your control, but the canopic jars give the bearer certain powers over it. Anyone holding one of the jars can communicate with the mummy as if the two shared a common language. The bearer gains the benefits of protection from evil and sanctuary, but only against that mummy. Unsealing or breaking a jar is a standard action, which dissipates its power (and protection) but lets the bearer issue a short command to the mummy, similar to a suggestion spell (Will DC 23 negates). You (and only you) may unseal all 4 jars in a 10-minute ritual to control the mummy with an similar to geas (Will DC 23 negates); most casters typically include a restriction that the mummy will not harm them, as unsealing the jars leaves them vulnerable.

The pharaohs of ancient Osirion sometimes used this spell to punish their enemies. The Risen Guard does not use the spell (though they have access to it), preferring trustworthy living guardians for the Ruby Prince. The Whispering Way may have copies of the spell.}
        
\DeclareSpell{Chastise}{transmutation|V|1 standard action|personal|Targets: you|1 minute/level||}[]
    \DeclareSpellDescription{Chastise}{You gain a +5 bonus on Bluff, Diplomacy, and Intimidate checks to convince a listener that they will get in trouble with their superiors or with the law if they don't do what you ask.

This spell is primarily used by the Eagle Knights, though they did not create it; Hellknights also use it extensively to gather information and foster cooperation.}
        
\DeclareSpell{Summon Elemental Steed}{conjuration (summoning) [see text]|V,  S,  DF|10 minutes|close (25 ft. + 5 ft./level)|Effect: one chariot|10 minutes/level|none|no}[]
    \DeclareSpellDescription{Summon Elemental Steed}{You summon a greater elemental (air, earth, fire, or water) bound in the form of a mighty chariot that moves at your command and can carry up to nine Medium creatures.

Passengers inside are not harmed by the elemental, and can see, breathe, and act normally, suffering no environmental damage even when the elemental chariot is flying, underwater, burrowing, or using earth glide. Passengers are not otherwise protected, though the elemental chariot does provide cover like a normal chariot.

The elemental keeps all of its statistics and abilities, except it loses its slam attacks and gains a trample attack (Pathfinder RPG Bestiary 305).

When you use a summoning spell to summon an air, earth, fire, or water creature, it is a spell of that type.

This spell is almost exclusively used by the Green Faith.}
        
\DeclareSpell{Hibernate}{necromancy|V,  S,  DF|1 standard action|touch|Targets: creature touched|1 minute/level|none|no}[]
    \DeclareSpellDescription{Hibernate}{You place a willing subject into a cataleptic state. It remains aware of its surroundings but is paralyzed, appearing dead unless observers make a DC 20 Heal check. Hibernate delays the effects of poison, disease, and bleed effects for the spell's duration, and half of any hit point damage suffered by the subject is converted to nonlethal damage.

This spell originated with the Green Faith, but has spread to all other druidic religions.}
        
\DeclareSpell{Sotto Voce}{necromancy [fear,  mind-affecting,  sonic]|V|1 standard action|close (25 ft. + 5 ft./level)|Targets: one humanoid creature of 4 HD or less|1 round|Will negates|yes}[]
    \DeclareSpellDescription{Sotto Voce}{Your dry, rasping whisper fills a living creature of 4 or fewer Hit Dice with unnatural dread. The affected creature must make a Will save or be shaken for 1 round.

This spell originated among the followers of the Whispering Way, but necromancers and other intimidating folk outside that group are known to use it.}
        
\DeclareSpell{Tomb Legion}{necromancy [evil]|V,  S|1 standard action|medium (100 ft. + 10 ft./level)|Effect: Three or more advanced mummies, no two of which can be more than 30 ft. apart; see text|7 days or 7 months ; see text|none|no}[]
    \DeclareSpellDescription{Tomb Legion}{This spell functions like shambler, except that it calls into existence 1d4+2 advanced mummies rather than shambling mounds.

Though the spell has fallen out of favor among the Risen Guard because the Ruby Prince frowns on the use of undead, this spell is popular among followers of the Whispering Way.\\\\

{\centering\bf Shambler\hrule}

The shambler spell creates 1d4+2 shambling mounds with the advanced template (see the Pathfinder RPG Bestiary). The creatures willingly aid you in combat or battle, perform a specific mission, or serve as bodyguards. The creatures remain with you for 7 days unless you dismiss them. If the shamblers are created only for guard duty, however, the duration of the spell is 7 months. In this case, the shamblers can only be ordered to guard a specific site or location. Shamblers summoned to guard duty cannot move outside the spell's range, which is measured from the point where each first appeared. You can only have one shambler spell in at one time. If you cast this spell while another casting is still in effect, the previous casting is dispelled. The shamblers have resistance to fire as normal shambling mounds do only if the terrain where they are summoned is rainy, marshy, or damp.}
        
\DeclareSpell{Tripvine}{transmutation|V,  S|1 standard action|touch|Targets: 10-foot-long rope or vine|10 minutes/level|Reflex negates|no}[]
    \DeclareSpellDescription{Tripvine}{You animate a rope, vine, or similar object so that it attempts to trip any creature that comes near. The rope attacks anything in a 10-foot-square area you designate. The rope does not provoke an attack of opportunity. Its CMB is equal to your caster level +2. A tripped target that was running, jumping, or charging takes 1d6 points of nonlethal damage. Creatures aware of the tripvine gain a +4 bonus to their CMD against it.

The Green Faith created this spell, but it is now common.}
    
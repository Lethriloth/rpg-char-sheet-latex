    
\DeclareSpell{Irriseni Mirror Sight}{divination (scrying)|V,  S,  F (a mirror)|10 minutes|see text|Effect: magical sensor|1 minute/level|none|no}[]
    \DeclareSpellDescription{Irriseni Mirror Sight}{This spell lets you look into a mirror near you and see an image that is reflected in another specific mirror (chosen by you) or an individual reflected in any other mirror. This works like a scrying spell, except you can only view creatures on the same plane as you. Each time you cast the spell, you can choose to see one of three types of reflections in your mirror.  Known Mirror: The current reflection in another mirror with which you are familiar.  Known Person: The current reflection of a person you know well, assuming that person is near a mirror.  Known Place: The current reflection of a place you know well, assuming the location is being reflected in a mirror.  You receive only visual information through this ability. You can choose to transmit information both ways so that a person reflected in the remote mirror can view whatever appears in the mirror you are using.  For example, Urion Petresky knows that Queen Elvanna keeps a mirror in a hall near her throne room. He can look through his own handheld mirror and see into this hall, even if the queen is not there. Alternatively, he can attempt to find the queen (wherever she is) by looking into his mirror; if, at that moment, the queen is near any mirror at all, he can see her. He may instead cast the spell and try to see into her throne room, hoping that someone has brought a mirror there. If any of these conditions fails, Urion sees nothing but his own reflection.  This spell works with intentionally fabricated mirrors only; it is not effective with other reflective surfaces, such as still pools or polished metal shields. Effects that block scrying block this spell.}
        
\DeclareSpell{Flurry Of Snowballs}{evocation [cold,  water]|V,  S|1 standard action|30 ft.|Area: cone-shaped burst|instantaneous|Reflex half|no}[]
    \DeclareSpellDescription{Flurry Of Snowballs}{You send a flurry of snowballs hurtling at your foes. Any creature in the area takes 4d6 points of cold damage from being pelted with the icy spheres.}
        
\DeclareSpell{Ice Spears}{conjuration [cold]|V,  S,  M (a small stalagmite-shaped crystal)|1 standard action|close (25 ft. + 5 ft./2 levels)|Effect: 1 ice spear/4 levels|instantaneous|Reflex half and see below|no}[]
    \DeclareSpellDescription{Ice Spears}{Favored by the spellcasters of Irrisen, this potent spell can disrupt spellcasters, topple enemies, and break even seemingly unstoppable charges.  One or more giant spears of ice lance up out of the ground. Each stalagmite-like icicle affects a 5-foot square and tapers to a height of 10 feet. You may cause a number of ice spears equal to one spear for every 4 caster levels you possess to burst from the ground. A creature that occupies a square from which a spear extends (or that is within 10 feet of the ground below) takes 2d6 points of piercing damage and 2d6 points of cold damage per square-creatures that take up more than 1 square can be hit by multiple spears if your caster level is high enough. The explosive growth can also trip foes. When the spears erupt from the ground, they attempt a combat maneuver check to trip any targets that take damage from the spears, with a total bonus equal to your caster level plus your Intelligence, Wisdom, or Charisma modifier, whichever is highest. Each additional ice spear beyond the first that strikes a single foe grants a +10 bonus on this combat maneuver check. If the check is successful, the ice spears knock the foe prone. A successful Reflex save halves the damage and prevents the trip attempt.  If you cast this spell upon an area covered with ice or snow, such as a glacier, frozen lake, or snow-covered field, the spears strike with additional force. Saves against the effect take a -2 penalty, and the spell effect gains a +4 bonus on the combat maneuver check to trip foes.  Ice spears created by this spell remain after they do their damage. They melt as normal depending on the surrounding environment. They no longer damage foes in their square, but can provide cover. An ice spear has hardness 5 and 30 hit points.}
    
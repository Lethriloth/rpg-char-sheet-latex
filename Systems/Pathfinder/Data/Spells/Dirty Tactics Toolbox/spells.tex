    
\DeclareSpell{Drain Poison}{transmutation [poisonUM]|V,  S,  M/DF (the fang of a poisonous creature)|1 standard action|touch|Targets: one weapon or a single piece of ammunition|24 hours|none|no}[]
    \DeclareSpellDescription{Drain Poison}{By touching a weapon against the fang of a poisonous creature and casting this spell, you drain 1 dose of the creature's poison, which is magically applied to your weapon without risk of poisoning yourself. The poison remains on the weapon until either it strikes a creature, you touch the weapon, or you wipe off the poison. It otherwise functions exactly like a dose of a manufactured poison applied to a weapon. You can use this spell with natural as well as manufactured weapons. This spell does not prevent you from exposing yourself to the poison if you roll a natural 1 on an attack roll while the poison is applied to your weapon.  The spell has no effect if the creature whose fang is touched is a summoned creature, if it has been dead for more than 1 minute, or if its poison has already been extracted from it.}
        
\DeclareSpell{Garden Of Peril}{conjuration (creation) [poisonUM]|V,  S,  M/DF (a pinch of spores)|1 standard action|medium (100 ft. + 10 ft./level)|Effect: one poisonous mushroom/level, no two of which can be more than 30 ft. apart|1 round/level (D)|Fortitude negates|no}[]
    \DeclareSpellDescription{Garden Of Peril}{Vividly colored poisonous mushrooms instantly spring into existence in squares you select. The mushrooms can appear on any solid surface, even growing horizontally from walls or upside down from the ceiling.  Once per round as a move action, starting the round after you cast this spell, you can command the mushrooms to release poisonous spores. Each creature adjacent to a mushroom or in a mushroom's square must then succeed at a Fortitude save or become sickened for 1d4 rounds. The condition of creatures that fail multiple saving throws does not worsen, but each additional failed saving throw adds 1d4 rounds to the duration the creature is sickened.  The mushrooms are Tiny and cannot provide cover, but they are fairly sturdy (AC 7, hp 30, break DC 20). Creatures can  move through squares containing mushrooms as normal. When destroyed, a mushroom evaporates and releases one last cloud of spores.}
        
\DeclareSpell{Invigorating Poison}{transmutation|V,  S,  M/DF (an apple seed)|1 standard action|touch|Targets: creature touched|10 minutes/level|Fortitude negates (harmless)|yes (harmless)}[]
    \DeclareSpellDescription{Invigorating Poison}{The body of the target creature gains a metabolic response that allows it to benefit from normally deadly toxins. When a poison would cause ability damage to the target creature, the target instead gains a +4 alchemical bonus to that ability score. The spell then immediately ends, but the bonus lasts for a number of minutes equal to the amount of ability damage the poison would have caused. If the poison would deal more than one type of ability damage, each bonus has a separate duration. If the poison has effects other than ability damage (such as unconsciousness or ability drain), these effects apply normally. This spell affects only a single poison; if multiple poisons affect the target simultaneously, this spell prevents only ability damage and grants the appropriate bonuses for the poison that would cause the most damage.}
        
\DeclareSpell{Greater Neutralize Poison}{conjuration (healing)|V,  S,  M/DF (a lump of charcoal)|1 standard action|touch|Targets: creature or object of up to 1 cu. ft./level touched|instantaneous or 1 hour/level (D)|Will negates (harmless, object)|yes (harmless, object)}[]
    \DeclareSpellDescription{Greater Neutralize Poison}{This spell functions as neutralize poison, except as noted here. You automatically succeed at all caster level checks to neutralize any poisons affecting the target creature. The spell also reverses all instantaneous or permanent effects caused by poisons, such as temporary ability damage or permanent ability drain (it does not heal ability drain caused by anything other than poison, however). If you use the spell to neutralize the poison in a poisonous creature or object, the duration is 1 hour per level rather than 10 minutes per level. If the spell is cast on a creature, the creature can negate the effect by succeeding at a Will saving throw.}
        
\DeclareSpell{Poison Breath}{evocation [poisonUM]|V,  S,  M/DF (a spider's mandible)|1 standard action|15 ft.|Area: cone-shaped burst|instantaneous|Fortitude negates|yes}[]
    \DeclareSpellDescription{Poison Breath}{You expel a cone-shaped burst of toxic mist from your mouth, subjecting everyone caught in the area to a deadly poison, as per the poison spell.}
        
\DeclareSpell{Sword To Snake}{transmutation [poisonUM]|V,  S,  M/DF (a hair from a spider)|1 standard action|close (25 ft. + 5 ft./2 levels)|Targets: one Medium or smaller item|1 round/level (D)|Will negates (object)|yes (object)}[]
    \DeclareSpellDescription{Sword To Snake}{When you make an undulating gesture using your hand, the target item changes slightly in shape and appearance to resemble a cross between its original form and that of a venomous animal. For example, a staff might change to resemble a snake, a shield might change to resemble an enormous venomous beetle, or the hilt of a sword might change to resemble a scorpion's tail.  If this spell is cast on a held item, the wielder adds his Will save bonus to the saving throw. You must be able to see the item to cast this spell on it.  The item retains all its properties, but whenever a creature picks up, attacks with, activates, or otherwise manipulates the item (with the exception of dropping the item) after this spell is cast, it animates and bites the creature. The attack is resolved before the action that triggered it. This spell otherwise does not affect the creature's ability to use the item.  If the item is a nonmagical weapon, it has an attack bonus equal to your caster level plus your spellcasting ability score modifier, and its attack deals 1d4 points of piercing damage. If the target item is a magic weapon, it retains any magical qualities, and its enhancement bonus is still added to both the attack roll and the damage roll.  A creature that takes damage from the bite is affected as if by a venomous snake's poison (frequency 1/round for 6 rounds; effect 1d2 Con; cure 1 save). The poison's save DC is equal to this spell's DC.}
        
\DeclareSpell{Toxic Rupture}{necromancy [poisonUM]|V,  S,  M/DF (a viper's fang)|1 standard action|close (25 ft. + 5 ft./2 levels)|Targets: one poisonous creature|instantaneous|Fortitude negates; see text|yes}[]
    \DeclareSpellDescription{Toxic Rupture}{With a squeezing hand gesture, you cause internal bleeding in the target creature's venom glands or similar organs. The target must attempt a saving throw against its own poison using the DC for this spell. If it fails, the creature is poisoned and suffers the full effect of its poison. Any subsequent saving throws that the target must attempt against the poison use the normal DC for the poison rather than this spell's DC.}
        
\DeclareSpell{Venomous Bite}{transmutation [poisonUM]|V,  S,  M (a vial of injury poison worth at least 75 gp),  DF|1 standard action|touch|Targets: creature touched|1 round/level (D)|Fortitude negates (harmless)|yes (harmless)}[]
    \DeclareSpellDescription{Venomous Bite}{When you touch the poison vial against the target creature and cast this spell, the vial drains and the target's teeth become envenomed with the same poison. The creature gains the use of 1 dose of the poison for every 5 caster levels you possess (maximum 4 doses). The creature gains immunity to that poison while this spell is in effect. The poison retains its normal properties, and this spell does not alter the DC to resist or cure the poison.  Each successful bite attack counts as an expended dose. When all the poison is used, the spell ends. If the target creature doesn't have a bite attack, or if the target's bite attack is already poisonous, the spell has no effect. If the target has multiple heads, the spell affects only one bite attack.  When this spell ends, any unused poison is harmlessly expelled from the target's mouth and cannot be used again.}
        
\DeclareSpell{Blend With Surroundings}{illusion (glamer)|V,  S,  M/DF (a chess piece)|1 round|close (25 ft. + 5 ft./2 levels)|Targets: one creature|10 minutes/level|Fortitude negates (harmless) or Will disbelieves (if interacted with)|yes (harmless)}[]
    \DeclareSpellDescription{Blend With Surroundings}{This spell changes the appearance of the affected creature so that it better blends in with its surroundings. As chosen by you, the affected creature takes on the appearance of a statue, furniture, a tree, a bush, a rock, or another object of similar size. As long as the target stays still, it gains a +20 bonus on Stealth checks, and it can use Stealth even if it is being observed. The target's armor blends in perfectly with the illusory shape, and the target's armor check penalty on Stealth checks is negated for the duration of the spell. If the target moves at all while this spell is in effect, the spell ends.}
        
\DeclareSpell{Body Double}{illusion (glamer)|V,  S,  M (two glass beads)|1 standard action|close (25 ft. + 5 ft./2 levels)|Targets: two creatures; see text|1 round/level|Will negates|yes}[]
    \DeclareSpellDescription{Body Double}{Upon casting this spell, choose a primary target and a secondary target, both of which must be within range. If the primary target fails or forfeits its saving throw, its appearance, scent, sounds, and mannerisms change to match those of the secondary target.  As long as the two targets are of the same size category, they are indistinguishable. As a consequence, if the targets are adjacent and a creature takes an action that would affect one of the targets (such as an attack, a targeted spell, or an area effect), that action has a 50\% chance of affecting the other target instead.  Any action that would affect both of the creatures affects them both normally.  This spell does not deceive creatures that have true seeing. Likewise, a creature that can't perceive one of the targets is not fooled by this spell (even if the spell fooled that creature earlier), and its attacks, targeted spells, and other actions affect targets as normal.}
        
\DeclareSpell{Grasping Tentacles}{conjuration (creation)|V,  S,  M (octopus or squid tentacle)|1 standard action|medium (100 ft. + 10 ft./level)|Area: 20-ft.-radius spread|1 round/level (D)|none|no}[]
    \DeclareSpellDescription{Grasping Tentacles}{This spell functions as black tentacles, except the tentacles blindly grasp at the targets' eyes and ears, and tug at their hair, clothes, and equipment. Instead of grapple attempts, the tentacles attempt dirty trick combat maneuver checks (CMB = your caster level + 4 [the tentacles' Strength bonus] + 1 [the tentacles' size bonus]). Roll on the following table to determine the effect of a successful dirty trick combat maneuver check.  d\%Effect1-20Blinded. The tentacles suction over the victim's eyes.21-40Deafened. The tentacles clap themselves around the victim's ears.41-60Entangled. The tentacles weave around the victim's limbs.61-80Shaken. The tentacles menace the victim from all sides. This is a fear effect.81-100Sickened. The tentacles give off sickening vapors. This is a poisonUM effect.}
        
\DeclareSpell{Greater Grease}{conjuration (creation)|V,  S,  M (butter)|1 standard action|medium (100 ft. + 10 ft./level)|Targets: one object/2 levels or 1 10-ft. square/2 levels; see text|1 min./level (D)|see text|no}[]
    \DeclareSpellDescription{Greater Grease}{This spell functions like grease, except as noted here. When you use this spell to cover a solid surface with slippery grease, the 10-foot squares must form a continuous area, each part of which must be within the spell's range. When you use this spell to create a greasy coating on items, no two of the targeted items can be more than 30 feet apart. You cannot target both items and surfaces with a single casting of this spell.\\\\

{\centering\bf Grease\hrule}

A grease spell covers a solid surface with a layer of slippery grease. Any creature in the area when the spell is cast must make a successful Reflex save or fall. A creature can walk within or through the area of grease at half normal speed with a DC 10 Acrobatics check. Failure means it can't move that round (and must then make a Reflex save or fall), while failure by 5 or more means it falls (see the Acrobatics skill for details). Creatures that do not move on their turn do not need to make this check and are not considered flat-footed.

The spell can also be used to create a greasy coating on an item.

Material objects not in use are always affected by this spell, while an object wielded or employed by a creature requires its bearer to make a Reflex saving throw to avoid the effect. If the initial saving throw fails, the creature immediately drops the item. A saving throw must be made in each round that the creature attempts to pick up or use the greased item. A creature wearing greased armor or clothing gains a +10 circumstance bonus on Escape Artist checks and combat maneuver checks made to escape a grapple, and to their CMD to avoid being grappled.}
        
\DeclareSpell{Hide Weapon}{transmutation|V,  S|1 standard action|personal|Targets: you|1 hour/level (D)||}[]
    \DeclareSpellDescription{Hide Weapon}{Upon casting this spell, a light or one-handed melee weapon in your hand melds with your flesh, accompanied by a disgusting sucking sound. The weapon disappears completely inside your arm, and thereafter for the duration of the spell, you can extend and retract the weapon as a move action.  While extended, the weapon remains partially melded with your hand and arm, providing a +5 bonus to your CMD against disarm combat maneuvers. While the weapon is retracted, its shape adjusts to fit inside your arm, and you retain the use of your hand. Spotting the weapon on casual inspection is impossible, but anyone frisking you can attempt a DC 25 Perception check  to notice a light weapon or a DC 20 Perception check to notice a one-handed weapon hidden inside your arm. If the weapon grants a bonus on Sleight of Hand checks to hide it (as does a dagger), the bonus is added to the DC to find the item. Anyone who sees you casting this spell doesn't need to succeed at a Perception check to know you have the hidden weapon.  If you cast this spell a second time, you can hide another weapon inside your other arm. You can extend and retract both weapons as part of the same action.}
        
\DeclareSpell{Selective Invisibility}{illusion (glamer)|V,  S,  M/DF (three translucent pebbles)|1 standard action|close (25 ft. + 5 ft./2 levels)|Targets: you and one other creature|1 round/level (D)|none|yes}[]
    \DeclareSpellDescription{Selective Invisibility}{Upon casting this spell, you turn invisible, as per the invisibility spell, and you choose a second target. That creature can still see you as though you were not invisible, and you can attack that creature without ending the invisibility effect on you. However, if you attack any other creature, this spell ends. If you cast this spell multiple times, you can attack any of the affected creatures without ending the spell.}
        
\DeclareSpell{Sense Vitals}{divination|V,  S,  M (a drop of blood)|1 standard action|personal|Targets: you|1 round/level||}[]
    \DeclareSpellDescription{Sense Vitals}{This spell makes your eyes shine blood red and allows you to see the vital areas and weak points of creatures within 30 feet of you as a warm glow. This allows you to use any manufactured weapon to make sneak attacks, as the rogue ability of the same name, dealing an additional 1d6 points of damage; this additional damage increases by 1d6 for every 3 caster levels you possess beyond 3rd, to a maximum of +5d6 at 15th level. This additional damage stacks with other sources of precision damage.}
        
\DeclareSpell{Shifting Shadows}{evocation [darkness]|V,  S,  M (a drop of ink)|1 standard action|medium (100 ft. + 10 ft./level)|Area: 20-ft.-radius emanation centered on a point in space|1 round/level (D)|none, see text|no}[]
    \DeclareSpellDescription{Shifting Shadows}{Waves of shifting shadows obscure the appearance of all creatures caught within the emanation, making it harder to tell friends from foes.  This spell does not cause affected creatures to risk accidentally attacking their allies, but it does require them to concentrate harder to keep track of their allies and foes. While this spell is in effect, every creature in the area can treat any other creature in the area as an allied threatening creature for the purpose of flanking. Further, whenever a creature in the area would provoke an attack of opportunity from an enemy, that creatures' allies in the area must succeed at a Reflex save or lose one attack of opportunity for that round as they suppress the instinct to attack a vulnerable target that might be an ally. The shadows do not hinder precision-based attacks or create areas of dim light.  Creatures under the effect of true seeing are not affected by this spell. Creatures with scent or keen scent, creatures that are blind, and creatures that operate effectively without vision (such as creatures with blindsight or blindsense) are also not affected by this spell.}
    
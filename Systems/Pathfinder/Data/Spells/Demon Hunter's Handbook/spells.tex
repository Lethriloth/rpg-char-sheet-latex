    
\DeclareSpell{Anti-Summoning Shield}{abjuration () []|V|1 standard action|medium (100 ft. + 10 ft./level)|Area50-ft.-radius|1 minute/level (D)|Will negates|yes}[]
    \DeclareSpellDescription{Anti-Summoning Shield}{Within the area of effect, this spell impedes the use of spells of the summoning subschool and other effects that summon creatures. Any such spells, spell-like abilities, or similar summoning effects used within the area have a percent chance of failure equals 5\% x your caster level, to a maximum 75\% chance of failure. If the summoning effect already has a percent chance of failure (as is often the case with the summon spell-like ability of outsiders), these percentages stack. This spell does not affect summoners attempting to summon their eidolon, but it does affect summoners casting other summoning spells.}
        
\DeclareSpell{Burst With Light}{evocation () [light]|V,  S|1 standard action|close (25 ft. + 5 ft./2 levels)|Targetsone creature|1 round/4 levels|see text|yes}[]
    \DeclareSpellDescription{Burst With Light}{The target becomes filled with intense magical light, taking 2d6 points of damage as the light bursts from its wounds and orifices (if the target is an undead creature, it instead takes 2d8 points of damage). In addition, the creature radiates bright light in a 30-foot radius and increases the light level by one step for an additional 30 feet beyond that area-darkness becomes dim light, dim light becomes normal light, and normal light becomes bright light. Creatures that take penalties in bright light take them while within the 30-foot radius of this magical light. A successful Will save halves the damage and negates the light effect.  For every 4 character levels you possess, the light continues to fill the creature for another round (to a maximum of 5 rounds at 20th level), though the target may make a Will save each round to halve the damage and end the effect. Any creature adjacent to the target that fails its save and takes damage takes half as much damage and is blinded for 1 round. A successful Reflex save halves this damage (to a total of one-quarter the damage taken by the target) and negates the blindness effect.}
        
\DeclareSpell{Detect Demon}{divination () []|V,  S,  DF|1 standard action|60 ft.|Areacone-shaped emanation|concentration, up to 10 minutes/level (D)|none (see text)|no}[]
    \DeclareSpellDescription{Detect Demon}{You sense the presence of a specific kind of evil-that of demons, their servants, and the Abyss. The amount of information revealed depends on how long you study a particular area or subject.  1st Round: Presence or absence of creatures with the demon subtype, creatures possessed by demons, creatures under the effects of spells or spell-like abilities cast by demons, or creatures otherwise tainted by demons. Creatures tainted by demons include tieflings with demonic heritages, sorcerers with abyssal bloodlines, creatures affected by a succubus's profane gift, creatures with demonic implants (see page 44 of Pathfinder Campaign Setting: Lords of Chaos, Book of the Damned, Vol. 2), or creatures who have the Demonic Obedience feat (Lords of Chaos 8), and those under significant demonic influence as determined by the GM. This spell does not detect creatures of chaotic evil alignment who are not demons or significantly influenced by demons.  Additionally, this spell detects whether or not a portal or similar magical passage leads to the Abyss.  2nd Round: Number of evil auras shed by creatures with the demon subtype in the area, as well as the power of the most potent evil aura present. If you are of good alignment, and the strongest evil aura's strength is overwhelming; if the creature has HD equal to at least twice your character level, you are stunned for 1 round and the spell ends.  3rd Round: The power and location of each aura, and what demon lord, if any, a demon is most closely affiliated with. If an aura is outside your line of sight, you discern the direction but not its exact location. Affiliation to a demon lord is only revealed when the creature detected is a demon (not merely a creature tainted by a demon). Demons receive a Will saving throw to resist revealing what demon lord they are affiliated with. If the demon succeeds at this saving throw or is not forsworn to a demon lord, you know only that this aspect of the spell returned no information.  Aside from what is detailed above, this spell otherwise functions similarly to detect evil in terms of aura power, lingering auras, overwhelming auras, and so forth.}
        
\DeclareSpell{Protection From Outsiders}{abjuration () [see text]|V,  S,  DF|1 standard action|touch|Targetscreature touched|1 minute/level (D)|Will negates (harmless)|no; see text}[]
    \DeclareSpellDescription{Protection From Outsiders}{This spell wards a creature from attacks by outsiders with a specific racial subtype, from mental control exerted by creatures of the chosen subtype, and from summoned creatures of that subtype. Only the subtypes of specific outsider races-angel, azata, demon, oni, psychopomp, protean, and so on-can be chosen as the subtype this spell protects against. Alignment subtypes or other general subtypes (like elemental, extraplanar, or native) cannot be selected. Outsiders without an outsider racial subtype (like genies, night hags, yeth hounds, or xills) are not affected by this spell.  This spell creates a magical barrier around the subject at a distance of 1 foot. The barrier moves with the subject and has three major effects.  First, the subject gains a +4 deflection bonus to AC and a +4 resistance bonus on saves when targeted by creatures of the chosen subtype.  Second, the subject immediately receives another saving throw (if one was allowed to begin with) against any spells or effects that possess or exercise mental control over the target creature. This functions in the same fashion as protection from evil, but only when the effect stems from outsiders of the chosen subtype, and the target's saving throw is made with a +4 morale bonus (using the same DC as the original effect).  Third, the spell prevents bodily contact by summoned creatures of the chosen subtype in the same manner as detailed in protection from evil.  This spell's descriptor varies depending on the outsider race selected, gaining the alignment descriptors opposite to the alignment of the outsider race-for example, lawful and good if the race is chaotic and evil, chaotic if the selected race is lawful, or none if the selected race is neutral.}
        
\DeclareSpell{Righteous Blood}{abjuration () [good]|V,  S,  DF|1 standard action|touch|Targetsone creature of good alignment|10 minutes/level (D)|Will negates (harmless)|yes (harmless)}[]
    \DeclareSpellDescription{Righteous Blood}{The target creature's innate goodness infuses its body with holy energy. While this energy does not directly empower the target, it can harm embodiments of evil. Any creature that damages the target with a slashing or piercing melee weapon is sprayed by the target's holy blood. If the attacker is a creature with the evil subtype, it takes 1d6 points of damage from divine power each time it successfully hits the target. If the target has the good subtype or an ability that grants it an aura of good (like paladins or some clerics), its blood instead deals 2d6 points of damage. Creatures that don't have the evil subtype or that are using reach weapons are not subject to this damage.}
        
\DeclareSpell{Telepathic Censure}{abjuration () []|V|1 standard action|medium (100 ft. + 10 ft./level)|Targetsone creature (see text)|1 minute/level (D)|Will negates|yes}[]
    \DeclareSpellDescription{Telepathic Censure}{This spell creates an invisible psychic interference that inhibits telepathic communication. When cast upon a creature that can communicate via telepathy, this spell prevents that ability's use- either to receive or project thoughts. When cast upon a creature without telepathy, the spell merely prevents the target from receiving telepathic communication. Those affected by this spell or that attempt to telepathically communicate with creatures under its effects are not innately aware that their communication is being inhibited. This spell temporarily disrupts spells and effects like telepathic bond or telepathic messages sent by a helm of telepathy, but not attacks and effects unrelated to communication like detect thoughts or a neothelid's psychic crush.}
    
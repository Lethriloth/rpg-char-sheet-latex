    
\DeclareSpell{Calculated Luck}{divination|V,  S|1 standard action|personal|Targets: you|1 round/level (D)|none|no}[]
    \DeclareSpellDescription{Calculated Luck}{You are able to use the Path of Numbers to substantially boost your combat preparedness, but your foresight comes at a price.

Roll 3d8 and refer to the Eight Basic Energies table on page 51.

After rolling ,assign each die to one of the following.

 Energy Type: You gain vulnerability to the energy type that matches the die's result for the duration of the spell. You take half again as much damage (+50\%) from that energy type, regardless of whether you successfully saved against the damaging spell or effect.

 Magic School: For the spell's duration, you cast all spells from the school that matches the die's result at +1 caster level.

 d20 Roll Type: You receive a +2 luck bonus on the d20 roll that matches the die's result for the duration of the spell.}
        
\DeclareSpell{Cleromancy}{divination|V,  S,  F/DF (a set of 64 chicken bones)|full-round action|personal|Targets: you|1 round/caster level|none|no}[]
    \DeclareSpellDescription{Cleromancy}{Cleromancy involves casting bones and interpreting the results. Those able to arrive at the proper interpretation are granted knowledge of coming events. Roll 1d4 per caster level. Group the dice by like results, and choose one of the groups. For the duration of cleromancy, you can apply a luck bonus equal to the result of the selected dice to any d20 roll.

You can apply this bonus to a number of rolls equal to the number of dice in the group. If cleromancy expires before you are able to allocate the total number of allotted bonuses, the remaining bonuses are lost.}
        
\DeclareSpell{Lucky Number}{transmutation|V,  S|1 standard action|touch|Targets: one willing creature|24 hours or until discharged|none|no}[]
    \DeclareSpellDescription{Lucky Number}{You are able to tweak tiny variables affecting a creature's immediate future in order to grant the target a bit of luck at the right time. Roll a d20; once during the duration of lucky number, when the target creature rolls that result (regardless of what type of dice the target rolls), the creature has the option to either reroll the result or add a +2 luck bonus to the result. The creature must decide to use this ability before the success or failure of the original roll is known. A creature can have only one lucky number at a time. If lucky number is cast on a creature already affected by that spell, the new number replaces the previous one.}
        
\DeclareSpell{Mathematical Curse}{necromancy [curse]|V,  S,  M/DF (a full set of 10 fingernails)|1 standard action|touch|Targets: one living creature|see text|Will negates|yes}[]
    \DeclareSpellDescription{Mathematical Curse}{Using the Path of Numbers, you are able to influence the seemingly random elements in the environment around a creature, reducing that creature's efficacy. Roll 3d8 and choose one of the dice-this die's result is the penalty mathematical curse imparts. Next, choose either of the two remaining dice; the d20 roll corresponding to that result on the Eight Basic Energies table is the roll the spell's penalty applies to. The result of the final die represents the number of rounds that mathematical curse lasts. A creature can be under the influence of only one mathematical curse at a time. If mathematical curse is cast on a creature already affected by that spell, the new curse replaces the previous one.}
        
\DeclareSpell{Numerological Evocation}{evocation|V,  S|1 standard action|see text|Targets: see text|instantaneous|Reflex half|yes}[]
    \DeclareSpellDescription{Numerological Evocation}{You can use the Path of Numbers to calculate the latent magical energies all around you and fire customized rays of elemental energy that damage nearby targets. Roll 1d6 per 2 caster levels you possess. This is your dice pool. You will use each die in the pool exactly once to customize numerological evocation. Choose a number of dice in the pool, and match their results to the values in the first column of the Eight Basic Energies table to determine the types of energy damage the spell deals. You can allocate any number of dice in this manner, provided you still have enough dice remaining for the subsequent steps.

Next, allocate one of the dice to represent the total number of target creatures you can affect, including the initial target. No matter how many creatures you are eligible to affect, no two affected creatures can be more than 15 feet apart. Next, allocate any number of the dice to represent the range of numerological resistance. The range of the spell is 10 x the sum of the allocated dice. Finally, add up the results of the dice remaining in the pool. This is the amount of damage that numerological evocation deals to the first creature hit. A successful Reflex save halves this damage. Secondary targets take 50\% of the damage dealt to the first creature hit (rounded up), and successful Reflex saves negate this damage.}
        
\DeclareSpell{Numerological Resistance}{abjuration|V,  S|1 standard action|close (25 ft. + 5 ft./2 levels)|Targets: see text|see text|none|no}[]
    \DeclareSpellDescription{Numerological Resistance}{Numerological resistance allows numerologists to use the Path of Numbers to grant their allies resilience in the face of elemental dangers. Roll 5d8; this is your dice pool. You will use each die in the pool exactly once to customize numerological resistance.

Choose one of the dice, and match its result to the appropriate value in the first column of the Eight Basic Energies table to determine the type of energy to which numerological resistance grants resistance. Next, allocate one of the dice to represent the number of creatures you can affect with numerological resistance (including yourself). All of these creatures must be within the spell's range. Then allocate one of the dice to represent the number of rounds numerological resistance lasts. Finally, sum the remaining two dice. This is the number of points of resistance that numerological resistance grants.}
        
\DeclareSpell{Spectral Saluqi}{necromancy|V,  S,  F (a precious metal canine statue worth 100 gp)|1 round|close (25 ft. + 5 ft./2 levels)|Effect: one spectral dog|10 minutes/level (D)|none|no}[]
    \DeclareSpellDescription{Spectral Saluqi}{This spell creates a spectral saluqi, an undead canine resembling an oversized hound with black fur, a gray ruff and tail, and milky gray eyes. The hound shares your alignment and can converse with you telepathically. It can see and attack ethereal creatures, and otherwise has the same statistics as a yeth hound (Pathfinder RPG Bestiary 286). It usually leaves no tracks because it prefers to fly an inch above the ground. You are immune to the hound's bay ability. The hound's bite is considered aligned to any single alignment you possess for the purposes of overcoming damage reduction.}
    
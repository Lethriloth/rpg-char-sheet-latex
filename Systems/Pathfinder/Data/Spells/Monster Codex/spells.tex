    
\DeclareSpell{Aura Of Cannibalism}{necromancy [evil]|V,  S,  M (a piece of flesh from another creature of your species)|1 standard action|personal|Area: 20-ft.-radius emanation centered on you|1 minute/level (D)|Fortitude negates|yes}[]
    \DeclareSpellDescription{Aura Of Cannibalism}{You emanate an aura that saps the strength of others of your kind and channels their energy into you. Each round that a creature starts its turn in the spell's area and shares both your creature type and at least one subtype you possess (if any), that creature must succeed at a Fortitude save or take 1d4 points of damage. You gain a number of temporary hit points equal to the amount of damage you deal with this spell (maximum 10 + your caster level). These temporary hit points go away when the spell ends.  As long as you have at least 5 temporary hit points while this spell is in effect, you gain a +1 competence bonus on attack rolls, saving throws, and skill checks. If you have 15 or more temporary hit points, this competence bonus increases to +2.}
        
\DeclareSpell{Isolate}{illusion (glamer)|V,  S|1 standard action|touch|Targets: creature touched|1 round/level (D)|Will partial|yes}[]
    \DeclareSpellDescription{Isolate}{You cause the target to become invisible and silent, but only to his allies. Any creature with an attitude of indifferent or worse toward the target, and any creature that wishes the target harm, can see the target normally. The target can see and hear himself, can cast spells with verbal components, and can use command words normally, but any effect that requires allies to see or hear the target does not function. An ally that can see invisible creatures can both see and hear the target of isolate.  If the target succeeds at its save, the duration of the spell is reduced to 1 round.}
        
\DeclareSpell{Control Vermin}{transmutation|V,  S,  DF/M (a beetle)|1 standard action|close (25 ft. + 5 ft./2 levels)|Targets: up to 1 HD/level of vermin, no two of which can be more than 30 ft. apart|1 hour/level|Will negates|no}[]
    \DeclareSpellDescription{Control Vermin}{You and a number of allies less than or equal to your caster level designated upon casting can use Handle Animal and Ride checks to influence or control the targeted vermin as if they were animals and had animal-level intelligence.}
        
\DeclareSpell{Dust Ward}{abjuration|V,  S,  M (a pinch of pumice)|10 minutes|touch|Targets: one magic item|permanent|none|no}[]
    \DeclareSpellDescription{Dust Ward}{You ward a magic item against other creatures who try to learn to use or copy it. When you cast the spell, you designate one creature type, subtype, or a specific individual. If the item is worn or carried for 1 continuous hour or more by a creature that doesn't match the designation, the dust ward disintegrates the item into worthless gray dust. This destruction also occurs if the creature attempts to study the item in order to learn its properties  or how to magically craft it (a miracle or wish spell used on the gray dust can still reveal this information). The destruction of the item doesn't harm the creature wearing or carrying the item (although the item's destruction may put the creature in harm's way, such as if a magical rope were being used to cross a chasm at the time). If the offending creature wears or carries the item for less than 1 hour and passes it to a different creature, the countdown to the item's destruction starts over.  The spell cannot affect a magic item with a caster level greater than half your caster level.}
        
\DeclareSpell{Ironskin}{transmutation [earth]|V,  S,  DF/M (a pinch of forge soot)|1 standard action|personal|Targets: you|1 minute/level (D; see text)||}[]
    \DeclareSpellDescription{Ironskin}{Your skin hardens and takes on the color and texture of rough iron. You gain a +4 enhancement bonus to your existing natural armor bonus (if you do not have a natural armor bonus, you are considered to have an effective natural armor bonus of +0). This enhancement bonus increases by 1 for every 4 caster levels above 4th, to a maximum of +7 at 15th level.  While you're under the effects of this spell, if an opponent confirms a critical hit or sneak attack against you with a physical weapon (not a spell or magical effect), you can dismiss this spell to negate the critical hit or sneak attack and treat it is as a normal hit. Dismissing the spell in this way is not an action, but you must be conscious and aware of the attack to do so.}
        
\DeclareSpell{Ice Slick}{evocation [cold]|V,  S|1 standard action|close (25 ft. + 5 ft./2 levels)|Area: 5-ft.-radius burst|instantaneous (see text)|Reflex partial (see text)|see text}[]
    \DeclareSpellDescription{Ice Slick}{You create a blast of intense cold, coating all solid surfaces in the area with a thin coating of ice. Any creature in the area when the spell is cast takes 1d6 points of cold damage + 1 point per caster level (maximum +10) and falls prone; creatures that succeed at a Reflex save take half damage and don't fall prone. Spell resistance applies to this initial effect.  A creature can walk within or through the area of ice at half its normal speed with a successful DC 10 Acrobatics check. Failure by 4 or less means the creature can't move that round (and must succeed at a Reflex save or fall); failure by 5 or more means it falls (see the Acrobatics skill on page 87 of the Pathfinder RPG Core Rulebook for details). Creatures that do not move on their turn do not need to attempt this check.  A 5-foot square of ice has hardness 0 and 3 hit points. The ice is an instantaneous effect, but persists as nonmagical ice. Under temperate conditions, the ice lasts 1 minute per level. In tropical environments, it might last only half as long. In cold environments where ice and snow persist without melting, it could last indefinitely.}
        
\DeclareSpell{Magic Boulder}{transmutation [earth]|V,  S,  DF|1 standard action|touch|Targets: up to three boulders touched|30 minutes or until discharged|Will negates (harmless, object)|yes (harmless, object)}[]
    \DeclareSpellDescription{Magic Boulder}{This spell works like magic stone, except you transmute as many as three boulders (rocks up to two size categories smaller than yourself) to use with the rock throwing ability or as siege engine ammunition. The boulder's damage increases by one step, and the boulder gains a +1 enhancement bonus on attack and damage rolls.}
        
\DeclareSpell{Fleshy Facade}{transmutation (polymorph)|V,  S|1 standard action|touch|Targets: corporeal undead creature touched|10 minutes/level (D)|yes (harmless)|yes (harmless)}[]
    \DeclareSpellDescription{Fleshy Facade}{The target's flesh fills out and gains a healthy, natural color. This gives the target the appearance of a living creature of the type it was when it was still alive (if applicable). Creatures casting spells such as detect undead must succeed at a saving throw (with a DC equal to the spell's save DC) to detect the target's presence, and if the target is intelligent, it gains a +10 bonus on Disguise checks to appear alive or recently deceased. If the undead has any features different from those of the type of living creature it most resembles (such as a ghoul's elongated teeth and claws), those features shrink and become less prominent, and the subject deals damage as though it were one size smaller. This spell has no effect on creatures that are skeletal or otherwise lack flesh.}
        
\DeclareSpell{Hungry Earth}{transmutation|V,  S|1 standard action|medium (100 ft. + 10 ft./level)|Area: 20-foot-radius spread|1 round/level|none|no}[]
    \DeclareSpellDescription{Hungry Earth}{The ground attempts to pull creatures beneath its surface as if hungry for the flesh of mortals.  Immediately, and at the beginning of each of your turns, every creature touching the ground within the area of the spell is the target of a grapple combat maneuver. Creatures that enter the area of effect are also automatically attacked. The ground does not provoke attacks of opportunity. The earth's CMB is equal to 5 + your caster level for the purpose of this combat maneuver check. Attempt the combat maneuver check only once each round and apply the result to all creatures in the area of effect.  Each time the ground succeeds at a combat maneuver check to grapple a foe, it drags the creature farther down, eventually forcing the creature below its surface. With the first successful check, the target gains the grappled condition. Grappled opponents can't move without first breaking the grapple (doing so requires a successful DC 20 combat maneuver or Escape Artist check). The ground receives a +5 bonus on combat maneuver checks to grapple opponents it is already grappling. After the second successful grapple combat maneuver check, the grappled creature is pulled to the ground and becomes prone. On the third successful grapple combat maneuver check, the creature is fully buried and must hold its breath or begin suffocating.  A buried creature can't attempt to escape unless the effect ends or it breaks the grapple. The DC to escape the grapple increases to 25 for a creature that has been pulled beneath the earth.}
        
\DeclareSpell{Bouncy Body}{transmutation|V,  S|1 standard action|touch|Targets: creature touched|10 minutes/level||}[]
    \DeclareSpellDescription{Bouncy Body}{The target's flesh becomes flexible and rubbery. It gains a +2 circumstance bonus on grapple combat maneuver checks and Escape Artist checks, as well as to its CMD against combat maneuver checks to grapple. Anytime the target would take falling damage, it treats falls as 20 feet shorter (minimum 0) for the purpose of determining damage. In addition, if the target falls against a hard surface, it can attempt an Acrobatics check (DC = the distance fallen) to attempt to bounce upward; success means the creature bounces upward half the distance fallen.}
        
\DeclareSpell{Mud Buddy}{conjuration (creation)|V,  S,  M (1 pint of water)|1 standard action|close (25 ft. + 5 ft./2 levels)|Targets: 5 cubic feet of earth or mud|1 hour/level (D) (see text)||}[]
    \DeclareSpellDescription{Mud Buddy}{You create a Small minion out of mud, and it obeys your commands. The mud buddy has AC 12, 10 hit points, Strength 5, and a speed of 30 feet. It can perform any tasks an unseen servant can, plus any similar tasks its Strength allows (it's able to lift up to 50 pounds), but instead of walking on water, it gains a swim speed of 30 feet.  You can command a mud buddy to move up to 5 feet and trip an opponent (CMB = your caster level + your spellcasting ability score modifier). After the trip attempt is resolved, the spell ends.  When the spell ends, the mud buddy reverts to a patch of wet earth.}
        
\DeclareSpell{Endothermic Touch}{transmutation|V,  S,  M/DF (a small bit of snakeskin)|1 standard action|touch|Targets: one living creature that has the dragon type or the reptilian subtype|1 round/level|Fort negates|yes}[]
    \DeclareSpellDescription{Endothermic Touch}{This spell slows the metabolism and other bodily functions of a creature for a short amount of time. The target is staggered and moves at half its normal speed (round down to the next 5-foot increment), but it can hold its breath for twice as long as normal.}
        
\DeclareSpell{Scale Spikes}{transmutation|V,  S,  M/DF (a small thorn)|1 standard action|close (25 ft. + 5 ft./2 levels)|Targets: one living creature/level that is a reptile, has the dragon type, or has the reptilian subtype, and that also has a natural armor bonus of at least +1|1 minute/level|Fort negates (harmless)|yes (harmless)}[]
    \DeclareSpellDescription{Scale Spikes}{When the target is affected by this spell, its scales grow jagged spikes. These spikes act like +1 armor spikes. The subject is automatically considered proficient with these scale spikes.}
        
\DeclareSpell{Greater Scale Spikes}{transmutation|V,  S,  M/DF (a bit of thistle)|1 standard action|close (25 ft. + 5 ft./2 levels)|Targets: one living creature/level that is a reptile, has the dragon type, or has the reptilian subtype, and also has a natural armor bonus of at least +1|1 hour/level|Fort negates (harmless)|yes (harmless)}[]
    \DeclareSpellDescription{Greater Scale Spikes}{This spell functions like scale spikes, except that the spikes growing out of the scales have an enhancement bonus on attack and damage rolls equal to +1 for every 4 caster levels (maximum +5). This bonus does not allow the spikes to bypass damage reduction aside from magic.\\\\

{\centering\bf Scale Spikes\hrule}

When the target is affected by this spell, its scales grow jagged spikes. These spikes act like +1 armor spikes. The subject is automatically considered proficient with these scale spikes.}
        
\DeclareSpell{Air Breathing}{transmutation|V,  S,  M/DF (flower or piece of grass)|1 standard action|touch|Targets: living aquatic creatures touched|2 hours/level; see text|Will negates (harmless)|yes (harmless)}[]
    \DeclareSpellDescription{Air Breathing}{The transmuted creatures can breathe air freely. Divide the duration evenly among all the creatures you touch. The spell doesn't make creatures unable to breathe water.}
        
\DeclareSpell{Blood In The Water}{necromancy [emotion]|V,  S,  DF|1 standard action|20 ft.|Area: 20-ft.-radius emanation centered on you|1 round/level|Will negates (harmless)|yes (harmless)}[]
    \DeclareSpellDescription{Blood In The Water}{As part of the casting of this spell, you must deal 1 point of piercing or slashing damage to yourself to release your blood. This causes you to take 1 point of bleed damage. While the spell is in effect, all sharks, feeders in the depths, and creatures with the blood frenzy ability in the area gain a +2 bonus to Strength and Constitution and take a -2 penalty to AC. This is treated as blood frenzy for the purposes of other feats and effects, and doesn't stack with the effects of actual blood frenzy. If you cease bleeding, the spell immediately ends.}
        
\DeclareSpell{Gift Of The Deep}{transmutation (polymorph)|V,  S,  DF|1 standard action|close (25 ft. + 5 ft./2 levels)|Targets: one non-mutated sahuagin/level, no two of which can be more than 30 ft. apart|1 minute/level (D)|Fortitude negate (harmless)|yes (harmless)}[]
    \DeclareSpellDescription{Gift Of The Deep}{You give the targets the appearance and many of the abilities of sahuagin mutants, with effects as described below. Choose one benefit for all targets of this spell. This spell has no effect on sahuagin that are already mutants or already under the effects of gift of the deep.  Four-Armed: The sahuagin sprouts an extra pair of arms-which can be used to make claw attacks (dealing 1d4 points of damage), or to wield weapons or hold items. It gains the benefits of the Multiattack and Multiweapon Fighting feats.  Malenti: The sahuagin's features shift to resemble those of an aquatic elf. It loses its light blindness as well as its claw and bite attacks. The sahuagin gains a +4 enhancement bonus to Dexterity and Charisma, and a +10 circumstance bonus on Disguise checks to appear to be an aquatic elf.  Prehistoric: The sahuagin grows in size, as enlarge person. It also gains a +2 enhancement bonus to its natural armor.  Shark-Blooded: The sahuagin's tail elongates and melds with its legs. Its mouth enlarges, increasing its bite damage by one size category (to 1d6 for a typical sahuagin). Its swim speed increases by 20 feet, but its land speed is reduced to 5 feet. The sahuagin can't be tripped.  Sightless: The sahuagin is blinded, but gains the benefits of the Blind Fight feat and blindsense with a range of 90 feet.  Spined: Spines grow on the sahuagin's scales. Any creature that successfully grapples with it, is grappled by it, or hits it with an unarmed strike or natural weapon takes 1d4 points of piercing damage. The sahuagin also gains the benefits of the Improved Grapple feat.}
        
\DeclareSpell{Spellsteal}{abjuration|V,  S|1 standard action|medium (100 ft. + 10 ft./level)|Targets: one creature|instantaneous and see text|Will negates|yes}[]
    \DeclareSpellDescription{Spellsteal}{You create a discordant blast of energy that disrupts the target's available magic and transfers knowledge of that magic to you.  If the target prepares spells, it must choose one of its prepared 3rd-level spells, which is immediately lost. If the target has no 3rd-level spells prepared, it loses a 2nd-level spell it has prepared. This progresses down to a 1st-level spell if the target has no 2nd-level spells prepared, and this spell has no effect if the target also has no 1st-level spells prepared. If the spell is on your spell list, you can cast this lost spell (using your caster level) on your next turn.  If the target is a spontaneous spellcaster, it loses one of its available 3rd-level spell slots. If the target has no available 3rd-level spell slots, it must lose a 2nd-level spell slot (progressing as above). Randomly select one of the target's spells known of that spell level; if that spell is on your spell list, you can cast it (using your caster level) on your next turn.  You must provide any focus or material components to cast the stolen spell.  If the target has more than one spellcasting class, choose one at random to be affected. This spell has no effect on spell-like abilities.  Any spell or spell slot lost because of this spell is treated as if the caster had failed a concentration check while trying to cast-the spell or spell slot is wasted and has no effect, but it is recovered normally the next time the character prepares spells or regains spell slots.}
        
\DeclareSpell{Sundered Serpent Coil}{conjuration (creation)|V,  S,  M (a snake scale)|1 standard action|medium (100 ft. + 10 ft./level)|Area: one 5-foot square|1 round/level (D)|none|no}[]
    \DeclareSpellDescription{Sundered Serpent Coil}{This spell functions like black tentacles, except it creates a Large decapitated snake, which erupts from the ground and grapples a creature you specify within its 5-foot reach. As a standard action, you can command the snake to release its grappled target and direct it to attack a different creature.\\\\

{\centering\bf Black Tentacles\hrule}

This spell causes a field of rubbery black tentacles to appear, burrowing up from the floor and reaching for any creature in the area.

Every creature within the area of the spell is the target of a combat maneuver check made to grapple each round at the beginning of your turn, including the round that black tentacles is cast. Creatures that enter the area of effect are also automatically attacked. The tentacles do not provoke attacks of opportunity. When determining the tentacles' CMB, the tentacles use your caster level as their base attack bonus and receive a +4 bonus due to their Strength and a +1 size bonus. Roll only once for the entire spell effect each round and apply the result to all creatures in the area of effect.

If the tentacles succeed in grappling a foe, that foe takes 1d6+4 points of damage and gains the grappled condition. Grappled opponents cannot move without first breaking the grapple. All other movement is prohibited unless the creature breaks the grapple first. The black tentacles spell receives a +5 bonus on grapple checks made against opponents it is already grappling, but cannot move foes or pin foes. Each round that black tentacles succeeds on a grapple check, it deals an additional 1d6+4 points of damage. The CMD of black tentacles, for the purposes of escaping the grapple, is equal to 10 + its CMB.

The tentacles created by this spell cannot be damaged, but they can be dispelled as normal. The entire area of effect is considered difficult terrain while the tentacles last.}
        
\DeclareSpell{Amplify Stench}{transmutation|V,  S|1 standard action|personal|Targets: you|10 minutes/level||}[]
    \DeclareSpellDescription{Amplify Stench}{You amplify your natural stench special ability-its save DC increases by 2, and creatures that fail their saving throws against your stench become nauseated rather than sickened. If your stench ability normally causes a creature to become nauseated (such as with the foul stench ability), the radius of your stench doubles instead. This spell has no effect if you don't possess the stench special ability.}
        
\DeclareSpell{Mark Of The Reptile God}{transmutation [curse]|V,  S,  DF|1 standard action|close (25 ft. + 5 ft./2 levels)|Targets: one creature|permanent|Fortitude negates|yes}[]
    \DeclareSpellDescription{Mark Of The Reptile God}{If you succeed at a ranged touch attack, you burn your handprint onto the flesh of a creature, dealing 1d6 points of acid damage. The mark can be placed on any exposed portion of the creature, typically the head or forearm. The flesh around the handprint becomes rough and scaly, like the hide of a lizard. It also glows with a green radiance (shedding light as a torch) when brought within 60 feet of you. While the handprint glows, the target takes a -2 penalty to AC against your attacks and on saving throws to resist any spell you cast or spell-like ability you use.  Additionally, each day the target remains cursed, more and more of its flesh becomes covered in reptilian scales. The target must succeed at a Fortitude save each day or take 1d4 points of Charisma damage, 1 point of which is Charisma drain instead. A creature reduced to 0 Charisma by this effect is immediately transformed into a small, harmless cave lizard, as the baleful polymorph spell.  As with the effects of bestow curse, the curse inflicted by this spell cannot be dispelled, but it can be removed with a break enchantment, limited wish, miracle, remove curse, or wish spell.}
        
\DeclareSpell{Swarm Of Fangs}{conjuration (summoning)|V,  S,  M (a lizard's tooth)|1 round|close (25 ft. + 5 ft./2 levels)|Effect: one swarm of animate teeth|1 round/level|none|no}[]
    \DeclareSpellDescription{Swarm Of Fangs}{You summon a swarm consisting of thousands of animate, flying teeth in a 10-foot-by-10-foot cube. These fangs attack all creatures within the swarm's area. You can summon the swarm so that it shares an area with other creatures, and you can move the swarm up to 40 feet each round as a move action. If you choose not to move the swarm, it automatically moves up to 40 feet to envelop the nearest creature (including you) if it has not already done so.  Creatures caught inside the swarm's area of effect take 2d6 points of damage. The fangs deal damage to all creatures sharing their area when they first appear, and at the end of their movement each round.}
        
\DeclareSpell{Transfer Regeneration}{transmutation|V,  S|1 standard action|close (25 ft. + 5 ft./2 levels)|Targets: 1 willing living creature|1 minute|none|no}[]
    \DeclareSpellDescription{Transfer Regeneration}{You bestow your regenerative abilities on the target. Your regeneration stops functioning for the duration of the spell, and the target gains your regeneration. For example, if you have regeneration 5 (acid or fire), your target gains regeneration 5 (acid or fire). This regeneration overlaps (does not stack) with any regeneration the creature already has, including other castings of this spell. This spell has no effect if you don't have the regeneration ability or your regeneration isn't functioning when you cast the spell.}
        
\DeclareSpell{Trial Of Fire And Acid}{evocation [acid,  fire]|V,  S|1 standard action|touch|Targets: creature touched|1 round/level|Fortitude half (see text)|no}[]
    \DeclareSpellDescription{Trial Of Fire And Acid}{The target creature is covered in burning acid that deals 1d6 points of acid damage and 1d6 points of fire damage each round. The subject can attempt a Fortitude saving throw each round to reduce the damage by half. Dousing the target in water ends the effect (both the acid and the fire), but rolling on the ground does not extinguish the fire or affect the acid.}
    
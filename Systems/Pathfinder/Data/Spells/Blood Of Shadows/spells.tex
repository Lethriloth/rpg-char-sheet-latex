    
\DeclareSpell{First World Revisions}{transmutation (polymorph)|V,  S|1 standard action|touch|Targets: willing wayang touched|24 hours (D)|Will negates (harmless)|yes (harmless)}[]
    \DeclareSpellDescription{First World Revisions}{This spell functions as ancestral regressionARG, except as noted above and as follows. The wayang loses her darkvision racial trait and gains the low-light vision racial trait in its place. The alignment and personality of the wayang are not affected by the transformation, but the spell conceals her alignment as per undetectable alignment. Unlike ancestral regression, this spell grants the target a +20 bonus on Disguise checks to pass as a gnome; even though the wayang appears as a gnomish analog of herself, the differences between gnomes and wayangs are great enough that she cannot be mistaken for a gnome by other wayangs who know her.}
        
\DeclareSpell{Darkvault}{abjuration|V,  S,  M (a stone that has never seen sunlight)|1 standard action|close (25 ft. + 5 ft./2 levels)|Area: 30-ft.-radius emanation|24 hours|none|no}[]
    \DeclareSpellDescription{Darkvault}{You ward an area's shadows such that light cannot penetrate them. The illumination level in the affected area no longer changes when nonmagical light enters it. Any magical effect must succeed at a caster level check (DC = 10 + your caster level) in order to change the light level within the spell's radius. Darkvault has no effect on spells or effects that would make the spell's area darker.  The spell must be cast on an area, such as a cave or room. A spellcaster of 11th level or higher can make darkvault permanent with a permanency spell, at a cost of 7,500 gp.}
        
\DeclareSpell{Fear The Sun}{transmutation|V,  S,  M (a drow eyelash)|1 standard action|medium (100 ft. + 10 ft./level)|Targets: up to one creature/level, no two of which can be more than 30 ft. apart|1 round/level|Fortitude negates|yes}[]
    \DeclareSpellDescription{Fear The Sun}{Each target that fails its saving throw gains light blindness, as per the universal monster rule. When exposed to bright light, affected targets are blinded for 1 full round and are dazzled in successive  rounds. If you cast this spell in the presence of bright light, any target that fails its save is blinded immediately, and dazzled starting at the beginning of its first turn.}
        
\DeclareSpell{Ignoble Form}{transmutation (polymorph)|V,  S,  M (a half-elf ear)|1 standard action|touch|Targets: one drow|24 hours|Fortitude negates (harmless)|no}[]
    \DeclareSpellDescription{Ignoble Form}{The target takes on the form of a half-elf from the surface world. Its skin, hair, and eyes change to match a specific human ethnicity. You can even change the target's facial features or produce light facial hair or stubble. The target loses its darkvision, light blindness, and light sensitivity traits, if it normally has them. The target gains low-light vision; a +3 racial bonus on a single Craft, Knowledge, Perform, or Profession skill of its choice; and both a +4 bonus on Bluff checks and a +10 bonus on Disguise checks to pass itself off as a half-elf.}
        
\DeclareSpell{Shadowmind}{illusion [phantasm]|V,  S,  M (a small square of black silk)|1 standard action|medium (100 ft. + 10 ft./level)|Targets: up to one creature/level, no two of which can be more than 30 ft. apart|1 minute/level|Will negates|yes}[]
    \DeclareSpellDescription{Shadowmind}{You dim your targets' perceptions of light and shadow, convincing them the space they occupy is dark. Each creature that fails its save perceives the world around it as one light level darker than its true illumination level. The spell does not change the light level outside of the targets' perception, and does not create magical darkness. However, the spell creates an illusion of darkness rather than actual darkness, so low-light and darkvision don't allow a target to see in the conditions created by the spell. Even targets that see normally through magical darkness suffer a loss of vision from this spell.}
        
\DeclareSpell{Umbral Strike}{necromancy (shadow) [darkness]|V,  S,  M (a black crossbow bolt)|1 standard action|medium (100 ft. + 10 ft./level)|Targets: 1 creature|1 round/level|Fortitude partial|yes}[]
    \DeclareSpellDescription{Umbral Strike}{You create a bolt of dark energy and use it to make a ranged touch attack that ignores concealment (but not total concealment).  If you hit, the target takes 1d6 points of damage per caster level (maximum 20d6). Half of this damage is cold damage and half of it is negative energy. The bolt's shadow expands and covers the target, rendering him blind for the duration of the spell. A successful Fortitude save halves the damage and negates the blind condition.}
        
\DeclareSpell{Dancing Darkness}{evocation [darkness,  shadowUM]|V,  S|1 standard action|medium (100 ft. + 10 ft./level)|Effect: Up to four spheres, all within a 10-ft.-radius area|1 minute/level (D)|none|no}[]
    \DeclareSpellDescription{Dancing Darkness}{You create either up to four spheres of darkness that each reduce the illumination level by one step within a 20-foot-radius, or one dimly lit, vaguely humanoid shape. Each sphere of dancing darkness must stay within a 10-foot-radius area of one another but can otherwise move as you desire (no concentration required): forward or back, up or down, straight or turning corners, or the like. The darkness can move up to 100 feet per round. The effect winks out if the distance between you and it exceeds the spell's range.  Dancing shadows can be made permanent with a permanency spell.}
        
\DeclareSpell{Motes Of Dusk And Dawn}{evocation [darkness,  light]|V,  S|1 standard action|medium (100 ft. + 10 ft./level)|Effect: Up to four motes, all within a 10-ft.-radius area|1 minute (D)|none|no}[]
    \DeclareSpellDescription{Motes Of Dusk And Dawn}{When you cast this spell, you create up to four motes that shed light or darkness in a 20-foot-radius, increasing or decreasing the illumination level by up to two categories. You decide whether each individual mote sheds light or darkness when the spell is cast.  The motes of dusk and dawn must stay within a 10-foot-radius area of one another but otherwise move as you desire (no concentration required): forward or back, up or down, straight or turning corners, or the like. The motes can move up to 100 feet per round. A mote winks out if the distance between you and it exceeds the spell's range.}
        
\DeclareSpell{Mydriatic Spontaneity}{evocation [darkness,  light]|V,  S|1 standard action|close (25 ft. + 5 ft./2 levels)|Targets: one living creature|1 round/level|Will negates|yes}[]
    \DeclareSpellDescription{Mydriatic Spontaneity}{You overstimulate the target with alternating flashes of light and shadow within its eyes, causing its pupils to rapidly dilate and contract. While under the effects of this spell, the target is racked by splitting headaches and unable to see clearly, becoming nauseated for the spell's duration. Each round, the target's pupils randomly become dilated or contracted for 1 round. During any round that its eyes are dilated, the target is blinded if exposed to bright light or dazzled if exposed to normal light. During any round its eyes are contracted, the target is blinded if exposed to darkness  or dazzled if exposed to dim light. In addition, any creature can attempt a Stealth check to avoid detection from the target, even if the creature lacks cover or concealment.}
        
\DeclareSpell{Mass Mydriatic Spontaneity}{evocation [darkness,  light]|V,  S|1 standard action|close (25 ft. + 5 ft./2 levels)|Targets: one or more living creatures, no two of which can be more than 30 ft. apart|1 round/level|Will negates|yes}[]
    \DeclareSpellDescription{Mass Mydriatic Spontaneity}{This spell functions as mydriatic spontaneity, except it can affect multiple creatures.}
        
\DeclareSpell{Penumbral Disguise}{conjuration [shadowUM]|V,  S|1 standard action|touch|Targets: creature touched|10 minutes/level (D)|none|no}[]
    \DeclareSpellDescription{Penumbral Disguise}{You mask your features with shadowy illumination, gaining a competence bonus equal to your caster level on Disguise checks and Stealth checks attempted while in normal light, dim light, or darkness. In addition, creatures that see you while you are in dim light or darkness are unable to discern any but the most general information about your appearance or actions. For example, they can determine your general shape (such as humanoid), as well as the gist of your actions (such as, "She was trying to break into the store"), but cannot determine your precise actions, your appearance, or any identifying information about you. In bright light, your normal appearance is revealed.}
        
\DeclareSpell{Shield Of Darkness}{evocation [darkness,  shadowUM]|V|1 standard action|personal|Targets: you|1 round/level (D)|none|no}[]
    \DeclareSpellDescription{Shield Of Darkness}{You shield yourself with darkness, reducing the illumination level in your space to magical darkness and granting you total concealment. Your opponents are automatically aware of which squares you occupy, preventing you from attempting Stealth checks using this concealment unless every square adjacent to you has an illumination level of darkness or lower. Shield of darkness does not hinder your vision, and creatures that can see in magical darkness ignore this effect.}
        
\DeclareSpell{Spotlight}{evocation [darkness,  light]|V,  S|1 standard action|long (400 ft. + 40 ft./level)|Targets: one creature|1 minute/level (D)|Reflex partial|yes}[]
    \DeclareSpellDescription{Spotlight}{You create a mobile area of bright light centered on one target while simultaneously suppressing other light sources surrounding it. The light level in the target's space increases to bright light, causing the target to take any penalties that it would normally take in bright light. In addition, all mundane light sources (and magic light sources of 3rd spell level or lower) within 20 feet of the target's space are suppressed, shedding no light as long as they remain within this spell's affected area and reverting the area normally affected by those light sources to their unmodified illumination levels.  The effects of spotlight are centered on the target and move as the target does. As a result, the target takes a -20 penalty on all Stealth checks for the spell's duration and cannot benefit from concealment normally provided by darkness, as though illuminated with faerie fire.  If the target succeeds at its Reflex save, the spotlight is created in the target's square but does not move with the target, and it hinders the Stealth checks only of creatures within that square.}
        
\DeclareSpell{Touch Of Blindness}{necromancy [darkness,  shadowUM]|V|1 standard action|touch|Targets: creature or creatures touched (up to one/level)|1 round/level (see text)|Fortitude negates|Yes}[]
    \DeclareSpellDescription{Touch Of Blindness}{A touch from your hand, which is engulfed in darkness, disrupts a creature's vision by coating its eyes in supernatural darkness. Each touch causes the target to become blinded for 1 round unless it makes a successful Fortitude saving throw. You can use this melee touch attack up to one time per caster level. Any touch attack not used after 1 round per caster level is lost.}
        
\DeclareSpell{Wall Of Split Illumination}{evocation [darkness,  light]|V,  S|1 standard action|medium (100 ft. + 10 ft./level)|Effect: 10-ft.-high vertical sheet of illumination up to 5 ft. long/level|1 minute/level (D)|none|no}[]
    \DeclareSpellDescription{Wall Of Split Illumination}{An immobile curtain of illumination springs into existence. When created, one side of the wall (designated by you) radiates bright light to a range of 60 feet away from that side while the other side radiates darkness to an equal distance. This effect alters the illumination level by up to two steps toward either bright light (the light side) or darkness (the dark side). The wall also obstructs vision through it, regardless of which side of the wall the viewer is on.}
        
\DeclareSpell{Baleful Shadow Transmutation}{illusion (shadow) [shadowUM]|V,  S|1 standard action|close (25 ft. + 5 ft./2 levels)|Targets: one creature|permanent|Will disbelief, then Fortitude negates|yes}[]
    \DeclareSpellDescription{Baleful Shadow Transmutation}{You infuse a target's shadow with energies from the Shadow Plane, shaping the shadow into one that appears to belong to a different creature, and tricking the target into believing it actually is that creature. When you cast this spell, choose one Huge or smaller creature of the animal type or one Medium or Small creature of the humanoid type. If the chosen creature is ill suited to the target's current environment, such as an aquatic creature not in water, the subject gains a +4 bonus on all saving throws against baleful shadow transmutation. If the subject fails its Will save, it believes that it is the chosen creature, causing it to lose its extraordinary, supernatural, and spell-like abilities, lose its ability to cast spells (if it had the ability), and gain the alignment, special abilities, and Intelligence, Wisdom, and Charisma scores of its new form in place of its own. It retains any class features (other than spellcasting) that aren't extraordinary, supernatural, or spell-like abilities.  When the subject is first targeted by this spell, and once every 24 hours thereafter, the subject must attempt a Will save in order to disbelieve this effect. If the save succeeds, the spell's effect ends. The first time the subject fails this save, it must also attempt a Fortitude save. If it also fails this Fortitude save, the subject permanently assumes the form of the chosen animal or humanoid, as per polymorph. This is a polymorph effect. Successfully disbelieving the spell returns the subject to its true form. If the subject fails its Fortitude save against the effects of baleful shadow transmutation, any further polymorph effects cast on the target automatically fail.  Incorporeal or gaseous creatures are immune to baleful shadow transmutation, and a creature with the shapechanger subtype can revert to its natural form as a standard action.}
        
\DeclareSpell{Masochistic Shadow}{necromancy [evil,  shadowUM]|V,  S|1 standard action|close (25 ft. + 5 ft./level)|Targets: one creature|1 round/level (D)|Will negates, then Reflex partial; see text|yes}[]
    \DeclareSpellDescription{Masochistic Shadow}{You animate the target's shadow with semi-living energies drawn from the Shadow Plane, instilling a maddening hunger for its owner's life energy within it. If the target fails its Will save, it takes 1d4 points of Strength damage as a quasi-real shadow manifests in its space and attacks it. This shadow remains attached to the target and moves wherever the target moves.  At the start of each subsequent round, the target must succeed at a Reflex save or take 1d4 additional points of Strength damage; a successful save reduces the Strength damage to 1 point. If its Strength score is reduced to 0 by this spell's effects, the target dies. If the target is in bright light, it gains a +2 bonus on Reflex saves against this spell. If the target is in darkness, it takes a -2 penalty on Reflex saves against this spell.}
        
\DeclareSpell{Shadow Transmutation}{illusion (shadow) [shadowUM]|V,  S|1 standard action|see text|Targets: see text|see text|Will disbelief (if interacted with); varies; see text|yes; see text}[]
    \DeclareSpellDescription{Shadow Transmutation}{You suffuse one subject's body with energy from the Shadow Plane, altering its form to match a creature from the Shadow Plane. Shadow transmutation can mimic any of the following spells: animal growth, anthropomorphic animalUM, enlarge person, fins to feetARG, longarmACG, polymorph, reduce person, and stone fist. If using shadow transmutation as polymorph, the target does not gain any sensory abilities that its new form has (such as low-light vision or darkvision) and the speed of any movement types gained from the spell cannot exceed the target's base speed or natural speed with those movement types (whichever is higher). A creature under the effects of shadow transmutation deals normal damage and has all the normal abilities and weaknesses of whatever form it assumes using the spell.  Any creature that interacts with a target under the effects of the spell (including attacking or being attacked by that creature) can attempt a Will save to recognize the target's true nature. Creatures that succeed at their Will saves to disbelieve  the illusion take one-fifth (20\%) of the normal damage from the target's natural attacks or special abilities granted by the target's shadowy form (if any), and the target's special abilities that don't deal damage have only a 20\% chance of working against them. Creatures that succeed at their saves see the shadow transmutation as transparent images superimposed over the target.}
        
\DeclareSpell{Greater Shadow Transmutation}{illusion (shadow) [shadowUM]|V,  S|1 standard action|see text|Targets: see text|see text|Will disbelief (if interacted with); varies; see text|yes; see text}[]
    \DeclareSpellDescription{Greater Shadow Transmutation}{This spell functions like shadow transmutation, except it can mimic greater polymorph instead of polymorph. The illusory attacks and special abilities of any shape assumed using greater shadow transmutation deal three-fifths (60\%) damage to nonbelievers, and nondamaging effects are 60\% likely to work against nonbelievers.\\\\

{\centering\bf Shadow Transmutation\hrule}

You suffuse one subject's body with energy from the Shadow Plane, altering its form to match a creature from the Shadow Plane. Shadow transmutation can mimic any of the following spells: animal growth, anthropomorphic animalUM, enlarge person, fins to feetARG, longarmACG, polymorph, reduce person, and stone fist. If using shadow transmutation as polymorph, the target does not gain any sensory abilities that its new form has (such as low-light vision or darkvision) and the speed of any movement types gained from the spell cannot exceed the target's base speed or natural speed with those movement types (whichever is higher). A creature under the effects of shadow transmutation deals normal damage and has all the normal abilities and weaknesses of whatever form it assumes using the spell.  Any creature that interacts with a target under the effects of the spell (including attacking or being attacked by that creature) can attempt a Will save to recognize the target's true nature. Creatures that succeed at their Will saves to disbelieve  the illusion take one-fifth (20\%) of the normal damage from the target's natural attacks or special abilities granted by the target's shadowy form (if any), and the target's special abilities that don't deal damage have only a 20\% chance of working against them. Creatures that succeed at their saves see the shadow transmutation as transparent images superimposed over the target.}
        
\DeclareSpell{Shadow Trap}{illusion (shadow)|V,  S|1 standard action|close (25 ft. + 5 ft./level)|Targets: one creature|1 round/level (D)|Will negates|yes}[]
    \DeclareSpellDescription{Shadow Trap}{You pin the target's shadow to its current location, causing the target to become entangled and preventing it from moving farther than 5 feet from its original position, as if its shadow were anchored to the terrain. Each round on its turn, the target can attempt a new saving throw to end the effect as a full-round action. A flying creature can only hover in place or fall while entangled in this manner. This spell automatically fails when cast on a creature that doesn't throw a shadow, and it ends if the creature is entirely in an area with no illumination.}
        
\DeclareSpell{Shadowform}{illusion (shadow) [shadowUM]|V,  S|1 standard action|touch|Targets: creature touched|1 round/level (D)|Will negates (see text)|yes}[]
    \DeclareSpellDescription{Shadowform}{You replace the target's body with mystic shadow material drawn from the Shadow Plane, rendering the target's physical form only quasi-real. Whenever a foe tries to directly attack the target of the spell (for instance, with a weapon or a targeted spell), that foe must attempt a Will save to disbelieve. If successful, the opponent can attack the target normally and is unaffected by shadowform for 1 round. If the foe fails, the target takes only one-fifth the normal amount of damage from the foe's successful attack or effect, and if the attack has a special effect other than damage, that effect is one-fifth as strong as normal (if applicable) or only 20\% as likely to occur. Objects automatically succeed at their Will saves against this spell.}
        
\DeclareSpell{Umbral Infusion}{necromancy [shadowUM]|V,  S|1 standard action|close (25 ft. + 5 ft./2 levels)|Targets: one mindless undead creature|1 minute/level|Will negates|yes}[]
    \DeclareSpellDescription{Umbral Infusion}{You infuse the target mindless undead creature with power drawn from the Shadow Plane, immediately granting it the advanced creature simple template. It gains a +2 bonus on all rolls, including damage rolls, a +2 bonus to special ability DCs, a +4 bonus to AC and CMD, and 2 additional hit points per Hit Die. The undead creature's destructive instincts take hold for the duration of this spell, and any attempts to control or command the undead creature have a 50\% chance of failing; if uncontrolled, the undead creature attacks any living creatures it sees. This spell has no effect on undead creatures that already have the advanced creature template.}
        
\DeclareSpell{Mass Umbral Infusion}{necromancy [shadowUM]|V,  S|1 standard action|close (25 ft. + 5 ft./2 levels)|Targets: one mindless undead creature/level, no two of which can be more than 30 ft. apart|1 minute/level|Will negates|yes}[]
    \DeclareSpellDescription{Mass Umbral Infusion}{This spell functions as umbral infusion, except it can affect multiple mindless undead creatures.}
    
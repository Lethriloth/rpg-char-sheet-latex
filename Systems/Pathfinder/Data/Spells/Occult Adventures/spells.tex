    
\DeclareSpell{Akashic Form}{necromancy () []|V,  S|1 hour|personal|Targetsyou|24 hours|no|no}[Store a copy of your body in the Akashic Record, and restore yourself to that form upon your death.]
    \DeclareSpellDescription{Akashic Form}{You create a perfect record of your physical body in the Akashic Record (see page 244) at the time the spell is cast. This record includes your current hit point total, physical ability scores (Strength, Dexterity, and Constitution), and any conditional modifiers or conditions such as ability damage to physical ability scores, disease, negative levels, and poison. If at any point within the duration of the spell you are reduced to fewer than 0 hit points or are slain by a death effect that is not mind-affecting, you can immediately let your current physical body die and assume the record of your physical body on your next turn. When this happens, your corpse disappears, and you can either resume the place of your dead body (already wearing any clothing still attached to the corpse) or appear in any place you've been within 500 feet of where your corpse lies (but without any of your gear).  You still retain your original mind, and therefore don't regain any spells. You are still under any mental influences and energy drain effects you were under when you recalled the record of your physical body, but don't retain any physical effects such as bleed damage or poison (unless you suffered from these conditions at the time the spell was cast). Spells affecting you when you store a record of yourself abide by their normal durations. For example, if you were affected by cat's grace when you created the record and you restore your body 1 hour later, you won't be under that effect anymore since its duration has already expired. This spell doesn't allow you to avoid dying  of old age. Casting this spell again replaces any previous version you cast; you can't store more than one copy of yourself in the Akashic Record.}
        
\DeclareSpell{Analyze Aura}{divination () []|V,  S|1 standard action|close (25 ft. + 5 ft./2 levels)|Targetsone creature or object|concentration, up to 1 round/level|none|no}[Read a creature’s or an object’s alignment, emotion, health, and magic auras.]
    \DeclareSpellDescription{Analyze Aura}{You peer into the aura of one target creature or object, gaining valuable information about its condition and nature. Each round, choose one of the target's four auras. This spell functions similarly to the read aura occult skill unlock (see page 197), but doesn't require checks and returns results on all the target's auras in an instant.  Alignment Aura: You study the target's spiritual nature to determine its alignment. You also learn the type and power of its alignment aura, as detect evil.  Emotion Aura: The colors playing within the target's aura reveal its emotional state, granting valuable insight into its psyche. You learn a general summary of the target's current disposition, as well as its attitude toward any other creatures within 30 feet of it. For the duration of the spell and for 1 hour afterward, you gain a +2 circumstance bonus on Bluff, Diplomacy, Intimidate, and Sense Motive checks you attempt against the target. Objects do not have emotion auras, except intelligent weapons and similar sentient oddities.  Health Aura: The flow of vital force animating the target becomes plainly visible, giving you insight into its physical condition. You know whether the target is unharmed or wounded, whether it is poisoned or diseased, and whether it is under any of the following conditions: confused, disabled, dying, nauseated, panicked, staggered, stunned, or unconscious. Further, your insight into the overall strength of the target's vital force reveals the total number of points available in its ki pool, grit pool, or similar resource. Objects and most undead creatures don't have health auras.  Magic Aura: You determine the number and power of all magical auras on the target (as detect magic to determine a magic aura's power), as well as the school of each aura. You can attempt Knowledge (arcana) or Spellcraft skill checks to determine the school or identify properties of a magic item, as normal. If cast on an item, analyze aura cuts through the obfuscation of the magic aura spell.}
        
\DeclareSpell{Anticipate Thoughts}{divination () [mind-affecting]|V|1 standard action|close (25 ft. + 5 ft./2 levels)|Targetsone creature|1 round/level|Will partial|yes}[Gain increasing bonuses to AC and on attack rolls and damage rolls against one creature]
    \DeclareSpellDescription{Anticipate Thoughts}{This spell taps into the target's mind so you get an impression of the actions it will take. You gain a +2 insight bonus to AC against the target's attacks. If the target fails its Will save, you also see how the target will react to your attacks, and the bonus applies on your attack rolls and damage rolls against the target. These bonuses apply only while the target is within range of the spell, though if it goes out of range, the bonuses return once it's back in range. Whenever the target misses you with an attack, the spell's bonuses increase by 1 until the spell ends (to a maximum of +5).}
        
\DeclareSpell{Apport Object}{conjuration (teleportation) []|V|1 standard action|touch|Targetsone touched object of up to 1 lb. and 1 cu. ft.|instantaneous or 1 hour/level|Will negates (object)|yes (object)}[Send or receive a small object via teleportation.]
    \DeclareSpellDescription{Apport Object}{This spell allows you to instantaneously transport a small nonliving object from one location to another. There are two ways to use the spell: sending allows you to immediately send an object held in your hands to a nearby location, while receiving permits you to cast the spell ahead of time on an object and summon it to your location at a later time.  Sending: If you choose to send the object elsewhere, the spell functions like teleport object, except the size of the object is limited and the distance it can travel is equal to only 25 feet + 5 feet per 2 levels. You can't transport an object to the Ethereal Plane. You can send the held object to any square within range, and you don't need line of sight to the target location. You can place the object in the open or inside a container, a pocket, or even someone's hand. If there isn't room in the space you select (either because the space chosen is too small or because there is already something else there), or if the person doesn't want or isn't expecting the object in his hands, it appears on the ground within the target's square instead. You can transport the object to an elevation above the floor as long as the destination is within the spell's total distance limit.  Receiving: You can prepare an object ahead of time to apport it to yourself by casting the spell upon it and assigning a mental trigger to complete the spell. You don't need line of sight to the object to apport it to you, but the object must be within a distance equal to 25 feet + 5 feet per 2 levels. Completing the spell is a swift action that has the same restrictions as a thought spell component. You can apport an object in this way even if someone holding the object is unwilling to let you take it. Once you apport the object, the spell ends.  You can't send or receive an object into a space that is protected by an antimagic field, globe of invulnerability, or similar effect that keeps magical effects out; if you attempt to do so, the spell is lost.}
        
\DeclareSpell{Apport Animal}{conjuration (teleportation) []|V|1 standard action|touch|Targetsone touched animal of Tiny or smaller size|instantaneous or 1 hour/level|Will negates|yes}[Send or receive a Tiny or smaller animal via teleportation.]
    \DeclareSpellDescription{Apport Animal}{This spell functions like apport object except the target is an animal. Only normal, nonmagical creatures of the animal type can be teleported.}
        
\DeclareSpell{Aura Alteration}{illusion () []|V,  S|1 standard action|touch|Targetsone object or willing creature|1 day/level (D)|Will negates (harmless, object)|yes}[Masks a creature’s or an object’s alignment, emotion, health, and magic auras.]
    \DeclareSpellDescription{Aura Alteration}{You mask and manipulate the target creature's or object's aura, confounding those who would attempt to discern helpful information from it using the read aura occult skill unlock (see page 197) or the analyze aura spell. You can change each of the following four auras with one casting of aura alteration.  Alignment Aura: You can change the target's apparent alignment to thwart spells that detect alignment, such as detect evil. You can alter the alignment by up to one step on each alignment axis, but can't make the creature appear to have a diametrically opposed alignment. For instance, you couldn't make a lawful evil character appear to be chaotic or good. You can also adjust the power of the alignment aura up or down in level by a number of steps equal to 1/2 your level or fewer.  Emotion Aura: Your manipulations mask the true emotions  of the target, instead depicting any combination of colors on the emotional spectrum that you wish. Creatures attempting Bluff, Diplomacy, Intimidate, or Sense Motive checks against the target receive no bonuses on their checks based on information gleaned from the target's aura.  Health Aura: You adulterate the target's aura to mask the condition of its physical body, revealing instead a wounded status, poison or disease status, and conditions of your choice. You likewise mask the target's available ki points or similar resources, manipulating its aura to suggest any number within the range normally available to the target.  Magic Aura: You alter a creature or item's magic aura so that it registers to detect magic (and spells with similar capabilities, like analyze aura) as though it were nonmagical, a magic item of a kind you specify, or the subject of a spell you specify. A single casting of aura alteration is sufficient to mask all of the magic auras on the target. If an object bearing aura alteration has identify cast on it or is similarly examined, the examiner recognizes the aura is false and detects the object's actual qualities if he succeeds at a Will save. If a targeted item's own aura is overwhelming, aura alteration can't alter it.}
        
\DeclareSpell{Aversion}{enchantment (compulsion) [mind-affecting]|V,  S|1 standard action|close (25 ft. + 5 ft./2 levels)|Targetsone creature|1 day/level|Will partial|yes}[Cause the target to avoid an object or location.]
    \DeclareSpellDescription{Aversion}{You plant a revulsion in the mind of the subject, causing her to avoid an object or location. You must choose a specific object or place. A location chosen in this way can be no larger than a cube measuring 50 feet on a side. The aversion is entirely in the target's mind, so the chosen object or location itself isn't subject to any magical effect. If the target fails her saving throw, she can't come within 60 feet of the chosen object or place. She makes every reasonable effort to avoid the object of the aversion, but will not put herself in danger in order to maintain the aversion. For example, if the object of the aversion is a bridge but a forest fire is closing in and will likely kill the target, she ignores the aversion and crosses the bridge to save herself. If the target must ignore the conditions of the aversion, she is nauseated until she is no longer violating the aversion.  If the target succeeds at her saving throw, she is instead sickened while within 60 feet of the object or place, but isn't compelled to stay away from it.}
        
\DeclareSpell{Awaken Construct}{transmutation () []|V,  S,  M (herbs and oils worth 2, 000 gp per HD of target),  DF|24 hours|touch|Targetsmindless construct touched|instantaneous|Will negates|no}[Grant a construct humanlike sentience.]
    \DeclareSpellDescription{Awaken Construct}{You amplify the animating force of a construct to more closely resemble a true soul, granting the construct humanlike sentience. To do so, you must succeed at a Spellcraft check (DC = 15 + the construct's current Hit Dice). If the construct's master (if any) is present, this is an opposed Spellcraft check. You have no special empathy or connection with a creature you awaken-it is a free-willed creature. Golems previously under another creature's control, either as shield guardians or a golem crafter's creations, break all connections with that creature.  Roll 3d6 to determine the Intelligence score of the awakened construct, and increase its Charisma score by 2d6. It gains feats and skill points according to its new Intelligence score, and the skill points are assigned appropriately for its function, as determined by the GM.  An awakened construct speaks one language that its creator spoke, plus one additional language that its creator knew per point of the construct's Intelligence bonus (if any). Its alignment is determined by the GM, but is usually within one step of its creator's alignment. This spell doesn't function on a construct with an Intelligence score.}
        
\DeclareSpell{Babble}{enchantment (compulsion) [mind-affecting]|V,  S|1 standard action|close (25 ft. + 5 ft./2 levels)|Targetsone creature; see text|1 round/level|Will negates|yes}[Target becomes nauseated and causes nearby creatures to become fascinated.]
    \DeclareSpellDescription{Babble}{This spell causes the target to break into a fit of bizarre, uncontrollable babbling. The target also becomes nauseated. If the target succeeds at its save, the effects end. If not, the creature continues babbling and is nauseated for the entire duration.  Creatures within 30 feet of the subject that can hear the target's babbling must succeed at a Will save or become fascinated for as long as the babbling persists. Once a creature's fascination ends, it can't become fascinated by the same instance of babble again.  Creatures with an Intelligence score of 2 or lower aren't affected by this spell.}
        
\DeclareSpell{Bilocation}{conjuration (creation) []|V|1 standard action|close (25 ft. + 5 ft./2 levels)|Effectone duplicate|1 round/level (D)|none|no}[Exist in two places at once.]
    \DeclareSpellDescription{Bilocation}{The spell creates an identical copy of you, along with everything you wear and carry, anywhere you choose within range; you exist in two places at once until the spell ends. You and the duplicate use the same statistics and share the same resources. If the duplicate takes damage, for example, you deduct the damage from your hit point total. Similarly, if your duplicate expends a charge or daily use from a magic item you both wield, the charge or daily use is expended from the item you carry. If you or the duplicate drops or gives away something you're carrying, the item disappears from the other body as well. This spell doesn't duplicate artifacts; any you possess remain on you.  You perceive sensory information from your body and that of your duplicate simultaneously. The spell enables you to process the sensations so you don't find them disorienting.  When you take any action, you choose which of your bodies performs the action, but both bodies share the same pool of actions. For instance, if you take a standard action to cast a spell,  you can use either body as the point of origin, but the other body can't also take a standard action that round. Likewise, if your duplicate takes a full-round action, you couldn't take a standard or move action. Both bodies can take any number of free actions as usual; for example, both bodies could say different things or each drop a different item. If either body doesn't move during the round, that body can either take a 5-foot step or move your speed once as a free action.  The two bodies are affected by attacks, spells, and effects as though they were one person, taking the worse effect when applicable (for example, if the bodies would be subject to differing effects due to being at different ranges). If both bodies are in the area of the same fireball, you would attempt the saving throw only once and take the damage only once. If one body is targeted by hold person, both would become paralyzed on a failed save. Both bodies count as a single creature for effects that target a specific number of creatures, and they can't be chosen more than once for such effects. You do count your other body as another creature for most combat effects, such as flanking or determining cover. However, you don't count as two unique creatures for the purposes of teamwork feats or effects you use that can target only creatures other than yourself.  Any magical effect with a duration affecting you has its duration halved while you're bilocating. For example, the hold person spell mentioned above would lose 2 rounds' worth of duration per round until your duplicate disappeared. If you were under an eagle's splendor e ffect t hat h ad 1 minute remaining when you cast bilocation, the effect would end after 5 rounds instead of 10. An effect shortened in this way lasts a minimum of 1 round total, and if an effect that lasts an odd number of rounds has 1 round remaining, it has its full effect on both of your bodies for that round.  When the spell ends, you decide whether you or your duplicate disappears. If you disappear, you become your duplicate. If you are carrying artifacts when you do this, they transfer with your consciousness.}
        
\DeclareSpell{Burst Of Adrenaline}{transmutation () []|V,  S|1 immediate action|personal|Targetsyou|instantaneous|none|no}[Gain a +8 bonus to Str, Dex, or Con for one roll, then be fatigued for 1 round.]
    \DeclareSpellDescription{Burst Of Adrenaline}{You draw upon your body's inner reserves of strength, leaving you winded. When you are about to make a d20 roll based on Strength, Dexterity, or Constitution, you can cast this spell to gain a +8 enhancement bonus to Strength, Dexterity, or Constitution for that roll, but you are fatigued for 1 round afterward.}
        
\DeclareSpell{Burst Of Insight}{transmutation () []|V,  S|1 immediate action|personal|Targetsyou|instantaneous|none|no}[Gain a +8 bonus to Int, Wis, or Cha for one roll, then be dazed for 1 round.]
    \DeclareSpellDescription{Burst Of Insight}{You plumb the depths of your mind for insight, leaving you momentarily frazzled. When you are about to make a d20 roll based on Intelligence, Wisdom, or Charisma, you can cast this spell to gain a +8 enhancement bonus to Intelligence, Wisdom, or Charisma for that roll, but you are dazed for 1 round afterward.}
        
\DeclareSpell{Call Spirit}{conjuration (calling) []|V,  S|10 minutes|10 ft.|Effectcall the spirit of a single deceased humanoid creature|concentration|Will negates; see text|no}[Make the spirit of one creature manifest.]
    \DeclareSpellDescription{Call Spirit}{You attempt to cause the spirit of a specific individual to manifest from its final resting place. You must request the spirit's presence by speaking its name.  Unwilling spirits can resist the summons by succeeding at a Will save. If the chosen spirit resists your call, another spirit with malevolent intent almost always takes its place, intent on deceiving you. The difficulty of the save depends on how well you know the subject and what sort of physical connection (if any) you have to the creature whose spirit you wish to call.     KnowledgeWill Save ModifierName only*+15Secondhand (you have heard of the subject)+10Firsthand (you met the subject in life)+5Familiar (you knew the subject well)+0* You must have at least a name the creature was called in life. ConnectionWill Save ModifierLikeness or picture-2Possession or garment-4Body part, lock of hair, nail clipping, etc.-10Different alignment+4     A successfully called spirit manifests as a wispy, vaporous form that vaguely resembles the form the deceased creature had in life. The spirit has the physical attributes of an unseen servant and is capable of minor physical manipulations, with the ability to speak in quiet, ghostly whispers in whatever languages the creature knew while it was alive. The spirit isn't an undead creature, and isn't beholden to its caller. Whether you summoned the chosen spirit or a deceitful replacement, the spirit can refuse to answer your questions or attempt to deceive you using Bluff, though in either case these spirits can speak about only what they knew in life and have no knowledge of events that occurred after their deaths. Malevolent spirits might take advantage of their limited physical abilities to cause terrifying spectacles designed to scare the caster.  You must concentrate on maintaining the spell (as a standard action) in order to ask questions at the rate of one per round. You can ask a total of one question per caster level; the spirit answers each during the same round. When the spell ends, the spirit's ectoplasmic form fades and the soul returns to its rest. This spell can't call the spirits of creatures that are currently undead.}
        
\DeclareSpell{Calm Spirit}{necromancy () []|V,  S|1 standard action|close (25 ft. + 5 ft./2 levels)|Targetsone incorporeal undead creature or haunt|1 minute or 1 round/level; see text|Will negates or none; see text|yes}[Postpone hostile action by a haunt or incorporeal undead.]
    \DeclareSpellDescription{Calm Spirit}{This spell temporarily calms agitated haunts and incorporeal undead such as ghosts. You have no control over the affected creatures, but calm spirit postpones hostile action by the affected spirits for the duration of the spell. Entities so affected cannot take violent actions or do anything destructive, including triggering persistent haunt abilities, though they can defend themselves. Any aggressive action against or damage dealt to a calmed spirit or haunt immediately ends the effect.  Haunts do not receive a saving throw against the spell, but the caster must succeed at a caster level check whose difficulty is equal to at least 10 + the haunt's CR in order to temporarily calm the angry entity. The spell's duration changes to concentration (up to 1 round/level) when affecting a haunt.}
        
\DeclareSpell{Catatonia}{necromancy () []|S|1 standard action|touch|Targetswilling creature touched|1 hour/level (D)|none|yes}[Make a willing target appear to be dead.]
    \DeclareSpellDescription{Catatonia}{You touch the target and place it into a deathlike state that persists for the duration. The target appears to be dead, and any creature that interacts with the target must succeed at a DC 20 Heal check to recognize it is actually alive.  Until the spell ends, the target counts as if it were dead for the purpose of resolving any effects that target or affect only living creatures, but it doesn't count as undead. The subject is helpless, and can still be killed normally.  Any effect that would bring the creature back to life or animate it as an undead fails, but ends the catatonia. The target can be affected by spells that affect only objects, including animate objects (if the creature is Small) and teleport object. However, anything that would cause the body to change form (such as shrink item) fails and ends the catatonia. This doesn't prevent the effects of spells that simply deal damage or otherwise destroy objects.}
        
\DeclareSpell{Charge Object}{transmutation () []|V,  S|10 minutes|touch|Targetsobject touched|permanent|none; see text|no}[Infuse psychic energy and ownership history into an item.]
    \DeclareSpellDescription{Charge Object}{You charge an item with minor psychic energy. The item can be detected by the detect psychic significance spell. If you wish, you can imprint the item with your ownership history. Spells such as object reading a nd u ses o f t he p sychometry o ccult skill unlock (see page 196) can reveal any information about yourself you impart into the object, including your name, alignment, profession, and a summary of your experiences with the item. You can't impart false information into the object, but you can omit any details you'd prefer not to divulge. If the target object is already psychically charged, you can add more information to it, but you can't use charge object t o e rase existing psychic information.}
        
\DeclareSpell{Cognitive Block}{enchantment (compulsion) [mind-affecting]|V|1 standard action|close (25 ft. + 5 ft./2 levels)|Targetsone creature|1 round/level (D)|Will negates|yes}[Add a thought component to all of the target’s spells.]
    \DeclareSpellDescription{Cognitive Block}{You create a mental block in the target's mind, impeding the flow of his mental spellcasting process and forcing him to incorporate a thought component into any spell or spell-like ability he uses. This is in addition to any other components the spell already requires, and doesn't replace the verbal component. Spell-like abilities require this thought component, even though they normally don't need spell components.  As usual with thought components, this addition increases the DC of any concentration checks the target attempts by 10 unless the target takes a move action to center his mind and satisfy the spell's thought component.}
        
\DeclareSpell{Condensed Ether}{transmutation () []|V,  S,  M (crushed amber)|1 standard action|medium (100 ft. + 10 ft./level)|Targets20-ft.-radius spread|1 minute/level|none|no}[Creates a planar conjunction that slows movement, penalizes AC and Reflex saves, and imposes a miss chance on ranged attacks.]
    \DeclareSpellDescription{Condensed Ether}{You condense the substance of the Ethereal Plane as it interpenetrates the Material Plane. This thickened planar conjunction slows movement through the area to a crawl. Creatures moving through condensed ether (even incorporeal creatures), move at only half their normal speed and can't take 5-foot steps. This slowing of movement doesn't stack with solid fog or similar effects. Creatures within condensed ether take a -2 penalty to Armor Class and on Reflex saves, and condensed ether prevents effective ranged weapon attacks. Even magic rays and similar ranged attacks suffer a 20\% miss chance on attacks into or passing through condensed ether. This miss chance is not based on concealment, and Blind-Fight, true seeing, and similar effects do not reduce it.}
        
\DeclareSpell{Contagious Zeal}{enchantment (compulsion) [emotion,  mind-affecting]|V,  S|1 standard action|close (25 ft. + 5 ft./2 levels)|Targetsone creature|1 round/level|Will negates (harmless)|yes (harmless)}[Grant bonuses and temporary hit points that spread from creature to creature.]
    \DeclareSpellDescription{Contagious Zeal}{The target gains a +2 morale bonus on attack rolls and weapon damage rolls, 1d6 temporary hit points, and a +4 morale bonus on saving throws against fear effects and to the DC of  Intimidate checks attempted against her. Once per round, the target can select one other creature to gain this bonus as well. The chosen creature can be no farther from the target than 25 feet + 5 feet for every 2 caster levels you possess, and a creature can't be selected more than once in this way. Such allies gain only the bonuses and temporary hit points; they don't continue to spread it to other creatures. The additional creatures' bonuses and temporary hit points share the original spell's duration, so when that duration ends, all affected creatures lose their bonuses and any remaining temporary hit points from this spell.}
        
\DeclareSpell{Create Mindscape}{illusion (phantasm) [mind-affecting]|V,  S|1 round|long (400 ft. + 40 ft./level)|Targetsone creature|10 minutes/level|Will disbelief; see text|yes}[Form an immersive mindscape.]
    \DeclareSpellDescription{Create Mindscape}{You create an immersive mindscape (see page 235) that the minds of both you and another creature enter together. You choose whether the mindscape is overt or veiled, and whether it's harmless or harmful. You can choose any shape and size trait, as well as any gravity trait (though you can't make it so the gravity is so strong it harms creatures within). The mindscape has normal time, no alignment traits, and normal magic.  You designate where both you and your subject appear. You don't need line of sight to draw the subject creature into the mindscape, but you must be aware of its presence within range. If you target an area with more than one sentient creature and you have never seen any of the creatures before (for example, if you know a group of soldiers is inside a barracks but you can't see them through the door), the subject of this spell is selected at random. If you have seen firsthand the creature you wish to target (continuing the example, if you spied the sergeant entering the room moments before), you can select it unerringly from among all the creatures. You must appear somewhere in your mindscape, though it's relatively easy to shield yourself from the view of any other creatures inside if they don't realize they're in a mindscape. You must also create a method of exit from the mindscape when you cast this spell, and that method must be possible to achieve based on the traits of the mindscape, even if it is obscure or difficult. The GM decides whether a method of escape is reasonable. Anything that would be a reasonable method of waking from a dream during deep sleep could allow one to leave a mindscape.  More creatures than the initial two can enter an existing mindscape, typically through the use of the mindscape door spell. You can create illusory creatures within the mindscape  but you're able to direct or concentrate on only one at a time. Only this creature is believable; any others largely remain silent or speak generic, repetitive phrases.  As is normal with mindscapes, you and any other creatures within the mindscape are unable to take actions within the real world, and your bodies lose their Dexterity bonuses to AC.  If you choose to create a veiled mindscape, the first time another creature interacts with the environment, it can attempt a Will save to disbelieve the effect. Disbelieving a mindscape reveals to that creature that it's within a mindscape and gives it the knowledge needed to leave the mindscape, but doesn't free it from the mindscape. For example, if you create a mindscape that takes the form of a parlor inside a stately mansion, and your target creature succeeds at its Will save, it gains the understanding that walking out the front door of the mansion allows it to return to its physical body, but it must actually move through the mental landscape of the mansion to reach the front door and exit in order to flee the mindscape. If the mindscape is overt, the creature automatically knows how to exit if it so chooses.  This spell ends at any time that you choose to depart the mindscape, freeing the target as well. Any creature that drops below 0 hit points while inside the mindscape returns to its body. If this happens to you, the spell also ends, freeing anyone still inside.  Create mindscape can be made permanent with a permanency spell by a caster of 12th level or higher at a cost of 10,000 gp. You and other creatures aware of a permanent mindscape can come and go using the mindscape door spell.}
        
\DeclareSpell{Create Mindscape, Greater}{illusion (phantasm) [mind-affecting]|V,  S|1 round|long (400 ft. + 40 ft./level)|Targetsone creature/level|1 day/level|Will disbelief; see text|yes}[As create mindscape, but affecting more creatures, having a longer duration, and allowing magic alteration.]
    \DeclareSpellDescription{Create Mindscape, Greater}{This spell functions like create mindscape, except it can affect more creatures, the mindscape lasts longer, and you can choose the mindscape's magic trait. You can also direct a number of believable creatures at a time equal to your caster level.  Greater create mindscape can be made permanent with a permanency spell by a caster of 15th level or higher at a cost of 17,500 gp. You and other creatures aware of a permanent mindscape can come and go using the mindscape door spell.}
        
\DeclareSpell{Decrepit Disguise}{illusion (glamer) []|V|1 standard action|close (25 ft. + 5 ft./level)|Targetsone object of no more than 10 cu. ft./level|1 day/level|none (object) or Will disbelief (if interacted with)|no}[Make an object seem worthless.]
    \DeclareSpellDescription{Decrepit Disguise}{You make an object seem like a worthless version of itself. A masterwork or magic sword could seem to be a useless, rusting, discarded blade, and a luxurious throne could appear to be a decrepit wooden chair. If used against an attended object, the wielder can immediately attempt a Will save to disbelieve the effect. Decrepit disguise counters and dispels quintessence.  Decrepit disguise can be made permanent with a permanency spell by a caster of 9th level or higher for the cost of 500 gp.}
        
\DeclareSpell{Deja Vu}{enchantment (compulsion) [mind-affecting]|V|1 standard action|medium (100 ft. + 10 ft./level)|Targetsone creature|2 rounds|none|yes}[Make a creature repeat its actions.]
    \DeclareSpellDescription{Deja Vu}{You reach into the target's mind and put its thought processes into a temporary loop. Whatever full-round, standard, or move actions the creature takes on its first turn after you cast this spell, it must repeat on the turn after that. The creature must take the same type of actions in the same order (for example, making a full attack, casting a specific spell, withdrawing, attempting a bull rush combat maneuver, or activating a magic item) and must act against the same target or targets, but doesn't have to make exactly the same choices (such as using Power Attack when attacking, moving exactly 15 feet, or choosing "drop" for the command spell). If the circumstances would prevent the target from repeating an action, such as if the target of its attack is dead or the target cannot cast the same spell again, the target instead becomes confused until the spell ends. A creature currently affected by deja vu can't be targeted with another deja vu spell. A creature affected by deja vu can't delay, and if it readies an action on its first turn, it must ready the same action on its second turn.}
        
\DeclareSpell{Demand Offering}{enchantment (compulsion) [mind-affecting]|V,  S|1 standard action|5 ft.|Targetsone creature|instantaneous or 1 round|Will negates|yes}[Make a creature give you an object it’s holding.]
    \DeclareSpellDescription{Demand Offering}{A creature that fails its save uses an immediate action to hand you whatever object it's currently wielding or holding. If the target doesn't have an immediate action available, it uses a move action at the beginning of its next turn to hand you the object. If it's currently holding or wielding more than one item, determine randomly which item it gives you. If you're no longer adjacent to the creature when it has to give you the item, the spell ends with no effect. A creature that isn't holding an item when you cast this spell is unaffected.}
        
\DeclareSpell{Detect Mindscape}{divination () []|V,  S|1 standard action|60 ft.|Areacone-shaped emanation|concentration, up to 1 minute/level (D)|Will negates; see text|no}[Sense the presence and attributes of mindscapes.]
    \DeclareSpellDescription{Detect Mindscape}{This spell functions similarly to detect thoughts, allowing you to sense when one or more creatures' consciousnesses are  inside a mindscape (see page 234). The amount of information revealed depends on how long you study a particular subject.  First Round: You sense the presence or absence of a mindscape. At least one of the creatures inside the mindscape must be within the cone-shaped area for you to detect the mindscape.  Second Round: You sense the number of consciousnesses inside the mindscape. Not all of the creatures within the mindscape must be standing inside the cone-shaped area of the spell for you to count them; only one is required to get this reading. However, you can't determine the location of the creatures if you can't see the creatures themselves. You sense their presence only in an abstract way.  Third Round: You sense an image of what's taking place inside the mindscape. The creature controlling the mindscape can sense your efforts, and if it does not wish you to view the mindscape, it can prevent this viewing with a successful Will save. Otherwise, you perceive the visuals remotely, as if watching through a window. If a creature has disguised itself, you see only the creature's mental mask (not its true form).  Each round, you can turn to detect mindscapes in a new area. The spell can penetrate many barriers, but 1 foot of stone, 1 inch of common metal, a thin sheet of lead, or 3 feet of wood or dirt blocks it.  If you are within a mindscape when you cast this spell, you become aware you're in a mindscape. On the second round, you can still detect how many consciousnesses are inside, but you don't get the image of the mindscape in the third round.}
        
\DeclareSpell{Detect Psychic Significance}{divination () []|V,  S|1 standard action|40 ft.|Area40-ft.-radius burst, centered on you|instantaneous|none|no}[Find psychically charged items.]
    \DeclareSpellDescription{Detect Psychic Significance}{You detect the presence of psychically significant items in your vicinity. Such items are those that might have significant psychic imprints or histories that can be read by the psychometry occult skill unlock (see page 196), or items under the effects of the charge object or implant false reading spells. Items within range that contain significant psychic energy spark a recognition in your mind's eye, but no other information is imparted. This spell doesn't automatically detect magic items or strongly aligned items, though such items often have storied histories and might have had previous owners who possessed psychic abilities.}
        
\DeclareSpell{Divide Mind}{enchantment () [mind-affecting]|V|1 standard action|personal|Targetsyou|1 minute|none|no}[Partition your mind, allowing you to roll twice on Will saves and Int checks, and to take extra mental actions.]
    \DeclareSpellDescription{Divide Mind}{You partition your mind to maximize your mental power. Until the spell ends, you roll twice and use the higher result for all Will saves, Intelligence checks, and Intelligence-based skill checks. In addition, as a swift action you can have your second mind perform any purely mental action that normally requires a standard action or a move action. This includes casting psychic spells, using spell-like abilities, and concentrating on spells. Spells and spell-like abilities cast or used by your secondary mind this way can't exceed 5th level.}
        
\DeclareSpell{Dream Council}{illusion (phantasm) [mind-affecting]|V,  S|1 standard action|unlimited|Targetsone or more living creatures|see text|Will negates|yes}[Communicate with multiple sleeping creatures.]
    \DeclareSpellDescription{Dream Council}{This spell functions as dream, but you and the target of your dream can converse in a limited fashion as long as the recipient is also asleep. You can send or receive a number of dream messages equal to your caster level. Each message can be up to 25 words long or a single vague image that can't convey fine details such as words. You can send and receive these dream messages with a single target or multiple targets, but each message you send or receive counts against the total number of messages allowed. Sending a message takes 1 round. The spell ends and you wake up when you run out of messages.  If you use dream council to send a message to a sleeping creature that has dream or dream council prepared (or is able to cast it spontaneously or as a spell-like ability), the recipient can expend one of your allotted messages to cast that spell while remaining asleep. This uses up that creature's prepared spell, spell per day, or spell-like ability use as normal. Instead of replying to your message, that character is able to enter your dreamscape. If the sleeping recipient has the Lucid Dreamer feat, it can enter your dreamscape without casting either of those spells. While in the original caster's dreamscape, those involved in the council can interact with one another and that dreamscape for 10 minutes for each message remaining. When that time elapses, the spell ends.}
        
\DeclareSpell{Dream Scan}{divination () [mind-affecting]|V,  S|1 standard action|unlimited|Targetsone living creature|see text|Will negates|yes}[Read a dreaming creature’s thoughts.]
    \DeclareSpellDescription{Dream Scan}{This spell functions as dream, but rather than sending a message to a sleeping target, you can instead read the target's thoughts. The target must be asleep for you to perform a dream scan, though if the target isn't asleep, you can wait in a trance until your target falls asleep. Once the target is asleep, you can concentrate in order to read its surface thoughts as if using detect thoughts. You can continue concentrating for up to 1 minute per caster level.  Instead of reading surface thoughts, you can choose to scan the target's dreams and subconscious mind for the answers to questions. For every minute you spend concentrating, you can obtain the answer to one question,  though the answers might be brief, cryptic, or repetitive. The target is entitled to a new Will save to end the dream scan each time you ask it a question. Otherwise, the creature can attempt a Bluff check with a DC equal to 11 + your Sense Motive modifier. If it fails its Bluff check, you gain the information you desire. If it succeeds at its check, you gain no information. If it succeeds at its check by 5 or more, you misinterpret the target's dreams- you draw a false conclusion of the target's choice and believe that wrong answer to be true.  Unlike with dream, you can't cast this spell on another creature to have it serve as your messenger.}
        
\DeclareSpell{Dream Travel}{conjuration (teleportation) [mind-affecting]|V,  S|1 standard action|touch|Targetsyou and one creature/level|1 hour/level (D)|Will negates|yes}[Venture into the Dimension of Dreams to enter the dreams of a designated creature, then exit near that creature’s body on the plane where it lies sleeping.]
    \DeclareSpellDescription{Dream Travel}{You and the other targets of the spell are physically drawn from the Material Plane into the Dimension of Dreams on a voyage into the dreams of a creature you designate. In the Dimension of Dreams, you move through a swirling sea of thoughts, desires, and emotions created by the minds of dreamers everywhere to reach your destination dreamscape. Reaching the destination dreamscape takes 1 hour. At any point before the spell's duration ends, you can dismiss the spell to return to where you started on the Material Plane. The connection between dreams and reality is inherently tenuous, and your ability to arrive precisely where you mean to is dependent on your familiarity with the dreamer you're trying to find. To determine how accurate your arrival is at the end of your dream travel, roll d\% on the following table.     FamiliarityOn TargetOff TargetOtherDreamscape MishapVery familiar1-9798-99100-Somewhat familiar1-9495-9798-99100Known creature1-8889-9495-9899-100Not well known1-7677-8889-9697-100False identity--81-9293-100 Familiarity: "Very familiar" means that you have had contact using dream, dream council, dream scan, or a similar spell within the past 24 hours with the creature whose dreamscape you are trying to locate. "Somewhat familiar" means that you have had contact with the dreaming character using one of those spells at least once in the past. "Known creature" means that while you know the creature whose dreamscape you're trying to locate, you have not connected to its dreams with those spells. "Not well known" is a creature you have heard of, know by name and true identity, but have never met.  "False identity" means that whether or not you have met the creature, you know it only through a false identity. When trying to locate the dreamscape of a creature known to you through a false identity, roll 1d20 + 80 on the table to obtain your results (instead of rolling d\%), since there is no way for you to pinpoint the correct dreamscape.  On Target: You travel to the correct creature's current dreamscape.  Off Target: You travel to an area near the target dreamscape on the Ethereal Plane.  Other Dreamscape: You travel to a similar creature's dreamscape on the Dimension of Dreams.  Mishap: You and anyone else traveling with you experience a mishap during travel; each character takes 1d10 points of damage and must reroll on the table to see where it ends up. For these rerolls, roll 1d20 + 80. Each time "Mishap" comes up, the travelers take more damage and must reroll to see where they end up.  Regardless of the accuracy of your dream travel, you and your companions all arrive at the same location (except in the case of a mishap). Mindless creatures can't use dream travel, nor can creatures that can't dream.  You might be able to exit the dream near the creature on its home plane. If you and the other creatures are still in a dream when the dreamer wakes up, the dreamer can decide to bring you out onto its plane within 1d10 miles of itself. If it chooses not to, you and the other travelers are pushed into another dreamscape or onto the Ethereal Plane. The spell ends when the creature wakes up, so you no longer have the option to dismiss the spell and return home. However, you do get enough warning that a spell is about to end that you can dismiss it before the dreamer awakens and decides whether to allow you to arrive at its location.  You can use dream travel to travel to the dream of a creature on another plane, but this requires trekking through the dreams of outsiders. It takes 1d4+1 hours of uninterrupted travel to reach the destination dream, and each traveler must succeed at a Will save each hour spent traveling in this fashion. The DC of this save is 10 for the first hour, and increases by 5 for each hour thereafter. If a creature fails this saving throw, it becomes shaken for the remainder of the dream travel and for a number of hours thereafter equal to the amount of time spent using dream travel to reach the destination plane. After the first time a creature fails the saving throw, one of the following effects occurs (determined randomly by the GM). These effects are considered emotion effects, in addition to any other descriptors that apply.   d6Result1Creature contracts cackle fever or mindfire (equal chance of either; see page 557 of the Pathfinder RPG Core Rulebook).2Creature is cursed, as bestow curse; if the creature is a spellcaster, it instead acquires a random minor spellblight (Pathfinder RPG Ultimate Magic 94).3Creature is attacked by a phantasmal killer, and is treated as having failed the initial saving throw to disbelieve it. The Fortitude save DC is equal to the Will save DC that the creature failed.4Creature is possessed as possession (see page 181) by an outsider (50\% chance to be a creature from the destination plane; otherwise, it is a random hostile outsider).5Creature is affected as feeblemind.6Creature is affected as insanity.   Traveling to a different planar destination decreases the accuracy of your dream transit. If you exit a creature's dream onto a plane different from the plane you started on, you end up in a random location on the destination plane.  If the target of your dream travel isn't dreaming, you and the other travelers can wait on the Dimension of Dreams until that creature begins sleeping. For each hour that passes during this wait, each dream traveler must attempt a Will save as if it were traveling to reach another plane.}
        
\DeclareSpell{Dream Voyage}{conjuration (teleportation) [mind-affecting]|V,  S|1 standard action|touch|Targetsyou and one creature/level|1 hour/level (D)|Will negates|yes}[As dream travel, but with more flexibility and able to affect more creatures.]
    \DeclareSpellDescription{Dream Voyage}{This spell functions as dream travel, but you and your companions travel through the Dimension of Dreams in a fantastical vehicle of your own devising that halves the travel time. Only you can pilot the vessel, and you can do so even if you aren't inside it. Your psychic vehicle buffers minds from the intense emotional tides of the Dimension of Dreams, rendering everyone inside immune to harmful emotion and fear effects. Unlike with dream travel, there is no chance the vessel arrives off target. In addition, you need not have met the target creature, but in that case you must have at least a reliable description of it. However, if you attempt to travel there with insufficient or incorrect information, you do need to roll, using the false identity entry from dream travel.  Creatures that disembark from your vehicle can either enter the dreamscape or exit to the plane where the dreamer's body lies, arriving within 1 mile of the dreamer's body. All creatures that disembark from the vehicle onto a plane at the same time arrive in the same place. The dreamer can't prevent the dream voyagers from entering its dream, but can attempt a Will save to prevent a group of creatures from exiting to the plane where its body resides.  The psychic vehicle remains in the dream for the remainder of the spell's duration, even if it's unoccupied. A creature adjacent to the dreamer can reenter the dream vessel as a full-round action through an illusory portal only subjects of the spell can see. You can return the vessel to the point where you cast dream voyage to allow any number of passengers to return to that plane, and can return it to the dream again. Either of these trips takes the same amount of time as the initial travel, and can be done any number of times within the spell's duration. If the vessel moves out of a dreamscape, companions left behind can't return to the vessel by being near the dreamer. If a companion is killed, its body doesn't return to the dream voyage unless you carry it with you, just like any other object.  If the dreamer awakens while the dream vessel is still in its dream, it can allow the dream travelers to exit or not (as with dream travel). The dream vessel remains intact in the Dimension of Dreams for the spell's duration even if it's untethered from a  dream. It drifts through dreams when this happens, and you can't pilot it to another dream (even a dream of the initial dreamer if that creature begins dreaming again). If you dismiss the spell at any time, you and all subjects within the dream vessel return to where you started on the Material Plane.  You can increase the number of companions you can bring on a dream voyage to 10 creatures per caster level, though to do so you must increase the casting time, maintaining your concentration and touching up to 6 creatures per round as a full-round action. If your concentration is disrupted before you touch all of the targets, the spell is wasted and has no effect. Once you have touched all of your companions, the dream voyage begins; however, its duration is reduced to 1 hour.}
        
\DeclareSpell{Ectoplasmic Eruption}{evocation () []|V,  S|1 standard action|medium (100 ft. + 10 ft./level)|Area30-ft.-radius burst|1 round/level|Reflex half and Will partial; see text|yes}[Deal 6d6 points of damage and entangle creatures in a 30-ft. radius, and push ethereal and incorporeal creatures onto the Material Plane.]
    \DeclareSpellDescription{Ectoplasmic Eruption}{A cascading avalanche of pale, swirling ectoplasmic matter erupts from a point you select. All creatures in the area when the spell is cast take 6d6 points of bludgeoning damage and are entangled for a number of rounds equal to your caster level. This spell passes between planes, so it affects ethereal and incorporeal creatures normally.  In addition, each ethereal or incorporeal creature in the area must succeed at a Will save or be pushed partially onto the Material Plane for a number of rounds equal to your caster level. It must attempt this Will save regardless of whether it succeeded at the Reflex save. An incorporeal creature pushed partially onto the Material Plane can't enter or pass through solid objects, takes half damage from nonmagical attack forms, and takes full damage from magic weapons, spells, spell-like effects, and supernatural effects. Corporeal spells and effects that don't cause damage affect the creature normally instead of having a 50\% chance of affecting it. The creature still gains the other benefits of being incorporeal, and retains its attack bonuses and Armor Class.}
        
\DeclareSpell{Ectoplasmic Snare}{evocation () []|V,  S|1 standard action|close (25 ft. + 5 ft./2 levels)|Effectentangling web of ectoplasm|concentration, up to 1 round/level (D)|Reflex partial; see text|yes}[Tendril of ectoplasm grapples a creature and tethers you to it.]
    \DeclareSpellDescription{Ectoplasmic Snare}{You unleash a writhing tendril of ectoplasm to grapple  or entangle a target creature. You must make a ranged touch attack to strike a target. If you hit, the target can attempt a Reflex save. On a successful save, the target is entangled for the duration of the spell and suffers no other effects. If the target fails this saving throw, the tendril is more restrictive, making the target grappled and dealing it 1d6+4 points of bludgeoning damage. Each round when you concentrate to maintain the spell, the snare attempts a grapple combat maneuver check to maintain the grapple. As normal when grappling, the snare gains a +5 bonus on grapple combat maneuver checks against opponents it is already grappling. The snare's CMB is equal to 6 + your caster level, and its CMD is equal to 16 + your caster level. Since the ectoplasm passes between planes, this spell affects incorporeal and ethereal creatures normally.  You remain tethered to the target for the duration of the spell. You can shrink or extend the snare, but if the distance between you and the target exceeds the spell's range, the snare disappears. The ectoplasmic tether has hardness 10 and a number of hit points equal to 10 + your caster level + your Constitution modifier, and the tether can be damaged or sundered anywhere along its length. You are not considered to have the grappled condition while tethered to the target. You can perform the move or damage action on a successfully grappled target, moving it up to half your speed or dealing an additional 1d6+4 points of bludgeoning damage to it on a successful grapple combat maneuver check. You can't pin your target.}
        
\DeclareSpell{Ego Whip I}{enchantment (compulsion) [emotion,  mind-affecting]|S|1 standard action|close (25 ft. + 5 ft./2 levels)|Targetsone creature|1 round/level|Will partial|yes}[Cause a creature to take a –2 penalty to Int, Wis, or Cha and be staggered for 1 round.]
    \DeclareSpellDescription{Ego Whip I}{You can use your psychic power to overwhelm the target's ego, leaving the target feeling hopeless and unsure of itself. Choose Intelligence, Wisdom, or Charisma. The target takes a -2 penalty to that ability score, and is also staggered for the first round it's affected. A successful Will save negates the staggered effect and reduces the duration of the penalty to 1 round.}
        
\DeclareSpell{Ego Whip II}{enchantment (compulsion) [emotion,  mind-affecting]|S|1 standard action|close (25 ft. + 5 ft./2 levels)|Targetsone creature|1 round/level|Will partial|yes}[As ego whip I, but a –4 penalty and staggered for 1d4 rounds.]
    \DeclareSpellDescription{Ego Whip II}{This functions as ego whip I, but the target takes a -4 penalty to the chosen ability score and is staggered for 1d4 rounds on a failed Will save. This spell can be undercast.}
        
\DeclareSpell{Ego Whip III}{enchantment (compulsion) [emotion,  mind-affecting]|S|1 standard action|close (25 ft. + 5 ft./2 levels)|Targetsone creature|1 round/level|Will partial|yes}[As ego whip I, but a –6 penalty and staggered for 1d6 rounds.]
    \DeclareSpellDescription{Ego Whip III}{This functions as ego whip I, but the target takes a -6 penalty to the chosen ability score and is staggered for 1d6 rounds on a failed Will save. This spell can be undercast.}
        
\DeclareSpell{Ego Whip IV}{enchantment (compulsion) [emotion,  mind-affecting]|S|1 standard action|close (25 ft. + 5 ft./2 levels)|Targetsone creature|1 round/level|Will partial|yes}[As ego whip I, but a –8 penalty and staggered for 1d8 rounds.]
    \DeclareSpellDescription{Ego Whip IV}{This functions as ego whip I, but the target takes -8 penalty to the chosen ability score and is staggered for 1d8 rounds on a failed Will save. This spell can be undercast.}
        
\DeclareSpell{Ego Whip V}{enchantment (compulsion) [emotion,  mind-affecting]|S|1 standard action|close (25 ft. + 5 ft./2 levels)|Targetsone creature|1 round/level|Will partial|yes}[As ego whip I, but a –10 penalty and staggered for 1d10 rounds.]
    \DeclareSpellDescription{Ego Whip V}{This functions as ego whip I, but the target takes -10 penalty to the chosen ability score and is staggered for 1d10 rounds on a failed Will save. This spell can be undercast.}
        
\DeclareSpell{Emotive Block}{enchantment (compulsion) [emotion,  mind-affecting]|S|1 standard action|close (25 ft. + 5 ft./2 levels)|Targetsone creature|1 round/level (D)|Will negates|yes}[Add an emotion component to all of the target’s spells.]
    \DeclareSpellDescription{Emotive Block}{You create an emotional block in the target's mind, adding an emotion component to each spell or spell-like ability he uses. This is in addition to any other components the spell already requires, and doesn't replace the somatic component. The target's spell-like abilities require this emotion component, even though they normally don't need spell components.  As usual with emotion components, the target can't cast spells (or use spell-like abilities, in this case) while affected by a non-harmless emotion or fear effect.}
        
\DeclareSpell{Enshroud Thoughts}{abjuration () [mind-affecting]|V,  S|1 standard action|personal|Targetsyou|10 minutes/level (D)|Will negates (harmless)|yes (harmless)}[Ward yourself against thought detection and memory alteration.]
    \DeclareSpellDescription{Enshroud Thoughts}{You become warded against the mental prying of others, including the effects of divination spells such as detect thoughts and seek thoughtsAPG, as well as enchantment spells such as modify memory and memory lapseAPG. If another creature attempts to target you with a mind-affecting spell that  detects or alters your thoughts or memories, the caster must succeed at a caster level check against a DC equal to 11 + your caster level or the spell fails. Only spells that detect or alter your thoughts or memories are blocked by this spell; effects such as clairaudience/clairvoyance, d etect e vil, and locate creature continue to affect you as normal.}
        
\DeclareSpell{Entrap Spirit}{necromancy () []|V,  S,  F (a small silver mirror)|1 standard action|close (25 ft. + 5 ft./2 levels)|Targetsincorporeal creature or haunt|1 hour/level (D)|Will negates|yes}[Trap an incorporeal creature or a haunt in a mirror.]
    \DeclareSpellDescription{Entrap Spirit}{You trap the target in the mirror used as the spell's focus. The target cannot be affected by any means while inside the mirror. If the mirror is destroyed, any incorporeal creature within is immediately freed and any haunt within returns to its original location. Creatures that assumed incorporeal form through a spell or other means remain incorporeal while trapped, even if the duration of the effect that rendered them incorporeal expires. If your focus mirror belonged to the target, the target takes a -2 penalty on its saving throw.}
        
\DeclareSpell{Erase Impressions}{abjuration () []|V,  S|1 round|touch|Targetsobject touched|instantaneous|Will negates (object)|yes (object)}[Erase psychic impressions from an object.]
    \DeclareSpellDescription{Erase Impressions}{You banish psychic impressions from the object touched, rendering it devoid of recent history. You choose how much time to erase, up to a maximum of 1 day per caster level. You must erase impressions from the present time back; you cannot choose to leave recent events untouched. Impressions erased in this manner cannot be recovered via object reading, legend lore, or similar measures.  Casting erase impressions o n a corpse r emoves recent psychic impressions from when the creature was alive as well, making those memories unavailable to speak with dead. For example, a 5th-level caster casting the spell on a 2-day-old corpse could erase the past 2 days of psychic impressions from the corpse and 3 additional days of information from just before the creature died.}
        
\DeclareSpell{Ethereal Envelope}{conjuration (teleportation) []|S,  M (empty crystal box)|1 standard action|personal|Targetsyou|1 hour/level (D)||}[Shroud your unconscious self in a cocoon on the Ethereal Plane.]
    \DeclareSpellDescription{Ethereal Envelope}{You place yourself into a cataleptic meditative state and shunt yourself to the Ethereal Plane, where your body is cocooned within a folded ripple of misty space, with total cover from creatures on the Ethereal Plane. While inside your ethereal envelope, you are treated as being asleep for most purposes. You can choose an amount of time within the duration when casting this spell, and you automatically dismiss the spell when that time elapses. Your ethereal envelope counts as an object of your size with AC 10, hardness 5, and hit points equal to twice your level, and it is immune to bludgeoning damage. An attacker must be on the Ethereal Plane or have a way to see ethereal objects in order to detect your ethereal envelope. The envelope can also be broken open with a successful Strength check with a DC equal to 15 + your caster level (to a maximum DC of 30). If the ethereal envelope is destroyed, the spell ends immediately.  When the spell ends, you remain asleep without the protection of the envelope for 1 round. At the end of that round, you are shunted back to the Material Plane at the location where you cast the spell, and you are staggered for 1 round as you reorient yourself to the Material Plane. You can't use this spell to travel from your location on the Ethereal Plane. If the space you formerly occupied is now occupied by another creature or object, you are shunted to the nearest open space and take 1d10 points of damage.}
        
\DeclareSpell{Ethereal Envelopment}{conjuration (teleportation) []|S,  M (empty crystal box)|1 standard action|close (25 ft. + 5 ft./2 levels)|Targetsone creature|1 hour/level (D)|Will negates, see text|yes}[As ethereal envelope, but able to affect an unwilling creature.]
    \DeclareSpellDescription{Ethereal Envelopment}{This spell is similar to ethereal envelope, but it allows you to shunt an unwilling creature to the Ethereal Plane and seal it there within a cocoon of misty ethereal fibers. If the target fails its save, it's forced into a cataleptic meditative state as if unconscious. However, each round at the end of its turn, it can attempt a Will save to awaken itself from this meditation. Once the target awakens, it is considered pinned, treating your caster level as your CMB and your CMD as 10 + your caster level + your ability modifier for your primary spellcasting ability score. A successful grapple combat maneuver check or Escape Artist check leaves the creature grappled rather than pinned but doesn't allow it to escape. However, a trapped creature can attack the cocoon as described in the ethereal envelope spell description.  If the creature breaks free of the ethereal envelopment or  you dismiss the spell, the target is returned to its prior location on the Material Plane, as described for ethereal envelope, but if the target is conscious at the time, it doesn't spend a round asleep or staggered.}
        
\DeclareSpell{Ethereal Fists}{transmutation () []|S|1 standard action|personal|Targetsyou|1 minute/level (D)||}[Your claws, unarmed strikes, and touch spells affect ethereal creatures.]
    \DeclareSpellDescription{Ethereal Fists}{Your hands reach simultaneously into the Ethereal and Material planes, allowing you to deal normal damage with claws, unarmed strikes, and touch spells or effects to ethereal creatures. Miss chance due to etherealness (such as from the blink spell) doesn't apply, though miss chance based on concealment does.}
        
\DeclareSpell{Etheric Shards}{evocation () [force]|S,  M (broken glass)|1 standard action|medium (100 ft. + 10 ft./level)|Areaone 10-ft. cube/level|1 hour/level (D)|Reflex partial or Reflex negates (see text)|no}[Fill an area with invisible shards that slow movement and damage creatures.]
    \DeclareSpellDescription{Etheric Shards}{You harden interpenetrated ethereal substance into deadly transdimensional razors that are invisible to normal sight. Movement through an area of etheric shards is halved, even for incorporeal creatures, and creatures entering a 5-foot cube filled with etheric shards take 1d8 points of piercing and slashing damage and must succeed at a Reflex save or take 1 point of bleed damage. This bleed damage stacks with itself and other sources of bleed damage. A creature standing within an area of etheric shards takes no damage as long as it remains completely motionless, but even the minor movements involved in attacking or defending in combat force a stationary creature to attempt a Reflex save. If a stationary creature succeeds at this save, it avoids damage completely for that round, but if it fails, it takes hit point damage and bleed damage as though it had moved through the square.  A creature forcibly moved through an area of etheric shards, such as by a bull rush or drag combat maneuver, takes a -4 penalty on its saving throw, but a creature able to see invisible or ethereal objects gains a +4 bonus, and damage to it is completely negated on a successful Reflex save.  Etheric shards are considered a magical trap, and a creature with trapfinding can use Perception to find them. The DC of this check is equal to 25 + the spell's level. Etheric shards can't be disabled with the Disable Device skill.}
        
\DeclareSpell{Explode Head}{evocation () []|V,  S|1 standard action|close (25 ft. + 5 ft./2 levels)|Targetsone living corporeal creature|instantaneous|special; see text|yes}[Explode the head of a creature with 20 hp or fewer and deal damage in a 10-ft. radius around it.]
    \DeclareSpellDescription{Explode Head}{You cause incredible pressure to build in the target's head; this spell works only on a creature that has a head and would die from the loss of a head. The spell kills any such target that has 20 hit points or fewer, exploding its head and spreading debris in a 10-foot radius. Each creature in the area must succeed at a Reflex saving throw or take 2d6 points of slashing damage from the flying debris.}
        
\DeclareSpell{Foster Hatred}{enchantment (compulsion) [emotion]|S,  DF|1 standard action|close (25 ft. + 5 ft./2 levels)|Targetsany number of living creatures, no two of which can be more than 30 ft. apart|1 hour/level (D)|Will negates|yes}[Cause creatures to hate one type of creature.]
    \DeclareSpellDescription{Foster Hatred}{You build upon your targets' innate prejudice, stoking it from a simmer to the full flame of hatred. Choose a creature type (if you choose humanoid or outsider, you must also choose a subtype). Targets who hold no ill will toward and have a completely positive opinion of the chosen creatures automatically succeed at their saving throws, while targets with a particular hatred toward the chosen creature (such as the dwarf's hatred racial ability) take a -4 penalty on their saving throws. Targets that fail their saving throws become hostile toward all creatures of the chosen type and never treat them as allies for the spell's duration. Affected creatures take everything creatures of the chosen type do in the worst possible light.  Since most creatures don't hold ill-will toward their own race, you can use foster hatred toward a smaller subset of creatures of your chosen type, such as worshipers of Shelyn, in order to focus the hatred onto your chosen targets.}
        
\DeclareSpell{Ghost Whip}{evocation () []|V,  S|1 standard action|0 ft.|Effectwhip of flexible ectoplasm|1 round/level (D)|none|yes}[Create a ghost touch whip that passes through objects.]
    \DeclareSpellDescription{Ghost Whip}{A white, 15-foot-long lash of ghostly evanescence appears from your hand. This weapon is treated as a ghost touch whip with no enhancement bonus. You can wield it as a whip as if you were proficient. Attacks with a ghost whip are resolved as incorporeal touch attacks. The whip affects only creatures you attack with it, passing through objects and other creatures in its path and thereby allowing you to ignore cover between you and your target. When a ghost whip attack passes through total cover, the target is treated as having total concealment (50\% miss chance). Against incorporeal and undead creatures,  a ghost whip deals lethal damage and can be used to perform drag or reposition combat maneuvers (in addition to a whip's normal disarm and trip maneuvers). A ghost whip cannot be disarmed or sundered.}
        
\DeclareSpell{Grave Words}{necromancy () [language-dependent]|S|1 minute|touch|Targetsone dead creature|1 round||}[Force a corpse to babble.]
    \DeclareSpellDescription{Grave Words}{With this spell and a touch, you can force a corpse talk to you, but you can't ask it specific questions or communicate with it at all. The corpse will start babbling for 1 round, spitting out random sentences. There is a 10\% chance this information is of some use to the caster, but it is difficult to distinguish whether the information is useful (the GM makes the percentile roll in secret).  Useful information may include warnings about dangers deeper in a dungeon, the command word to a magic item, or even vague and spectral warnings of your or your companions' future. The GM decides what information, useful or not, the corpse spews out in its babbling.  Once a corpse has been subject to grave words b y a ny caster, any new attempt to cast grave words on that corpse fails. You can cast this spell on a corpse that has been deceased for any amount of time, but the head of the corpse must have a mouth in order to speak at all. This spell doesn't affect a corpse that has been turned into an undead creature.}
        
\DeclareSpell{Hypercognition}{divination () []|V|1 standard action|personal|Targetsyou|see text||}[Rapidly recall everything you know about a subject.]
    \DeclareSpellDescription{Hypercognition}{You rapidly catalog and collate all available data on a person, place, thing, or event, calling to mind scraps of memory and assembling clues in a logical and systematic order. Immediately after casting the spell, you can begin spending the necessary time to perform an Intelligence check, a Linguistics check to detect a forgery or decipher a hidden message, or a Knowledge check. If the check requires at least 3 rounds, you can perform it five times as quickly (minimum 1 round). If the check required fewer than 3 rounds, you can perform it instantly. Either way, you gain an insight bonus on the check equal to your caster level (maximum bonus +10), but this bonus can't raise your result above what you could have achieved on a natural 20.}
        
\DeclareSpell{Id Insinuation I}{enchantment (compulsion) [mind-affecting]|S|1 standard action|close (25 ft. + 5 ft./2 levels)|Targetsone creature|concentration + 1 round|Will negates|yes}[Confuse one creature for the duration of your concentration + 1 round.]
    \DeclareSpellDescription{Id Insinuation I}{By invading the mind of a creature with your psychic presence, you can isolate parts of its mind, preventing the target from functioning in a coherent manner. The target is confused as long as you concentrate on it plus 1 additional round. A successful Will save negates this effect.}
        
\DeclareSpell{Id Insinuation II}{enchantment (compulsion) [mind-affecting]|S|1 standard action|close (25 ft. + 5 ft./2 levels)|Targetstwo creatures|concentration + 1 round|Will negates|yes}[As id insinuation I, but affects two creatures and has a stronger confusion effect.]
    \DeclareSpellDescription{Id Insinuation II}{This functions as id insinuation I, except as noted above. In addition, on the first round of the effect, the targets of this spell must roll twice to determine the result of their confused condition. You select which roll is used. This spell can be undercast.}
        
\DeclareSpell{Id Insinuation III}{enchantment (compulsion) [mind-affecting]|S|1 standard action|close (25 ft. + 5 ft./2 levels)|Targetsthree creatures|concentration + 1 round|Will negates|yes}[As id insinuation I, but affects three creatures and has a stronger confusion effect.]
    \DeclareSpellDescription{Id Insinuation III}{This functions as id insinuation I, except as noted above. In addition, on the first 2 rounds of the effect, the targets of this spell must roll twice to determine the result of their confused conditions. You select which roll is used. This spell can be undercast.}
        
\DeclareSpell{Id Insinuation IV}{enchantment (compulsion) [mind-affecting]|S|1 standard action|close (25 ft. + 5 ft./2 levels)|Targetsfour creatures|concentration + 1 round|Will negates|yes}[As id insinuation I, but affects four creatures and you select the confusion effect.]
    \DeclareSpellDescription{Id Insinuation IV}{This functions as id insinuation I, except as noted above. In addition, on the first round of the effect, you can select the result of each creature's confused condition without having to roll, but this doesn't allow you to make any decisions about how that result is applied. On the following rounds of the effect, the targets of this spell must roll twice to determine their confused effect. You select which roll is used. This spell can be undercast.}
        
\DeclareSpell{Implant False Reading}{illusion (glamer) []|V,  S|1 minute|touch|Targetsobject touched|1 day/level|none; see text|no}[Instill false psychic impressions into an object.]
    \DeclareSpellDescription{Implant False Reading}{You implant a false psychometric reading into an object, imbuing the item with misleading information that might be learned from the object reading spell or through the psychometry occult skill unlock (see page 196). This information might fool enemies, create a false history, or pass on secret messages through objects to psychically attuned allies. You can implant any information you wish that could be implanted by the charge object spell, as well as up to one piece of information per 3 levels that could be attained using psychometry or the object reading spell. You can detail this information in any way you see fit, either making up a previous owner and circumstances involving the object  from whole cloth, or attempting to emulate the biographical information and fictional accounts of a specific individual's experience with the object.  During the spell's duration, creatures reading the object through the object reading spell or the psychometry occult skill unlock receive this false information in place of an item's true history if they fail a Will saving throw. Creatures that succeed at their saves still detect and receive the false information, but recognize it for what is and are able to discern the fictitious information from the item's true history without difficulty.  This spell masks its own magical aura, causing the target to not register as magical to detect spells and similar effects. This masking doesn't conceal any other magical auras on the object, and can be overcome in the same ways as magic aura.}
        
\DeclareSpell{Incorporeal Chains}{evocation () [force]|V,  S|1 standard action|close (25 ft. + 5 ft./2 levels)|Targetsone incorporeal creature/level, no two of which can be more than 30 ft. apart|concentration|none|yes}[Grapple incorporeal creatures and deal damage equal to 1d8 + Int.]
    \DeclareSpellDescription{Incorporeal Chains}{You summon ghostly chains to bind incorporeal creatures. The chains' CMB is equal to your caster level + your Intelligence modifier. Roll only once for the entire spell effect and apply the result to all targeted incorporeal creatures.  If the chains succeed in grappling an incorporeal creature, the creature takes an amount of damage equal to 1d8 + your key spellcasting ability score modifier and gains the grappled condition. Grappled opponents can't move without first breaking the grapple. The spell prohibits all other movement unless the creature breaks the grapple first (although dimensional magic and the like still work as normal). Each round you concentrate, the chains continue grappling their current targets (though they don't seek out targets they aren't currently grappling). The incorporeal chains receive a +5 bonus on grapple combat maneuver checks against opponents the chains are already grappling, but can't move foes or pin foes. Each round the incorporeal chains succeed at a grapple combat maneuver check, they deal an additional amount of damage equal to 1d8 + your key spellcasting ability score modifier. The CMD of incorporeal chains, for the purposes of escaping the grapple, is equal to 10 + its CMB. If you move out of range of any of the grappled creatures, the chains cease grappling those creatures.}
        
\DeclareSpell{Inflict Pain}{enchantment () [mind-affecting,  painUM]|V,  S|1 standard action|close (25 ft. + 5 ft./2 levels)|Targetsone creature|1 round/level|Will partial; see text|yes}[Target takes a –4 penalty on attack rolls, skill checks, and ability checks.]
    \DeclareSpellDescription{Inflict Pain}{You telepathically wrack the target's mind and body with agonizing pain that imposes a -4 penalty on attack rolls, skill checks, and ability checks. A successful Will save reduces the duration to 1 round.}
        
\DeclareSpell{Inflict Pain, Mass}{enchantment () [mind-affecting,  painUM]|V,  S|1 standard action|close (25 ft. + 5 ft./2 levels)|Targetsone creature/level, no two of which can be more than 30 ft. apart.|1 round/level|Will partial; see text|yes}[As inflict pain, but affecting one creature per level.]
    \DeclareSpellDescription{Inflict Pain, Mass}{This spell functions like inflict pain except as noted above.}
        
\DeclareSpell{Instigate Psychic Duel}{illusion (phantasm) [mind-affecting]|V,  S|1 standard action|medium (100 ft. + 10 ft./level)|Targetsone creature|1 minute/level|Will negates|yes}[Start a psychic duel between yourself and another creature]
    \DeclareSpellDescription{Instigate Psychic Duel}{You begin a psychic duel (see page 202) with a creature. This psychic duel takes place on a binary mindscape (see page 235) that includes only two minds: yours and the target's. This spell ends and both minds return to their respective bodies if one of you drops below 0 hit points or if both of you agree to end the psychic duel (a free action that you can perform even if it isn't your turn).  Even if you cast the spell, you must succeed at a Will save to escape your own psychic duel if your opponent doesn't want to allow you to leave. Spells like mindscape door can also free you from the mindscape (and mindscape door is one of the few spells you can cast in a binary mindscape). A binary mindscape is clearly illusory, and disbelieving the illusion doesn't alter how the mindscape works.}
        
\DeclareSpell{Intellect Fortress I}{abjuration () []|V|1 immediate action|20 ft.|Area20-ft.-radius emanation centered on you|1 round|none|yes (harmless)}[Suppress emotion and fear effects in a 20-ft. radius as an immediate action.]
    \DeclareSpellDescription{Intellect Fortress I}{Using the power of pure logic, you disrupt mental attacks. Intellect fortress I suppresses all effects with the emotion and fear descriptors for its duration.}
        
\DeclareSpell{Intellect Fortress II}{abjuration () []|V|1 immediate action|20 ft.|Area20-ft.-radius emanation centered on you|1 round|none|yes (harmless)}[As intellect fortress I, plus reduce the damage of mind-affecting effects.]
    \DeclareSpellDescription{Intellect Fortress II}{This spell functions as intellect fortress I, but in addition, any mind-affecting effects that deal damage have their damage reduced by half (before any reduction due to a successful save or other effect). This applies to both hit point damage and ability score damage. This spell can be undercast.}
        
\DeclareSpell{Intellect Fortress III}{abjuration () []|V|1 immediate action|20 ft.|Area20-ft.-radius emanation centered on you|1 round|none|yes (harmless)}[As intellect fortress II, plus remove partial effects of fear and emotion effects.]
    \DeclareSpellDescription{Intellect Fortress III}{This spell functions as intellect fortress II, but creatures within the area who succeed at a saving throw against a fear or emotion effect suffer no effects, even if there is usually a partial effect on a successful saving throw. This spell can be undercast.}
        
\DeclareSpell{Mental Barrier I}{abjuration () []|V|1 immediate action|personal|Targetsyou|1 round||}[Gain a +4 shield bonus to AC and resist mind thrust for 1 round as an immediate action.]
    \DeclareSpellDescription{Mental Barrier I}{You put a barrier of mental energy that protects you from harm. This barrier grants you a +4 shield bonus to AC. In addition, you take half damage from mind thrust I and II (reduced to one-quarter on a successful Will save).}
        
\DeclareSpell{Mental Barrier II}{abjuration () []|V|1 immediate action|personal|Targetsyou|1 round||}[As mental barrier I, but +6 to AC.]
    \DeclareSpellDescription{Mental Barrier II}{This spell functions as mental barrier I, but the shield bonus to AC increases to +6. In addition, you take half damage from mind thrust I, II, and III (reduced to one-quarter on a successful Will save). This spell can be undercast.}
        
\DeclareSpell{Mental Barrier III}{abjuration () []|V|1 immediate action|personal|Targetsyou|1 round||}[As mental barrier I, but +8 to AC.]
    \DeclareSpellDescription{Mental Barrier III}{This spell functions as mental barrier I, but the shield bonus to AC increases to +8. In addition, you are immune to mine thrust I, and you take half damage from mind thrust II, III, and IV (reduced to one-quarter on a successful Will save). This spell can be undercast.}
        
\DeclareSpell{Mental Barrier IV}{abjuration () []|V|1 immediate action|personal|Targetsyou|1 round||}[As mental barrier III, plus 25\% chance to prevent critical hits and sneak attacks.]
    \DeclareSpellDescription{Mental Barrier IV}{This spell functions as mental barrier III, but if you are struck by a critical hit or sneak attack, there is a 25\% chance that the additional damage is negated (as light fortification). This does not stack with similar effects that negate the additional damage from a critical hit or sneak attack. In addition, you are immune to mind thrust I and II, and you take half damage from mind thrust III, IV, and V (reduced to one-quarter on a successful Will save). This spell can be undercast.}
        
\DeclareSpell{Mental Barrier V}{abjuration () []|V|1 immediate action|personal|Targetsyou|1 round||}[As mental barrier III, plus 50\% chance to prevent critical hits and sneak attacks.]
    \DeclareSpellDescription{Mental Barrier V}{This functions as mental barrier III, but if you are struck by a critical hit or sneak attack, there is a 50\% chance that the additional damage is negated (as moderate fortification). This does not stack with similar effects that negate the additional damage from a critical hit or sneak attack. In addition, you are immune to mind thrust I, II, and III, and you take half damage from mind thrust IV, V, and VI (reduced to one-quarter on a successful Will save). This spell can be undercast.}
        
\DeclareSpell{Mental Block}{divination () [mind-affecting]|V|1 standard action|close (25 ft. + 5 ft./2 levels)|Targetsone creature|1 round/level (D)|Will negates|yes}[Prevent the target from using its skill ranks, spells, feats, and abilities.]
    \DeclareSpellDescription{Mental Block}{You lock access to the target's procedural memories, preventing it from drawing upon its experience and expertise. The target loses all skill ranks, spells known, spells prepared, and activated feats, as well as its extraordinary, supernatural, and spell-like abilities. Each round at the end of the target's turn, the target can attempt another Will save to end this effect.}
        
\DeclareSpell{Microcosm}{illusion (phantasm) [mind-affecting]|V,  S|1 standard action|close (25 ft. + 5 ft./2 levels)|Targetsone or more creatures|permanent|Will partial (see text)|yes}[Trap creatures in a veiled mindscape permanently, causing their bodies to waste away in the real world.]
    \DeclareSpellDescription{Microcosm}{You plunge the targets' minds into a veiled immersive mindscape (see page 235) of your own design. You can affect any number of creatures whose combined total Hit Dice do not exceed 30. A creature of 10 HD or fewer gets no saving throw, one of 11-15 HD escapes after 10 minutes per level you possess on a successful save, and a creature of 16 HD or more negates the effect on a successful save. Given time, the bodies of creatures  whose minds are trapped in a microcosm can die of starvation and thirst without care. Creatures can neither escape from nor enter a microcosm, even via a mindscape door. Microcosm can be ended only by miracle or wish.  Creatures inside a microcosm can be affected by mind-affecting effects targeted against their real bodies. However, they perceive them as natural events within their inner mindscape. Multiple creatures affected by the same microcosm can interact with one another within the mindscape.}
        
\DeclareSpell{Mind Probe}{divination () [mind-affecting]|V,  S|1 minute|close (25 ft. + 5 ft./2 levels)|Targetsone creature|1 round/level (D)|Will negates|yes}[Learn answers from a subject’s memories.]
    \DeclareSpellDescription{Mind Probe}{You gain access to the subject's memories and knowledge. You can attempt to learn the answer to one question per round. A successful Will save ends the spell. Otherwise, the creature can attempt a Bluff check with a DC equal to 11 + your Sense Motive modifier. If it fails its Bluff check, you gain the answer you desire. If it succeeds at its check, you gain no information. If it succeeds by 5 or more, it answers whatever it chooses, and you believe that wrong answer to be true. Your questions are purely telepathic inquiries, and the answers to those questions are imparted directly to your mind. You and the target don't need to speak the same language, though less intelligent creatures may yield up only appropriate sensory images in answer to your questions.}
        
\DeclareSpell{Mind Swap}{enchantment (compulsion) [mind-affecting]|V,  S|1 round|medium (100 ft. + 10 ft./level)|Targetsone creature|1 hour/level (D)|Will negates|yes}[Switch minds with another creature for 1 hour per level.]
    \DeclareSpellDescription{Mind Swap}{This spell is similar to possession, except you switch minds with the target, so instead of the target's soul sharing its body with yours and being unable to act, the target's mind moves to your body and can control it as if you each had cast possession on the other.}
        
\DeclareSpell{Mind Swap, Major}{enchantment (compulsion) [mind-affecting]|V,  S,  M (diamonds worth 10, 000 gp)|1 minute|medium (100 ft. + 10 ft./level)|Targetsone creature of your same race|instantaneous|Will negates|yes}[Swap minds with another creature forever.]
    \DeclareSpellDescription{Mind Swap, Major}{The spell functions as mind swap, except as noted above. In addition, after 24 hours, instead of using each other's hit points, you each revert to your original hit points, modified by your new Constitution modifiers. This spell can be reversed only with miracle, wish, or another major mind swap.}
        
\DeclareSpell{Mind Thrust I}{divination () [mind-affecting]|S|1 standard action|close (25 ft. + 5 ft./2 levels)|Targetsone creature|instantaneous|Will half|yes}[Mentally deal 1d6 points of damage per level.]
    \DeclareSpellDescription{Mind Thrust I}{You divine the most vulnerable portions of your opponent's mind and overload it with a glut of psychic information. This attack deals 1d6 points of damage per caster level (maximum 5d6). The target receives a Will save for half damage. This attack has no effect on creatures without an Intelligence score.}
        
\DeclareSpell{Mind Thrust II}{divination () []|S|1 standard action|close (25 ft. + 5 ft./2 levels)|Targetsone creature|instantaneous|Will half|yes}[As mind thrust I, but deal 1d8 points of damage per level (maximum 5d8).]
    \DeclareSpellDescription{Mind Thrust II}{This functions as mind thrust I, but the target takes 1d8 points of damage per caster level (maximum 5d8). This spell can be undercast.}
        
\DeclareSpell{Mind Thrust III}{divination () []|S|1 standard action|close (25 ft. + 5 ft./2 levels)|Targetsone creature|instantaneous|Will half|yes}[As mind thrust II, but deal a maximum of 10d8 points of damage.]
    \DeclareSpellDescription{Mind Thrust III}{This functions as mind thrust I, but the target takes 1d8 points of damage per caster level (maximum 10d8). This spell can be undercast.}
        
\DeclareSpell{Mind Thrust IV}{divination () []|S|1 standard action|close (25 ft. + 5 ft./2 levels)|Targetsone creature|instantaneous|Will half|yes}[As mind thrust II, but a maximum of 15d8 points of damage and target is fatigued for 1 round.]
    \DeclareSpellDescription{Mind Thrust IV}{This functions as mind thrust I, but the target takes 1d8 points of damage per caster level (maximum 15d8) and is fatigued for 1 round if it fails its save. This spell can be undercast.}
        
\DeclareSpell{Mind Thrust V}{divination () []|S|1 standard action|close (25 ft. + 5 ft./2 levels)|Targetsone creature|instantaneous|Will half|yes}[As mind thrust IV, but target is exhausted or fatigued for 1 round.]
    \DeclareSpellDescription{Mind Thrust V}{This functions as mind thrust IV, but the target is also exhausted for 1 round if it fails its save and fatigued for 1 round if it succeeds at its save. This spell can be undercast.}
        
\DeclareSpell{Mind Thrust VI}{divination () []|S|1 standard action|close (25 ft. + 5 ft./2 levels)|Targetsone creature|instantaneous|Will half|yes}[As mind thrust IV, but maximum 20d8 points of damage and target is exhausted and stunned for 1 round.]
    \DeclareSpellDescription{Mind Thrust VI}{This functions as mind thrust IV, but the target takes 1d8 points of damage per caster level (maximum 20d8) and is exhausted and stunned for 1 round if it fails its save. This spell can be undercast.}
        
\DeclareSpell{Mindlink}{divination () [mind-affecting]|V|1 standard action|touch|Targetscreature touched|instantaneous|Will negates (harmless)|yes}[Communicate a great deal of information in an instant.]
    \DeclareSpellDescription{Mindlink}{You link your mind to that of a touched creature to swiftly communicate a large amount of complex information in an instant. You decide what the target learns, limited to any amount of information that otherwise could be communicated in 10 minutes. This information comes in a series of visual images and emotional sensations, and isn't language-dependent.}
        
\DeclareSpell{Mindscape Door}{illusion (phantasm) [mind-affecting]|V,  S|1 standard action|close (25 ft. + 5 ft./2 levels)|Effectone illusory portal|10 minutes/level|Will negates (see text)|no}[Create a portal allowing entry to and exit from a mindscape.]
    \DeclareSpellDescription{Mindscape Door}{You or other creatures enter into or escape from a mindscape (see page 234) through an imaginary doorway. This doorway takes on any form of your choosing, but can be no larger than a 5-foot cube. This spell has different effects depending on whether you are inside a mindscape when you cast it.  When you cast this spell outside a mindscape, the doorway connects to one mindscape inhabited by a creature of your choice within close range (25 feet + 5 feet per 2 levels) of the door. That creature becomes the door's conduit, but can attempt a Will save to deny access to the mindscape. You must be aware of the mindscape to connect a mindscape door to it. If you connect to the mindscape, you designate any number of creatures to be able to see and pass through the door as though it were a normal doorway. You can name specific creatures or categories of creatures, or allow all creatures access. Creatures other than those you designate can't perceive or use the door, nor can mindless creatures or those immune to mind-affecting effects. Once anyone enters the mindscape through the door, a duplicate of the door appears inside the mindscape next to the creature you used as a conduit. Anyone inside a mindscape is able to perceive and use a mindscape door within that mindscape. Anyone who enters is subject to all rules of the mindscape, but is aware she is in a mindscape if you informed her where the portal leads.  When you cast this spell inside a mindscape, a duplicate of the door appears next to your body in the real world. The creator of the mindscape can attempt a Will saving throw to prevent you  from creating the door if she is within the mindscape. The two doors operate as if you'd cast the spell outside a mindscape. You still designate who can use the door from the real world to get into the mindscape, but you can't prevent creatures within the mindscape from exiting to the real world.  Typically, a mindscape door operates in both directions, but you can create it as a one-way door if you so choose. Unlike with other spells, you can cast this spell as a full-round action while engaged in a psychic duel. This spell can be used to exit a binary mindscape, but not to enter one. If you successfully cast mindscape door within a veiled mindscape, you learn that you are in a mindscape, but other creatures observing the door don't automatically realize they are.}
        
\DeclareSpell{Mindwipe}{enchantment (compulsion) [mind-affecting]|V|1 standard action|close (25 ft. + 5 ft./2 levels)|Targetsone creature|instantaneous; see text|Will negates|yes}[Erase a portion of the target’s mind and experiences, inflicting negative levels.]
    \DeclareSpellDescription{Mindwipe}{You erase a portion of the target's mind and experiences, inflicting 2 temporary negative levels on the target for 1 day per caster level. If the target is a spellcaster who must choose and prepare spells in advance, each negative level imposed by mindwipe also causes the target to lose one prepared spell from her highest level of spells known for each of her spellcasting classes for which she must prepare spells. These lost spells are no longer considered known by the target until the corresponding negative levels are removed.  If the target is a spontaneous spellcaster, each negative level inflicted by mindwipe causes her to lose one spell slot of her highest level of spells for each spontaneous spellcasting class in which she has levels and to lose knowledge of one random spell known of that level (or the next lowest level if she has already lost knowledge of all spells known of that level, and so on). The target cannot use the lost spell slots and spells known as long as the corresponding negative levels last. When the target loses a spell known, the spell remains on her class list but she cannot prepare or cast the spell.  If the negative levels from mindwipe cause the target's total number of negative levels to equal or exceed her character level, instead of dying, she enters a catatonic state as long as the negative levels from mindwipe continue to cause her total number of negative levels to equal or exceed her character level.}
        
\DeclareSpell{Node Of Blasting}{abjuration () [mind-affecting]|V|1 standard action|touch|Targetsone touched object weighing no more than 10 lbs.|permanent until discharged (D)|Will partial; see text|no}[Place a trap on an object to mentally damage a creature that touches it.]
    \DeclareSpellDescription{Node Of Blasting}{You imbue an object with psychic energy. The node of blasting unleashes a mental blast when a creature with a mind touches the object, dealing 6d6 points of damage to the creature touching the object and causing the creature to be staggered for 1 minute. A successful saving throw halves the damage and negates the staggered condition.  Magic traps such as node of blasting are hard to detect and disable. A character with the trapfinding class feature can use Disable Device to thwart node of blasting. The DCs to find a node of blasting using Perception and to disable it using Disable  Device are both equal to 25 + the spell's level. Additionally, a creature with the read aura occult skill unlock (see page 197) can attempt the same Perception check to notice a node of blasting.}
        
\DeclareSpell{Object Possession}{necromancy () []|V,  S|1 standard action|close (25 ft. + 5 ft./2 levels)|Targetsunattended Large or smaller object (minimum Tiny)|10 minutes/level (D)|None|No}[As lesser object possession, but with a larger object.]
    \DeclareSpellDescription{Object Possession}{This spell functions as lesser object possession, except as noted above. The possessed animated object has a number of Construction Points appropriate for its size (up to 3 CP for Large objects).  You can return your consciousness to your body as a standard action. On your next turn, you can attempt to possess a different object as a standard action instead of ending the spell.}
        
\DeclareSpell{Object Possession, Greater}{necromancy () []|V,  S|1 standard action|medium (100 ft. + 10 ft./level)|Targetsunattended Gargantuan or smaller object (minimum Tiny) or construct|10 minutes/level (D)|None|No}[As possess object, but the object is more powerful and you can possess a construct.]
    \DeclareSpellDescription{Object Possession, Greater}{This spell functions as object possession, except as noted above. The possessed animated object has a number of Construction Points appropriate for its size (up to 5 CP for Gargantuan objects). Your silver cord extends to medium range (100 ft. + 10 ft./level).  You can attempt to possess a construct instead of an unattended object as your first possession with this spell. If you do, this spell acts as control constructUM, except as noted above.}
        
\DeclareSpell{Object Possession, Lesser}{necromancy () []|V,  S|1 standard action|touch|Targetsunattended Tiny or Small object|1 minute/level (D)|None|No}[Project your soul into an object, animating it.]
    \DeclareSpellDescription{Object Possession, Lesser}{This spell functions as possession, except you possess an object instead of a creature.  A possessed object becomes capable of animation, gaining the statistics of an animated object of its size (including 1 CP worth of abilities; Pathfinder RPG Bestiary 14). You can't use any spells or other abilities while possessing an object.  A spiritual tether connects your body to the possessed object, in a manner similar to a silver cord (see page 244). If the possessed object and your body are ever farther than close range (25 ft. + 5 ft./2 levels) from one another, this tether breaks, returning your consciousness to your body.}
        
\DeclareSpell{Object Reading}{divination () []|V,  S|1 standard action|touch|Targetsone touched object|concentration, up to 1 round per level (D)|none|no}[Read psychic impressions left on an object.]
    \DeclareSpellDescription{Object Reading}{You read the psychic impressions left upon an object by emotionally or psychically charged events in the item's history, or the information imprinted by the charge object or implant false reading spells. T his spell returns the s ame i nformation a s the psychometry occult skill unlock (see page 196), but gives 1 piece of information when cast and requires 1 round of concentration per additional piece of information instead of 1 minute. You must still attempt an Appraise check to see how much information you gain. You gain a +10 competence bonus on the check.}
        
\DeclareSpell{Oneiric Horror}{illusion (phantasm) [mind-affecting]|V,  S|1 standard action|medium (100 ft. + 10 ft./level)|Targetsone living creature|1 round/level (D)|Will negates|yes}[Distract and fatigue the target with a creature from its nightmares.]
    \DeclareSpellDescription{Oneiric Horror}{You cause the subject to believe it is being attacked by a creature out of its nightmares. Each round, the subject makes a full-attack action against the creature. A flying creature can still attempt a Fly check to hover. Each round on its turn after making a full attack against the imaginary creature, the subject can attempt a new saving throw to end the effect. The subject is fatigued for 1 minute after the spell ends.}
        
\DeclareSpell{Oneiric Horror, Greater}{illusion (phantasm) [mind-affecting]|V,  S|1 standard action|medium (100 ft. + 10 ft./level)|Targetsone living creature|1 round/level (D)|Will negates, Fortitude negates, see text|yes}[As oneiric horror, plus Str damage.]
    \DeclareSpellDescription{Oneiric Horror, Greater}{This spell functions as oneiric horror except each round the subject fails its Will save, it takes 1 point of Strength damage, and after the spell ends, the subject must succeed at a Fortitude save or be fatigued for a number of minutes equal to the number of rounds spent under the spell's effect.}
        
\DeclareSpell{Paranoia}{illusion (phantasm) [mind-affecting]|V,  S|1 standard action||Targetsone creature|1 round/level (D)|Will negates|yes}[Target becomes hostile to all creatures.]
    \DeclareSpellDescription{Paranoia}{The target believes everyone it sees is an enemy. The target becomes hostile to all creatures, treating all creatures as enemies and only itself as an ally. The target must attempt attacks of opportunity whenever any creature provokes them. Finally, whenever the target is adjacent to two or more creatures, its paranoia overwhelms it, and it takes a -2 penalty on attack rolls, weapon damage rolls, ability checks, skill checks, and saving throws.}
        
\DeclareSpell{Parchment Swarm}{transmutation () []|S,  M (blank parchment or magic scroll; see text)|1 standard action|close (25 ft. + 5 ft./2 levels)|Targetsone creature|instantaneous|Reflex half; see text|yes}[Shredded parchment deals 1d6 points of damage per level, and has a spell effect if you shred a scroll.]
    \DeclareSpellDescription{Parchment Swarm}{When you cast this spell, you quickly tear a parchment into shreds, releasing the flying fragments to swarm around a target creature and deliver thousands of tiny paper cuts. Using normal parchment, the spell deals 1d6 points of magical slashing damage per caster level (maximum 15d6).  If you use a magic scroll as the material component, choose one of the spells stored in the scroll. If the spell is 1st level, you can choose to apply the effect of that spell to the parchment swarm's target on a failed Reflex save (if the spell on the scroll has its own saving throw, the target then attempts that save as normal). If the spell on the scroll is at least 2nd level, you can choose to instead change parchment swarm to affect a 20-foot-radius spread instead of a single target.}
        
\DeclareSpell{Placebo Effect}{illusion (phantasm) [mind-affecting]|V,  M (a sugar cube)|1 standard action|touch|Targetscreature touched|1 minute/level|Will disbelief (harmless)|Yes (harmless)}[Temporarily suppress an affliction or condition.]
    \DeclareSpellDescription{Placebo Effect}{The subject temporarily ceases to feel the ill effects of a single ongoing affliction or condition from the following list: blinded, cursed, dazed, deafened, diseased, fatigued, frightened, nauseated, panicked, paralyzed, poisoned, shaken, sickened, staggered, or stunned. If that affliction or condition has a duration, it is suspended until this spell expires. If the subject has multiple instances of the same type of affliction (such as multiple diseases), a single casting of this spell can suspend only one of them. Placebo effect doesn't remove or temporarily negate any damage that the affliction may have already caused, nor does it provide protection against receiving such conditions again.}
        
\DeclareSpell{Possession}{necromancy () []|V,  S|1 standard action|medium (100 ft. + 10 ft./level)|Targetsone creature|1 hour/level (D)|Will negates|yes}[Project your soul into a creature’s body.]
    \DeclareSpellDescription{Possession}{You attempt to possess a creature by projecting your soul into its  body. The target creature must be within spell range and you must know where it is, though you do not need line of sight or line of effect to it. When you transfer your soul upon casting, your body appears to be dead. Failure to take over a host ends the spell.  If you are successful, your life force occupies the host body. The host's soul is imprisoned with you, but can still use its own senses (though it can't assert any influence or use even purely mental abilities). You can communicate telepathically with the host as if you shared a common language, but only with your consent. You keep your Intelligence, Wisdom, Charisma, level, class, base attack bonus, base save bonuses, alignment, and mental abilities. The body retains its Strength, Dexterity, Constitution, hit points, natural abilities, and automatic abilities. A body with extra limbs doesn't allow you to make more attacks (or more advantageous two-weapon attacks) than normal. You can't activate the body's extraordinary or supernatural abilities, nor can you cast any of its spells or spell-like abilities.  As a standard action, you can shift freely back to your own body regardless of range, so long as it remains on the same plane. If the host's body is killed, you return to your own body and the life force of the host departs (it is dead). If your body is slain, when the spell expires or the host's body is killed, you are slain. It is possible to cast this spell on a new target from within another creature's body; this resets the duration. You still return to your original body (not any intermediate body) when you take a standard action to do so or the spell's duration expires. Returning to your body ends the spell. Creatures whose souls are their bodies, such as incorporeal undead and non-native outsiders, use the effects of greater possession instead.}
        
\DeclareSpell{Possession, Greater}{necromancy () []|V,  S|1 standard action|medium (100 ft. + 10 ft./level)|Targetsone creature|1 hour/level (D)|Will negates|yes}[As possession, but your body vanishes.]
    \DeclareSpellDescription{Possession, Greater}{This spell functions as possession, but when you possess a host, you enter the host's body and your physical body vanishes. You are ejected to the closest empty square upon expiration of the spell or upon the host's death.}
        
\DeclareSpell{Primal Regression}{enchantment (compulsion) [mind-affecting]|S|1 standard action|close (25 ft. + 5 ft./2 levels)|Targetsone creature/2 levels, no two of which can be more than 30 feet apart|1 minute/level|Will negates|yes}[Make a creature become bestial and unintelligent.]
    \DeclareSpellDescription{Primal Regression}{This spell sequesters the targets' ability for higher reasoning, allowing their darker impulses to come to the fore. An affected target becomes a ravening monster-savage, bloodthirsty, and brutish. Until the spell ends, the targets have all of the following benefits and drawbacks.   Targets' Intelligence and Charisma scores drop to 3 (if the scores were higher), and they are unable to use Intelligence-or Charisma-based skills, cast spells, understand language, or communicate coherently.   Targets take a -4 penalty on Will saves.   Targets gain a +6 enhancement bonus to Strength, a +2 natural armor bonus to AC, and 2d8 temporary hit points. These temporary hit points disappear at the end of the spell's duration.  When the spell ends, each affected creature must succeed at a Will saving throw or take 1d4 points of Intelligence drain and 1d4 points of Charisma damage.}
        
\DeclareSpell{Psychic Asylum}{illusion (phantasm) [mind-affecting]|V,  S|1 swift action|personal|Targetsyou|instantaneous; see text|none|no}[Perform a lengthy mental task in a private mindscape.]
    \DeclareSpellDescription{Psychic Asylum}{You retreat into a mindscape (see page 234) of your own making that allows you to perform a lengthy mental task in an instant. You create a mental landscape that provides succor and calm for you, such as a library, sitting room, garden, or childhood tree house. You can spend up to 15 minutes in your psychic asylum. While you are within, no time passes for your body, and when you emerge you can continue with your turn. The mindscape is overt, finite, and has a rapid passage of time.  While within the mindscape, you can consult any text, recall any conversation, or remember any image that you have been exposed to with perfect clarity and recollection, as if you had an eidetic memory. You must have seen or heard the material within a time frame of 1 week per level. For example, you might wish to reexamine a passage of text from an ancient manuscript you rapidly scanned in a library 2 weeks prior. While in the psychic asylum, you can perfectly recall the precise layout of the text  within the ancient manuscript and read it word for word. Once you emerge from the psychic asylum, you can recall the details of what you had just studied as clearly as if you had just looked at it, but you no longer have total recall of the material.  If you're able to prepare spells, you can use the time to prepare a single spell. For example, if you were poisoned, you could use your swift action to cast psychic asylum, pray and meditate for the full 15 minutes to gain neutralize poison, then emerge from the mindscape and immediately cast the spell upon yourself as your standard action.  Any mind-affecting spell that was affecting you before you cast the spell continues to expend its duration on you while you are within the psychic asylum, so you can use this spell to wait out the effects of such a condition. Effects that are currently affecting your body do not expend additional duration while you are in the psychic asylum, but you also don't experience their effects during the time spent there.}
        
\DeclareSpell{Psychic Crush I}{necromancy () [mind-affecting]|S|1 standard action|close (25 ft. + 5 ft./2 levels)|Targetsone creature|instantaneous|Will partial and Fortitude partial; see text|yes}[Sicken a target and cause it to start dying, or deal 3d6 + 1 points of damage per level on a save.]
    \DeclareSpellDescription{Psychic Crush I}{Using your psychic power, you invade the mind of the target and tear it asunder, causing massive internal damage to both its mind and body. If the target succeeds at the initial Will save, it is sickened for 1 round. If the target fails its Will save, it must attempt a Fortitude save (with a +4 circumstance bonus on this save if it has more than half its total hit points remaining). If it also fails the Fortitude save, the target is reduced to -1 hit points and is dying. If the target succeeds at its Fortitude save, it instead takes 3d6 points of damage + 1 point of damage per caster level, which cannot reduce the target below -1 hit point, and the target is sickened for 1 round. This attack has no effect on creatures without an Intelligence score.}
        
\DeclareSpell{Psychic Crush II}{necromancy () [mind-affecting]|S|1 standard action|close (25 ft. + 5 ft./2 levels)|Targetsone creature|instantaneous|Will partial and Fortitude partial; see text|yes}[As psychic crush I, but deal 5d6 + 1 points of damage per level on a save and harder to resist.]
    \DeclareSpellDescription{Psychic Crush II}{This functions as psychic crush I, but on a successful Fortitude save, the target takes 5d6 points of damage + 1 point of  damage per caster level. In addition, the target receives a +4 circumstance bonus on the Fortitude save only if it is at full hit points; otherwise, it gains a +2 bonus if it has more than half its total hit points remaining. This spell can be undercast.}
        
\DeclareSpell{Psychic Crush III}{necromancy () [mind-affecting]|S|1 standard action|close (25 ft. + 5 ft./2 levels)|Targetsone creature|instantaneous|Will partial and Fortitude partial; see text|yes}[As psychic crush I, but deal 7d6 + 1 points of damage per level on a save and harder to resist.]
    \DeclareSpellDescription{Psychic Crush III}{This functions as psychic crush I, but the target takes 7d6 points of damage + 1 point of damage per caster level on a successful Fortitude save and 1 point of damage per caster level on a successful Will save. The target receives a +2 circumstance bonus on the Fortitude save if it is at full hit points, and no bonus if it has taken any damage. This spell can be undercast.}
        
\DeclareSpell{Psychic Crush IV}{necromancy () [mind-affecting]|S|1 standard action|close (25 ft. + 5 ft./2 levels)|Targetsone creature|instantaneous|Will partial and Fortitude partial; see text|yes}[As psychic crush I, but deal 9d6 + 1 points of damage per level on a save and no Fort save at 1/2 hp or fewer.]
    \DeclareSpellDescription{Psychic Crush IV}{This functions as psychic crush I, but the target takes 9d6 points of damage + 1 point of damage per caster level on a successful Fortitude or Will save. The target does not receive any saving throw bonus because of its hit points. If it is at fewer than half its total hit points, it doesn't gain a Fortitude save to resist this spell but instead proceeds as if it had automatically failed its Fortitude save. This spell can be undercast.}
        
\DeclareSpell{Psychic Crush V}{necromancy () [mind-affecting]|S|1 standard action|close (25 ft. + 5 ft./2 levels)|Targetsone creature|instantaneous|Will partial and Fortitude partial; see text|yes}[As psychic crush IV, but deal 11d6 + 1 points of damage per level on a save.]
    \DeclareSpellDescription{Psychic Crush V}{This functions as psychic crush IV, but on a successful Fortitude or Will save, the target takes 11d6 points of damage + 1 point of damage per caster level. If it is at fewer than half its total hit points, the target takes a -2 penalty on the Will save to resist this spell. This spell can be undercast.}
        
\DeclareSpell{Psychic Image}{illusion (shadow) []|V|1 standard action|long (400 ft. + 40 ft./level)|Effectone shadow duplicate|1 round/level (D)|Will disbelief (if interacted with)|yes}[Create a perfect illusion of yourself that is incorporeal and capable of casting psychic spells, and switch between it and your body at will.]
    \DeclareSpellDescription{Psychic Image}{You envelop your consciousness in a quasi-real image of yourself. Your psychic image looks, sounds, and smells like you, but is intangible. While your mind occupies the image, you control it as though it were your own body, but you cannot directly affect physical objects. Your image moves with a fly speed of 60 feet and perfect maneuverability. Your senses perceive only what the image can see and hear while occupying it, and your own body is considered blind, deaf, and helpless. You can switch between the image and your body as a swift action. While your mind is in your body, the image is similarly helpless.  Your image can pass through solid objects as though you are incorporeal. It cannot go farther into a solid object than  your space (5 feet for a Small or Medium creature). It can't be damaged by most attacks or effects, whether or not they affect incorporeal creatures. However, mind-affecting effects targeted against or affecting your psychic image have their full effect on you whenever your mind occupies it.  If you desire, you can cast any psychic spell or spell-like ability with a range of touch or greater while your mind occupies your image. You can't cast non-psychic spells through the image, even if you possess them. The psychic image can cast only psychic illusion spells on itself. The spells affect other targets normally, despite originating from the psychic image.  Objects are affected by the psychic image as if they had succeeded at their Will saves. You need not maintain line of effect to your psychic image, but if you cross into another plane even for an instant, such as via blink, dimension door, or similar spells, the spell ends.}
        
\DeclareSpell{Psychic Reading}{divination () []|V|1 standard action|close (25 ft. + 5 ft./2 levels)|Targetsone humanoid creature|1 round|none|yes}[Read surface thoughts to learn information about a subject.]
    \DeclareSpellDescription{Psychic Reading}{You are able to read a person's surface thoughts and take cues from the person's appearance, body language, and manner of speech to infer a great deal of information about the person, even if that person is in disguise (including polymorph effects). When casting the spell, attempt a DC 20 Sense Motive check with a bonus equal to your caster level. A successful skill check reveals to you one of the following pieces of information of your choice, plus one additional piece of information for every 5 points by which your check result exceeds 20: age, alignment, class†, feats†, gender, native language, place of origin, race or ethnicity, racial traits†, religion, sexual orientation, or training in a Craft, Perform, or Profession skill. For items marked with a cross (†), if the target has more than one of these features, each one you discover counts as a piece of information.  There's a 70\% chance that all information you receive is correct. This roll is made secretly. Otherwise, on a roll of 71-80, you receive one false piece of information, on a roll of 81-90 you receive two, and on a roll of 91-100 you receive three. You can't get more false information than the total number of pieces of information you discover.}
        
\DeclareSpell{Psychic Surgery}{enchantment () [mind-affecting]|V,  M (diamond dust worth 250 gp)|10 minutes|touch|Targetsone willing and living creature|instantaneous|none|no}[Cure all Int, Wis, and Cha damage and drain, plus remove other mental afflictions and conditions.]
    \DeclareSpellDescription{Psychic Surgery}{Psychic surgery cures the target of all Intelligence, Wisdom, and Charisma damage and restores all points permanently drained from the target's Intelligence, Wisdom, and Charisma scores. It also eliminates all ongoing insanity, confusion, and fear effects. Psychic surgery can also remove other mental afflictions, including enchantment spells and abilities, and even instantaneous effects, but in this case, if dispel magic couldn't remove the effect, psychic surgery works only if the spell level or equivalent spell level of the effect was 6th level or lower. Psychic surgery removes all effects magically altering the target's memory, even instantaneous effects, and it can restore a memory to perfect clarity like the second use of modify memory.}
        
\DeclareSpell{Purge Spirit}{necromancy () []|V,  S|1 standard action|medium (100 ft. + 10 ft./level)|Targetsone creature or haunt|instantaneous|Will partial|yes}[Deal 1d6 points of damage per level to one haunt or spirit creature and stagger it.]
    \DeclareSpellDescription{Purge Spirit}{Purge spirit rips away at the target's spiritual substance, scattering it over a wide area and hampering the target's ability to reform. The target takes 1d6 points of damage per caster level (maximum 10d6) and is staggered for 1 round. On a successful saving throw, the target takes half damage and is not staggered. This spell affects astrally projected creatures, ethereal creatures, haunts, incorporeal creatures, mediums channeling a spirit, and phantoms, and at the GM's discretion can affect other spirits or creatures made of ectoplasm. Incorporeal creatures take full damage from purge spirit.}
        
\DeclareSpell{Quintessence}{illusion (glamer) []|V,  S|1 standard action|touch|Targetscreature or object touched|10 minutes/level (D)|Will negates (harmless) and Will disbelief; see text|see text}[Mask any flaws of or damage to a creature or object.]
    \DeclareSpellDescription{Quintessence}{You draw forth the idealized image of the target creature or object, masking any flaws or damage. An injured or ill creature appears healthy, and a damaged object or one with the broken condition appears intact. However, a corpse  masked by quintessence remains obviously dead, and a completely destroyed object can't be made to seem whole. This illusion has visual and tactile components. Careful examination of or handling the target grants a saving throw to disbelieve, but casual observation does not. Using a broken object for its intended purpose automatically reveals the deception. Appraise checks to assess the value of an object affected by quintessence estimate the value as an item of its type in perfect condition, unless the appraiser disbelieves the illusion.  Unwilling targets can negate the spell's effect on them with successful Will saves or with spell resistance. Those who interact with the target can attempt Will saves to see through the glamer, but spell resistance doesn't apply. Quintessence counters and dispels decrepit disguise.}
        
\DeclareSpell{Remote Viewing}{divination (scrying) []|V,  S,  M (incense)|1 hour|see text|Targetsyou|instantaneous||}[Gain psychic impressions from a distant location.]
    \DeclareSpellDescription{Remote Viewing}{Your body enters a trance as you send your psychic senses to a distant location and gain psychic impressions of that location. You must specify the distance and direction to the location you desire to view remotely. This spell doesn't allow you to see the visual appearance of the location's surface, so it isn't useful for casting spells like teleport, but it grants you a psychic impression of the location, which could give you deeper information. For example, a forest that is home to a tight-knit community of fey might appear as a city in the trees, or a beautiful palace ruled by an evil king and warded by forbiddance might appear as a dark fortress encased in insubstantial chains.  During the 1 hour casting time of remote viewing, your real body is unconscious and helpless, and you are unaware of its surroundings.}
        
\DeclareSpell{Repress Memory}{enchantment (compulsion) [mind-affecting]|V,  S|1 round|personal|Targetsyou|instantaneous||}[Remove a piece of knowledge from your mind.]
    \DeclareSpellDescription{Repress Memory}{This spell allows you to safeguard important knowledge, even from yourself. When casting this spell, you recount one piece of knowledge you possess (up to a maximum of 50 words). This knowledge disappears utterly from your mind, and you might not realize you forgot something. The magic of the spell patches omissions in your memory with indistinct haze  Repress memory protects against detect thoughts, discern lies, zone of truth, and similar spells, though careful questioning may reveal the gaps in your memory, or that your memory has been affected by the spell.  A repressed memory can be restored only by break enchantment, psychic surgery, limited wish, miracle, or wish. If you use this spell to negate the memory of a magical compulsion, it doesn't remove the compulsion, nor does it remove procedural memories that might affect your skills or class abilities.  At the GM's discretion, multiple castings of this spell might erase memories of a lengthier event or all memory of a place or individual from your memory.}
        
\DeclareSpell{Retrocognition}{divination () []|V,  S|1 minute|personal|Targetsyou|concentration, up to 1 minute/level||}[Gain psychic impressions from past events in a location.]
    \DeclareSpellDescription{Retrocognition}{This spell allows you to gain psychic impressions from past events that occurred in your current location. Retrocognition reveals psychic impressions from events that occurred over the course of the last hour throughout the first minute of the duration, followed by impressions from the next hour back the next minute you concentrate, and so on. If a psychically traumatic or turbulent event happened during that time period, you must succeed at a concentration check (DC = 20, 30, or 40, depending on the severity of the psychic disturbance) or lose concentration on the spell.  At caster level 13th and higher, you can choose to collect impressions from over the course of a longer interval of time than an hour, beginning at 1 week per minute of concentration (as listed on the table below). The amount of detail you receive diminishes, so this eventually makes it harder to distinguish impressions left by anything but the most major events.     Caster LevelTime Period13th-15th1 week per minute16th-18th1 year per minute19th+1 century per minute}
        
\DeclareSpell{Riding Possession}{necromancy () []|V,  S|1 standard action|medium (100 ft. + 10 ft./level)|Targetsone creature|1 hour/level (D); see text|Will negates|yes}[As possession, but you observe instead of control the subject.]
    \DeclareSpellDescription{Riding Possession}{You stealthily project your soul into the host's body as an observer, with limited ability to influence the target. This functions as possession, except the host is still in full control of its body and is unaware you are possessing it. You can't communicate with the host directly, but you can cast mind-affecting spells or riding possession on the host as long as you can cast these spells as purely mental actions. Even if the spell you cast would normally affect more than one target or an area, it can affect only the host. If the host succeeds at a saving throw against a spell that you cast in this way, it immediately becomes aware that it is possessed, and if it was already aware, it receives another saving throw against riding possession. Protection from evil and similar effects don't expel you from the host, but they do prevent you from casting further spells from within your host until their durations expire.}
        
\DeclareSpell{Sealed Life}{abjuration () []|S|1 standard action|close (25 ft. +5 ft./2 levels)|Targetsone creature|1 round/level|none|yes}[Prevent a creature from transferring life force to or from others.]
    \DeclareSpellDescription{Sealed Life}{You seal the life force within the target, preventing it from sharing its vitality with others. Affected creatures can't transfer damage to or from another, such as through shield other, a spiritualist's life bond, or a summoner's life link. Effects such as vampiric touch that steal vitality from others deal damage normally, but provide no benefits.}
        
\DeclareSpell{Sealed Life, Greater}{abjuration () []|S|1 standard action|close (25 ft. +5 ft./2 levels)|Targetsone creature|1 minute/level|none|yes}[As sealed life, plus death ward and immunity to soul transference.]
    \DeclareSpellDescription{Sealed Life, Greater}{This spell functions as sealed life, except it additionally provides the benefits of death ward and renders its targets immune to effects that extract or transfer its soul, such as possession and trap the soul. This spell prevents a soul from returning to its body if it has already departed, until the spell's duration expires.}
        
\DeclareSpell{Sessile Spirit}{necromancy () []|V,  S|1 standard action|medium (100 ft. + 10 ft./level)|Targetsone creature or object (see text)|1 round/level (D)|Will negates (harmless)|yes}[Cause a spirit inhabiting a creature or an object to go dormant.]
    \DeclareSpellDescription{Sessile Spirit}{You reach into the target creature and cause that spirits within to become dormant and inactive. If the target is a medium with a spirit inhabiting its body or a spiritualist with a phantom inhabiting its consciousness (or a member of another class using the spirit or phantom class features), a failed save renders that spirit or phantom powerless, suppressing any benefits the spirit or phantom normally provides to its host, including bonus feats and spells known.  If a spirit within the target is a creature using a possession effect, including possess object, possession, or a ghost's malevolence, the spell targets the possessing creature rather than the creature it inhabits. On a failed save, the possessing spirit isn't exorcised from the target but is dazed for the duration of the spell.}
        
\DeclareSpell{Shadow Body}{transmutation (polymorph) []|V,  S|1 standard action|personal|Targetsyou|1 minute/level (D)||}[Turn your body into a living shadow.]
    \DeclareSpellDescription{Shadow Body}{You exchange the crude matter of your material body with the insubstantial essence of the Plane of Shadow, becoming a living shadow yourself. You are visible as an unattached shadow in bright light or normal light, but you gain total concealment in dim light or darkness. Against creatures with darkvision, you gain concealment rather than total concealment. Your shadow body is incorporeal for most purposes, though you can't fly or pass through solid objects or creatures. However, you can move at your normal speed along any surface, including horizontal and vertical surfaces and liquids, and you are never slowed by difficult terrain. Your size doesn't change.  You can speak and cast spells and perform mental actions, but you have no physical substance and cannot manipulate objects or attack physically. You can deliver touch spells and effects as if making an incorporeal touch attack. Your equipment merges with your shadow body, so you can't cast spells with a material component unless those spells are prepared with Eschew Materials.}
        
\DeclareSpell{Spirit-Bound Blade}{evocation () []|S|1 standard action|touch|Targetsweapon touched|1 minute/level|Will negates (harmless, object)|no}[Give a weapon ghost touch and another ability tied to an emotion.]
    \DeclareSpellDescription{Spirit-Bound Blade}{You focus emotional energy and weave it into a shroud of hardened ectoplasm around the weapon you touch, infusing it with a ghostly glow and great power. The weapon becomes a ghost touch weapon, and gains one of the following additional benefits, depending on the type of emotion you  infuse into the weapon. If a special ability wouldn't apply to the chosen weapon (such as vicious on a ranged weapon), the weapon doesn't gain that benefit.  Anger: The weapon also gains the vicious special ability.  Dedication: The weapon also gains the returning special ability.  Despair: The weapon also gains the cruelUE special ability.  Fear: The weapon also gains the menacingUE special ability.  Hatred: The weapon also gains the cunningUE special ability.  Jealousy: The weapon also gains the mimeticUE special ability.  Zeal: The weapon also gains the keen special ability.  If you are a spiritualist and the type of emotional energy you choose matches the emotional focus of your phantom, the weapon grants its wielder a +2 bonus on skill checks with both skills you gain Skill Focus in from your phantom's emotional focus.}
        
\DeclareSpell{Synapse Overload}{divination () [mind-affecting]|V|1 standard action|touch|Targetsliving creature touched|instantaneous|Fortitude partial (see text)|yes}[Deal 1d6 points of damage per level and stagger target for 1 minute.]
    \DeclareSpellDescription{Synapse Overload}{You cause the target's mind to unleash a vast overflowing torrent of information throughout the target's body, causing the target's synapses to violently trigger. The target takes 1d6 points of electrical damage per caster level (maximum 15d6) and is staggered for 1 minute. A successful Fortitude saving throw doesn't reduce the damage, but it negates the staggered effect.}
        
\DeclareSpell{Synaptic Pulse}{enchantment (compulsion) [mind-affecting]|V|1 standard action|30 ft.|Area30-ft.-radius spread centered on you|1 round|Will negates|yes}[Stun creatures in a 30-ft. radius.]
    \DeclareSpellDescription{Synaptic Pulse}{You emit a pulsating mental blast that stuns all creatures in range of your psychic shriek for 1 round.}
        
\DeclareSpell{Synaptic Pulse, Greater}{enchantment (compulsion) [mind-affecting]|V|1 standard action|30 ft.|Area30-ft.-radius spread centered on you|1d4 rounds; see text|Will negates|yes}[As synaptic pulse, but for 1d4 rounds.]
    \DeclareSpellDescription{Synaptic Pulse, Greater}{You emit a pulsating mental blast that stuns all creatures in range of your psychic shriek for 1d4 rounds. On a successful save, a creature is instead sickened for 1 round.}
        
\DeclareSpell{Synaptic Scramble}{enchantment (compulsion) [mind-affecting]|V,  S|1 standard action|medium (100 ft. + 10 ft./level)|Targetsone creature|1 round/level|Will negates|yes}[Prevent the target from communicating and cause it to act randomly.]
    \DeclareSpellDescription{Synaptic Scramble}{You scramble the synaptic connections of your target, causing it to lose the ability to coherently communicate and to take unintended actions when it meant to perform others. The target cannot effectively communicate while under the spell's influence, and thus cannot engage in acts like speaking, attempting Bluff checks to pass secret messages, writing, or using telepathy, although the spell doesn't prevent verbalizations made for purposes other than communication, such as command words or the verbal component of spellcasting.  The target's mind is too scrambled to take full-round actions or longer actions, as the target's attention inevitably wanders before the action completes. Whenever the target attempts to take a move action or a standard action, roll on the following table instead. If the result on the table is an action that the target cannot take, the target takes no action instead.     d\%Action1-20Move action (must use move action to move)21-40Other move action (perform a move action other than moving)41-60Attack action (spend a standard action on the attack action, attacking once with a weapon)61-80Other standard action (perform a standard action that isn't the attack action)81-100Desired action (perform whatever standard or move action the creature desires to perform) }
        
\DeclareSpell{Synesthesia}{illusion (phantasm) [mind-affecting]|V,  S|1 standard action|close (25 ft. + 5 ft./2 levels)|Targetsone living creature|1 round/level|Will negates|yes}[Target moves at half speed, takes penalties, and has trouble casting spells.]
    \DeclareSpellDescription{Synesthesia}{You overstimulate the senses of the affected creature, causing its senses to interfere with another. While a creature is under the effects of this spell, sensory input is processed by the wrong  senses, such that noise triggers bursts of colors, smells create sounds, and so on. The affected creature moves at half speed, has a 20\% miss chance on attacks, and takes a -4 penalty to AC and on skill checks and Reflex saves. Successful spellcasting while affected requires a concentration check with a DC equal to this spell's save DC plus the level of spell being cast.  In addition, the affected creature is considered distracted for the purpose of attempting Perception checks. Effects that negate or reduce concealment do not affect the miss chance from synesthesia.}
        
\DeclareSpell{Synesthesia, Mass}{illusion (phantasm) [mind-affecting]|V,  S|1 standard action|medium (100 ft. + 10 ft./level)|Targetsone or more living creatures, no two of which can be more than 30 ft. apart|1 round/level|Will negates|yes}[As synesthesia, but affecting multiple creatures.]
    \DeclareSpellDescription{Synesthesia, Mass}{This spell functions like synesthesia, except as noted above.}
        
\DeclareSpell{Talismanic Implement}{evocation () []|V|10 minutes|personal|Targetsyou|1 hour/level (D) or until discharged||}[As contingency, but invests a spell into one of your implements.]
    \DeclareSpellDescription{Talismanic Implement}{This spell functions as contingency, but you invest a spell whose level doesn't exceed one-quarter of your level (rounded down, maximum spell level 3rd) into one of your implements that holds 2 or more points of mental focus. The spell's school must match that of the implement, and the spell must be one that affects only your person. Casting talismanic implement reduces the selected implement's mental focus by 2, including for the purposes of determining the effects of resonant powers. When you refresh your mental focus, talismanic implement ends automatically, even if the duration hasn't elapsed. Unlike contingency, you may have multiple talismanic implements simultaneously, though a single implement can hold only one talismanic implement spell, and if you have any talismanic implements, you can't have any other kind of contingency in effect.}
        
\DeclareSpell{Telekinetic Maneuver}{transmutation () []|V,  S|1 standard action|close (25 ft. + 5 ft./2 levels)|Targetsone creature|concentration (up to 1 round/level)|none|yes}[Perform a telekinetic combat maneuver.]
    \DeclareSpellDescription{Telekinetic Maneuver}{This spell functions as telekinesis, but it can be used only to perform a bull rush, disarm, drag, grapple (including pin), reposition, steal, or trip combat maneuver.}
        
\DeclareSpell{Telekinetic Storm}{evocation () [force]|V,  S|1 standard action||Area40-ft.-radius burst centered on you|instantaneous|Fortitude partial (see text)|yes}[Deal 1d6 points of damage per level plus daze and stun in a 40-ft. radius.]
    \DeclareSpellDescription{Telekinetic Storm}{You generate a storm of telekinetic energy that emanates from you, ripping through the spell's area of effect with devastating force. Any creature caught in the spell's radius takes 1d6 points of damage per caster level (maximum 20d6) and is dazed and stunned for 1 round. A successful Fortitude save reduces the damage by half and negates the dazed and stunned effects.  The telekinetic storm damages objects in the area. If the damage caused to an interposing barrier shatters or breaks through it, the telekinetic storm continues beyond the barrier if the spell's area permits; otherwise, it stops at the barrier just as any other spell effect does.  You can designate any number of creatures to be immune to the spell's effect, though you must be capable of targeting those creatures.}
        
\DeclareSpell{Telempathic Projection}{enchantment (compulsion) [emotion,  mind-affecting]|V,  S|1 standard action|medium (100 ft. + 10 ft./level)|Targetsone creature|1 minute/level|Will negates (see text)|yes}[Alter the target’s attitude or give bonuses to those interacting with the target.]
    \DeclareSpellDescription{Telempathic Projection}{You alter the target's mood, adjusting its attitude toward you or another creature you designate by one step either positively or negatively (see the Diplomacy skill on page 94 of the Pathfinder RPG Core Rulebook). You can instead use this spell to assist your own or an ally's Bluff, Diplomacy, Intimidate, Perform, or Sense Motive check, granting that check a +5 insight bonus against the target of your telempathic projection with no save. In this case, the spell's duration expires immediately when the skill check is complete.}
        
\DeclareSpell{Telepathy}{divination () [mind-affecting]|V|1 standard action|personal|Targetsyou|1 minute/level||}[Communicate mentally with creatures within 100 ft.]
    \DeclareSpellDescription{Telepathy}{You can mentally communicate with any other creature within 100 feet that has a language. It is possible to address multiple creatures at once telepathically, although maintaining a telepathic conversation with more than one creature at a time is just as difficult as speaking and listening to multiple people simultaneously.}
        
\DeclareSpell{Thaumaturgic Circle}{abjuration () []|V,  S,  M (a 3-ft.-diameter circle of powdered silver),  DF|1 standard action|touch|Area10-ft.-radius emanation from touched creature|10 min./level|Will negates (harmless)|no; see text}[As magic circle, but affecting a non-alignment subtype or outsider race.]
    \DeclareSpellDescription{Thaumaturgic Circle}{This spell functions as magic circle against chaos, evil, good, or law, but rather than stipulating an alignment descriptor, you can designate any one non-alignment subtype of outsider, including air, earth, fire, and water, as well as outsider races such as angels and devils.}
        
\DeclareSpell{Thought Echo}{illusion (glamer) []|V|1 round|touch|Targetsliving creature touched|1 minute/level (D)|Will negates (harmless)|yes (harmless)}[Replace surface thoughts with a mental echo.]
    \DeclareSpellDescription{Thought Echo}{When you cast this spell, you establish a mental echo of up to 25 words. Any attempt to read the surface thoughts of the target creature reads only this mental echo unless the caster succeeds at a caster level check (DC = 11 + your caster level). If you cast thought echo on yourself, the DC is instead equal to 15 + your caster level. In addition, when casting this spell upon yourself, you can change the echoed thoughts by taking a standard action and concentrating.  Each time you change the words echoed by the spell in this way, you reduce the spell's remaining duration by 1 minute. Thought echo d oes n ot i nterfere w ith t elepathy o r m ind reading effects that are capable of reaching thoughts deeper than surface thoughts.}
        
\DeclareSpell{Thought Shield I}{abjuration () []|V|1 immediate action|personal|Targetsyou|1 round||}[As an immediate action, gain a +4 bonus on Will saves against mind-affecting effects.]
    \DeclareSpellDescription{Thought Shield I}{Sensing an intrusion, you throw up a defense to protect your mind from attack or analysis. This grants you a +4 circumstance bonus on Will saving throws against mind-affecting effects. As long as this spell lasts, spells and effects that allow a creature to read your thoughts receive no information from you.}
        
\DeclareSpell{Thought Shield II}{abjuration () []|V|1 immediate action|personal|Targetsyou|1 round||}[As thought shield I, but +6 on Will saves.]
    \DeclareSpellDescription{Thought Shield II}{This functions as thought shield I, but the circumstance bonus on Will saves to resist mind-affecting effects increases to +6. This spell can be undercast.}
        
\DeclareSpell{Thought Shield III}{abjuration () []|V|1 immediate action|personal|Targetsyou|1 round||}[As thought shield I, but +8 on Will saves and stun creatures that read your thoughts for 1 round.]
    \DeclareSpellDescription{Thought Shield III}{This functions as thought shield I, but the circumstance bonus on Will saves to resist mind-affecting effects increases to +8. In addition, any creature that tries to read your thoughts while this spell lasts must succeed at a Will save or be stunned for 1 round. This spell can be undercast.}
        
\DeclareSpell{Thought Shield IV}{abjuration () []|V|1 immediate action|personal|Targetsyou|1 round||}[As thought shield III, but stun for 1d4 rounds and resist psychic crush spells.]
    \DeclareSpellDescription{Thought Shield IV}{This functions as thought shield III, but any creature that tries to read your thoughts while this spell lasts must succeed at a Will save or be stunned for 1d4 rounds. Additionally, if you succeed at your save against a psychic crush spell, you take no damage. This spell can be undercast.}
        
\DeclareSpell{Thought Shield V}{abjuration () []|V|1 immediate action|personal|Targetsyou|1 round/level; see text||}[As thought shield IV, but lasts 1 round per level.]
    \DeclareSpellDescription{Thought Shield V}{This functions as thought shield IV, except as noted above. This spell immediately ends if you fail a Will saving throw against a mind-affecting effect. This spell can be undercast.}
        
\DeclareSpell{Thoughtsense}{divination () [mind-affecting]|V,  M (a bit of dried brain tissue)|1 standard action|personal|Targetsyou|1 minute/level||}[Automatically detect nearby conscious creatures.]
    \DeclareSpellDescription{Thoughtsense}{You automatically detect and locate conscious creatures within 60 feet, as if you possessed the blindsight ability. Nondetection, mind blank, and similar effects can block this effect. Thoughtsense can distinguish between sentient (Intelligence 3 or greater) and non-sentient (Intelligence 1-2) creatures, but otherwise provides no information about the creatures it detects.}
        
\DeclareSpell{Tower Of Iron Will I}{abjuration () []|V|1 immediate action|10 ft.|Area10-ft.-radius emanation centered on you|1 round|none|yes (harmless)}[As an immediate action, gives creatures in a 10-ft. radius spell resistance against psychic magic and mind-affecting effects.]
    \DeclareSpellDescription{Tower Of Iron Will I}{You project a fortress of mental power that blocks out the psychic energy of others, granting mental strength and resiliency to all inside the area. All creatures inside the area gain spell resistance equal to 10 + double this spell's level against psychic magic and any mind-affecting effects. Creatures inside the area don't receive this protection against your spells or special abilities.}
        
\DeclareSpell{Tower Of Iron Will II}{abjuration () []|V|1 immediate action|10 ft.|Area10-ft.-radius emanation centered on you|2 rounds|none|yes (harmless)}[As tower of iron will I, but lasts 2 rounds.]
    \DeclareSpellDescription{Tower Of Iron Will II}{This functions as tower of iron will I, except as noted above. This spell can be undercast.}
        
\DeclareSpell{Tower Of Iron Will III}{abjuration () []|V|1 immediate action|10 ft.|Area10-ft.-radius emanation centered on you|3 rounds|none|yes (harmless)}[As tower of iron will I, but lasts 3 rounds.]
    \DeclareSpellDescription{Tower Of Iron Will III}{This functions as tower of iron will I, except as noted above. This spell can be undercast.}
        
\DeclareSpell{Tower Of Iron Will IV}{abjuration () []|V|1 immediate action|10 ft.|Area10-ft.-radius emanation centered on you|4 rounds|none|yes (harmless)}[As tower of iron will I, but lasts 4 rounds.]
    \DeclareSpellDescription{Tower Of Iron Will IV}{This functions as tower of iron will I, except as noted above. This spell can be undercast.}
        
\DeclareSpell{Tower Of Iron Will V}{abjuration () []|V|1 immediate action|10 ft.|Area10-ft.-radius emanation centered on you|5 rounds|none|yes (harmless)}[As tower of iron will I, but lasts 5 rounds.]
    \DeclareSpellDescription{Tower Of Iron Will V}{This functions as tower of iron will I, except as noted above. This spell can be undercast.}
        
\DeclareSpell{Unshakable Zeal}{enchantment (compulsion) [emotion,  mind-affecting]|S,  F (a silver circlet)|1 standard action|touch|Targetscreature touched|1 hour/level|Will negates (harmless)|yes (harmless)}[Grant benefits on future attempts after failed checks, and protect against fear and emotion effects.]
    \DeclareSpellDescription{Unshakable Zeal}{You fill the target with boundless enthusiasm and faith in its ultimate triumph. Whenever the target fails an attack roll, a save, a skill check, a concentration check, or an ability check, the target receives a +4 morale bonus on its next attempt at the failed check within 1 round (this includes attack rolls against the same foe, saving throws against the same ability from the same foe, and so on). In addition, when the target would be affected by a fear or emotion effect, it can instead dismiss unshakable zeal without spending an action to negate the effect on itself.}
        
\DeclareSpell{Wall Of Ectoplasm}{evocation () []|V,  S,  M (small bit of gauze)|1 standard action|close (25 ft. + 5 ft./2 levels)|Effectopaque sheet of ectoplasm up to 10 ft. square/level or a sphere or hemisphere with a radius of up to 1 ft./level|1 minute/level|none, and Will negates; see text|yes}[Wall of spirits blocks movement on Material Plane and Ethereal Plane and causes fear.]
    \DeclareSpellDescription{Wall Of Ectoplasm}{You draw forth a massive veil of ectoplasm that roils with restless spirits. Immovable once formed, the wall of ectoplasm is 1 inch thick per caster level and covers up to a 10-foot-square area per caster level (so a 10th-level wizard can create a wall of ectoplasm 100 feet long and 10 feet h igh, a wall 25 feet long and 40 feet high, or any other combination of length and height that does not exceed 1,000 square feet). The plane can be oriented in any fashion and need not be anchored, but must be created continuous and unbroken. The wall can't include squares that have creatures within them, even if the creatures are on the Ethereal Plane. The wall exists on both the Material Plane and Ethereal Plane, and blocks ethereal and incorporeal creatures from passing through it.  One side of the wall, selected by you, radiates a deeply foreboding and menacing aura from the writhing spirits within. The range of this mind-affecting fear effect is 10 feet from the wall's surface, and creatures that are in range when the wall is created or that later approach to within 10 feet must succeed at a Will save or become shaken (or panicked if they have 4 Hit Dice or fewer) for 1 round per your caster level.  Each 10-foot square of the wall has 2 hit points per inch of thickness. A section of the wall whose hit points drop to 0 is breached, but if a section is destroyed, the remaining ectoplasm in the wall immediately fills in any such hole created, reducing the wall's overall size by one 10-foot square but remaining a contiguous barrier. The wall can also take the form of a sphere or hemisphere whose maximum radius is 1 foot per caster level, and that is as hard to break through as the ectoplasmic plane form.}
        
\DeclareSpell{Withdraw Affliction}{necromancy () []|V,  S|1 standard action|touch|Targetsone afflicted creature|instantaneous|none|yes}[Remove an affliction and inflict it on another creature.]
    \DeclareSpellDescription{Withdraw Affliction}{You push your hand into the subject, then withdraw an affliction from the body of the sufferer as a tangible object. This extraction appears as a slimy mass of flesh. The target creature is cured as if affected by remove disease, remove curse, or neutralize poison.  In addition, this slimy mass allows you to deliver the affliction to another as a touch attack, as if holding the charge for a touch spell. It has the same effect as the original affliction, with the same saving throw and DC.}
    
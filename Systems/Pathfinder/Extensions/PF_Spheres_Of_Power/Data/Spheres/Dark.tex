\defineSphere{Dark}{You may create and manipulate darkness.}{You may create and manipulate darkness.}
\defineBaseAbility{Dark}{Darkness}{Create a \domath{5 * \value{charCLCount} / 2 + 10} ft radius sphere of darkness within medium range that lowers light by two steps. The sphere remains for as long as you concentrate, or for \arabic{charCLCount} minute\s{} for the cost of a spell point}{
	As a standard action, you may create a sphere of darkness with a radius of up to 10 ft + 5 ft per 2 caster levels, centered anywhere within Medium range. This darkness radiates from a central point, and cannot extend through walls. You must concentrate to maintain this sphere, but you may always spend 1 spell point as a free action to make the darkness last for 1 minute per level without concentration. You must remain within Medium range of the darkness to sustain it through concentration.\par
	In this darkness, bright light (including daylight) becomes dim light, imposing a 20\% miss chance to attacks. Normal light and dim light become absolute darkness. Sources of normal light only produce dim light in a 5-foot radius, and sources of dim light disappear. This is not subject to Spell Resistance, but creatures with Darkvision may see in this darkness as normal. You may always dismiss one of your \textit{darkness} effects as a free action.\par 
	Talents designated as (darkness) add additional effects to a sphere of \textit{darkness}. Only one such talent may be applied to an individual sphere of \textit{darkness}, but areas of \textit{darkness} with different effects may overlap. Individual \textit{darkness} effects do not stack with themselves.\par
	If a Light sphere effect is created inside a \textit{darkness} effect, the creator of the Light sphere effect must pass a magic skill check against the creator of the \textit{darkness} effect. If the check succeeds, the Light sphere effect functions normally. Otherwise, it functions as any other light source, as described above.}
\defineBaseAbility{Dark}{Meld}{Touch a target to grant them a boon while within your \textit{darkness}}{(Meld) talents allow the caster to grant himself and others the ability to interact with darkness in new ways. The caster must touch a target as a standard action to grant them the benefits of a (meld) talent.}
\defineTalent{Clearsight (meld)}{Spend a spell point to grant a target immunity to negative effects of your (darkness) talents for \arabic{charCLCount} hour\s.}{You may spend a spell point to grant a target immunity to all negative effects from your (darkness) talents for 1 hour per caster level. This does not grant the target the ability to see in your \textit{darkness} if it does not already possess the means to do so, but it does allow creatures with Darkvision to see in Pure Darkness.}{}
\defineTalent{Dark Slaughter (meld)}{Spend a spell point to grant a target 1d6 sneak attack damage while within your \textit{darkness} for \arabic{charCLCount} hour\s}{You may spend a spell point to grant the target the ability to make precision strikes for 1 hour per caster level. Whenever the target attacks a creature it is flanking, that is denied its Dexterity bonus to AC, or that is not able to perceive the creature (such as through a successful Stealth check), it deals an additional 1d6 points of precision damage to the target. This damage is not multiplied on a critical hit, but it does stack with a rogue's sneak attack. This only functions while within an area of your \textit{darkness}.}{}
\defineTalent{Darkvision (meld)}{Spend a spell point to grant Darkvision \arabic{chardarkvisiontalentdistance} ft., or increase existing darkvision range by \domath{\value{chardarkvisiontalentdistance}-30} ft. for \arabic{charCLCount} hour\s.}{You may spend a spell point to grant the target Darkvision 60 ft for 1 hour per caster level. If the target already possesses Darkvision, this instead increases the range of their Darkvision by 30 ft. You may gain this talent multiple times. Each time this is taken beyond the first, it increases the range of the granted Darkvision by 30 ft.}{\newCounter{chardarkvisiontalentdistance}\setCounter{chardarkvisiontalentdistance}{30*#1+30}}
\defineTalent{Disorienting Darkness (darkness)}{Creatures within \textit{darkness} must make a will save or become disoriented, determining the direction of movement randomly.}{You may create a \textit{darkness} effect that has a chance to disorient anyone who enters it. When a creature within this area attempts to move, or when a creature first enters this area, they must pass a Will save or become disoriented. A creature must attempt this Will save every time they enter the area of \textit{darkness}. If the creature fails this saving throw, they must roll a d8 to determine direction: 1 is their intended direction, with 2-8 rotating around the creature in a clockwise direction. The target moves that direction as if it were their intended course. The target does not realize they are off-course until their next turn or until they leave the area of \textit{darkness}.}{}
\defineTalent{Fearful Darkness (darkness)}{Creatures in \textit{darkness} must make a will save or become shaken for as long as they remain inside the \textit{darkness}.}{You may create a \textit{darkness} effect that plays with the fears of any who enter it. Creatures within this area of \textit{darkness} must pass a Will save or become shaken. Creatures who succeed at this save but remain in the \textit{darkness} must save again at the end of your subsequent turns. If any creature enters this area of \textit{darkness}, they must immediately save or become shaken. When a creature fails their saving throw, they remain shaken for as long as they remain within the area of \textit{darkness}. This is a mind-altering effect.}{}
\defineTalent{Feed on Darkness (meld)}{Spend a spell point to grant a creature fast healing 1 for \arabic{charCLCount} minute\s, which functions only when they are within your \textit{darkness}}{You may spend a spell point to grant a target fast healing 1 for 1 minute per caster level. This only functions so long as the target remains within an area of your \textit{darkness}.}{}
\defineTalent{Greater Darkness}{Spend an additional spell point to increase the radius of \textit{darkness} to \domath{5 * \value{charCLCount}+20} ft.\ifinlist{Wall of Darkness}{\charDarkTalentsList}{, or \domath{6+2*\value{charCLCount}} 10-ft. cubes with Wall of Darkness}{}.}{When creating an area of \textit{darkness}, you may spend an additional spell point to increase the affected area. This allow you to create a sphere of \textit{darkness} with a radius of up to 20 ft + 5 ft per caster level. If combined with the Wall of Darkness talent, this allows you to create six 10-ft cubes, + 2 cubes per caster level.}{}
\defineTalent{Hide In Darkness (meld)}{Spend a spell point to grant Hide in Plain Sight while within dim or lower light in your \textit{darkness}}{You may spend a spell point to grant the target the ability to make Stealth checks to hide in areas of dim light or darkness even while being observed. This lasts for 1 hour per caster level, and only functions when within an area of your \textit{darkness}.}{}
\defineTalent{Hungry Darkness (darkness)}{Creatures within \textit{darkness} must make a Fortitude save each round or take 1 CON damage.}{You may create a \textit{darkness} effect that saps away the lifeforce of those inside. Any creature inside the \textit{darkness} must pass a Fortitude save or suffer 1 point of Constitution damage. Creatures who remain within this area of \textit{darkness} must save at the end of your subsequent turns or suffer another point of Constitution damage. If a creature enters this area of \textit{darkness} after it is created, they must immediately save or also suffer this damage. A creature may only be affected by Hungry Darkness once per round, regardless of how many times they enter or exit the area that turn.}{}
\defineTalent{Lingering Darkness}{When you cease concentrating on a \textit{darkness} effect, you may choose to have the \textit{darkness} remain for two rounds before dissipating.}{When you cease concentrating on a \textit{darkness} effect, you may choose to have the \textit{darkness} remain for two rounds before dissipating.}{}
\defineTalent{Looming Darkness (darkness)}{Creatures within \textit{darkness} must make a Will save or take -\domath{\value{charCLCount}/ 5 + 1} to all saving throws while they remain in the \textit{darkness}.}{You may create a \textit{darkness} effect that erodes the resolve of those who enter it. Creatures within this area of \textit{darkness} must pass a Will save or suffer a -1 penalty to all saving throws so long as they remain within this area of \textit{darkness}. This penalty increases by 1 per 5 caster levels. If any creature enters this area of \textit{darkness}, they must immediately save or suffer this penalty. This is a mind-affecting effect.}{}
\defineTalent{Pure Darkness (darkness)}{Creatures within \textit{darkness} loose low-light vision, darkvision beyond 5 ft., and half the range of all other senses}{You may create a \textit{darkness} effect that negates low-light vision. Darkvision is reduced to 5 ft. In addition, all other senses (blindsight, scent, etc.) are reduced by half.}{}
\defineTalent{Quick Meld}{You may use (meld) talents on yourself as a swift action instead of a standard action.}{You may use (meld) talents on yourself as a swift action instead of a standard action.}{}
\defineTalent{Silent Darkness (darkness)}{\textit{Darkness} muffles sound, imposing a -\arabic{charCLCount} penalty to Perception checks to hear sounds from within}{You may create an area of darkness that dims sound as well as light. All Perception checks made to hear noises originating from within the area of your darkness suffer a penalty equal to your caster level.}{}
\defineTalent{Snagging Darkness (darkness)}{Creatures within \textit{darkness} must make a Reflex save each round or become entangled.}{You may create a darkness effect filled with dark tendrils that snare anything that passes. Creatures within this area of darkness must pass a Reflex save or become entangled. Creatures who succeed at this save but remain in the darkness must save again at the end of your subsequent turns. If any creature enters this area of darkness, they must immediately save or suffer the effects. An ensnared creature can attempt to escape its entanglement by making a Strength or Escape Artist check as a move action, with a DC equal to the Reflex save DC.}{}
\defineTalent{Step Through Darkness (meld)}{Spend a spell point to allow target to teleport up to 30 ft. as a move action between areas of your \textit{darkness} for \arabic{charCLCount} hour\s.}{You may spend a spell point to grant the target the ability to step into one patch of darkness and emerge in another. As a move action, the target may teleport up to 30 ft. This lasts for 1 hour per caster level. Both the location they are in and the location they are teleporting to must be within an area of your darkness.}{}
\defineTalent{Thick Darkness (darkness)}{\textit{Darkness} counts as difficult terrain.}{You may create a \textit{darkness} effect that counts as difficult terrain. Creatures move at half speed through your darkness, cannot run or charge, and cannot make 5-ft steps.}{}
\defineTalent{Wall of Darkness}{Instead of a sphere, you can spread your \textit{darkneess} over \domath{3+\value{charCLCount}} contiguous 10-ft. cubes.}{Rather than create a sphere of darkness, you may arrange your darkness as up to three 10-ft cubes, +1 cube per caster level. These cubes must be arranged contiguously, but otherwise may assume any shape. You must be able to perceive all areas your darkness will inhabit.}{}
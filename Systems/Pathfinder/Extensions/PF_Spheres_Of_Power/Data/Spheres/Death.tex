\defineSphere{Death}{You may command the powers of unlife.}{You may command the powers of unlife.}
\defineBaseAbility{Death}{Ghost Strike}{Ranged touch attack against Medium range target inflicts conditions. You begin with \textit{Exhausting Strike}, which causes the target to becomes fatigued for \arabic{charCLCount} round\s{} (fort negates), or exhausted for \arabic{charCLCount} minutes\s{} if you spend a spell point. (fort reduces to 1 round of fatigue)}{
	As a standard action, you may make a ghost strike, summoning negative energy and throwing it at a target within Medium range as a ranged touch attack. A ghost strike is considered a negative energy death effect, and as such has no effect on undead, constructs, elementals, and other creatures immune to such things (although some talents provide exceptions). Ghost strike effects do not stack with themselves.\par
	Some Death talents are designated (ghost strike), which provide you with additional types of ghost strikes. You gain the following ghost strike when you gain the Death sphere:\par
	\textit{Exhausting Strike:} The subject of your ghost strike becomes fatigued for 1 round per caster level (Fortitude negates). You may spend a spell point to increases this effect to making the target exhausted for 1 minute per caster level (Fortitude negates). On a successful save, the target is still fatigued for 1 round. Unlike regular fatigue and exhaustion, these conditions end as soon as the duration expires.}
\newcommand{\charReanimateduration}{\arabic{charCLCount} minute\s}
\newCounter{charReanimatedHD}\setCounter{charReanimatedHD}{0}
\defineBaseAbility{Death}{Reanimate}{Reanimate a corpse for \charReanimateduration{} as a skeleton or zombie. Control a maximum of \addToCounter{charReanimatedHD}{\value{charCLCount} * 2}\arabic{charReanimatedHD} HD}{
	As a standard action, you may touch an intact dead body and spend a spell point to \textit{reanimate} it as a zombie or skeleton (depending on the composition of the body in question) for 1 minute per caster level. This creature gains the zombie or skeleton template and obeys your commands, although only simple commands such as “go”, “stay”, “attack”, or “guard” are understandable. A \textit{reanimated} body cannot speak and has no knowledge or ability to think and so cannot answer questions or reveal anything it knew in life. When the duration expires, the body collapses until \textit{reanimated} again. It does not regain hit pints between reanimations. If reduced to 0 hp, the body collapses and is destroyed; it cannot be \textit{reanimated} again.\par
	You may have a total number of \textit{reanimated} creatures active at any one time whose combined Hit Dice does not exceed twice your caster level. If you attempt to \textit{reanimate} a creature that would push your total beyond this limit, your \textit{reanimated} creatures cease to be \textit{reanimated} sequentially from oldest to newest until the Hit Dice total is low enough to permit the new \textit{reanimated} creature. You cannot \textit{reanimate} a creature with more Hit Dice than twice your caster level, and creatures with more than 20 racial Hit Dice can never become skeletons or zombies.}
\defineTalent{Bleeding Wounds (ghost strike)}{You can \textit{ghost strike} to deal \HalfCLMinOne{} bleed damage, or \CLMinTwo{} bleed damage if you spend a spell point.}{You may make a ghost strike that deals 1 bleed damage per 2 caster levels (minimum: 1; no save). Targets take damage on the round they are hit, plus each additional round until the bleed effect stops (usually through the Heal skill or an application of magical healing). You may spend a spell point before making this ghost strike to improve this effect to 1 bleed damage per caster level (minimum: 2)}{}
\defineTalent{Command Undead (ghost strike)}{You may spend a spell point to \textit{ghost strike} to control an unintelligent undead or make an intelligent undead friendly to you for \arabic{charCLCount} minute\s{} (will negates).}{
	You may spend a spell point to make a ghost strike that grants you a measure of control over an undead creature (Will negates). For 1 minute per caster level, an unintelligent undead creature falls under your control or an intelligent undead become friendly toward you. You can give an intelligent undead creature orders, but you must win an opposed Charisma check to convince it to do anything it wouldn't ordinarily do. Retries are not allowed. An intelligent commanded undead never obeys suicidal or obviously harmful orders, but it might be convinced that something very dangerous is worth doing. \par
	Any act by you or your apparent allies that threatens the commanded undead (regardless of its Intelligence) breaks this effect. Your commands are not telepathic; the undead creature must be able to hear you. Intelligent undead remember they were manipulated and may seek revenge.}{}
\defineTalent{Curse (ghost strike)}{You may spend a spell point to \textit{ghost strike} causing a permanent curse. (see text for examples)}{
	You may spend a spell point to make a ghost strike that bestows	a permanent curse on the target (Will negates). Curses may be	removed with the Break Enchantment Life talent, upon your death, or by your choice as a free action, but otherwise cannot be dispelled. Choose one of the following effects to bestow upon the target (with GM permission you may invent your own curse, but it should not be more powerful than these): the target suffers 1d6 points of non-lethal damage every minute spent in bright light; the target becomes blind except when in areas of dim light or darkness; the target must eat and drink twice as much as normal or begin suffering from starvation; the target becomes vulnerable to a single energy type (this cannot affect a creature already immune to that energy type—apply vulnerability before protection or resistance); the target suffers the penalties (but not bonuses) of advancing to the next age category.}{}
\defineTalent{Cryptic Strike}{Make \textit{ghost strike} through a single weapon attack}{As a standard action, you may make a single ranged or melee attack coupled with a ghost strike. If the attack hits, the target is also affected by the ghost strike.}{}
\defineTalent{Drain (ghost strike)}{Spend a spell point to make a \textit{ghost strike} that deals \ifnum \value{charCLCount} > 19 1d6\else \ifnum \value{charCLCount} > 14 1d4 \else \ifnum \value{charCLCount} > 9 1d3 \else \ifnum \value{charCLCount} > 4 1d2 \else 1 \fi\fi\fi\fi{} negative level\ifnum \value{charCLCount} > 4 s \fi{} (fort negates) which last for \arabic{charCLCount} hour\s.}{
	You may spend a spell point to make a ghost strike that imposes 1 temporary negative level on the target for one hour per caster level (no save). This increases by 1 die size per 5 caster levels (1d2, 1d3, 1d4, and 1d6). Unlike with other ghost strikes, negative levels stack. While normally negative levels have a chance to become permanent and can kill a target whose negative levels equal its Hit Dice, these negative levels do not last long enough to become permanent, and if a negative level would reduce the creature to 0 Hit Dice, the creature instead takes 4 points of Constitution drain for the duration of the effect.\par
	If a negative level lasts longer than 1 day, the target must pass a
	Fortitude save per negative level or have the negative level become permanent. If this ability is used on an undead creature, it
	instead grants the creature 5 temporary hit points per negative
	level, which last for 1 hour.}{}
\defineTalent{Empowered Reanimate}{All creatures you \textit{reanimate} gain a +4 enhancement bonus to their Strength and Dexterity}{All creatures you \textit{reanimate} gain a +4 enhancement bonus to their Strength and Dexterity}{}
\defineTalent{Expanded Necromancy}{Can \textit{reanimate} a creature as a bloody or burning skeleton or a fast or plague zombie, but it counts as twice its usual HD}{When you \textit{reanimate} a creature, you may \textit{reanimate} it as a bloody skeleton, burning skeleton, fast zombie, or plague zombie. When reanimating a creature in this way, they count as twice their Hit Dice against the total amount you may have \textit{reanimated} at once.}{}
\defineTalent{Greater Ghost Strike}{When making a ghost strike, you may spend an additional spell point to form your ghost strike into a Close-range cone, allowing you to make an attack roll against every target within this area.}{When making a ghost strike, you may spend an additional spell point to form your ghost strike into a Close-range cone, allowing you to make an attack roll against every target within this area.}{}
\defineTalent{Greater Reanimate}{Increase the total Hit Dice of creatures you may have \textit{reanimated} at once by an additional \setCounter{charReanimatedHD}{\value{charReanimatedHD} / \value{charCLCount} - 2}\arabic{charReanimatedHD} per caster level.}{Increase the total Hit Dice of creatures you may have \textit{reanimated} at once by an additional 1 per caster level. You may select this talent up to 3 times.}{\addToCounter{charReanimatedHD}{#1*\value{charCLCount}}}
\defineTalent{Inflict Disease (ghost strike)}{Spend a spell point to \textit{ghost strike} to inflict a disease}{You may spend a spell point to make a ghost strike that causes the target to contract a disease (Fortitude negates). The subject contracts one of the following diseases: blinding sickness, bubonic plague, cackle fever, filth fever, leprosy, mindfire, red ache, shakes, or slimy doom. The disease is contracted immediately (the onset period does not apply). Use the disease's listed frequency and save DC to determine further effects.}{}
\defineTalent{Killing Curse}{Any target that fails 3 saves against your \textit{ghost strikes} within one minute instantly dies (fort negates)}{Your ghost strike can rip the very soul from the living. If a target fails their saving throws against your ghost strike 3 times within a 1-minute period, they immediately die (Fortitude negates). If a ghost strike does not allow a save, it is not usable with this talent.}{}
\defineTalent{Lingering Necromancy}{When you \textit{reanimate} a corpse or corpses, they remain for 1 hour per caster level instead of 1 minute per caster level.}{When you \textit{reanimate} a corpse or corpses, they remain for 1 hour per caster level instead of 1 minute per caster level.}{\def\charReanimateduration{\arabic{charCLCount} hour\s}}
\defineTalent{Manipulate Undeath (ghost strike)}{You may make a ghost strike that harms undead, dealing \if \value{HalfCL} > 1 \arabic{HalfCL}\else1\fi d8 damage (Will half). You may spend a spell point to instead heal the undead for this amount.}{You may make a ghost strike that harms undead, dealing 1d8 damage per 2 caster levels (minimum: 1d8, Will half). You may spend a spell point to instead heal the undead for this amount.}{}
\defineTalent{Mass Reanimate}{You may spend an additional spell point to \textit{reanimate} multiple creatures at once.}{When using your \textit{reanimate} ability, you may spend an additional spell point to \textit{reanimate} multiple creatures at once. All corpses to be \textit{reanimated} must be within Close range. Your Hit Dice limits apply to the total number you may \textit{reanimate} at once with this ability.}{}
\defineTalent{Necrotic Feeding (ghost strike)}{\textit{Ghost strike} kills a dying creature, gains you temporary HP and enhancements to STR and DEX}{You may spend a spell point to make a ghost strike that, when it strikes a target with -1 or fewer hit points, kills it instantly (Will negates). If the target fails their saving throw, you gain temporary hit points equal to twice the target's Hit Dice, as well as a +2 enhancement bonus to Strength and Dexterity, which increases to +4 if the target has 8 HD or more and +6 if the target has 16 HD or more. These effects last for 10 minutes per HD of the slain creature. Bonuses from multiple creature do not stack; only the highest bonuses apply.}{}
\defineTalent{Necrotic Senses}{See through the eyes of controlled undead}{As a full-round action, you may concentrate on one undead creature under your control. This allows you to perceive that creature's surroundings as if you were standing where that creature was. While you may use the creature's special sense (i.e., Darkvision, etc.), you must use your own Perception skill if making a check. Only targets completely under your control are valid; charmed undead are not truly under your control, and as such as such do not qualify.}{}
\defineTalent{Sickening (ghost strike)}{\textit{ghost strike} causes the target to becomes sickened for \arabic{charCLCount} round\s{} (fort negates), or nauseated for \arabic{charCLCount} minutes\s{} if you spend a spell point. (fort reduces to 1 round of sickened)}{You may make a \textit{ghost strike} that causes the target to be sickened for 1 round per caster level (Fortitude negates). You may spend a spell point to cause the target to instead become nauseated. On a successful save, the target is still sickened for 1 round.}{}
\defineTalent{Vampiric Strike (ghost strike)}{Spend a spell point to make a \textit{ghost strike} that deals \HalfCLMinOne d6 damage and gives you the same number of temporary HP}{
	You may spend a spell point to make a ghost strike that deals 1d6 damage per 2 caster levels to the target (minimum: 1d6) and grants yourself an equal number of temporary hit points that last 1 minute per caster level (Fortitude half). You cannot gain more temporary hit points in this manner than the subject's current hit points + their Constitution score. If you strike multiple targets at once with the same vampiric strike (for example, through the Greater Ghost Strike talent) you cannot gain more temporary hit points than 3 per caster level (minimum: 6). As always, temporary hit points do not stack.}{}
\defineTalent{Weakening (ghost strike)}{\textit{Ghost strike} inflicts a 1d4 point penalty to STR or DEX (fort negates) for \arabic{charCLCount} round\s. You may spend a spell point to increase the penalty by \HalfCLMinOne and change it to (fort half)}{You may make a ghost strike that inflicts a 1d4 point penalty to the target's Strength or Dexterity (your choice, Fortitude negates) for 1 round per level. You may spend a spell point to increase this reduction by half your caster level (minimum: 1) and cause a successful Fortitude save to only halve the effect instead of negate it. This cannot reduce the target's Strength or Dexterity scores to less than 1.}{}